\documentclass{book}\usepackage{knitr}

% Preamble

%%%%%%%%%%%%%%%%%%%%%%
%% PACKAGES
%%%%%%%%%%%%%%%%%%%%%%
\usepackage[twoside,letterpaper,width=6in,height=8in]{geometry}
\usepackage{siunitx} % format units properly
\usepackage{wrapfig}
\usepackage[margin=10pt,font=small,labelfont=bf]{caption} % format captions
\usepackage{booktabs} % nicer tables
\usepackage{subcaption} 
\usepackage{csquotes} % block quotes
\usepackage{tikz}
\usepackage[inline, shortlabels]{enumitem} % inline enumeration
%\usepackage[version=4]{mhchem}
\usepackage{graphicx} % packages are used to modify the text and create bling.
%\includegraphics{{Home/CAMPUS/mwl04747/github/Environmnental-Sciences-in-East-Asia/images/}}
\usepackage{textcomp}
\usepackage{gensymb}
\usepackage{natbib}
\usepackage{glossaries}
\usepackage{amsmath}%
\usepackage{amsfonts}%
\usepackage{amssymb}%

%\usepackage[super,square,comma]{natbib}
%\usepackage{float}
%\usepackage{appendix}
%\usepackage{chngcntr}
%\usepackage{etoolbox}
%\usepackage[usenames]{xcolor}% for commenting in color!

\RequirePackage{hyperref} % For hyperlinked cross-references
\hypersetup{
    colorlinks,
    citecolor=blue,
    filecolor=blue,
    linkcolor=blue,
    urlcolor=blue
}


%----------------------------------------------------------
\newtheorem{theorem}{Theorem}
\newtheorem{acknowledgement}[theorem]{Acknowledgement}
\newtheorem{definition}[theorem]{Definition}
\newtheorem{example}[theorem]{Example}
\newtheorem{exercise}[theorem]{Exercise}

\newtheorem{problem}[theorem]{Problem}
\newtheorem{remark}[theorem]{Remark}
\newtheorem{solution}[theorem]{Solution}
\newtheorem{summary}[theorem]{Summary}
\newenvironment{proof}[1][Proof]{\textbf{#1.} }{\ \rule{0.5em}{0.5em}}
%----------------------------------------------------------

\AtBeginEnvironment{subappendices}{%
\chapter*{Appendix}
\addcontentsline{toc}{chapter}{Appendices}
%\counterwithin{figure}{section}
%\counterwithin{table}{section}
}

\makeatletter
\newcommand{\chapterauthor}[1]{%
  {\parindent0pt\vspace*{-25pt}%
  \linespread{1.1}\large\scshape#1%
  \par\nobreak\vspace*{35pt}}
  \@afterheading%
}
\makeatother

\renewcommand{\glstextformat}[1]{\textbf{\color{blue}\em #1}}

\newcommand{\R}{\mathbb{R}}
\newcommand{\carbondioxide}{CO$_2$~}



\title{Environmental Issues in East Asia}
\author{EA30e Spring 2021}
\date{\today}
\IfFileExists{upquote.sty}{\usepackage{upquote}}{}
\begin{document}

\maketitle
\makeglossaries

\frontmatter
\tableofcontents


\chapter*{Preface}

\section{Guiding Principles}

Environmental issues in East Asia are not unique or particularly more prevasive than other parts of the world. However, the issues are born from particular histories that may contrast with other parts of the world and other parts of the world may be able to learn from. 

In this project, the students in EA030e (Spring 2021) have written a textbook that highlights examples of environmental processes. Each student contributed to one theme, composed of two examples that highlight environmental issues of East Asia. 

\subsection{Context and Positionality}

As students in a college course located in Southern California, we approach the project with...


Our goal is not to call out environmental issues in East Asia, but to point to linkages of how a range of globalized economy contribute to these environmental problems. 

In the end, it would be useful for us to acknowledge we have some capacity to address these how these global linkages could be modified to reduce these environmental issues.

We are not experts, but learning... if there are errors please let us know... We recommend that suggestions be submitted via a github pull request.

\subsection{Goals}

Processes across horizontal boundaries define many environmental patterns that frame human interactions with the environment. How do humans impact processes that cross these boundaries and how do humans influence these ecosystem interface?

\subsection{Rationale}

We hope to learn more about the how environmental issues are expressed in different parts of the world and to what extent can we learn from this work. 

\subsection{Activity}

Each group will be composed of two students, that will become experts and teach their classmates on the topic. 

\section{East Asia and the World}







\section{Acknowledgments}

Everyone in the world!




\chapter{\LaTeX Guide}\label{ch:guide}

\subsection*{Why Learn \LaTeX?}

In the past, I used \LaTeX to make publication quality text. In fact, many prefer writing in \LaTeX because they can focus on the text and avoid worrying about formatting. However, it is NOT WYSIWYG (``what you see is what you get'') word processor. In reality, the processing or compiling is a separate step. 

Nevertheless, the quality of the output and ability to integrate with R (or Python) allows us to have an exceptional tool to make reproducible documents. 

\subsection*{How to Learn \LaTeX?}

There are several ways to learn \LaTeX. I suggest you find a decent tutorial to get the basics. For example, here are some suggestions:

\begin{itemize}
  \item \href{https://www.overleaf.com/learn/latex/Learn_LaTeX_in_30_minutes}{Learning \LaTeX in 30 minutes}
\end{itemize}

If you are like me and can't remember commands very well, then here's a \href{https://wch.github.io/latexsheet/latexsheet-0.png}{cheet sheet} that might be helpful. 

\subsubsection{R Chunks}

To create effective graphics, each chapter will have a rchunk that creates a graphic for the chapter. To review and learn R, here are some resources: 

\begin{itemize}
  \item \href{www.tbd.com}{Marc's Video Description}
  \item \href{https://rmd4sci.njtierney.com/}{RMarkdown for Scientists (super helpful!)}
  \item \href{https://rmarkdown.rstudio.com/lesson-1.html}{R Studio Tutorial}
  \item \href{https://rstudio.com/wp-content/uploads/2016/03/rmarkdown-cheatsheet-2.0.pdf?_ga=2.107420162.161662097.1613074083-214354297.1613074083}{R Studio's Cheatsheet}
  \item \href{https://bookdown.org/yihui/rmarkdown-cookbook}{R Markdown Cookbook -- Robust Source}
\end{itemize}


\subsection*{Noting Your Contribution}

Because this is an ongoing project, you should record your contribution to each chapter -- but also let go of these contributions at some point; Others might revise and their authorship might take some precedence, so you should both invest in the product but also be willing to detach from the final outcome as others contribute. This will feel uncomfortable at times, but please note from the beginning this is a social process and as such subject to negotiation. Please be generous to the authors that laid the foundation and be respectful of those that follow. 

\section{Setting Up Book Project--Type Setting w/ \LaTeX}\label{sec:settingup}

\subsection{Latex Book Class}

Currently, the text is written using the standard book class. %However, in 2019, I (Los Huertos) will convert the format to a Tufte book class. 

\subsection{Structuring the Text with Nested Hierarchies}

Contributors divide their contributions into sections and subsections. This format allows a consistent approach to structuring the text and forcing themes to be organized in blocks that can be used to organize the overall text. We use section, subsection, and subsubsection to break up the topic into bite sizes. 

To accomplish this, contributors use the \verb"\section{Section}" command for major sections, and the \verb"\subsection{Subsection}" command for subsections, and a similar approach for subsubsections. 

NOTE: for each nested level, it MUST be followed by the lowest level in the section before a paragraph is started -- in contrast to what is shown above!

NOTE: We may dispense with subsubsections in the future to provide a less blocky structure, but for now they remain useful. 

\subsection{Font Changes}

We can use various methods to alter the typeset: \emph{Emphasize}, \textbf{Bold}, \textit{Italics}, and \textsl{Slanted}. We can also typeset \textrm{Roman}, \textsf{Sans Serif}, \textsc{Small Caps}, and \texttt{Typewriter} texts.  Look online to see the commands to accomplish these changes. 

You can also apply the special, mathematics only commands $\mathbb{BLACKBOARD}$, $\mathbb{BOLD}$, $\mathcal{CALLIGRAPHIC}$, and $\mathfrak{fraktur}$. Note that blackboard bold and calligraphic are correct only when applied to uppercase letters A through Z.

You can apply the size tags -- Format menu, Font size submenu -- {\tiny tiny}, {\scriptsize scriptsize}, {\footnotesize footnotesize}, {\small small}, {\normalsize normalsize}, {\large large}, {\Large Large}, {\LARGE LARGE}, {\huge huge} and {\Huge Huge}.

You can use the \verb"\begin{quote} etc. \end{quote}" environment for typesetting short quotations. Select the text then click on Insert, Quotations, Short Quotations:

\begin{quote}
The buck stops here. \emph{Harry Truman}

Ask not what your country can do for you; ask what you can do for your
country. \emph{John F Kennedy}

I am not a crook. \emph{Richard Nixon}

I did not have sexual relations with that woman, Miss Lewinsky. \emph{Bill Clinton}
\end{quote}

The Quotation environment is used for quotations of more than one paragraph. Following is the beginning of description of \LaTeX from \emph{Wikipedia}:

\begin{quotation}
LaTeX (/ˈlɑːtɛx/ LAH-tekh or /ˈleɪtɛx/ LAY-tekh, often stylized as \LaTeX) is a software system for document preparation. When writing, the writer uses plain text as opposed to the formatted text found in ``What You See Is What You Get'' word processors like Microsoft Word, LibreOffice Writer and Apple Pages. The writer uses markup tagging conventions to define the general structure of a document (such as article, book, and letter), to stylise text throughout a document (such as bold and italics), and to add citations and cross-references. A \TeX distribution such as \TeX Live or MiK\TeX is used to produce an output file (such as PDF or DVI) suitable for printing or digital distribution.

LaTeX is widely used in academia for the communication and publication of scientific documents in many fields, including mathematics, statistics, computer science, engineering, physics, economics, linguistics, quantitative psychology, philosophy, and political science. It also has a prominent role in the preparation and publication of books and articles that contain complex multilingual materials, such as Sanskrit and Greek. \LaTeX uses the TeX typesetting program for formatting its output, and is itself written in the TeX macro language.''
\end{quotation}

Use the Verbatim environment if you want \LaTeX\ to preserve spacing, perhaps when
including a fragment from a program such as:
\begin{verbatim}
#read csv data  // read data into R
my.dataframe <- read.csv(file.choose())   // read data from a popup window.

str(my.dataframe) // display data structure

\end{verbatim}
(After selecting the text click on Insert, Code Environments, Code.)


\subsection{Mathematics and Specialized Characters}\label{sub:mathchar}

\subsubsection*{Warning: Special Characters}

When you use percent and ampersand symbols, hash tags, and other non-standard ASCII characters, \LaTeX will be very uncooperative. \LaTeX~doesn't like a range of characters or they reserved for special behavior. So, do yourself a favor and make sure you understand that these are used for special typesetting functions. To use them you have to ``escape'' and use commands to get them to do what you might usually expect!  

The following symbols \$, \%, \#, \&, \`e, \~n, `` and '' do not reflect the key stroke you might expect. 

For example, the \& is used for tabs in a table environment. \% is used to make comments, thus stuff behind a \% is ignored. There are lots of others, but these come up the most. If you want to show use the ampersand or one of these characters, put a backslash in front of the dollar sybmol, e.g. \textbackslash\$. See Table \ref{tab:tableofsymbols}.

If you want to a superscript (raised to 3nd power), we can create text in math mode, with \$ to start and end the text in math mode, e.g. m$^3$ is written in \LaTeX as m\$\^{}3\$. A subscript uses an underscore, x\$\_1\$ creates x$_1$. If you need more than one character as a subscript or superscript then enclose the content in curly brakets, e.g. x$^{2c}$ (x\$\^{}\{2c\}\$) and t$_{step}$ (t\$\_\{step\}\$).

\begin{table}[h]
\caption{Table of Symbols in \LaTeX}
\label{tab:tableofsymbols}
\begin{tabular}{|ll|ll|} \hline
Symbol  & \LaTeX code & Symbol & \LaTeX code \\ \hline\hline
\&  & \textbackslash\&  & 
\$  & \textbackslash\$ \\
``  & \`{}\`{} & 
''  & \'{}\'{} \\
mg~L$^{-1}$ & mg$\sim${}L\$\^{}\{-1\}\$ & 
    & \\ 
\hline
\end{tabular}

\end{table}

\subsection{Creating equations}

One of the most powerful parts of \LaTeX is how it can be used to write complex equations, with all those symbols and Greek letters! This can be done inline $y = mx + b + \epsilon$ for fairly simple equations, or set apart for more complex equations:

\begin{equation}
\int_0^\infty e^{-x^2} dx=\frac{\sqrt{\pi}}{2}
\end{equation}

\subsubsection{Theorems, etc}
\begin{theorem}
(The Currant minimax principle.) Let $T$ be completely continuous selfadjoint operator
in a Hilbert space $H$. Let $n$ be an arbitrary integer and let $u_1,\ldots,u_{n-1}$ be
an arbitrary system of $n-1$ linearly independent elements of $H$. Denote
\begin{equation}
\max_{\substack{v\in H, v\neq
0\\(v,u_1)=0,\ldots,(v,u_n)=0}}\frac{(Tv,v)}{(v,v)}=m(u_1,\ldots, u_{n-1})
\label{eqn10}
\end{equation}
Then the $n$-th eigenvalue of $T$ is equal to the minimum of these maxima, when
minimizing over all linearly independent systems $u_1,\ldots u_{n-1}$ in $H$,
\begin{equation}
\mu_n = \min_{\substack{u_1,\ldots, u_{n-1}\in H}} m(u_1,\ldots, u_{n-1}) \label{eqn20}
\end{equation}
\end{theorem}
The above equations are automatically numbered as equation (\ref{eqn10}) and
(\ref{eqn20}).


\subsection{Lists Environments: Making bulletted, numbered, description lists}

We use special commands to create an itemized list.

You can create numbered, bulleted, and description lists
(Use the Itemization or Enumeration buttons, or click on the Insert menu
then chose an item from the Enumeration submenu):

\begin{enumerate}
\item List item 1

\item List item 2

\begin{enumerate}
\item A list item under a list item.

\item Just another list item under a list item.

\begin{enumerate}
\item Third level list item under a list item.

\begin{enumerate}
\item Fourth and final level of list items allowed.
\end{enumerate}
\end{enumerate}
\end{enumerate}
\end{enumerate}

\begin{itemize}
\item Bullet item 1

\item Bullet item 2

\begin{itemize}
\item Second level bullet item.

\begin{itemize}
\item Third level bullet item.

\begin{itemize}
\item Fourth (and final) level bullet item.
\end{itemize}
\end{itemize}
\end{itemize}
\end{itemize}

\begin{description}
\item[Description List] Each description list item has a term followed by the
description of that term.

\item[Bunyip] Mythical beast of Australian Aboriginal legends.
\end{description}

\subsection{Theorem-Like Environments}

The following theorem-like environments (in alphabetical order) are available
in this style.

%\begin{acknowledgement}
%This is an acknowledgement
%\end{acknowledgement}

\begin{example}
This is an example
\end{example}

\begin{exercise}
This is an exercise
\end{exercise}


%\begin{proof}
%This is the proof of the lemma.
%\end{proof}

%\begin{notation}
%This is notation
%\end{notation}

%\begin{problem}
%This is a problem
%\end{problem}

%\begin{proposition}
%This is a proposition
%\end{proposition}

%\begin{remark}
%This is a remark
%\end{remark}

%\begin{summary}
%This is a summary
%\end{summary}

\begin{theorem}
This is a theorem
\end{theorem}

%\begin{proof}
%[Proof of the Main Theorem]This is the proof.
%\end{proof}

%\subsubsection{``Child'' Rnw Contributions}

%This is a chapter that we can input into the text... you will each create a chapter without the preamble and begin and end document... that can be integrated into a single book! 

\subsection{Peer Review Commenting}

You can put your comments in square brackets and in color for things that need help. \textcolor{red}{[This section is confusing, I am not sure what commenting means.]}

\subsection{Adding Figures, etc}

\subsubsection{Using Rnw Files}

To generate R figures, we use R chunks in and Rnw file, where the text is integreated. When we compile into a PDF, the program converts the files into TeX files and then combineds them into a single pdf. 

For each chapter, we create a ``child'' document and Marc will help you create that text when you begin. 

\subsubsection{Creating a floating figure}

This is my floating figure (Figure \ref{fig:plot}).

\begin{figure}

\caption{My plot's caption is here!}
\label{fig:plot}
\end{figure}

\subsubsection{Using R to Create Effective Figures}

R Markdown can be a very powerful tool to integrate R code, figures and text. Making high quality figures that are both clear and aestically pleasing will be something that we need to think about it. 

\begin{itemize}
  \item Axis Labels -- Labelled with clarity 
  \item Axis Text -- Size, Orientation 
  \item Captions (usually better than titles)
  \item References connecting labels to references
  \item ADA accessible (e.g. color impairment mitigation)
\end{itemize}

For example, here's code to generate a pretty good figure: 



\begin{knitrout}
\definecolor{shadecolor}{rgb}{0.969, 0.969, 0.969}\color{fgcolor}\begin{kframe}


{\ttfamily\noindent\bfseries\color{errorcolor}{\#\# Error in file(file, "{}rt"{}): cannot open the connection}}

{\ttfamily\noindent\bfseries\color{errorcolor}{\#\# Error in createDataPartition(., p = 0.8, list = FALSE): object 'maunaloa' not found}}

{\ttfamily\noindent\bfseries\color{errorcolor}{\#\# Error in eval(expr, envir, enclos): object 'maunaloa' not found}}

{\ttfamily\noindent\bfseries\color{errorcolor}{\#\# Error in eval(expr, envir, enclos): object 'maunaloa' not found}}

{\ttfamily\noindent\bfseries\color{errorcolor}{\#\# Error in eval(expr, envir, enclos): object 'maunaloa' not found}}

{\ttfamily\noindent\bfseries\color{errorcolor}{\#\# Error in is.data.frame(data): object 'maunaloa' not found}}

{\ttfamily\noindent\bfseries\color{errorcolor}{\#\# Error in summary(model): object 'model' not found}}

{\ttfamily\noindent\bfseries\color{errorcolor}{\#\# Error in predict(., test.data): object 'model' not found}}

{\ttfamily\noindent\bfseries\color{errorcolor}{\#\# Error in mean((pred - obs)\textasciicircum{}2, na.rm = na.rm): object 'predictions' not found}}\end{kframe}
\end{knitrout}

In the case of Figure~\ref{fig:co2-graphic}, we can a create a figure that has all of the characteristics listed above, except perhaps ADA. Creating a "alt text" for the figure is something we might want to consider -- For now a decent caption about what the reader is seing is super helpful. 

\begin{figure}
\begin{knitrout}
\definecolor{shadecolor}{rgb}{0.969, 0.969, 0.969}\color{fgcolor}\begin{kframe}


{\ttfamily\noindent\bfseries\color{errorcolor}{\#\# Error in ggplot(train.data, aes(decimal.date, average)): object 'train.data' not found}}\end{kframe}
\end{knitrout}
\caption{Carbon Dioxide Concentrations (Mauna Loa, HI). Data demonstrate the CO$2$ concentrations are increases, but that a seasonal impact is embedded in the long-term trend. Source: Scripps/NOAA.}
\label{fig:co2-graphic}
\end{figure}

\subsection{Using Boxes}

\fbox{
\begin{minipage}[c]{.9\textwidth}
\subsection{minibox X}

Some text
\end{minipage}
}

\subsection{Cross-References, Citations, and Glossaries}

\subsubsection{Cross-References}

We can cross-reference sections (e.g. Section~\ref{ch:critical-zone}  or figures (Figure~\ref{fig:maunaloa}) using several methods. I suggest you look at the this Rmd file to see how I did it in these examples.

You can also create links to URLs or hyperlinks, e.g. \url{http://texblog.org}. However, if these addresses change, then the link will break, so I suggest you only link to internal references.

\subsubsection{Bibliography generation}

There will be two steps to cite our sources. First, we need to add the reference to a database, or bib file. This is titled 'References.bib' and is located in the main folder in our respository. When you add information to the bib file, be sure to paste in the reference using a bibTeX format. 

Second, we'll need to place in-line citations, using \verb"\citep{knitr}", which produces \citep{knitr}, by using a key, which is knitr in this case. 

For example, you might write, ``This document was produced in RStudio using the knitr package (\citep{knitr}). Also try \verb"\citet{LosHuertos2017OverviewR}" to create use the author name as the subject: \citet{LosHuertos2017OverviewR} wrote an guide to help students learn R. 

Note: You will see these citations automatically put in alphabetic order in the Bibliography at the end of the PDF. 

%Currently, we are using the ecology.bst, but it has trouble with misc type of references, so I will changing this in 2019. 

\subsubsection{Creating glossary words}
 
\newglossaryentry{peat}{
	name=peat, 
	description={is cool.}
}

\begin{definition}
This is a definition and the word is use in an glossary, e.g. \gls{peat}. \Gls{peat} is when you want to capitalize the defined word without having to re-define a capitalized version, the only downside of case sensitivity in \LaTeX.
\end{definition}


\chapter{Template Chapter Title}\label{ch:template}

\chapterauthor{Chapter Author Name}

\footnote{Statement of Contributions-- For example, ``The chapter was first drafted by Marc Los Huertos (2021). The author recieved valuable feedback from X, and Y and Z to improve the chapter. Slater revised the chapter in 2022 with suggestions from Cater.'' Note: I am still working on the formatting for this to improve it.}

\section{Section Heading}% Avoid putting text between section and subsection headings.

\subsection{Subsection Headings} % Avoid putting text between subsection and subsubsection headings. Not applicable if you don't have subsections!

Some text here...The hierarchy structure is described in the Author Guide, Section~\ref{sec:settingup} -- NOTE: This is a section cross reference.

if you cut and paste, be sure to make sure you don't include formatted characters outside the ASCII values. See Author Guide, Section~\ref{sub:mathchar}. NOTE: This is a subsection cross reference.


\subsubsection{Optional Subsubsection Headings}\label{subsub:optionalsubsub} % Again try to avoid putting text between the subheadig and the subsubheading to main a structural consistency.

some text here.... and a subsubsection cross reference (See Section \ref{subsub:optionalsubsub}).

\section{Goals of this template}

This template will NOT teach you how to use \LaTeX! To accomplish that, we'll rely on some great online resources that you can find on in Chapter~\ref{ch:guide}. 

Instead this section of the document is designed to demonstrate how our textbook will look, feel, and ultimately how we contribute to the project.

This document also compiles all of our projects into a single PDF, where each chapter is composed of a input tex file.

\section{Here's figure}

\subsection{R Created Figures}

First we create an R chunk and add some code. In this case, I created a floating figure which can be referenced (Figure~\ref{fig:pressure})!  

\begin{figure}
\begin{knitrout}
\definecolor{shadecolor}{rgb}{0.969, 0.969, 0.969}\color{fgcolor}\begin{kframe}
\begin{alltt}
\hlkwd{plot}\hlstd{(pressure)}
\end{alltt}
\end{kframe}
\includegraphics[width=\maxwidth]{figure/fig:pressure-1} 

\end{knitrout}
\caption{Figure Caption...we should turn "echo=False" in the R chunk options, but I left it true for now. (source: ??)} % define the caption, then the label.
\label{fig:pressure}

\end{figure}

\subsection{Floating Figures from External Sources}

All figures and images that are imported should be put into the "images" sudirectory to keep stuff organized. Even better to create a subdirectory with your images, but we can naviagate as we go.

Figure \ref{fig:vadose} is a good example of inserting an image from an external source.

\begin{figure}
\includegraphics[width=\linewidth]{images/Lee-Vadose}
\caption{Vadose zone is neato (Source: \citet{lee2017fifty}).}
\label{fig:vadose}
\end{figure}

In this case, I had to specify the width so it would fit on the page!  See the Rnw file for the code. Notice, I was also abel to ``reference'' the figure in the text.

\section{Adding Citations}

See the Guide, as well, but my video is probably the most helpful.


Generally, there are many environmental trends in Asia \citep{imura2005urban}.

\citet{imura2005urban} describes the how urbanization has affected the hydrology of East Asia. 
 

\chapter{Plastic}

\chapterauthor{Nora}

$\rightarrow$

chekcing on this today, 4-020-2021
pull request test 1.2 

changes at 3 pm, 4-1-21

changes at 3:20 pm, 4-1-21

changes at 3:29 pm 4-1-21

changes at 3:33 pm

\section{What the Polar Vortex and why do we care?}

test commit and pull request 


\subsection{What Factors Drive Land Use Change?}





\mainmatter


\chapter{The Earth System}\label{earthsystem}

\chapterauthor{Marc Los Huertos}

\section{The Sun's Energy and the Earth's Temperature}

The temperture of the Earth's surface is the result of a balance -- the energy entering the atmosphere and the leaving the atmosphere. Most of this energy is in the form of light or electromagnetic radiation (Figure~\ref{fig:earthbudget}). 

\begin{figure}
\includegraphics[width=\linewidth]{images/earth-system/earth-rad-budget-nasa-erbe.png}
\caption{caption}
\label{fig:earthbudget}
\end{figure}

Light enters the atmosphere, where some is absorbed and some is reflected. Light interacts in different ways with land, oceans, and vegetation, which is beyond the scope of our project. The ``quality'' of light changes through these processes. 

\subsection{The Spectrum of Light Entering and Exiting the Earth's Surface}

As the sun's electromagnetic radiation interacts with the Earth's Atmosphere, certain wavelengths are absorbed and filtered out (Figure~ \ref{fig:em-entering}).

\begin{figure}
\includegraphics[width=\linewidth]{images/earth-system/em-radiation-atmosph-depth-stsci.jpg}
\caption{Various wavelengths of solar electromagnetic radiation penetrate Earth's atmosphere to various depths. Fortunately for us, all of the high energy X-rays and most UV is filtered out long before it reaches the ground. Much of the infrared radiation is also absorbed by our atmosphere far above our heads. Most radio waves do make it to the ground, along with a narrow `window' of IR, UV, and visible light frequencies. Source: STCI/JHU/NASA.}
\label{fig:em-entering}
\end{figure}

\subsection{The Atmosphere and Greenhouse Effect}



\section{Carbon Biogeochemistry}

\subsection{Long and Short Time Scales}

The carbon cycle processes occur at wide range of temporal scales from hundreds of millions of years to seasons of the year. These have been referred to as long and short carbon cycles. However, for our purposes, I will call them ``geologic carbon'' and ''biosphere carbon'' processes. 

\subsection{Rock Cycle and Geologic Carbon}

The carbon cycle describes changes in the fluxes and reservoirs of carbon in the Earth system. On very long time-scales, millions of years, the primary reservoirs of carbon are the atmosphere, ocean, and rocks (limestone). Carbon moves between these reservoirs through volcanic outgassing, silicate weathering, and limestone sedimentation. The carbon cycle is linked to Earth's energy balance through atmospheric carbon in the form of \carbondioxide, a greenhouse gas.

\subsubsection{Mountains and Erosion}

\ref{fig:carbonpools}

\begin{figure}
\includegraphics[width=\linewidth]{images/earth-system/Carbon-reservoirs-and-cycles-in-the-Earth.jpg}
\caption{Carbon reservoirs and cycles in the Earth. The figure shows short-and long-term cycles; biosphere and geologic carbon reservoirs and fluxes, and the relative sizes and residence times (y axis) of respective carbon. Numbers in brackets refer to the total mass of carbon in a given reservoir, in Pg C (1Pg C = 10$^{15}$ g carbon). All reservoirs are pre-industrial. Abbreviations: C org = organic carbon; DIC = dissolved inorganic carbon; MOR = mid ocean ridge; seds = sedimentary rocks. Adapted from Lee et al. (2019 And references therein).}
\label{fig:carbonpools}
\end{figure}

\subsubsection{Subduction Burial and Carbon Recycling}

Figure~\ref{fig:longtermcarbon}

\begin{figure}
\includegraphics[width=\linewidth]{images/earth-system/long-term-carbon.png}
\caption{Schematic of the long-term carbon cycle (from Bice, 2001)}
\label{longtermcarbon}
\end{figure}

\subsection{Photosynthesis, Respiration, and Biosphere Carbon}

\subsubsection{Soil Respiration and the Soil Profile}

Carbon in soils is respired -- but different pools might have different rates of respiration. Sometimes these pools are distinquished as an active soil organic carbon pool and slow soil organic carbon pool. Although the reference of ``slow'' causes confusion with long-term, geologic carbon, but soil organic carbon remains a component of what we are refering to as biosphere carbon. 

The surface of the soil tends to have more SOC and microbes that can use that carbon for respiration. Lower down in the soil profile, we tend to see lower amounts of SOC and lower microbial biomass (Figure~\ref{fig:soilcarbon}. In addition, soils in the lower part of the profile tend to have more aggregation that protects SOC from microbial attack, thus a key area that soil carbon can seqeustor carbon. 

In addition to these microbial biomass and aggregate patterns, the microbes aree more senstive to temperature changes near the surface as measured by Q10 -- the rate of biochemical processes with a 10 degree C increase in temperature. Thus, soil processes, such as respiration, is likely to increase more near the surface with global warming that the lower part of the soil profile.  

\begin{figure}
\includegraphics[width=\linewidth]{images/earth-system/Q10-SOC-Regulation.jpg}
\caption{Regulatory Mechanisms of the Temperature Sensitivity of Soil Organic Matter Decomposition in Alpine Grasslands (Source: \citet{Qineaau1218, CAS2021researchers}).}
\label{fig:Q10-SOC}
\end{figure}


\section{Fossil Fuels and Carbon Dioxide Trends}\label{sec:fossilfuels}

As part if the industrial revolution, our energy sources have put more \carbondioxide from the biosphere (soils and forests) and geologic carbon (coal, petroleum). 

\subsection{The Signal of Geologic and Biosphere Carbon in Atmosphere}

The combined contribution from geologic and biosphere carbon in the atmosphere is clearly documented from numerous sources. First, look at data collected at the Mauna Loa where \carbondioxide measurements have been taken continuously since the late 1950s. 

Figure~\ref{fig:maunaloa2}

\begin{figure}
\begin{knitrout}
\definecolor{shadecolor}{rgb}{0.969, 0.969, 0.969}\color{fgcolor}\begin{kframe}


{\ttfamily\noindent\bfseries\color{errorcolor}{\#\# Error in ggplot(train.data, aes(decimal.date, average)): object 'train.data' not found}}\end{kframe}
\end{knitrout}
\caption{Carbon Dioxide Measure on Mauna Loa, HI}
\label{fig:maunaloa2}
\end{figure}






\chapter{Monsoons and East Asia Climates}

\section{Temperature Gradients and Latitude}





\chapter{Critical Zone}\label{ch:critical-zone}

\chapterauthor{Marc Los Huertos}\footnote{The chapter was first drafted by Marc Los Huertos (2021). The author recieved valuable feedback from X, and Y and Z to improve the chapter.}

\section{What is the Critical Zone}

The crticical zone refers the the portion of the Earth's skin where the zone where rock meets life. The Critical Zone supports all terrestrial life.

The critical zone includes the following:

\begin{itemize}
  \item A permeable layer from the tops of the trees to the bottom of the groundwater;
  \item An environment where rock, soil, water, air, and living organisms interact and shape the Earth's surface;
  \item Water and atmospheric gases move through the porous Critical Zone, and living systems thrive in its surface and subsurface environments, shaped over time by biota, geology, and climate.
\end{itemize}

All this activity transforms rock and biomass into the central component of the Critical Zone - soil; it also creates one of the most heterogenous and complex regions on Earth.

Its complex interactions regulate the natural habitat and determine the availability of life-sustaining resources, such as food production and water quality.

These are but two of the many benefits or services provided by the Critical Zone. Such `Critical-Zone Services' expand upon the benefits provided by ecosystems to also include the coupled hydrologic, geochemical, and geomorphic processes that underpin those ecosystems.

\begin{figure}
\includegraphics[width=\textwidth]{images/critical-zone/criticalzone.jpg}
\caption{The Critical Zone is an interdisciplinary field of research exploring the interactions among the land surface, vegetation, and water bodies, and extends through the pedosphere, unsaturated vadose zone, and saturated groundwater zone. Critical Zone science is the integration of Earth surface processes (such as landscape evolution, weathering, hydrology, geochemistry, and ecology) at multiple spatial and temporal scales and across anthropogenic gradients. These processes impact mass and energy exchange necessary for biomass productivity, chemical cycling, and water storage.}
\label{fig:criticalzone}
\end{figure}

\subsection{What are the environmental implications of the Critical Zone?}

The critical zone as a concept and as a material space pushes us to think of the porousity of the Earth's surface --- the gas and fluid flows through rocks, soils, and plants. We can begin to appreciate the complexity of the transport and fate of chemical pollutants as they enter the soil and become part of the vadose zone and perhaps the ground water table -- moving with water and diffusing through the water, simultaneously.

\section{Hydrologic Aspects}

\subsection{The Vadose Zone}

Jeji is a volcanic island is located some XX km south of the Korean Penisula. Water runs off the steep slopes quickly and water supplies are limited on the island. To adddress this...\citet{lee2017fifty}.

\begin{figure}
\includegraphics[width=\linewidth]{images/critical-zone/Lee-Vadose.png}
\caption{... (Source: \citep{lee2017fifty}).}
\label{fig:vadose2}
\end{figure}



\chapter{Land Use in East Asia}

chapterauthor{Samantha Beaton}

What is Land Use Change?

What Factors Drive Land Use Change?

How Land Use Change is Measured and Quantified

Integration of sociology

with data science: spatial data compiled from aerial photos, Landsat satellite images, topographic maps, GPS data, etc.

Requires classification and division of land-space types

Ecological Effects of Land Use Change on Soil, Air, and Water

\section{Impacts on Soil}

Deforestation and soil degradation

lack of stability (erosion) and loss of carbon sequestration potential

Forests


coupled with monoculture agriculture

Example Case Study: representative of monoculture agriculture-rice paddies in SE Asia (potentially\ldots)


Impacts on Local Watersheds

hydrology 

infiltration/pollution, groundwater recharge, flow of river basins, runoff

Higher risk of flooding and droughts

\section{Conclusion \& Prospect of Sustainable Urbanization/Land Use Change}


\chapter{Invasive Species}

\chapterauthor{Soliel}

\footnote{Statement of Contributions-- For example, ``The chapter was first drafted by Marc Los Huertos (2021). The author recieved valuable feedback from X, and Y and Z to improve the chapter. Slater revised the chapter in 2022 with suggestions from Cater.'' Note: I am still working on the formatting for this to improve it.}

\section{Section Heading}% Avoid putting text between section and subsection headings.



\chapter{Nuclear Power and Nuclear Waste}

\section{Current and Future Energy Needs}



\chapter{Air Pollution \& Social Justice in Hong Kong}

\chapterauthor{Neenah Vittum}

\section{Science of Air Pollution}

\subsection{Overview of the layers of the atmosphere/atmospheric gases}

What part of the atmosphere does air pollution affect?

What is air pollution?

Overview of different types of air pollution

\section{Major Sources → Use as geographical overview}

\subsection{General common sources of air pollution all over the world}

\subsection{East Asian countries/communities and their prominent air pollution sources}

Shipping

Traffic Emissions

Commercial and otherwise

Coal

Urban Development

Manufacturing

Other

The transboundary issue and its implications in regulation and politics

Impacts

Human health

Environmental Health

Greenhouse gas emissions and global warming

Both

Visibility

Environmental Justice

Case Study: Hong Kong

The Intersection of Air Pollution and Other Environmental Issues

Many environmental issues are interconnected

Air pollution and deforestation

Air pollution and urbanization/industrialization

Other Issues (To Explore)

Goals/Other Ideas/Questions

Ground information in geography and relevant examples

Incorporate stories and person accounts

slow violence → environmental justice issues

Maybe activist or someone who has suffered the issues firsthand

Draw people into the empathy

Use stories and descriptions to describe places

What is the best way to section the chapter?


\chapter{Flood Pulse System in East Asia}

\chapterauthor{Kristin Gabriel}

\section{Introduction}

What is the flood pulse system?

Seasonality

Ecosystem Services

Fish stocks

Flooded forests

How the flood pulse system influences the Tonle Sap Ecosystem

Timing of Flood Pulse

Magnitude of Flood Pulse

Duration of Flood Pulse

Influence of flood pulse system on people and their livelihoods

Fisheries

Immigration and emigration

Human Impacts on the flood pulse system

Climate change

Dam development

Case Study: Cambodia and the Tonle Sap


\chapter{Hydroelectric Dams in East Asia}

\section{Introduction}

Basic facts about dams in East Asia


Statistics on how many, size, scale, location etc.

Function of the Dam 

How it generates electricity/how much

Different types of dams (multi/single use etc.) 

Immediate ecological impacts 

Positive: 

Flood control, electricity generation, improved water quality 

Negative: 

Decreased water quality, flooding, sedimentation, habitat loss, deforestation, salinization etc... *note: the ecological impacts may be too many to go completely in depth into so perhaps a paragraph or subsection of each as opposed to a 7 page explanation of each 

Anthropological impacts 

Supposedly positive (I.e. employment etc...)

Negative: displacement, loss of cultural sites, diseases 

Displacement

Policy/government action/regulation  (policies that exist or propose solutions)

\section{Conclusion}



\chapter{Climate Change and Food Security in Myanmar}



\section{Climate Change, Climate Change Response in Myanmar}


General history of rice production and food demand in Myanmar. 

Impact on credit policy on rice 

Impact of infrastructure development on rice production

Study of the constraints of rice production in Myanmar

The effect of a command economy on food production in Myanmar 

Overall review on demand for food in Myanmar 

Possible implementation of SRI (systemic rice intensification) in order to increase rice yields in Myanmar

Transition from talking about rice production

sea-level rise

subsidence

coastal erosion

coastal flooding

Impact of climate change on rice production in Southeast Asia

Monsoon Season effect on Ayeyarwady River Badin

Sea Level Rise 

Sea level rise effect on global markets/rice production


Subsidence

Subsidence in Yangon, Myanmar

interview segments/personal experiences of rice farmers


Roles of the Burmese government

\section{Conclusion}

Reminders/Areas of Focus




\chapter{Disasters, Typhoons and Phillipines}

\chapterauthor{Ian Horsburgh}

\section{What are Typhoons?}



\chapter{Climate Change Adaptation and Infrastructure in Vietnam}

\chapterauthor{Jay Scott}

\section{Introduction}

As a low-lying, coastal nation with heavy dependence on its two river deltas, Vietnam is a country with severe risk factors for climate-related disaster. Even without the added effects of sea level rise, Vietnam frequently experiences typhoons during its wet season, at an average of 4-6 times each year \citep{scff2009climate}. Current dike systems aren’t strong enough and their effectiveness will only worsen with increased storm surges \citep{garschagen2011urban}. An increase in runoff could have a catastrophic impact on rural rice economies, with an estimated reduction in yields of 12\% and 24\% in the Mekong and Red River Deltas, respectively \citep{evers2018adaptation}. Rural residents rely on the rivers as their main source of drinking water, and both rivers are at risk from the construction of hydroelectric dam projects, saltwater intrusion, and increased demand for irrigation \citep{evers2018adaptation}. Vietnam’s urban population is steadily on the rise as well, growing from 20\% of the population to 30\% from 1985 to 2009 \citep{margulis2010economics}. This number is expected to continue to rise, as people migrate to Vietnam’s cities for economic opportunities, with estimates expecting cities to account for 57\% of the country’s population by 2050 \citep{garschagen2011urban}. This unprecedented increase in Vietnam’s urban population has the potential to overwhelm local governments, which have struggled with a simultaneous decentralization and tight control from Vietnam’s federal government \citep{garschagen2011urban}. Additionally, Vietnam does not guarantee its citizens the right to free speech, making community input virtually nonexistent in environmental policy \citep{nguyen2015deltaic}. While climate change is a new issue for Vietnam, it is a country that is uniquely adapted to floods \citep{nguyen2015deltaic}. The future of infrastructure the country will either build on this history, or forge a new path as Vietnam seeks to improve its standing internationally through economic development. 

\section{Climate Change Impacts on Vietnam}

\subsection{Flooding}

Most of Vietnam has a wet and dry season, bar the northernmost regions of the country . Unlike the four distinct seasons experienced in other parts of the world, Vietnam’s close proximity to the equator means its temperature rarely fluctuates, making the idea of “summer” and “winter” inadequate to describe conditions. “Wet” and “dry” are used as descriptors instead, and the seasons directly relate to the sun’s position over either the Northern or Southern hemisphere \citep{centerforscienceeducation}.

During the wet monsoon season, the warm air holds more water droplets, and flooding occurs along the coast and river deltas. Flooding is something that Vietnam has been adapted to over centuries. In the Mekong River Delta, there are a variety of housing types adapted to floods, including boat houses, floating houses, and stilted houses \citep{nguyen2015deltaic}. Farmers almost exclusively planted rice resistant to floods until the 1990s, when funding from the World Bank helped build dikes in order to produce a second rice crop during the rainy season \citep{nguyen2015deltaic}.

While these adaptations have proved sufficient in the past, they may not be enough to protect against current and future conditions. A 2018 study found that 33\% of Vietnam’s population is currently exposed to a 25-year flood, and cautioned that climate change could increase this number of exposed to up to 46\% \citep{bangalore2019exposure}Figure~\ref{fig:floodrisk}

\begin{figure}
\includegraphics[width=\linewidth]{images/Vietnam/floodriskovertimemap}
\caption{This map shows the number of people impacted currently in the case of a 25-year flood on the left, shows a future number of people impacted on the right. \citep{bangalore2019exposure}}
\label{fig:floodrisk}
\end{figure}

Flooding can have a profound impact on health. Incidences of drowning are relatively low compared to indirect health effects of flooding, such as diarrheal and skin diseases \citep{who2004report}. Because floods can decrease access to clean water, families respond by either washing foods less, or using subpar sources of water, hence the effect on disease \citep{who2004report}. In addition, Commune Health Services (CHS), an important part of Vietnamese healthcare, can become damaged during floods, worsening epidemics \citep{who2004report}.

In urban areas, floods have a major economic impact, shutting down roadways and preventing people from leaving their homes \citep{margulis2010economics}. 

\subsection{Drought}
In addition to its wet season, Vietnam has a long dry season that is expected to become even more dry with the addition of climate change. While drought risk is everywhere, it is especially concentrated in Vietnam’s mountainous regions \citep{lohmann2015effect}. 

Drought has a negative impact on human health, that is especially pronounced amongst children and young girls in particular. A 2001 study found that children aged 12-24 months during a drought were an average of 1.5-2cm shorter than children born during average conditions \citep{lohmann2015effect}. A separate 2009 study found that women born during years with higher rainfall were taller and had higher academic achievement than those born under average conditions \citep{lohmann2015effect}. This suggests that there are long-term effects of drought on children that continue after the rains return. 

Many rural homes are constructed of highly flammable materials, such as the aforementioned stilted houses constructed using melaleuca trees, and fires can spread quickly during the dry season \citep{margulis2010economics}. 

Vietnam’s most important crop, rice accounts for 47\% of all agricultural production and is very water intensive \citep{margulis2010economics}. As a staple crop, many rural households' economic stability is highly dependent on the year’s harvest. Evidence shows that in areas most affected by the drought, yields dropped 40\% under drought conditions \citep{lohmann2015effect}. Those who plant successful rice crops that year benefit from rice’s higher selling price, but there is a net negative effect on rice growers \citep{lohmann2015effect}. Aside from rice, aquaculture, specifically of catfish and shrimp, is important to rural economies and drought poses a risk to their cultivation \citep{margulis2010economics}. 

\subsection{Sea Level Rise}
Vietnam’s long coast makes it particularly susceptible to the consequences of sea level rise. The coastline has been rising at about the global average of 3mm per year \citep{huong2013urbanization}. If this rate is stable, Vietnam is expected to experience 75 cm of sea level rise by the end of the 21st century \citep{monre2010climate}. This will have wide ranging effects, one of the most damaging being saltwater intrusion \citep{hens2018sea}. Salinization occurs when sea water, with it’s high salt content, vertically infiltrates through soil and contaminates underground sources of fresh water \citep{hens2018sea}. A higher sea level will bring seawater higher up the water table, causing this effect. Figure~\ref{fig:saltwater}

\begin{figure}[h!]
\includegraphics[width=\linewidth]{images/Vietnam/saltwaterintrusion}
\caption{This graphic by the National Environmental Education Foundation visualizes the process of saltwater intrusion. \citep{bradford}}
\label{fig:saltwater}
\end{figure}

A 1 meter rise in sea level is estimated to affect 11 percent of Vietnam’s population and 5\% of total land area \citep{scff2009climate}. To combat this, the government has invested 280 Million VAT into building and fortifying sea dikes \citep{scff2009climate}. While sea dikes can be important elements of adaptation strategy, especially in Northern Vietnam which does not have a long history of floods, this strategy is not always suitable \citep{nguyen2015deltaic}. In his dissertation “Deltaic Urbanism or Living With Flooding in Southern Vietnam”, Phuong Nga Nguyen argues that the ideology of nation building, along with the Vietnamese’ government’s interest in increasing the productivity of rice, has been a contributing factor in the construction of dams in the Mekong Delta \citep{nguyen2015deltaic}. Nguyen believes the communities along the delta are well-adapted to flooding and don’t require much in the way of physical barriers. 

A rise in sea level could increase the severity of floods \citep{huong2013urbanization}. When sea levels rise, the area flooded can creep inland \citep{hens2018sea}. This exposes areas that previously weren’t exposed to flooding and therefore less adapted to its effects \citep{hens2018sea}. 

\subsection{Urbanization}

As climate events, economic conditions, and other factors force people out of agricultural villages, more Vietnamese are migrating to cities \citep{margulis2010economics}. Generally, urbanization increases flood risk because it concentrates the population into small areas and forces quick land use changes \citep{huong2013urbanization}.  The growing proportion of urban Vietnamese poses an issue for infrastructure already vulnerable to weather events during the monsoon season. 

For example, the rapid development of former wetlands in Ho Chi Minh City (HCMC) has led to poor drainage, exacerbating flooding brought on by storms \citep{vachaud2019flood}. These areas, in particular Phu My Hung and Thu Thiem, are considered undesirable places to live and are occupied almost exclusively by poor migrants, creating environmental inequality \citep{vachaud2019flood}. This issue and others will be discussed later in this chapter in the section on Ho Chi Minh City.

The urbanization of rural areas can often damage local aquifers as new residents drill for drinking water \citep{margulis2010economics}. The urban poor are one of the most vulnerable populations to climate disaster as they often have substandard housing, and rely on jobs in the informal economy that come with varied levels of stability \citep{margulis2010economics}. 

\section{Current Adaptation Plans and Policies}

\subsection{Strengthening Barriers and Existing Infrastructure}

The World Bank and United Nations Development Program have both allocated funds to improve existing physical infrastructure in Vietnam. In 2009, the UNDP funded a 180 million dollar project to enhance climate infrastructure, with 168 million specifically dedicated to erecting barriers like seawalls and dikes \citep{scff2009climate}. 

The United Nations, the World Bank, and Vietnam’s federal government, all prioritize physical infrastructure, in their approach to climate adaptation. While physical infrastructure is crucial in many areas of Vietnam, large-scale projects such as dikes and seawalls can have the effect of evicting the poorest and most vulnerable residents of a community. In HCMC, new plans for a ring dike around the city could displace as many as 1500 people \citep{yarina2018}. Historically, small canals were used to redirect water to the Saigon river, and the city’s residents took advantage of flooding with small rice crops and aquaculture operations \citep{yarina2018}. This less invasive approach to infrastructure is important and should be taken into consideration by governments and NGOs. 

\subsection{Implementing Effective Policy and Encouraging Collaboration}

In addition to improving physical infrastructure, the UNDP also allocated funds to an exhaustive review of existing environmental policy, especially in rural coastal communities. The UNDP identified several social obstacles to effective climate policy in Vietnam \citep{scff2009climate}. 

Vietnam’s climate change policy is handled by three separate government agencies: the Ministry of Agricultural Development (MARD), the Ministry of Construction (MOC), and the Ministry of Natural Resources and the Environment (MONRE). Historically, there has been a lack of collaboration between the three agencies. Construction of climate-resilient infrastructure is handled by MOC, while natural disaster relief is the responsibility of MARD \citep{scff2009climate}. Climate change preparedness and mitigation is under the scope of MONRE \citep{scff2009climate}. These three agencies work, for the most part, independently of each other \citep{garschagen2011urban}Figure~\ref{fig:mocmonremard}. In addition, local MOCs and MONREs exist in each province that are under the purview of provincial governments, not the federal MOC and MONRE. These smaller bureaus often work independently of each other with only minimal communication between them and the larger federal ministries \citep{garschagen2011urban}.

\begin{figure}
\includegraphics[width=\linewidth]{images/Vietnam/mocmonremard}
\caption{This organizational map explains the structure and purpose of the three separate federal agencies responsible for climate infrastructure and adaptation.}
\label{fig:mocmonremard}
\end{figure}

The United Nations Development Program identified that institutional knowledge of climate change was lacking, and that administrators were somewhat unwilling to integrate climate into their policy and operations \citep{scff2009climate}. Local governments were also noted as being indifferent to climate change, not seeing it as a larger threat than already common monsoons and other extreme weather events \citep{scff2009climate}. 

\subsection{Ecosystems Based Adaptation}

Ecosystems Based Adaptation, also known as EBA, is an approach to climate adaptation that prioritizes strengthening existing ecosystems. In Vietnam, this usually means strengthening ecosystems to protect against floods, landslides, and land degradation \citep{nguyen2017integration}. 

Mangrove forests have historically provided protection during storm surges and their revitalization could be a key part of Vietnam’s EBA. Additionally, forest conservation can aid in retaining soil nutrients and prevent landslides \citep{nguyen2017integration}. 

EBA often comes with co-benefits that can be economic, sociocultural, and promote biodiversity \citep{nguyen2017integration}. Despite the effectiveness and affordability of EBA, it is often overlooked in favor of new physical infrastructure \citep{nguyen2017integration}\citep{nguyen2015deltaic}. A lack of coordination between provinces and between the aforementioned MONRE, MOC, and MARD can make it difficult to effectively integrate EBA into policy \citep{garschagen2011urban}.


\subsubsection{Three Facets of Adaptation Policy}

The Vietnamese government has outlined three approaches to climate change: full protection, adaptation, and withdrawal \citep{monre2010climate}. Full protection is the use of physical infrastructure to completely insulate an area. This is seen as an option for important economic centers in cities or cultural landmarks, but can often exclude the poorest residents of these cities \citep{nguyen2015deltaic}\citep{yarina2018}. Adaptation is the prediction and acceptance of some climate-related losses, and the design of new systems compatible with a changed climate. Adaptation is important in the agricultural sector, as farmers find solutions to integrate climate change into their practices \citep{scff2009climate}. Withdrawal is complete avoidance of climate events by vacating an area extremely at-risk for climate impacts \citep{scff2009climate}\citep{monre2010climate}. Withdrawal could result in the unequal displacement of poor Vietnamese.

\section{Case Studies in Two Cities}

\subsection{Can Tho}
Can Tho is the largest city on Vietnam’s Mekong river delta, currently at a population of 1.8 million \citep{huong2013urbanization}.  This is up from 1.2 million just eight years ago in 2013, and this rapid pace of urbanization has produced an urban heat island effect \citep{huong2013urbanization}. As discussed in the section on flooding, warmer air holds more water and in turn increases rainfall. This has already been recorded in Can Tho \citep{huong2013urbanization}. The city sits at a low elevation, an average of only 60-80 cm above sea level \citep{huong2013urbanization}. Can Tho City is unique in its use of water as a primary means of transport and way of life. Its residents are uniquely adapted to living with floods, but also at risk for increasingly worse floods brought on by climate change \citep{nguyen2015deltaic}. 

Currently, there are plans underway to build large concrete barriers along the Can Tho River, which is an important part of city life and culture \citep{nguyen2015deltaic}. There are two floating markets that take place on the river, and many people live in houseboats and floating houses on the river \citep{nguyen2015deltaic}Figure~\ref{fig:floatingmarket}. Barriers would invariably change the way residents interact with the river and move through the city.

\begin{figure}
\includegraphics[width=\linewidth]{images/Vietnam/floatingmarket}
\caption{This is a photo of one of Can Tho's Floating Markets.\citep{isderion_2013}}
\label{fig:floatingmarket}
\end{figure}

Forced evacuation is occurring in some sections of the city, with the government offering plots of land on higher ground to those living closest to the river \citep {evers2018adaptation}\citep{nguyen2015deltaic}. However, there is evidence that residents who are relocated move back to their previous homes. Nguyen (2015) interviewed relocated families and around 60\% of them had moved back to their homes along the riverside. 

The Vietnamese federal government classified Can Tho as a first class city in 2015. This gave the federal government more control over Can Tho’s development, and priorities are firmly on the side of economic development \citep{evers2018adaptation}. Flooding, which long-time residents accept as a way of life, is not conducive to the kind of economic development projects the government wants to undertake in order to attract tourists and foreign companies \citep{nguyen2015deltaic}. As a result, Can Tho is a city being pulled in two directions. On one side are residents who lack political representation, and on the other side is the Vietnamese government, seeking to increase economic opportunity in the country as a whole.

\subsection{Ho Chi Minh City}

Ho Chi Minh City, formerly known as Saigon, is the largest city in Vietnam and its main economic center \citep{margulis2010economics}. Much like Can Tho, the city’s lifeblood is the Saigon river, which serves transportation, recreational, and economic purposes \citep{vachaud2019flood}. During the Nguyen Dynasty from the late 18th to 19th centuries, canals were constructed across the city as flood management tools \citep{vachaud2019flood}. In the early 19th century, France colonized Vietnam and by the mid-1800s, the canals were being filled in and converted to tree-lined boulevards meant to mimic the landscape of the River Seine \citep{vachaud2019flood}\citep{yarina2018}. This was disastrous for flood management and has not been corrected. The remaining canals left from the dynastic era became an open air sewage system, and they still serve this purpose today \citep{vachaud2019flood}.

HCMC is a prime example of the quickly growing population of urban Vietnamese, as mentioned in the introduction. During the war era, HCMC, known as Saigon, was part of South Vietnam. South Vietnamese were encouraged to populate cities, and those original residents from the war era are some of the longest residents of the city \citep{bolay1997sustainable}. When Vietnam was reunified in 1975, this policy was reversed as the new government seeked to relieve pressure on the densely populated urban areas, and rebuild the decimated rural economy \citep{bolay1997sustainable}. This effort was largely unsuccessful, and today, HCMC has a population of almost 9 million (Census 2019). 

Today’s Ho Chi Minh City faces major problems in regards to flooding. 65\% of its land area is less than 1.5 meter above sea level \citep{vachaud2019flood}. Technical solutions call for seawalls and dikes in hopes of fully protecting the city; however, this full protection doesn’t extend to everyone \citep{yarina2018}. The Ho Chi Minh City Adaptation Strategy, produced by MONRE in partnership with the Dutch government, seeks to fashion HCMC in the image of Rotterdam \citep{yarina2018}. A major element of this plan is a 2 billion dollar ring dike ensconcing the heart of the city, a complex system including sluice gates and canals. This proposal drew criticism, however, for its exorbitant cost and its potential to worsen flooding in communities on the unprotected side of the dike \citep{yarina2018}. 

As mentioned in the section on urbanization, HCMC’s historical landscape consisted of wetlands and swamps that provided ecosystem services such as drainage, protection against coastal erosion, and flood control \citep{bolay1997sustainable}\citep{vachaud2019flood}. As the city has expanded, many of these former wetlands have been developed and can no longer serve this purpose. Their extremely low elevation makes them undesirable places to live, and as a result they are occupied by the poorest residents of the city \citep{margulis2010economics}.

In many ways, the environmental problems in Ho Chi Minh City can be seen as a microcosm of the complex challenges facing Vietnam’s growing cities. A lack of investment in basic public services, especially access to clean water and sewage treatment, has persisted in the city. The government also prioritizes the construction of new types of infrastructure over historically used methods of flood control \citep{nguyen2015deltaic}\citep{yarina2018}. 

\section{Climate Vulnerable Groups in Vietnam}

\subsection{Women and Climate}
Women in Vietnam are an especially climate-vulnerable group. 60\% of Vietnamese women rely on agriculture as their primary source of income, compared to a little under 50\% of men \citep{margulis2010economics}. Therefore, heavy rainfall and storms’ impact on agriculture is more severely felt. Furthermore, many women being the sole person carrying the financial burden in their households. Anecdotal evidence points to women being more likely to put other family members first during climate disasters, at the expense of their own well-being \citep{nellemann2011women}. Additionally, many women in Vietnam lack basic swimming skills as young girls are not encouraged to learn to swim. This leads many to die avoidable deaths in survivable flooding conditions \citep{margulis2010economics}. 

\subsection{Children and Climate}
Children are a climate-vulnerable group in Vietnam, not only because of the immediate threat of flooding, but also because of their still developing immune systems that are highly susceptible to water-borne illnesses that spread after floods \citep{margulis2010economics}\citep{pink2016vietnam}. Children’s natural inclination to play outside can expose them to pollutants in the air and water \citep{margulis2010economics}. Extreme weather events can interrupt schooling and impact the success of a child long-term \citep{lohmann2015effect}.

\subsection{Migration and Climate in Vietnam}
Not only is climate a major driver of internal migration in Vietnam, it also exposes migrants to environmental hazards caused by climate change \citep{margulis2010economics}. As Vietnam’s agricultural sector continues to produce diminishing returns, many Vietnamese people are leaving the countryside for large cities, with the hope of securing financial opportunities less reliant on the environment \citep{margulis2010economics}.

 In Vietnam, moving required permission from the federal government until the mid-90s, and the difficulty of receiving this permission led many Vietnamese to migrate to cities without it \citep{bolay1997sustainable}. Today, moving requires registration under the National Household Registration System, and many migrants never go through this step \citep{margulis2010economics}. As a result, many of the rural to urban migrants are considered “undocumented” and are more likely to hold exploitative, dangerous, or unstable jobs in informal economy \citep{margulis2010economics}. 

Migrants often live in substandard housing that is extremely vulnerable to weather events. The poorly managed nature of urban sprawl in Vietnam’s cities can eliminate ecosystem services formerly provided by surrounding wetlands or forests \citep{vachaud2019flood}.

\section{Conclusion}

Having persevered through colonialism and a war that literally split the country in two, Vietnam now faces yet another threat to its survival: climate change. How the country will adapt to climate change has yet to be seen. The government envisions high tech physical infrastructure, but hasn’t yet been able to make meaningful progress in implementing its ambitious ideas. While the lack of free speech in Vietnam makes it difficult to gauge the sentiments of its citizens, it seems many do not have the same vision for the country. What is clear is that unlike many countries that will be heavily impacted by climate change, Vietnam has experience in coping with extreme weather events.


\chapter{Waste Management for a Circular Economy}

\section{Life-Cycle}

\subsection{Collection}

\subsection{Transport}

Treatment

Disposal

Sectors:

Industrial

Household

Biological 

Types of Waste:

Solid:

Liquid

Gaseous waste

\section{Biomimicry}

\subsection{Circularity}

Examples in Nature

Education:

Teach people to be mindful and live sustainably

Social PsychologyProblems and New Approaches: 

Sustainability

Incineration \& Dumping

Recycle \& Reuse

Resource Recovery


\chapter{Plastic and Packaging in Japan}

\section{Introductiona and Goals?}

Plan: Use Japan's unique plastic packaging as a lens to view plastic waste management. I can bring in benefits of their plastic use, like cultural significance of beautiful wrapping and food safety, and then discuss plastic pollution as a larger issue in East Asia, bringing in examples of blame placing, and of course discussing potential solutions on both international and local scales. 

\section{Plastic Pollution and Waste Management in East Asia} 

\subsection{Statistics/comparisons}

graphs and images will help with perspective

\subsection{History of plastic waste issues in East Asia}

\subsubsection{Are specific companies/industries responsible responsible}

what kinds of plastic waste are there (sector break down)? 

\subsubsection{Where in the world did the ubiquitous usage of single use plastics come from?}

General blame placing/biases/rhetorical 

examples of discourse around plastic waste in East Asia. Why does any of this matter(needs its own section)?

Plastic waste trade? 

\url{https://link.springer.com/article/10.1007%2Fs10163-004-0115-0}

\url{https://www.sciencedirect.com/science/article/abs/pii/S0956053X20305602}

Blame placing through both rhetoric and scientific studies

(this source is a very data based study that concluded that the vast majority of plastic pollution comes from a few sources in Asia/Africa... I want to explore what they might not have taken into account when collecting data)

\url{https://science.sciencemag.org/content/347/6223/768}

\url{https://pubs.acs.org/doi/10.1021/acs.est.7b02368}

\url{https://www.dw.com/en/whose-fault-is-plastic-waste-in-the-ocean/a-49745660} (found the two above studies through this article)

Japan Specific (I need to break these into hierarchies of significance), some sections, the first  few will be more data based, the second half will be more rooted in sociological primary sources.

Waste management issue overview

Sector Break Down/ responsible parties in Japan

Impacts of plastic pollution on different groups within Japan

Cultural significance of wrapping

Food safety

Gov action/recycling/current efforts

Activism

Potential solutions moving forward rooted in current activist efforts/respect to culture

\url{https://www.pnas.org/content/117/33/19844.short}

\url{https://www.jstor.org/stable/432317?seq=1}

\url{https://onlinelibrary.wiley.com/doi/abs/10.1002/1099-1522(200003/04)13:2%3C45::AID-PTS496%3E3.0.CO;2-%23}






\backmatter

\part{Backmatter}

The back matter often includes one or more of an index, an afterword, acknowledgments, a bibliography, a colophon, or any other similar item. In the back matter, chapters do not produce a chapter number, but they are entered in the table of contents. If you are not using anything in the back matter, you can delete the back matter TeX field and everything that follows it.

\printglossary

\renewcommand\bibname{References}
\setlength{\bibsep}{2\baselineskip}
\setlength\bibindent{.5in}
\bibliographystyle{plainnat}
\bibliography{References}

\end{document}
