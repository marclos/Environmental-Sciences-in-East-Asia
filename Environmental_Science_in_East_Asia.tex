\documentclass{book}\usepackage{knitr}

% Preamble

%%%%%%%%%%%%%%%%%%%%%%
%% PACKAGES
%%%%%%%%%%%%%%%%%%%%%%
\usepackage[twoside,letterpaper,width=6in,height=8in]{geometry}
\usepackage{siunitx} % format units properly
\usepackage{wrapfig}
\usepackage[margin=10pt,font=small,labelfont=bf]{caption} % format captions
\usepackage{booktabs} % nicer tables
\usepackage{subcaption} 
\usepackage{csquotes} % block quotes
\usepackage{tikz}
\usepackage[inline, shortlabels]{enumitem} % inline enumeration
%\usepackage[version=4]{mhchem}
\usepackage{graphicx} % packages are used to modify the text and create bling.
%\includegraphics{{Home/CAMPUS/mwl04747/github/Environmnental-Sciences-in-East-Asia/images/}}
\usepackage{textcomp}
\usepackage{gensymb}
\usepackage[T1]{fontenc}
\usepackage{natbib}
%\usepackage{wasysym} %\smiley{}
% \usepackage{siunitx}
\usepackage{glossaries}
% \usepackage[automake]{glossaries} % automake option unknown.
\usepackage{amsmath}%
\usepackage{amsfonts}%
\usepackage{amssymb}%
\usepackage{soul}
\usepackage{comment}

%\usepackage[super,square,comma]{natbib}
%\usepackage{float}
%\usepackage{appendix}
%\usepackage{chngcntr}
%\usepackage{etoolbox}
%\usepackage[usenames]{xcolor}% for commenting in color!

\RequirePackage{hyperref} % For hyperlinked cross-references
\hypersetup{
    colorlinks,
    citecolor=blue,
    filecolor=blue,
    linkcolor=blue,
    urlcolor=blue
}

% Editorial Environments

\newcommand{\red}[1]{\textcolor{red}{[MLH: #1]}}
\newcommand{\blue}[1]{\textcolor{blue}{[MLH: #1]}}

%-----------------------------------------------------
\newtheorem{theorem}{Theorem}
\newtheorem{acknowledgement}[theorem]{Acknowledgement}

\newenvironment{proof}[1][Proof]{\textbf{#1.} }{\ \rule{0.5em}{0.5em}}
%----------------------------------------------------------

\AtBeginEnvironment{subappendices}{%
\chapter*{Appendix}
\addcontentsline{toc}{chapter}{Appendices}
%\counterwithin{figure}{section}
%\counterwithin{table}{section}
}

\makeatletter
\newcommand{\chapterauthor}[1]{%
  {\parindent0pt\vspace*{-25pt}%
  \linespread{1.1}\large\scshape#1%
  \par\nobreak\vspace*{35pt}}
  \@afterheading%
}
\makeatother

\renewcommand{\glstextformat}[1]{\textbf{\color{blue}\em #1}}

\newcommand{\R}{\mathbb{R}}
\newcommand{\carbondioxide}{CO$_2$~}
\newcommand{\nitrous}{N$_2$O~}
\newcommand{\methane}{CH$_4$~}
\newcommand{\kms}{km$^2$}
\newcommand{\micrograms}{$\mu$g~m$^{-3}$}

\newcommand{\Mypm}{\mathbin{\tikz [x=1.4ex,y=1.4ex,line width=.1ex] \draw (0.0,0) -- (1.0,0) (0.5,0.08) -- (0.5,0.92) (0.0,0.5) -- (1.0,0.5);}}%



\title{Environmental Issues in East Asia}
\author{EA30e Spring 2021}
\date{\today}
\IfFileExists{upquote.sty}{\usepackage{upquote}}{}
\begin{document}
\makeglossaries
\maketitle


\frontmatter
\tableofcontents


\chapter{Preface}

\section*{Guiding Principles}

Environmental issues in East Asia are not unique or particularly more prevasive than other parts of the world. However, the issues are borne from particular histories that may contrast with other parts of the world and other parts of the world may be able to learn from. 

In this project, the students in EA030e (Spring 2021) have written a textbook that highlights examples of environmental processes. Each student contributed to one theme, composed of two examples that highlight environmental issues of East Asia. 

\subsection*{Context and Positionality}

As students in a college course located in Southern California, we approach the project with...


Our goal is not to call out environmental issues in East Asia, but to point to linkages of how a range of globalized economy contribute to these environmental problems. 

In the end, it would be useful for us to acknowledge we have some capacity to address these how these global linkages could be modified to reduce these environmental issues.

We are not experts, but learning... if there are errors please let us know... We recommend that suggestions be submitted via a github pull request.

\subsection*{Goals}

Processes across horizontal boundaries define many environmental patterns that frame human interactions with the environment. How do humans impact processes that cross these boundaries and how do humans influence these ecosystem interface?

\subsection*{Rationale}

We hope to learn more about the how environmental issues are expressed in different parts of the world and to what extent can we learn from this work. 

\subsection*{Activity}

Each group will be composed of two students, that will become experts and teach their classmates on the topic. 

\section*{East Asia and the World}







\section*{Acknowledgments}

Everyone in the world!




\newpage

\section*{Permission for Continued Collaborations}

Congratulations, you have done a steller job describing environmental issues in East Asia and create a valuable resources for others. 

As a goal of the course, the content can continue to evolve and even improve with new research and contributions. Thus, the text you have written could be revised in the future, should you give permission. For example, I'd like to add a glossary and an index of topics -- more than we had time to do. At some point, I might fix citation issues, improve clarity, and perhaps add or move content. 

I have purposely given you the option after grades were submitted to ensure that how you proceed will have no bearing on the your course outcome. You may appreciate others going over and possibly improving your text with time and adding their names as contributors or co-authors as appropriate. Whether or not you decide to give permission has no bearing on your grade.

\section*{Form}

I, \rule{3in}{0.4pt}, give permission for other collaborators to edit my text to improve the clarity and add new information. Contributions should be explicitly documented in the chapter contributions section. Should the edits be substantial, additional co-authors may be added to the chapter as appropriate. 

\bigskip

$\square$ I do not give permission to have my chapter posted online with the rest of the text.

\bigskip

$\square$ I give permission to have my chapter posted online with the rest of the text.

\bigskip

$\square$ I give permission to have my chapter posted online with the rest of the text only with additional editing.

\medskip

$\square$ I give permission for others to edit my chapter without contacting me.

\medskip

$\square$ I give permission for others to edit my chapter after contacting me.

\medskip

$\square$ I prefer to collaborate with others in the editing of the chapter during the summer of 2021.

\medskip

$\square$ I prefer to collaborate with others in the editing of the chapter at some time, but not in the next year or two.

\bigskip

\noindent Sign: \rule{3.4in}{0.4pt}

\bigskip
\noindent Print Name: \rule{3in}{0.4pt}

\begin{comment}

\chapter{\LaTeX Guide}\label{ch:guide}

\section*{Why Learn \LaTeX?}

\subsection*{Creating Professional Typesetting}

In the past, I used \LaTeX to make publication quality text. In fact, many prefer writing in \LaTeX because they can focus on the text and avoid worrying about formatting. However, it is NOT WYSIWYG (``what you see is what you get'') word processor. In reality, the processing or compiling is a separate step. 

Nevertheless, the quality of the output and ability to integrate with R (or Python) allows us to have an exceptional tool to make reproducible documents with highly professional looking typesetting.

\subsection*{How to Learn \LaTeX?}

There are several ways to learn \LaTeX. I suggest you find a decent tutorial to get the basics. For example, here are some suggestions:

\begin{itemize}
  \item \href{https://www.overleaf.com/learn/latex/Learn_LaTeX_in_30_minutes}{Learning \LaTeX in 30 minutes}
\end{itemize}

If you are like me and can't remember commands very well, then here's a \href{https://wch.github.io/latexsheet/latexsheet-0.png}{cheet sheet} that might be helpful. 

\subsubsection{R Chunks}

To create effective graphics, each chapter will have a rchunk that creates a graphic for the chapter. To review and learn R, here are some resources: 

\begin{itemize}
  \item \href{www.tbd.com}{Marc's Video Description}
  \item \href{https://rmd4sci.njtierney.com/}{RMarkdown for Scientists (super helpful!)}
  \item \href{https://rmarkdown.rstudio.com/lesson-1.html}{R Studio Tutorial}
  \item \href{https://rstudio.com/wp-content/uploads/2016/03/rmarkdown-cheatsheet-2.0.pdf?_ga=2.107420162.161662097.1613074083-214354297.1613074083}{R Studio's Cheatsheet}
  \item \href{https://bookdown.org/yihui/rmarkdown-cookbook}{R Markdown Cookbook -- Robust Source}
\end{itemize}


\subsection*{Noting Your Contribution}

Because this is an ongoing project, you should record your contribution to each chapter -- but also let go of these contributions at some point; Others might revise and their authorship might take some precedence, so you should both invest in the product but also be willing to detach from the final outcome as others contribute. This will feel uncomfortable at times, but please note from the beginning this is a social process and as such subject to negotiation. Please be generous to the authors that laid the foundation and be respectful of those that follow. 

\section{Setting Up Book Project--Type Setting w/ \LaTeX}\label{sec:settingup}

\subsection{Latex Book Class}

Currently, the text is written using the standard book class. %However, in 2019, I (Los Huertos) will convert the format to a Tufte book class. 

\subsection*{Structuring the Text with Nested Hierarchies}

Contributors divide their contributions into sections and subsections. This format allows a consistent approach to structuring the text and forcing themes to be organized in blocks that can be used to organize the overall text. We use section, subsection, and subsubsection to break up the topic into bite sizes. 

To accomplish this, contributors use the \verb"\section{Section}" command for major sections, and the \verb"\subsection{Subsection}" command for subsections, and a similar approach for subsubsections. 

NOTE: for each nested level, it MUST be followed by the lowest level in the section before a paragraph is started -- in contrast to what is shown above!

NOTE: We may dispense with subsubsections in the future to provide a less blocky structure, but for now they remain useful. 

\subsection*{Font Changes}

We can use various methods to alter the typeset: \emph{Emphasize}, \textbf{Bold}, \textit{Italics}, and \textsl{Slanted}. We can also typeset \textrm{Roman}, \textsf{Sans Serif}, \textsc{Small Caps}, and \texttt{Typewriter} texts.  Look online to see the commands to accomplish these changes. 

You can also apply the special, mathematics only commands $\mathbb{BLACKBOARD}$, $\mathbb{BOLD}$, $\mathcal{CALLIGRAPHIC}$, and $\mathfrak{fraktur}$. Note that blackboard bold and calligraphic are correct only when applied to uppercase letters A through Z.

You can apply the size tags -- Format menu, Font size submenu -- {\tiny tiny}, {\scriptsize scriptsize}, {\footnotesize footnotesize}, {\small small}, {\normalsize normalsize}, {\large large}, {\Large Large}, {\LARGE LARGE}, {\huge huge} and {\Huge Huge}.

You can use the \verb"\begin{quote} etc. \end{quote}" environment for typesetting short quotations. Select the text then click on Insert, Quotations, Short Quotations:

\begin{quote}
The buck stops here. \emph{Harry Truman}

Ask not what your country can do for you; ask what you can do for your
country. \emph{John F Kennedy}

I am not a crook. \emph{Richard Nixon}

I did not have sexual relations with that woman, Miss Lewinsky. \emph{Bill Clinton}
\end{quote}

The Quotation environment is used for quotations of more than one paragraph. Following is the beginning of description of \LaTeX from \emph{Wikipedia}:

\begin{quotation}
LaTeX (/ˈlɑːtɛx/ LAH-tekh or /ˈleɪtɛx/ LAY-tekh, often stylized as \LaTeX) is a software system for document preparation. When writing, the writer uses plain text as opposed to the formatted text found in ``What You See Is What You Get'' word processors like Microsoft Word, LibreOffice Writer and Apple Pages. The writer uses markup tagging conventions to define the general structure of a document (such as article, book, and letter), to stylise text throughout a document (such as bold and italics), and to add citations and cross-references. A \TeX distribution such as \TeX Live or MiK\TeX is used to produce an output file (such as PDF or DVI) suitable for printing or digital distribution.

\LaTeX is widely used in academia for the communication and publication of scientific documents in many fields, including mathematics, statistics, computer science, engineering, physics, economics, linguistics, quantitative psychology, philosophy, and political science. It also has a prominent role in the preparation and publication of books and articles that contain complex multilingual materials, such as Sanskrit and Greek. \LaTeX uses the \TeX typesetting program for formatting its output, and is itself written in the TeX macro language.''
\end{quotation}

Use the Verbatim environment if you want \LaTeX\ to preserve spacing, perhaps when
including a fragment from a program such as:
\begin{verbatim}
#read csv data  // read data into R
my.dataframe <- read.csv(file.choose())   // read data from a popup window.

str(my.dataframe) // display data structure

\end{verbatim}
(After selecting the text click on Insert, Code Environments, Code.)


\subsection*{Mathematics and Specialized Characters}\label{sub:mathchar}

\subsubsection*{Warning: Special Characters}

When you use percent and ampersand symbols, hash tags, and other non-standard ASCII characters, \LaTeX will be very uncooperative. \LaTeX~doesn't like a range of characters or they reserved for special behavior. So, do yourself a favor and make sure you understand that these are used for special typesetting functions. To use them you have to ``escape'' and use commands to get them to do what you might usually expect!  

The following symbols \$, \%, \#, \&, \`e, \~n, `` and '' do not reflect the key stroke you might expect. 

For example, the \& is used for tabs in a table environment. \% is used to make comments, thus stuff behind a \% is ignored. There are lots of others, but these come up the most. If you want to show use the ampersand or one of these characters, put a backslash in front of the dollar sybmol, e.g. \textbackslash\$. See Table \ref{tab:tableofsymbols}.

If you want to a superscript (raised to 3nd power), we can create text in math mode, with \$ to start and end the text in math mode, e.g. m$^3$ is written in \LaTeX as m\$\^{}3\$. A subscript uses an underscore, x\$\_1\$ creates x$_1$. If you need more than one character as a subscript or superscript then enclose the content in curly brakets, e.g. x$^{2c}$ (x\$\^{}\{2c\}\$) and t$_{step}$ (t\$\_\{step\}\$).

\begin{table}[h]
\caption{Table of Symbols in \LaTeX}
\label{tab:tableofsymbols}
\begin{tabular}{|ll|ll|} \hline
Symbol  & \LaTeX code & Symbol & \LaTeX code \\ \hline\hline
\&  & \textbackslash\&  & 
\$  & \textbackslash\$ \\
``  & \`{}\`{} & 
''  & \'{}\'{} \\
mg~L$^{-1}$ & mg$\sim${}L\$\^{}\{-1\}\$ & 
    & \\ 
\hline
\end{tabular}

\end{table}

\subsection{Creating equations}

One of the most powerful parts of \LaTeX is how it can be used to write complex equations, with all those symbols and Greek letters! This can be done inline $y = mx + b + \epsilon$ for fairly simple equations, or set apart for more complex equations:

\begin{equation}
\int_0^\infty e^{-x^2} dx=\frac{\sqrt{\pi}}{2}
\end{equation}

\subsubsection{Theorems, etc}
\begin{theorem}
(The Currant minimax principle.) Let $T$ be completely continuous selfadjoint operator
in a Hilbert space $H$. Let $n$ be an arbitrary integer and let $u_1,\ldots,u_{n-1}$ be
an arbitrary system of $n-1$ linearly independent elements of $H$. Denote
\begin{equation}
\max_{\substack{v\in H, v\neq
0\\(v,u_1)=0,\ldots,(v,u_n)=0}}\frac{(Tv,v)}{(v,v)}=m(u_1,\ldots, u_{n-1})
\label{eqn10}
\end{equation}
Then the $n$-th eigenvalue of $T$ is equal to the minimum of these maxima, when
minimizing over all linearly independent systems $u_1,\ldots u_{n-1}$ in $H$,
\begin{equation}
\mu_n = \min_{\substack{u_1,\ldots, u_{n-1}\in H}} m(u_1,\ldots, u_{n-1}) \label{eqn20}
\end{equation}
\end{theorem}
The above equations are automatically numbered as equation (\ref{eqn10}) and
(\ref{eqn20}).


\subsection*{Lists Environments: Making bulletted, numbered, description lists}

We use special commands to create an itemized list.

You can create numbered, bulleted, and description lists
(Use the Itemization or Enumeration buttons, or click on the Insert menu
then chose an item from the Enumeration submenu):

\begin{enumerate}
\item List item 1

\item List item 2

\begin{enumerate}
\item A list item under a list item.

\item Just another list item under a list item.

\begin{enumerate}
\item Third level list item under a list item.

\begin{enumerate}
\item Fourth and final level of list items allowed.
\end{enumerate}
\end{enumerate}
\end{enumerate}
\end{enumerate}

\begin{itemize}
\item Bullet item 1

\item Bullet item 2

\begin{itemize}
\item Second level bullet item.

\begin{itemize}
\item Third level bullet item.

\begin{itemize}
\item Fourth (and final) level bullet item.
\end{itemize}
\end{itemize}
\end{itemize}
\end{itemize}

\begin{description}
\item[Description List] Each description list item has a term followed by the
description of that term.

\item[Bunyip] Mythical beast of Australian Aboriginal legends.
\end{description}

\subsection*{Theorem-Like Environments}

The following theorem-like environments (in alphabetical order) are available
in this style.

%\begin{acknowledgement}
%This is an acknowledgement
%\end{acknowledgement}

\begin{example}
This is an example
\end{example}

\begin{exercise}
This is an exercise
\end{exercise}


%\begin{proof}
%This is the proof of the lemma.
%\end{proof}

%\begin{notation}
%This is notation
%\end{notation}

%\begin{problem}
%This is a problem
%\end{problem}

%\begin{proposition}
%This is a proposition
%\end{proposition}

%\begin{remark}
%This is a remark
%\end{remark}

%\begin{summary}
%This is a summary
%\end{summary}

\begin{theorem}
This is a theorem
\end{theorem}

%\begin{proof}
%[Proof of the Main Theorem]This is the proof.
%\end{proof}

\subsubsection*{``Child'' Rnw Contributions to create a Textbook}

To create the texbook, we will each contribut chapters, which are referred to in knitr speak as a `child'. For each `child' or chapter, will be a sepreate Rnw fild, but without the need for the preamble, such as the \verb"\begin{document}" and \verb"\end{document}", which is pretty handy. Then each chapter has the exact same formatting as defined in the root document, which you have to select in the project options. 

\subsection*{Peer Review Commenting}

You can put your comments in square brackets and in color for things that need help. \textcolor{red}{[This section is confusing, I am not sure what commenting means.]} I have created a short cut command to do this with \verb"\red{}". 

\subsection*{Adding Figures, etc}

\subsubsection*{Using Rnw Files}

To generate R figures, we use R chunks in and Rnw file, where the text is integreated. When we compile into a PDF, the program converts the files into TeX files and then combineds them into a single pdf. 

For each chapter, we create a ``child'' document and Marc will help you create that text when you begin. 

\subsubsection*{Creating a floating figure}

This is my floating figure (Figure \ref{fig:plot}).

\begin{figure}

\caption{My plot's caption is here!}
\label{fig:plot}
\end{figure}

\subsubsection*{Using R to Create Effective Figures}

R Markdown can be a very powerful tool to integrate R code, figures and text. Making high quality figures that are both clear and aestically pleasing will be something that we need to think about it. 

\begin{itemize}
  \item Axis Labels -- Labelled with clarity 
  \item Axis Text -- Size, Orientation 
  \item Captions (usually better than titles)
  \item References connecting labels to references
  \item ADA accessible (e.g. color impairment mitigation)
\end{itemize}

For example, here's an example of a pretty good figure, and you can find the code in the 01-guide.Rnw file.





In the case of Figure~\ref{fig:co2-graphic}, we can a create a figure that has all of the characteristics listed above, except perhaps ADA. Creating a "alt text" for the figure is something we might want to consider -- For now a decent caption about what the reader is seing is super helpful. 

\begin{figure}
\begin{knitrout}
\definecolor{shadecolor}{rgb}{0.969, 0.969, 0.969}\color{fgcolor}
\includegraphics[width=\maxwidth]{figure/maunaloa-1} 

\end{knitrout}
\caption{Carbon Dioxide Concentrations (Mauna Loa, HI). Data demonstrate the CO$2$ concentrations are increases, but that a seasonal impact is embedded in the long-term trend. Source: Scripps/NOAA.}
\label{fig:co2-graphic}
\end{figure}

\subsection*{Using Boxes}

\fbox{
\begin{minipage}[c]{.9\textwidth}
\subsection{minibox X}

Some text
\end{minipage}
}

\section*{Cross-References, Citations, and Glossaries}

\subsection*{Cross-References}

We can cross-reference sections (e.g. Section~\ref{ch:critical-zone}  or figures (Figure~\ref{fig:maunaloa}) using several methods. I suggest you look at the this Rmd file to see how I did it in these examples.

You can also create links to URLs or hyperlinks, e.g. \url{http://texblog.org}. However, if these addresses change, then the link will break, so I suggest you only link to internal references.

\subsection*{Footnotes}

I usually recommend that you don't use footnotes, but sometimes you really need to add an explanation of something that isn't really necessary, so you can use a footnote.\footnote{this is a footnote!}
!}}

\subsection*{Bibliography generation}

There will be two steps to cite our sources. First, we need to add the reference to a database, or bib file. This is titled 'References.bib' and is located in the main folder in our repository. When you add information to the bib file, be sure to paste in the reference using a bibTeX format. 

Second, we'll need to place in-line citations, using \verb"\citep{knitr}", which produces \citep{knitr}, by using a key, which is knitr in this case. 

For example, you might write, ``This document was produced in RStudio using the knitr package (\citep{knitr}). Also try \verb"\citet{LosHuertos2017OverviewR}" to create use the author name as the subject: \citet{LosHuertos2017OverviewR} wrote an guide to help students learn R. 

Note: You will see these citations automatically put in alphabetic order in the Bibliography at the end of the PDF. 

%Currently, we are using the ecology.bst, but it has trouble with misc type of references, so I will changing this in 2019. 

\subsection*{Creating glossary words}
 
Not sure why, I haven't been getting this to work yet -- stay tuned!


\newglossaryentry{peat}{
	name=peat, 
	description={is cool.}
}

\begin{definition}
This would be used in a glossary entry at the end of the book and the word is use in an glossary, e.g. \gls{peat}. \Gls{peat} is when you want to capitalize the defined word without having to re-define a capitalized version, the only downside of case sensitivity in \LaTeX.
\end{definition}

\section*{Custom Environments}

For text that we rely on many times, but require odd key strokes and hard to remember code, we can create new commands as shortcuts. 

For example, I have created two so far:

\verb"\carbondioxide}", \carbondioxide

and 

\verb"\kms", \kms


\chapter{Template Chapter Title}\label{ch:template}

\chapterauthor{Chapter Author}

\footnote{Statement of Contributions-- For example, ``The chapter was first drafted by Marc Los Huertos (2021). The author recieved valuable feedback from X, and Y and Z to improve the chapter. Slater revised the chapter in 2022 with suggestions from Cater.'' Note: I am still working on the formatting for this to improve it.}

\section{Section Heading}% Avoid putting text between section and subsection headings.

\subsection{Subsection Headings} % Avoid putting text between subsection and subsubsection headings. Not applicable if you don't have subsections!

Some text here...The hierarchy structure is described in the Author Guide, Section~\ref{sec:settingup} -- NOTE: This is a section cross reference.

if you cut and paste, be sure to make sure you don't include formatted characters outside the ASCII values. See Author Guide, Section~\ref{sub:mathchar}. NOTE: This is a subsection cross reference.


\subsubsection{Optional Subsubsection Headings}\label{subsub:optionalsubsub} % Again try to avoid putting text between the subheadig and the subsubheading to main a structural consistency.

some text here.... and a subsubsection cross reference (See Section \ref{subsub:optionalsubsub}).

\section{Goals of this template}

\subsection{Learning \LaTeX}

This template will NOT teach you how to use \LaTeX! To accomplish that, we'll rely on some great online resources that you can find on in Chapter~\ref{ch:guide}. 

Instead this section of the document is designed to demonstrate how our textbook will look, feel, and ultimately how we contribute to the project.

This document also compiles all of our projects into a single PDF, where each chapter is composed of a input tex file.

\subsection{Placing figure}

\subsubsection{R Created Figures}

First we create an R chunk and add some code. In this case, I created a floating figure which can be referenced (Figure~\ref{fig:pressure})!  

\begin{figure}
\begin{knitrout}
\definecolor{shadecolor}{rgb}{0.969, 0.969, 0.969}\color{fgcolor}\begin{kframe}
\begin{alltt}
\hlkwd{plot}\hlstd{(pressure)}
\end{alltt}
\end{kframe}
\includegraphics[width=\maxwidth]{figure/fig:pressure-1} 

\end{knitrout}
\caption{Figure Caption...we should turn "echo=False" in the R chunk options, but I left it true for now. (source: ??)} % define the caption, then the label.
\label{fig:pressure}

\end{figure}

\subsubsection{Floating Figures from External Sources}

All figures and images that are imported should be put into the ``images'' sudirectory to keep stuff organized. Even better to create a subdirectory with your images, but we can naviagate as we go.

Figure \ref{fig:vadose} is a good example of inserting an image from an external source.

\begin{figure}
\includegraphics[width=\linewidth]{images/Lee-Vadose}
\caption{Vadose zone is neato (Source: \citet{lee2017fifty}).}
\label{fig:vadose}
\end{figure}

In this case, I had to specify the width so it would fit on the page!  See the Rnw file for the code. Notice, I was also abel to ``reference'' the figure in the text.

\subsection{Adding Citations}

See the Guide, as well, but my video is probably the most helpful.


Generally, there are many environmental trends in Asia \citep{imura2005urban}.

\citet{imura2005urban} describes the how urbanization has affected the hydrology of East Asia. 

\subsection{Adding Glossary Words}

\newglossaryentry{entry}
{
  name={entry},
  description={my entry description.}
}

\gls{entry} is defined in the glossary! 
 

\chapter{Nora's Practice Sessions}

\chapterauthor{Nora}

$\rightarrow$

4/23
1. made new project titled ERTA2020-NewEA30e21-Project
2. did the single branch clone for EA30e21
3. switched the Git brach pull down to EA30e21 in remote: ORIGIN
4. testing if I can push to my branch only 
5. save changes
6. commit - TEST NEW PROJECT PUSH TO BRANCH at 2:04
      did it work? 2:03 est
      YES 2:16 est


1. added Marc's repo as the "upstream" to be able to pull from his branch when things are changed there
2. tried to fetch from marc's EA30e21 branch. - SUCCESSFUL
3. no need to use the "git merge origin EA30e21 "code
4. going to test making a pull request to Marc after saving, committing and pushing to my repo. 
    did it work? 7:45 est
    YES 7:51 est

\section{Testing Various Push and Pull Options}

test commit and pull request 


\subsection{What Factors Drive Land Use Change?}





\end{comment}

\mainmatter


\chapter{The Earth System}\label{earthsystem}

\chapterauthor{Marc Los Huertos}

\section{The Sun's Energy and the Earth's Temperature}

The temperture of the Earth's surface is the result of a balance -- the energy entering the atmosphere and the leaving the atmosphere. Most of this energy is in the form of light or electromagnetic radiation (Figure~\ref{fig:earthbudget}). 

\begin{figure}
\includegraphics[width=\linewidth]{images/earth-system/earth-rad-budget-nasa-erbe.png}
\caption{caption}
\label{fig:earthbudget}
\end{figure}

Light enters the atmosphere, where some is absorbed and some is reflected. Light interacts in different ways with land, oceans, and vegetation, which is beyond the scope of our project. The ``quality'' of light changes through these processes. 

\subsection{The Spectrum of Light Entering and Exiting the Earth's Surface}

As the sun's electromagnetic radiation interacts with the Earth's Atmosphere, certain wavelengths are absorbed and filtered out (Figure~ \ref{fig:em-entering}).

\begin{figure}
\includegraphics[width=\linewidth]{images/earth-system/em-radiation-atmosph-depth-stsci.jpg}
\caption{Various wavelengths of solar electromagnetic radiation penetrate Earth's atmosphere to various depths. Fortunately for us, all of the high energy X-rays and most UV is filtered out long before it reaches the ground. Much of the infrared radiation is also absorbed by our atmosphere far above our heads. Most radio waves do make it to the ground, along with a narrow `window' of IR, UV, and visible light frequencies. Source: STCI/JHU/NASA.}
\label{fig:em-entering}
\end{figure}

\subsection{The Atmosphere and Greenhouse Effect}



\section{Carbon Biogeochemistry}

\subsection{Long and Short Time Scales}

The carbon cycle processes occur at wide range of temporal scales from hundreds of millions of years to seasons of the year. These have been referred to as long and short carbon cycles. However, for our purposes, I will call them ``geologic carbon'' and ''biosphere carbon'' processes. 

\subsection{Rock Cycle and Geologic Carbon}

The carbon cycle describes changes in the fluxes and reservoirs of carbon in the Earth system. On very long time-scales, millions of years, the primary reservoirs of carbon are the atmosphere, ocean, and rocks (limestone). Carbon moves between these reservoirs through volcanic outgassing, silicate weathering, and limestone sedimentation. The carbon cycle is linked to Earth's energy balance through atmospheric carbon in the form of \carbondioxide, a greenhouse gas.

\subsubsection{Mountains and Erosion}

\ref{fig:carbonpools}

\begin{figure}
\includegraphics[width=\linewidth]{images/earth-system/Carbon-reservoirs-and-cycles-in-the-Earth.jpg}
\caption{Carbon reservoirs and cycles in the Earth. The figure shows short-and long-term cycles; biosphere and geologic carbon reservoirs and fluxes, and the relative sizes and residence times (y axis) of respective carbon. Numbers in brackets refer to the total mass of carbon in a given reservoir, in Pg C (1Pg C = 10$^{15}$ g carbon). All reservoirs are pre-industrial. Abbreviations: C org = organic carbon; DIC = dissolved inorganic carbon; MOR = mid ocean ridge; seds = sedimentary rocks. Adapted from Lee et al. (2019 And references therein).}
\label{fig:carbonpools}
\end{figure}

\subsubsection{Subduction Burial and Carbon Recycling}

Figure~\ref{fig:longtermcarbon}

\begin{figure}
\includegraphics[width=\linewidth]{images/earth-system/long-term-carbon.png}
\caption{Schematic of the long-term carbon cycle (from Bice, 2001)}
\label{longtermcarbon}
\end{figure}

\subsection{Photosynthesis, Respiration, and Biosphere Carbon}

\subsubsection{Soil Respiration and the Soil Profile}

Carbon in soils is respired -- but different pools might have different rates of respiration. Sometimes these pools are distinquished as an active soil organic carbon pool and slow soil organic carbon pool. Although the reference of ``slow'' causes confusion with long-term, geologic carbon, but soil organic carbon remains a component of what we are refering to as biosphere carbon. 

The surface of the soil tends to have more SOC and microbes that can use that carbon for respiration. Lower down in the soil profile, we tend to see lower amounts of SOC and lower microbial biomass (Figure~\ref{fig:soilcarbon}. In addition, soils in the lower part of the profile tend to have more aggregation that protects SOC from microbial attack, thus a key area that soil carbon can seqeustor carbon. 

In addition to these microbial biomass and aggregate patterns, the microbes aree more senstive to temperature changes near the surface as measured by Q10 -- the rate of biochemical processes with a 10 degree C increase in temperature. Thus, soil processes, such as respiration, is likely to increase more near the surface with global warming that the lower part of the soil profile.  

\begin{figure}
\includegraphics[width=\linewidth]{images/earth-system/Q10-SOC-Regulation.jpg}
\caption{Regulatory Mechanisms of the Temperature Sensitivity of Soil Organic Matter Decomposition in Alpine Grasslands (Source: \citet{Qineaau1218, CAS2021researchers}).}
\label{fig:Q10-SOC}
\end{figure}


\section{Fossil Fuels and Carbon Dioxide Trends}\label{sec:fossilfuels}

As part if the industrial revolution, our energy sources have put more \carbondioxide from the biosphere (soils and forests) and geologic carbon (coal, petroleum). 

\subsection{The Signal of Geologic and Biosphere Carbon in Atmosphere}

The combined contribution from geologic and biosphere carbon in the atmosphere is clearly documented from numerous sources. First, look at data collected at the Mauna Loa where \carbondioxide measurements have been taken continuously since the late 1950s (Figure~\ref{fig:maunaloa2}).

\begin{figure}
\begin{knitrout}
\definecolor{shadecolor}{rgb}{0.969, 0.969, 0.969}\color{fgcolor}
\includegraphics[width=\maxwidth]{figure/maunaloa3-1} 

\end{knitrout}
\caption{Carbon Dioxide Measure on Mauna Loa, HI}
\label{fig:maunaloa2}
\end{figure}






\chapter{Monsoons and East Asia Climates}

\section{Temperature Gradients and Latitude}

\subsection{Monsoon -- Dominant Climate for SE Asia}

Monsoons are identified with copious seasonal rainfall as well as with steady seasonal winds, which reverse their direction in a persistent manner with surprising regularity. The term `monsoon' has its origin in the Arabic word `mausim', which means season. Seamen used this word for many centuries to describe a system of alternating winds over the Indian Ocean and Arabian Seawhich blow from the north-east for six months and from the south-west for the remaining six months. In general, the shift in seasonal wind is about 120$\degree$, which is mainly caused by the differential heating of land and ocean. Monsoon winds are most pronounced in the summer season of either hemisphere. The monsoons blow over East Asia, South Asia, Africa, and northern parts of Australia. Recently the presence of a monsoon has also been recognized over south-western parts of North American continent. Overall, about half the tropics or a quarter of the surface area of the entire globe may be defined as having a monsoon climate.

The annual cycle of the monsoon contains two strong components. One is the summer monsoon, which lasts almost from June to September. The second component, the winter monsoon, lasts roughly from December through February. The summer monsoon in southern Asia is a huge climate system in which the Indian monsoon and the East Asian monsoon are the principal constituents.

For millions of people in monsoon regions, monsoon rainfall is the only accessible source of water. If the monsoon fails, people's lifestyles can be severely disrupted. Accurate prediction of monsoon onset and quantum of monsoon rainfall for a few days up to a season can make a difference between agricultural success and failure. Accurate monsoon forecasts with sufficient lead times are also required for government action and public response. Similarly, prolonged droughts and floods can bring excess hardship to the people in the monsoon region. These may lead to loss of life and property, and a heavy economic burden on the people as well as on the government.



\chapter{Critical Zone}\label{ch:critical-zone}

\chapterauthor{Marc Los Huertos\footnote{The chapter was first drafted by Los Huertos (2021). The author recieved valuable feedback from X, and Y and Z to improve the chapter.}}

\section{What is the Critical Zone}

The crticical zone refers the the portion of the Earth's skin or where rock meets life. The Critical Zone supports all terrestrial life.

The critical zone includes the following:

\begin{itemize}
  \item A permeable layer from the tops of the trees to the bottom of the groundwater;
  \item An environment where rock, soil, water, air, and living organisms interact and shape the Earth's surface;
  \item Water and atmospheric gases move through the porous Critical Zone, and living systems thrive in its surface and subsurface environments, shaped over time by biota, geology, and climate.
\end{itemize}

All this activity transforms rock and biomass into the central component of the Critical Zone - soil; it also creates one of the most heterogenous and complex regions on Earth.

Its complex interactions regulate the natural habitat and determine the availability of life-sustaining resources, such as food production and water quality.

These are but two of the many benefits or services provided by the Critical Zone. Such `Critical-Zone Services' expand upon the benefits provided by ecosystems to also include the coupled hydrologic, geochemical, and geomorphic processes that underpin those ecosystems.

\begin{figure}
\includegraphics[width=.8\textwidth]{images/critical-zone/criticalzone.jpg}
\caption{The Critical Zone is an interdisciplinary field of research exploring the interactions among the land surface, vegetation, and water bodies, and extends through the pedosphere, unsaturated vadose zone, and saturated groundwater zone. Critical Zone science is the integration of Earth surface processes (such as landscape evolution, weathering, hydrology, geochemistry, and ecology) at multiple spatial and temporal scales and across anthropogenic gradients. These processes impact mass and energy exchange necessary for biomass productivity, chemical cycling, and water storage.}
\label{fig:criticalzone}
\end{figure}

\subsection{What are the environmental implications of the Critical Zone?}

The critical zone as a concept and as a material space pushes us to think of the porousity of the Earth's surface --- the gas and fluid flows through rocks, soils, and plants. We can begin to appreciate the complexity of the transport and fate of chemical pollutants as they enter the soil and become part of the vadose zone and perhaps the ground water table -- moving with water and diffusing through the water, simultaneously.

\section{Hydrologic Aspects}

\subsection{Permeability and the Water Cycle}

Except for a neglible fraction, all water on the planet is stable, thus the total amount is constant. But water can move between various ``pools'', for example between the oceans and atmosphere, via evaporation and precipitation. On land, water might evaporate from surface waters or the soil or transpired from plant surfaces. Evapotranspiration is the sum of soil evaporation and plant transpiration. Thus, the soil and plant surfaces are permeable surfaces. Water can infiltrate into soils, which is called infiltration, which can move move further downward through the soil and beyond or percolation. Thus, below the soil and below is also permeable. Water that does not infiltrate flows across the Earth's surface as runoff and is discharged into the ocean or lakes that have not outlet. 

The rate that water flows through the water cycle varies dramatically. For example, permeability surfaces slow the rate of water movement. A terminal lake or sink may fill up with discharge with spring runoff, only to evaporate in the late summer. Thus, the water cycle a set of overlapping rate limiting steps, which to me makes it more interesting than the water cycle that just identifies the processes of evaporation, transpiration, precipation, discharge, and infiltration. 

The travel time $\tau$ within a compartment or pool can be measured, about when it enters ($t_{in}$) and exits ($t_{out}$):

\begin{equation}
\tau = t_{out} - t_{in}
\end{equation}

\noindent but why is knowing these important?  

\red{Add: June 2021}

\subsection{Subsurface Hydrology}

As water infiltrates into the soil or flows in the pour space below the surface waters as it becomes part of the ground water. The study of this water might be called subsurface hydrology or ground water hydrology or hydrogeology. 

There are two main areas of subsurface waters: saturated zone and unsaturated zone (Figure~\ref{fig:groundwater}).

The saturated zone is the region where all the spaces between the particles or rocks are filled by water. The surface of the saturated zone is the water table. 

The unsaturated zone is also called the vadose zone has some percentage of the pour spaces have air. The vadose zone also includes an area called the capillary fringe. Because of the surface tension of water, water is found in between particles above the water table and this zone is referred to as the capillary fringe. 

\begin{figure}
\includegraphics{images/critical-zone/groundwater}
\caption{Diagram of Ground Water. }
\label{fig:groundwater}
\end{figure}

\subsection{Saturated Zone}

\subsubsection{Aquifers and Aquitards}

Aquifers are have different ages \citep{sprenger2019demographics}.

\citet{REF, penna2020water}

\subsubsection{Confined and Unconfined Aquifers}

\subsubsection{Ground Water Flow}

TAKE A LOOK at some flow net sketches that will help clarify the relationships between aquifer matrix, and groundwater movement.


In general, water flow is driven by potential energy, e.g.  where water flows down hill, driven by gravity. If water flows from point A to point B, the the potential energy for the flow is the height of the water at point A minus the height at point B, which can be symbolized as $dl$. The head potential is the heigh difference divided by the distance between the two points.  

DARCY'S LAW

\begin{equation}
Q = KIA
\end{equation}

In 1856, Henry Darcy studied the movement of water through porous material. He determined an equation that described groundwater flow. The following description tell how Darcy determined his equation:

A horizontal pipe filled with sand is used to demonstrate Darcy's experiment. Water is applied under pressure through end A, flows through the pipe, and discharges at end B. Water pressure is measured using piezometer tubes (thin vertical pipes installed at each end of the horizontal pipe). The difference in hydraulic head (between points A and B) is dh (change in height). Divide this by the flow length (i.e. the distance between the two tubes), dl, and you get the hydraulic gradient ( I ).

The velocity of groundwater is based on hydraulic conductivity (K), as well as the hydraulic head (I). Therefore, the equation determined by Darcy to describe the basic relationship between subsurface materials and the movement of water through them is Q = KIA where Q is the volumetric flow rate (or discharge) and A is the area that the groundwater is flowing through. This relationship is known as Darcy’s law.

DISCHARGE
%* symbol-Q * units-volume/time EX. (m^3/day) * volume of water flowing through an aquifer per unit time * FIND WITH DARCY'S LAW Q = KIA

AREA OF FLOW
%* symbol-A * units-distance squared EX. (m^2) * Cross-sectional area of flow. (i.e. aquifer width x thickness)

Now, rearrange the equation to Q/A = KI, which is known as the flux (v), which is an apparent velocity. Actual groundwater velocity is lower than that determined by Darcy, and is called Darcy Flux (vx)

FLUX
%* symbol-v * units-distance/time EX. (m/sec) * v=Q/A=KI * this is a velocity measure and gives the IDEAL velocity of groundwater (assumes that the water molecules can flow in a straight line through the subsurface). * this is ideal because it doesn't account for tortuosity of flow paths (this means that the water molecules actually follow a very windy path in an out of the pore spaces and so travel quite a bit slower in reality than the flux would indicate).

DARCY FLUX
%* symbol - vx * units - distance/time EX. (m/sec) * vx = Q/An = KI/n * This is the ACTUAL velocity of qroundwater and DOES account for tortousity of flow paths by including porosity in its calculation.

Darcy's law is used extensively in groundwater studies. It can help answer important questions such as what direction an aquifer pollution plume is moving in, and how fast it is traveling

\subsection{The Vadose Zone}

The vadose zone is the 

Jeji is a volcanic island is located some XX km south of the Korean Penisula. Water runs off the steep slopes quickly and water supplies are limited on the island. To adddress this...\citet{lee2017fifty}.

\begin{figure}
\includegraphics[width=\linewidth]{images/critical-zone/Lee-Vadose.png}
\caption{... (Source: \citep{lee2017fifty}).}
\label{fig:vadose2}
\end{figure}



\chapter[Land Use Change and Monitoring]{Science of Remote Sensing and Biogeochemical Changes to the Land}

\chapterauthor{Samantha Beaton}

\section{Example of a Place: Urbanization and Land Use Change in Shenzhen, China}

Following the late 20th century reform period, China has embarked on a mass urban renewal project transforming swaths of vegetated land into areas serving the country’s growing urban needs. Considered among China's most innovative, modern cities, Shenzhen has blossomed from the products of land use change. In 12 years between 1996 and 2006, the region saw a significant drop in arable land (covered by grassland or forests) from 51.36\% to 45.72\% \citep{qian2016urban} (Figure~\ref{fig:shenzhen}). Land use change, regardless of its impacts, is nevertheless occurring at a rapid pace. This phenomenon might be particularly felt in areas of Southeast Asia undergoing similar urbanization processes as China (the scope of this chapter), but it is certainly relevant worldwide. By paying attention to rising land use changes, we can understand how such processes, while a contributing factor to growing civilization, are nevertheless coupled with cascading environmental and social impacts. How does land use change both regionally and locally affect the environment, and thus its people?

\begin{figure}
\includegraphics[width=\linewidth]{images/land-use/Shenzhen-cityscape.png}
\caption{(A) Shenzhen cityscape, China. (B) Change in construction land within Shenzhen from 1979, 1986, 2005, and 2014 \citep{qian2016urban}.}
\label{fig:shenzhen}
\end{figure}

As most scholars would agree, we are no longer situated in the Holocene, but have entered into a new and still-evolving geological age. ``Anthropocene'' describes how human activities have made massive ecological impacts on the world like no other time in our world's history. It is a period of ``irreversible destabilization of the global climate system—an impending climate crisis'' \citep{whyte2018indigenous}. Another version of this age is described as the ``Capitalocene'' which highlights how it is not just human activity, but specifically the capitalist economic system our colonialist Western society depends upon, that has shaped the conditions for such global change. Yet another term circulating in academia is introduced by theorist Donna Haraway: the ``Chthulucene'' \citep{haraway2016staying}. With roots derived from the word \emph{chthonic}, describing the worlds beneath our feet, Chthulucene understands our place amongst a deeply connected, multi-species realm. It argues that, to mitigate the global change humans have caused, we must transcend beyond our human activities and economic structure to interact with the world differently.

Whether it goes by the name Anthropocene, Capitalocene, or Chthulucene, the vast scale of change instigated by human systems cannot be denied. The supposed ``age of man,'' as described by environmental artist and engineer Natalie Jeremijenko, asserts how humans are ``major biogeochemical forces in the world.'' By simple definition, humans have a true impact on the land, which then responds. Humans are entangled in the forces of nature.  

Traditional Chinese medicine (TCM), a thousand-year-old philosophy indigenous to China and practices throughout Asia, is fundamentally understood through a holistic, integrative perspective that values humans as deeply connected with the surrounding ecosystem. How might this perspective aid in the understanding of climate change and human land-use within the environment?

Within traditional Chinese thought, \emph{Qi} is the immaterial substance that creates the atmosphere around us and envelopes each element together into one system, both within individual bodies and the greater environment. Just like how human ailments are not understood on a microscopic level, but upon looking at the body as a whole, this scale of focus can be magnified to include the greater environment \citep{kelly2012yin}. Climate, then, represents the Qi of nature as the larger, encompassing body \citep{sun2020adaptation}. In this framework, humans act as cells within this body. As with the traditional understanding of the human body, when the dynamic between cells and body are disrupted, as with one dominating the other, the greater system suffers. In the context of yin and yang, in which two opposite factors are interrelated as one, everything exists in a finite balance where all actions yield a direct response. As climate change progresses, global warming increases (yang) proportionate to global cooling effects decreasing (yin) spurred by the collective imbalance of the most Western industrialized countries with proportionate negative effects falling most often upon the most marginalized communities \citep{kelly2012yin}. When a landscape is changed, the environment reacts to the stress in a proportionate amount, either positive or negative depending on the action. Thus yin-yang theory within TCM can be acutely associated with ideas of sustainability. When humans perform an action disharmonious to natural systems, they reflect proportionally back into the environment as negative responses for the entire body. Since the history of China has predicated on how well people have been able to thus adapt to environmental change, TCM and other forms of Indigenous knowledges are fitting to climate mitigation strategies. The big question thus becomes how to impact the environment and live within it ``sustainably,'' in a way progressing from our current practices with radical change, deconstructs power systems, and moreover reimagines our relationship \emph{with} nature.

``Land use changes'' encapsulates all the ways that humans change the land. Through this specific lens, we can delve deeper into questions of where nature belongs and how different people experience nature differently. What does nature look like in and around a populated region? How is the greater environment changed? While land use changes may only be implicated onto local land, they can be a contributing influence to places across the globe. Within the context of Southeast Asia, this chapter will specifically analyze land use changes through various places to understand its most prominent, intersecting issues.


\section{What is Land Use Change?}

Land use changes are the ways in which land is converted or transformed by humans to serve another purpose than its previous. Throughout history, land has been changed in large part for agricultural purposes, whether intentional or by consequence. Most recently, emerging systems of colonialism and capitalism have largely influenced how that land has been changed and what role it plays as an economic resource. It is also being increasingly used to satisfy urban needs. Regardless of the specific reason behind the action, though, ``land use change'' can be as simple as describing the shift land experiences from one purpose to another, one state of being to one marked with different characteristics. Ultimately, analyzing this concept can help us understand how spectrums of change can influence biogeochemical cycling, ecosystem stability, and balance dynamics within a given environment. 

\subsection{What Factors Drive Land Use Change?}

Land use changes are influenced by both \emph{proximite} (direct changes to environment from proximate sources) and \emph{underlying} (indicative of larger social/biophysical processes that drive change) forces that set certain conditions specific to the place. Underlying forces are largely considered to include economic, policy and institutional, technological, sociopolitical and cultural, demographic factors. As used in a study analyzing land-use change across 8 regions in Southeast Asia over 10 years, proximite causes focused on agricultural expansion, wood extraction, and infrastructural development as the main factors, with space for others in a miscellaneous category \citep{fox2005land}. In corroboration with previous research, \citet{fox2005land} found that multiple causes contributed to land use change, with agriculture being a common thread. On the underlying level, state policies (especially in relation to climate and property rights) influenced these motivations the strongest, as well as economic market pressures to establish commercial agriculture sites and the general impact of land tenure systems that determine land ownership.

\subsection{How Is Land Use Change Measured and Quantified?}

Land use change is primarily measured through \emph{remote sensing}. Temporal and spatial data are compiled from aerial photos, satellite images, and various radar sensing systems to be analyzed into topographic maps and more. Remote monitoring may also integrate interdisciplinary practices to further understand socioeconomic and institutional factors that affect a given landscape, as well as how that land has changed through history.

\section{The Science, Art, and History of Remote Sensing}

Socrates once wrote: ``Man must rise above the Earth to the top of the atmosphere and beyond, only thus will he fully understand the world in which he lives.'' The science of remote sensing attempts to do just that. The term ``remote sensing'' is attributed to have been first officially used in the early 1960s by Evelyn Pruitt, a geographer within the U.S. Office of Naval Research \citep{fussell1986defining}. In short, it is used to describe the science and art of identifying, observing, measuring, and analyzing a target object from a distance without direct contact \citep{graham_NASA}.  

\subsection{Invention of Aerial Photography: Pigeons, Kites, Balloons}

Roots of remote sensing began upon the invention of photography, with one of the first practical processes developed in 1839 by Louis Daguerre in France \citep{moore1979picture}. Aerial photography (that was also full-spectrum-sensitive) can be traced back to 1868 when Nadar aka Gaspard-F\'elix Tournachon captured the first from aboard a hot air balloon \citep{salomonson}. From then, aerial photography was captured with cameras attached to pigeons, kites, and even more balloons (Fig~\ref{fig:pigeons}). By the year 1909, the first photograph was finally taken from an airplane \citep{humboldt}. 

\begin{figure}
\includegraphics[width=\linewidth]{images/land-use/Pigeons.png}
\caption{Images taken by pigeons at the turn of the 20th century with timer cameras patented by German inventor Julius Neobrunner. The simultaneous development of aerial photography, however, soon trumped avian photography \citep{denhoed}.}
\label{fig:pigeons}
\end{figure}

\subsection{Remote Sensing in the Early-to-Mid 20th Century and Military Reconnaissance}

Soon after, aerial photography was quickly taken up by the U.S. military for various reconnaissance purposes. The progressive development of aerial photography is deeply tied to European military advances and demands beginning during World War I when cameras were mounted to German and American aircrafts to monitor positions of troops \citep{salomonson} (Figure~\ref{fig:WW2-aerial-photo}). Records show that in 1918, the French military captured and developed up to 10,000 photos per day from such camera mounts \citep{moore1979picture}. 

\begin{figure}
\includegraphics[width=\linewidth]{images/land-use/WW2-aerial-photo.png}
\caption{Aerial military reconnaissance during World War I. Trench patterns (left) and cameras mounted on military aircrafts for long-range photography (right) \citep{smithsonian}.}
\label{fig:WW2-aerial-photo}
\end{figure}

During World War II, non-photographic remote sensing methodologies were developed using radar (radio detection/ranging), thermal infra-red detection, and sonor (sound navigation) systems \citep{moore1979picture}. In particular, radar systems were developed by Britain and the U.S. to track ships and aircrafts \citep{salomonson}. In the 1950s, the University of Michigan spearheaded development of other systems including infrared radar and synthetic aperture radar (SAR) imagery, which was used (and later declassified) during the 1960s in experimental programs by the U.S. National Reconnaissance Office \citep{salomonson}; \citep{xiao2019remote}. At this time, the significant development of remote sensing, guided by conflict, occurred as U.S. and European military assets, so most Asian countries had not yet had the technology to similarly catch up; it would only be until the late 1990s into early 2000s when remote sensing technology was solidified as a facet of government use in Asian countries.

\subsection{Satellites in Space: Remotely Sensed Earth Observations}

The beginning of remote sensing from space began with V-2 rockets, the first long-range missiles developed during World War II \citep{salomonson}. As the late 1940s saw the end of WWII and the very beginnings of the Cold War, the Soviet Union and U.S. both took advantage of captured V-2 rockets to start research and development on launch vehicles for space programs \citep{britannicamissiles}. While the first picture taken of Earth from the sky (high enough to see its curvature) was captured in 1935 on a balloon 13.7 miles high, it was in March of 1947 when a camera placed in the nose shell of a V-2 rocket flew more than 100 miles above ground; once the series of pictures were stitched together, it clearly showed for the first time Earth against the black space, spanning more than a million miles together (Figure~\ref{fig:remote_earth}) \citep{dunbar2017}.These pictures set the stage for the potential of remote sensing to be used to monitor Earth's processes. Still, it was not until the mass development and launching of various satellites during the Space Race when this prospect was cemented as reality.

\begin{figure}
\includegraphics[width=\linewidth]{images/land-use/remote_earth.png}
\caption{First remote photographs of Earth from space pieced together as one image \citep{remote_earth}.}
\label{fig:remote_earth}
\end{figure}

In 1957, the Soviet Union launched the first satellite into space, the Sputnik 1, with the U.S. following shortly behind \citep{humboldt}. In 1964, photographs obtained from the Mercury-4 spacecraft were recognized as highly-valuable to understanding Earth sciences \citep{moore1979picture}. National institutions aside from NASA, such as the U.S. Department of Agriculture (USDA), U.S. Geological Survey (USGS), and National Oceanic and Atmospheric Administration (NOAA) all realized the potential application of remote sensing to fields of agriculture, hydrology, archaeology, geography, oceanography, and meteorology \citep{bauer2020remote}. In 1960, NASA launched its first low-orbital experimental weather satellite, the TIROS-1, who's success proved the feasibility of capturing images of Earth from space \citep{dunbartiros}. Thus welcomed the 1968 development of the Earth Resources Technology Satellite series, now called LANDSAT, sent to record worldwide images on a continual basis--a series that has continued ever since \citep{moore1979picture}. From these developments, images would from then on not only be turned towards enemy troops, nor towards space, but also towards the Earth--the beginning of modern remote sensing.

\subsection{Satellite Remote Sensing in Asia}

US and USSR sensor series launched in the 1960s beckoned in a new era of Earth observation monitoring. While operation of satellite imaging on Asian seas for military monitoring had been occurring throughout this time, it was only until the 1970s into 1980s when space organizations were also established in Asian countries themselves, including the People's Republic of China, India, Japan, and South Korea \citep{mitnik2019historical}. The primary motivations for these countries was not only to gain access to global ocean-atmospheric data, but to be able to monitor coastal zones surrounding their countries \citep{mitnik2019historical}. As described by Mitnik, 2018, due to their emphasis on regional coastal monitoring, countries in Asia were the first to develop satellite instruments specifically emphasizing low-orbit, sun-synchronous, and geostationary parameters with high resolution and sea surface temperature, marine pollution, wind speed measuring abilities. The People's Republic of China saw their first satellite in April of 1970, named Mao 1, sent to blast a patriotic song into space, and their first oceanic satellite in 2002 \citep{mitnik2019historical}; \citep{huadong2013earth}. In 2010, China officially launched its high-resolution Earth observation system, with its first series sent out to space in 2013 \citep{huadong2013earth}. Since then, China's remote monitoring system has progressively grown, but is still relatively just beginning.

\subsection{Critical Remote Sensing}

Still, the history of remote sensing cannot be understood without analyzing its use in producing and controlling knowledge to monitor and manage global resources and populations. After all, the purpose behind remote sensing is deeply tied to map-making and cartography, which itself has a fundamentally colonial history. During the Western world's Age of Exploration, or Age of Information, between the 15th and 18th centuries, Europe financed global expeditions not only to conquer and colonize, but to gather information on the land's resources and people \citep{bennett2020pixel}. In this way, maps were used to classify life for the purposes of exploitations. They were used to mark territory, regardless of local dynamics, for the use of settlers. Maps, then, were essential to the imperialist settler colonial state in dominating a land. As noted by University of Hong Kong Geography professor Mia Bennet, the quest for information is ``an activity critical to empire'' \citep{bennett2020pixel}. If so, what then must be true to our current quest for information? What must we be critical of if remote sensing is just another rendition, albeit modernized, version of the colonizing tactics of those in power?

Additionally, a key component to colonial mapmaking was the arbitrary creation of boundaries. Remote sensing analyses can fall into a similar ``territorial trap'' that disregards how human activities, resources, and environmental change occurs as natural flows that transcend transnational boundaries \citep{bennett2020pixel}. By working within discrete state boundaries, data and resulting maps can reproduce certain narratives, particularly ones that suppress developing countries over developed ones.We can see this in the map below (Figure~\ref{fig:Bonelli Erosion Map}) on soil erosion rates and vulnerabilities across the globe, dictated by national boundaries. While developing countries are illustrated in red, indicating high rates of soil erosion and land degradation, developed Western countries look perfectly clean. This example is just one in a long line of historical misrepresentations of local occurrences that are ever-more complex than what can (and molded to) be portrayed in bordered images. Systems of remote sensing must strive to actively dismantle the colonial, militaristic foundation it is so strongly built upon.

\begin{figure}
\includegraphics[width=\linewidth]{images/land-use/Bonelli_Erosion_Map.png}
\caption{Global land use change and soil erosion estimates, measured on a scale of severity by colors green (low erosion risk) to red (high erosion risk). As denoted by the red color, Southeast Asia is expected to experience a higher risk of soil erosion compared to other regions in the world \citep{borrelli2017assessment}.}
\label{fig:Bonelli Erosion Map}
\end{figure}

\subsubsection{Is Satellite Imagery Truly ``Rational''?}

In an analysis of China's ``One Belt, One Road'' policy (aka the Belt and Road Initiative or BRI), which as of 2019 lacked an official map shifting the focus to governmental remote sensing illustrations for accountability of the policy's efficacy, Bennet importantly points out an illusory distinction between traditional maps and those obtained through remote sensing: ``Whereas maps are considered malleable representations, satellite imagery is imagined as objective, neutral, and importantly, rational--a key word in Chinese narratives of development and modernization.'' In other words, while remote sensing may easily be seen as perfectly objective, it requires interpretation and representation. Therefore, it too is inherently political. While the goal of remote sensing is to accurately capture reality, the ``processing of billions of pixels can just as easily be edited or reframed to help turn policies into self-fulfilling prophecies'' \citep{bennett2020pixel}. Remote sensing produces entire analyses of landscape that may look `rational' or `self-evident'; the challenge thus requires a critical look into each trillion pixels and the spaces left unattended within its representation.

Bennett thus presents a methodology of three main principles to foster critical remote sensing:

\begin{enumerate}
\item Any analysis must ``be sensitive to the (geo)politics involved in the production and analysis of satellite imagery'' by looking closely at the motivations of goals of the controlling state
\item Analysis should not conform but rather act to contest ``dominant social meta-narratives and discourses about modernization and development'' that often use sensing to sustain development and paint a `pretty' picture of modernization. Critical remote sensing includings the encouragement of political debate.
\item Sensing should include mixed methods of qualitative and quantitative analysis at various research scales and levels throughout its production.
\end{enumerate}

Ultimately, the increased use of remote sensing systems within government agendas and policy initiatives requires the need for active critique on the politics and positionalities ingrained in satellite imagery, especially due to how ``images present the guise of the entire truth in a way that can dissuade debate'' \citep{bennett2020pixel}. Data must be transparent, management must be accessible, and analysis must be critical to rightly expose the complexities of the (sociopolitical) environment. Lastly, while remote sensing is without a doubt an advantageous to our modern society, it cannot entirely replace local observation and place-based understanding, but rather is best kept in balance.

One further point of hindrance to a more fair and just remote sensing system is that the large majority of imaging satellites currently being used are operated and controlled by the most powerful of countries. Increased participation in remote sensing can be created by giving smaller states, organizations, and universities the opportunity to launch their own satellites rather than needing to depend on powerful governments: a ``democratization of space'' \citep{bennett2020pixel}. While this potential is increasing in actuality, it is far from truly increasing representation. Still, it was promising in 2008 when the USGS decided to make all LANDSAT records available to the public for free, when previously it was kept behind a paywall \citep{xiao2019remote}. As remote sensing becomes increasingly popular, more data in greater varieties are being released and made available to the global public. Using this available data, researchers, scientists, and the public are dedicating more time to analyzing ecological effects of land use change within Earth's systems, a practice becoming evidently more important as land degradation and desertification intensifies, necessitating directive action with transparent sources of data.

\section{Ecological Effects of Land Use Change on Soil, Air, and Water}

Land use change is associated with devegetation and the general destruction of land, at least the land that existed upon conversion). Not only is land use change the biggest driver in terrestrial biodiversity loss, but it has extensive impacts on the soil. For instance, replacing native vegetation with crops or infrastructure initiates soil degradation, hinders hydrological cycles, and thus negatively impacts the terrestrial ecosystem. It is for this reason that anthropogenic land use change is the primary accelerant of soil erosion. Various erosional processes enhance soil degradation beyond the rate of natural soil regeneration. As a result of extensive sediment loss, critical soil properties including nutrient concentration, moisture content, and carbon sequestration are negatively impacted. Specifically related to the carbon cycle, uprooting the land and changing its function creates a loss potential for carbon sequestration. Estimates propose that approximately 12.5\% to 17\% of carbon emissions have originated from loss during land use and land cover changes \citep{houghton20122012};\citep{paustian2016climate}. Therefore, understanding the processes behind erosion is important before considering its long-term environmental effects. To do so, we will look at a region in China where human-accelerated erosion is prevalent: the Loess Plateau. Due to its long history within northeastern China, the Loess Plateau is an apt case study to analyze how a place so important to humanity is related to soil degradation, land management, the potential for restoration, as well as how to critically go about such a process.

\subsection{The Cradle of Eastern Civilization}

The Loess Plateau, with agriculture beginning in the region nearly 7000 years ago, was once the cradle of Chinese civilization. It receives its name from the loose, porous, easy-to-farm sedimentary deposits originating from the Gobi Desert that define the region. From southeast to northwest, native vegetation varies from forest to forest-steppe, typical steppe, desert-steppe, and steppe-desert with variation as a product of precipitation, rainfall, and wind gradients (Figure~\ref{fig:loess_climate_map}) \citep{qiao2018factors}. Covering approximately 640,000 sq km, the Loess Plateau was also once the largest distribution of such fertile soil on Earth.

\begin{figure}
\includegraphics[width=\linewidth]{images/land-use/loess_climate_map.png}
\caption{Loess Plateau climate map with annual precipiation and temperature averages \citep{luy2013apolicy}.}
\label{fig:loess_climate_map}
\end{figure}

However, due to intense human activity in the past 2500 years (largely agriculture as well as social conflicts), the high flat plain has experienced severe erosion, transforming it into a land marked by steep barren hills and deep-cutting ravines. Agricultural processes led to the overgrazing of land that, over a very long period of time, damaged the vegetation. Without trees or plants rooted in the earth, water no longer seeped into the soil but rather evaporated immediately or ran off the hillsides, eroding considerable amounts of topsoil away. The land of the plateau eroded seasonally, both by wind and water \citep{xiao2019remote}. Over time, massive amounts of silt swept into the Yellow River (thus giving the river its distinct name), with an estimation of 90\% river sediment originating from Loess Plateau erosion, hence the name of the river \citep{wang2016reduced}. Due to lack of water retention, this phenomenon contributed to increased flooding events: as quickly as water would come, it would just as quickly go away so that the region experienced severe droughts. While this is all true, it must be kept in balance with the knowledge that the Loess Plateau was, for thousands of years, a place that built civilization. Generations of people who lived within the region, a center for traditional pastoral life, gave and received a great deal from the soil. The longstanding history of humanity within the region should not be taken for granted beneath the eroded landscape we may see today.

\subsection{Principles of Pedology: Soil Processes and Erosion Types}
Soil is the combination of biotic organisms and dirt—it is the thick, rich, brown substance that gives rise to life on earth. As reviewed in Chapter~\ref{ch:critical-zone} on the Critical Zone, this region of soil acts as a contact zone between the pedosphere, biosphere, lithosphere, and atmosphere. In the case of China and its Loess Plateau, these soil principles are no different. However, compared to other regions around the world, the Loess plateau is certainly unique in how erosional processes sculpt its landscape. In this section, we will delve deeper into understanding factors of soil formation, soil forming processes, and types of water-wind erosional systems.

\subsubsection{Factors of Soil Formation}
Soil is a function of five main environmental factors \citep{jenny1994factors}; \citep{birkeland1984soils}; \citep{paz2016factors}:
\begin{enumerate}
\item \emph{regional climate} (Cl): characterized by average temperature and precipitation
\item \emph{topography} (R): describes the shape and slope of a landscape, direction of slope face, and proximity to water table—topography in particular directly influences erosional potential, predicts moisture content, and determines the stability of loose materials
\item \emph{parent material} (P): initial state of system (at time zero) i.e. the materials from which the soil is formed
\item \emph{organisms} (O): the biotic factor i.e. organic matter at site
\item \emph{time} (T): since formation of slope, deposition, exposure at surface, or any other point of change--soil theoretically reaches a steady state of equilibrium after a steep period of change
\end{enumerate}

All five factors are interdependent with one another. Soil is thus a function of all variables expressed in the following equation:

\begin{equation}
S  or  s = f(Cl, R, P, O, T...)
\end{equation}

where S is soil and s is soil property. The proceeding three dots leave flexibility for additional local or regional factors to soil formation, such as anthropogenic factors. This equation exhibits how soil formation, development, and maintenance is absolutely localized to a place; after all, within each variable are multiple sub-factors specific to each region, even each hillslope (Figure~\ref{fig:soil_influences}). In order to fully understand the processes involved in the production of the Loess Plateau landscape observed today, we must go deeper into its unique characteristics as an erosional landscape in northern China.

\begin{figure}
\includegraphics[width=\linewidth]{images/land-use/soil_influences.png}
\caption{Factors affecting soil formation across the biosphere, lithosphere, hydrosphere, and atmosphere \citep{paz2016factors}.}
\label{fig:soil_influences}
\end{figure}

\begin{figure}
\includegraphics[width=\linewidth]{images/land-use/pedogenic_processes.png}
\caption{Schematic illustration of the four pedogenic processes that impact soil development\citep{paz2016factors}.}
\label{fig:pedogenic_processes}
\end{figure}

First, however, we will zoom out slightly to learn about pedogenic processes, which describe erosion.

\subsubsection{Four Pedogenic Processes}
\citep{cornell_soils}

Layers of soil are constantly being acted upon by four main processes: additions, losses, translocations, and transformations (Figure~\ref{fig:pedogenic_processes}). These operations collectively act to develop (or degrade) the soil and evolve the landscape.

\renewcommand{\labelenumii}{\Roman{enumii}}
\begin{enumerate}
\item \emph{Additions} are factors that deposit material or other resources into the soil.
  \subitem Organic matter (plant material deposition); solutes (originating from groundwater); sediment (aerial or fluvial deposits); water (surface or groundwater discharge); organic or inorganic nutrients; energy from the sun.
\item \emph{Losses} describe processes that take away material or resources from the environment, ultimately altering the physical and/or chemical makeup of the soil.
  \subitem Erosion (sediment loss via wind, water, or gravity); groundwater leaching to bottom of soil profile (loss of dissolved or suspended solutes in higher layers); oxidation (loss of organic matter); uptake of nutrients by vegetation.
\item \emph{Translocation} processes occur across horizontal and vertical planes. Aside from gravity, they are most often determined by differential water or chemical gradients that facilitate movement of materials from areas of high concentration to areas of lower concentration.
  \subitem Clay is a good example of how material accumulates low in the profile; due to its relatively high density, clay deposits can form barriers where water accumulates.
\item \emph{Transformation} of soil components occurs via chemical or biological reactions. These processes influence the soil structure and other characteristics (i.e. color), largely dependent and relative to the parent material.
  \subitem One of the most common (and critical) transformation processes in soil is nitrogen fixation where nitrogen elements are taken up by microorganisms and converted into ammonia available for metabolic uptake by plants.
\end{enumerate}

\subsection{Types of Water Erosion}

Given the predominance of ``losses'' in forming China's contemporary Loess Plateau landscape, let us understand the different types of water erosion that can occur within the landscape. In doing so, we can appreciate how water erosion has the potential to become a persistent problem stuck in a positive-feedback loop, further exacerbated by other erosional processes (wind and gravity).

Erosion by water channels is a principal factor sculpting Earth's landscape, following the processes of particle detachment, transport, and deposition by forces of raindrops and runoff \citep{mccool_erosion}; \citep{shi2012soil}. The first stage of erosion is \textbf{splash erosion} in which raindrops make contact with the topsoil with enough transferred force to dislodge and displace soil particles. Rainfall erosivity, the capacity of rain to erode soil, depends on the kinetic energy of rain which is a function of raindrop mass and velocity, while soil erodibility describes soil particle detachment susceptibility, dependent upon its physical-chemical properties \citep{angulo2012splash}. Similarly, raindrop intensity determines the number of raindrops hitting a unit surface over a given amount of time \citep{angulo2012splash}. These descriptions are specific to each soil, and are important in predicting erosion events. The force of falling raindrops can be mitigated by vegetation cover, but on exposed soil, there is not much besides the forces holding particles together that protect soil from initiating erosion.

The second degree of erosion is \textbf{sheet erosion} instigated by a downhill slope gradient. Rainfall that does not permeate into the ground flows downhill, stripping away thin layers of topsoil, and carrying the sediment towards \textbf{interrill channels} i.e. the third degree of erosion. Interrills are regions in between rills where raindrops continue to detach soil particles in their transport \citep{gilley_erosion}. Interrill erosion is differentiated to \textbf{rill erosion} by their relative particle selectively. Where interrill erosion selectively removes finer particles, rill erosion is less or non-selective after a certain shear stress of the soil is exceeded \citep{shi2012soil}.

The fourth degree of erosion manifests as \textbf{gullies}, which form when the concentrated flow is so large that it cuts deep trenches into the landscape \citep{gilley_erosion}. Even if the water dries up, gullies remain permanent in the landscape and so deep that that cannot be crossed nor easily rectified. If the gully only appears seasonally in the same place after rain events and are small enough to be filled or otherwise rectified, they are described as ephemeral. Large runoff events can catalyze rapid development of gullies by expanding, deepening, and zipping them uphill, thus forcing ever-large amounts of sediment into the channel \citep{gilley_erosion}. 

Gully formation and its associated erosional impacts present serious threats to soil sustainability, worse further due to their permanent nature. Unfortunately, gullies are quite prevalent in regions across China. As reviewed by \citet{liu2020soil}, a 2015 study on a watershed in northeast China recorded that 92.8\% of total sediment yield originated from gully erosion of bare weathered stone \citep{zhao2015sediment}. In \citet{liu2020soil}'s own study on a region in southern China, they found the contribution of gully erosion to total sediment yield to be 52.27\% (averaged between 2005 to 2010). This value may be lower than their mentioned study because of the location as southern China's climate experiences less erosion than its northern counterpart. While the sediment contribution value obviously varies by region, topography, geology, land management practices (i.e. tillage), pre-existing gully development, and rainfall intensity, gully erosion is clearly more severe than sheet or rill erosion.

Beyond gullies, water eventually accumulates to form \textbf{streams}: bodies of water with a current that flows downhill by gravity within a channel. While streams might only cover a small area, the erosional effects they can have on a landscape can be severe. Streams begin to collect sediment loads, which includes suspended and dissolved particles, by scoring both the bottom of the waterbed and the sides of the channel. Meandering streams in particular have high sinuosity (amount a channel curves) that further enhances erosion. Made up of \textbf{tributaries} (smaller subsidiary channels that feed into main streams), streams grow larger by joining drainage networks, thus increasing in size, discharge, and load. The largest drainage basin associated with the Loess Plateau is, of course, the Yellow River Basin, granted its name because of the yellow loess sediment originating from the Plateau. Not only was the Yellow River historically the most sediment-laden river in the world, but 90\% of that sediment originated from the Loess Plateau \citep{wang2016reduced}. Thus, erosion is a significant process involving the Loess Plateau.

\subsection{Water-Wind Erosional Complex in the Loess Plateau}

Different types of erosions generate different amounts of sediment loss. In China's Loess Plateau, situated in what is known to be a ``water-wind criss-cross'' region, the predominant pedogenic process is the loss of sediment through erosion, specifically within a water-wind complex. In this environment, wind erosion accounts for approximately 39.7\% total average soil loss while water accounts for 60.3\%, substantially more thus making water erosion the predominant erosional process \citep{tuo2018relative}. This water-wind complex accelerates soil degradation beyond the rate of natural restoration. Ultimately, each landscape and location on erosional planes have different soil characteristics. What are characteristics of eroded soil within the Loess Plateau, and what can they tell us about erosional landscapes in general?

\subsubsection{Position on a Slope}
Slopes are divided between the top (shoulder), middle (back), and bottom (foot) in studies to compare soil organic matter (SOM) indices across positions and track erosional features. First and foremost, higher slope positions experience higher rates of erosion. As a measure of soil movement, Cs$^{137}$ inventories record Cs$^{137}$ greatest at foot positions than at upper positions: a clear indication of eroded material moving downhill with an erosional trend decreasing from top to bottom (Figure~\ref{fig:Cs_slope_position}) \citep{tuo2018effects}. In a graphical explanation, erosion rate and Cs$^{137}$ inventories are negatively correlated: the higher erosion rate, the more sediment is lost, resulting in lower amounts of Cs$^{137}$ (Figure~\ref{fig:gradient}). Since net soil loss consequently means net nutrient loss, other soil property indices can be used to further elucidate this trend.

\begin{figure}
\includegraphics[width=\linewidth]{images/land-use/Cs_slope_position.png}
\caption{Inventory of Cs$^{137}$ within the topsoil of three different positions on the same slope, in comparison to a reference field. Shows a clear relationship between slope position and Cs$^{137}$ inventory, and thus a path of erosion \citep{zhang2021characteristics}.}
\label{fig:Cs_slope_position}
\end{figure}

\begin{figure}
\includegraphics[width=\linewidth]{images/land-use/gradient.png}
\caption{Distributions of Cs$^{137}$ inventories and erosion rates on Loess Plateau slope in Northeast China, exhibiting active erosion. A negative correlation is observed between the two variables, with values shifted to show a correlative relationship in the gradient \citep{tuo2018effects}.}
\label{fig:gradient}
\end{figure}

For instance, depositional positions with high Cs$^{137}$ inventories (product of erosion) also display higher SOC and nitrogen (N) contents which is further correlated with soil aggregate size. Aggregates are primary particles of sand, silt, clay, and their associated nutrients bound up in clumps. In aqueous solution (soil with some amount of moisture content), sorptive particles (ions or molecules) bind to the sorbent compound to form sorbate functional groups on its surface (Figure~\ref{fig:sorption}) \citep{thompsonsorption}; \citep{patterson2003geochemistry}. Through these interactions, aggregates offer physical protection and biological protection to SOM from bacterial mineralization because soil bacteria do not have enzymes to decompose and break down the encased nutrients. Because of their protective potentials, aggregates therefore form from more unstable, labile soil materials encasing higher concentrations of nutrients such as SOC. As a result, higher proportions of aggregates are found at depositional positions with higher concentrations of SOC and labile soil \citep{zhang2021characteristics}. Clay particles often make up a large portion of aggregates because of how they stabilize C and other nutrients in its structure \citep{jia2017soil}. A correlated factor to aggregate production, because of its aqueous solution requirement, is moisture content which makes sense given how water often pools at depositional positions.

\begin{figure}
\includegraphics[width=\linewidth]{images/land-use/sorption.png}
\caption{Illustration of sorption process. While in this case the sorptive is a hydrated copper ion, it can be any ion, molecule, or other nutrient suspended within the soil solution \citep{thompsonsorption}.}
\label{fig:sorption}
\end{figure}

\subsubsection{Slope Position and Face Direction}

The position and topography of the actual slope also has significant implications to SOM characteristics. In the Loess Plateau, a semi-arid environment, wind erosion is dominant in the winter and spring, while water erosion is dominant during the monsoon rainy seasons of summer and autumn \citep{tuo2018relative}. Moreover, the dominant wind direction travels from the northwest. Therefore, northwest-facing slopes experience higher wind patterns on top of higher degrees of solar radiation. Compared to southeast-facing slopes, then, northwest slopes experience higher rates of wind erosion and subsequent water erosion. \citet{tuo2018effects} found this trend to be true in their study on vegetation and precipitation trends within the Loess Plateau. Specifically, in comparison of six slopes (three NW, three SE), northwest slopes exhibited erosion rates 58.8\% higher than those on southeast-facing slopes. As a combined consequence of more wind, solar radiation, and erosion than southeast slopes, northwest slopes exhibit lower soil moisture contents which then have direct impacts on vegetation viability.

A related consequence of erosion's particle selectivity is thus the soil makeup: northwest slopes have higher concentrations of sand (7.57\% higher than southeast), while southeast slopes have higher concentrations of clay and silt contents (21.01\% and 16.05\% more, respectively) than northeast slopes, in part because clay content is correlated with greater soil moisture. As with slope position, topography results in changes to nutrient and moisture contents. While upholding trends of higher erodibility on high slope positions, (Figure~\ref{fig:face_position_erosion}) showcases how northwest-facing slopes experience higher degrees of sediment loss signified by its lower Cs$^{137}$ inventories and thus higher erosion rates. As a result, northwest-facing slopes have lower concentrations of SOC and other nutrients, not to mention other soil indices such as soil moisture content.

\begin{figure}
\includegraphics[width=\linewidth]{images/land-use/face_position_erosion.jpg}
\caption{Cs$^{137}$ inventories and erosion rates by slope position on northwest-facing versus southeast-facing slopes \citep{tuo2018relative}.}
\label{fig:face_position_erosion}
\end{figure}


A couple conclusions can be drawn on soil characteristics in erosional landscapes:

\begin{itemize}
\item Erosion initiates movement of soil with its associated SOM and nutrient contents (i.e. SOC and TN stocks)  from erosional positions to depositional positions downslope.
\item Due to the historic geology of slopes, younger less-stable (labile) soil is preferentially removed from the topsoil for transport, producing higher concentrations of SOC-laden aggregates at depositional positions (thus smaller concentrations of more stable, finer particles) with lower bulk density.
\end{itemize}

From these two sections, we can therefore see how different slope positions and differential topographies between southeast and northwest slopes support different vegetation systems, implying a spatial heterogeneity of soil erosion, as well as soil indices such as carbon sequestration or soil moisture content. Still, how does the soil contents change moving down a profile within an erosional hillslope (beyond the normal horizon layers)? How might land use affect those characteristics?

\subsubsection{An Aside on Modelling and Estimation of Soil Erosion}

The \emph{Universal Soil Loss Equation} (USLE) was first published in 1965 through the USDA as an empirical model to estimate soil loss \citep{USDA_USLE}. It has since been revised (RUSLE) and used globally across a variety of landscapes. The RUSLE equation is as follows:

\begin{equation}
A = R \times K \times LS \times C \times P
\end{equation}

where A is the soil loss, R is rainfall erosivity factor, K is soil erodibility factor (function of sand/silt/clay content), LS includes slope length and steepness factor, C is vegetation cover, and P is a conservation support factor.


Another prominent data set used in soil erosion modelling is the use of Cs$^{137}$ as proxy data. Cs$^{137}$ is a radionuclide released into the environment during nuclear weapons tests during the 1950s to 1970s \citep{tuo2018relative}. The fallout was not only spatially and locally uniform, but the chemical rapidly (and irreversibly) fixed onto soil particles. Because of this factor, any redistribution of Cs$^{137}$ illuminates the movement of soil particles; erosion, then, can be monitored from Cs$^{137}$ when compared to an undisturbed control. Despite the negative socio-environmental consequences of nuclear weapons testing, it can nevertheless be used for the better in understanding erosion landscape characteristics.

As described  in \citet{zhang2021characteristics}'s study, the following equation can be used to measure how much Cs$^{137}$ is present at each slope position (As):

\begin{equation}
As or Ar = \sum_{i=1}^{n} C_{i} \times \rho_{i} \times h_{i}
\end{equation}

Where C is the mean concentration of Cs$^{137}$ in soil layer i, $\rho$ is the mean bulk density of the soil, and h is the soil layer thickness of that soil layer i.

\subsubsection{Profile Depth}

Not only do soil characteristics change by topography and slope position, but they also change through the soil profile \emph{within} those features. If it is important to understand processes across regional slopes, then it is just as critical to analyze the same conditions across soil horizons within those landscapes. In this subsection, we will delve even deeper into the complexities of SOM characteristics within erosional landscapes on a more detailed scale. As has been stated, given its importance to various processes within the critical zone, an apt measurement to examine spatial variation in soil characteristics is SOC.

At shallow depths, precipitation and temperature are primary drivers of SOC content. In a study conducted along an 860km transect line at 86 sites within the Loess Plateau region from south to north, precipitation did indeed have a clear effect on SOC \citep{jia2017soil}. On average, SOC decreases from south to north along the rainfall gradient, where both rainfall and thus SOC are lowest in the north. Two relationships to precipitation are responsible for this result: vegetation growth and clay content. Precipitation is a major factor of vegetation growth, which inputs organic C into the soil for greater average SOC. Similarly, clay content is associated with greater soil moisture from precipitation and contributes to greater SOC sequestration by stabilizing it in aggregate structures. Across all land use types from forestland, grassland, to cropland, precipitation thus is a strong determining factor to how much SOC is readily available in the topsoil's uppermost A horizon.

At deeper depths of 40-200 cm, soil properties and climate conditions--both longer-term agents--are larger drivers to SOC content. Below 200cm, in addition to the above mentioned variables, long-term land use also exerted a significant influence on SOC content \citep{jia2017soil}. While measuring the differential contributions of various factors to changing statistics across soil depths is difficult, as represented by the many research gaps, it is nonetheless important to gain a deeper understanding of how land use changes soil landscapes. Still, from papers discussed, there seems to be a slight consensus that land use change might more directly affect topsoil areas, but nevertheless strongly impact deeper soil levels over the long-term. In short, the deeper the soil layer, the more long-term and intense its influencing agents must be i.e. climate and land-use. Of course, these two factors are major interrelated issues in our modern world, which makes them all the more important to study.

Erosion not only contributes to sediment loss, but moreover results in both a physical and chemical degradation of soil. Working through a water-wind complex within the Loess Plateau, erosion presents significant risks to the environment. In northeast China where the Loess Plateau is located, \citet{zhang2021characteristics} recorded a 1 cm topsoil loss in just 5-13 years. Even though this study was conducted on a relatively small, shallow slope, the potential for erosion is nevertheless quite significant. This body of work highlights the nuance and importance of conducting studies in specific locations with high erosion vulnerabilities. After all, loss on this scale is felt not only environmentally, but also economically and socially as people depend on the land. Since different types of erosion processes generate different amounts of sediment loss, and each landscape to slope position have different soil characteristics, mitigation practices must be extremely localized with thorough understanding of the local landscape behind each decision and action.

\section{Efforts of Regeneration}

As analyzed above, the topsoil layer of soil is most directly impacted by daily variations in precipitation and temperature; land use can either help or hinder soil development based upon these parameters. This section will thus examine how land use is interrelated with those measurements.

\subsection{Regeneration Efforts: Afforestation}

Understanding shifting land uses on carbon cycling is particularly important to understand nutrient stock dynamics. As a result of growing socioeconomic pressures, natural vegetation cover has been steadily decreasing both globally and in China. More and more, forest and grasslands are extensively and intensively being converted to agricultural lands or other urban needs, while reforestation projects are also growing in popularity. \citet{poeplau2011temporal} reviewed shifting carbon stocks in soil following five different land use change types across nearly 350 sites in the temperate zone, in which China and much of Eastern Asia resides. Unsurprisingly, complete deforestation resulted in a -32${\pm}$20\% decrease in equilibrium SOC over 23 years, and grassland to cropland similarly resulted in a -36${\pm}$5\% equilibrium decrease after 17 years. These numbers reflect the vulnerability of topsoil to intensive change.

\begin{figure}
\includegraphics[width=\linewidth]{images/land-use/LUC-SOC-Table.png}
\caption{Comparison of land use change types on years to reach equilibrium, predicted change rates of soil organic carbon stock after 20 and 100 years, and average initial stock value \citep{poeplau2011temporal}. Note the rapid rates of deforestation (values 17 and 23) compared to longer predicted rates of reforestation land use changes.}
\label{fig:LUC-SOC-Table}
\end{figure}

In comparison with other land use change transitions, restoration of grasslands has one of the greatest potentials for carbon stock restoration. As displayed in (Figure \ref{fig:LUC-SOC-Table}), shifting croplands to grasslands can result in a 128$\pm$23\% increase in SOC--grasslands in particular are long-lasting nutrient sinks \citep{poeplau2011temporal}. This is due in part to how grasses form extensive fine root systems. Decomposing roots are considered to be major contributors to organic biomass in soil, further bolstering nutrient loads \citep{wei2009distribution}. While land degradation and loss of carbon can be a relatively rapid process compared to longer rates of restoration, native grassland restoration nonetheless presents a big potential for successful carbon sequestration, especially in comparison to other land use transitions.

\subsection{Beginnings of Modern-Day Restoration in the Loess Plateau}

Beginning in the 1960s, mass efforts have been made to restore forests and grasslands to the Loess Plateau. The primary goal of these endeavors were, and continue to be, to increase biomass and biodiversity in all ways within the environment. In 1999, the Chinese government implemented the Green for Grain Program (GFGP) to reforest its slopes on a large scale. Since then, restoration has been an active goal for the country. The Loess Plateau rehabilitation effort is representative of the true potential to restore severely degraded land from degraded agricultural land usage.

\begin{figure}
\includegraphics [width=\linewidth] {images/land-use/Loess-Comparison.png}
\caption {Comparison of Loess Plateau from 1995 to 2009 \citep{liu_documentary}.}
\label{fig:Loess-Comparison}
\end{figure}

\subsection{Shifting Practices: What do Effective Restoration Strategies Look Like?}

Much of the recent soil degradation within the Loess Plateau has been a result of intensive agricultural usages. Still, populations within and around the Plateau absolutely depend on it for agricultural land. Therefore, shifting land practices even if for agricultural purposes is important. More and more studies are providing evidence for both economic and environmental potentials in shifting management practices to return to a more regenerative strategy.

This is particularly important in a region like Northeast China where the land provides 30\% of total national maize production for a crop already being one of the biggest food crops in the country. Differences in production potentials can be measured by yield: potential yield is the theoretical ``ceiling'' yield for a given place, partially dependent upon its specific environmental conditions, under perfect management. In general, the yield gap between the potential and actual is usually constrained by 1) non-controllable factors (i.e. environmental conditions, access to technology) 2) agronomic and 3) socioeconomic factors \citep{liu2016narrowing}. Compared to other areas of improvement, such as chosen crop variety, the largest potential for increased yield is found to be associated with management practices. These practices move away from intensive systems towards regenerative or conservationist strategies, including the incorporation of crop residue and low-to-no-till practices that combat shallow topsoil.

One of the biggest biotic constraints on actual yields are diseases and pests, including root-parasitic nematodes. \citet{suong2019impact} contrasted conventional plough-based tillage rice farming practices to a direct-seeding mulch-based rice cropping system (described closely to practices of regenerative agriculture) based on population densities of root-parasitic nematodes. They found that while nematode population densities were significantly higher in the regenerative system than the conventional system, the rice yields were still higher. These results thus show how regenerative systems improve soil fertility and quality both for more productive plants and microbial communities; the higher-nutrient-dense soil not only provided a better environment for microbial biodiversity and nematofauna, but it was moreover productive enough to compensate for any plant damage from the nematodes. These results add to the immense body of research and knowledge that the level of microbidoviersity is a key indicator for soil health. Any land use system that supports this life, then, is most beneficial and sustainable.

Sustainable, regenerative, conservation, or permaculture-based land systems are all names for a common approach in using the land. This approach has the potential to not only ensure food security, but most importantly, has the ability to sustain soil health, promote carbon sequestration (along with other nutrients), decrease GHG emissions, and protect functional ecosystems that then can provide humanity various services. Regenerative agriculture presents an alternative approach characterized by no-to-minimal soil disturbance, diverse crop rotation, and residue retention to enhance nutrient cycling (Figure~\ref{fig:Climate-Smart-Soils}). In short, this integrated, holistic management of land is meant to mimic the natural world--rather than interfere with intensive systems, it is meant to work with existing natural systems.

\begin{figure}
\includegraphics[width=\linewidth]{images/land-use/Climate-Smart-Soils.png}
\caption{Integrated strategy of scientific research, management practices, and widespread implementation of GHG-mitigation-driven agricultural systems \citep{paustian2016climate}.}
\label{fig:Climate-Smart-Soils}
\end{figure}

\subsection{Restoration in the Loess Plateau}

In restoring the Loess Plateau, the Chinese government and other organizations have implemented similar principles. Not only does it provide an alternative avenue of producing food, but it is grounded in thousands-of-years long knowledge of how to best relate to the land, no matter the land use type. Practices within regenerative ways of management are foundational to restoration efforts on the Loess Plateau. Between the 1970s and 1990s, construction of check dams, terraces, reservoirs, and other erosional-mitigating structures have been successful for decreased runoff \citep{tuo2018relative}; \citep{jia2017soil}; \citep{wang2016reduced}. In the past 60 years, the Yellow River Basin, where the majority of Loess Plateau erosion leads to, has seen a consistently decreasing trend in erosion. Between 1980 and 2010, water yield and river discharge were recorded to have decreased by 26\%; sediment load of the basin similarly decreased by 21\% (Figure~\ref{fig:YRB_sediment_load}) \citep{wang2016reduced}. Between 2002 and 2008, yet another study recorded a decrease in runoff of 10.3 mm/yr across the entire Loess Plateau region \citep{luy2013apolicy}. What has helped produce such a significant decrease?

\begin{figure}
\includegraphics[width=\linewidth]{images/land-use/YRB_sediment_load.png}
\caption{Annual sediment load in the Yellow River Basin recorded at 12 catchment zones between 1950 and 2010 \citep{wang2016reduced}.}
\label{fig:YRB_sediment_load}
\end{figure}

\begin{figure}
\includegraphics[width=\linewidth]{images/land-use/restoration_retention.png}
\caption{Erosional retention capacity of three restoration measures: vegetation, terracing, and dams and reservoirs recorded. Time periods were as follows: P1=1951-1979; P2=1980-1999; P3=2000-2010 \citep{wang2016reduced}.}
\label{fig:restoration_retention}
\end{figure}

As shown in (Figure~\ref{fig:restoration_retention}), while dams, reservoirs, and terraces reduced erosion from 1951-1999, after the large-scale GFGP was implemented in 1999, vegetation has instead taken the lead role in increasing soil retention capacity; in fact, dams and reservoirs seem to have now taken a negative toll on erosion rates. So, while built structures certainly have helped reduce erosion and runoff, vegetation nevertheless holds the largest long-term potential in mitigating wind-water erosional forces. Therefore, in the following section we will look closer at the interaction between afforestation efforts and soil properties.

\subsubsection{Afforestation and the Importance of Grasslands within the Loess Plateau}

The Loess Plateau and its unique (yet universal) history of land degradation highlights the importance of maintaining native habitats, particularly grassland ecosystems. Historically, the Loess Plateau was dominated by a grassland ecosystem with the most common native grasses including bunge needlegrass (\emph{Stipa bungeana}) and Dahurian bush clover (\emph{Lespedeza daurica}) \citep{wei2009distribution}. Beginning predominantly in the 1970s-1990s when the push to ``green slopes'' reached its highest point in political agendas, large populations of non-grassland vegetation were introduced to reduce erosion. Some of the most popular included the Chinese Pine (\emph{P. armandii}), Chinese Red Pine (\emph{P. tabuliformis}), Korshinsk Peashrub (\emph{C. korshinskii}), Black Locust tree (\emph{R. pseudoacacia}), and the Sea-Buckthorn shrub (\emph{H. rhamnoides}) \citep{wei2009distribution}; \citep{jian2015effects}; \citep{jia2017soil}. As a result, vegetation cover on the Loess Plateau has increased from 31.6\% in 1999 to 59.6\% in 2013 \citep{chen2015balancing}. While these plants have helped increase vegetation cover in the Loess Plateau, providing an avenue of soil erosion control, how has it affected other ecosystem services?

At first glance, vegetation inputs no matter the species are beneficial to a degraded soil environment. They contribute to surface erosion control, cycling and sequestration of nutrient stocks, as well as providing a considerable increase in biomass within the environment. Compared to bare ground, vegetation not only reduces soil loss, but further works to improve porosity, water movement across horizons, and enhanced filtration \citep{jia2017soil}. Additionally, plant canopy is a strong regulator on the force of falling raindrops, which greatly diminishes in both size and velocity upon interception with vegetation \citep{tuo2018effects}. All together, revegetation has contributed to a decrease in average erosion rate from 3362 t/km$^{2}$ in 2002 to 2405 t/km$^{2}$ in 2008, as well as an additional 96.1 Tg of C sequestered during that time \citep{feng2017ecosystem}. Afforestation has similarly strong implications to the hydrological cycle, but those results are not always so beneficial.

Recent studies have shown that soil indices such as erosion control and C sequestration have improved in afforested areas, soil moisture content (SMC) has not. \citet{feng2017ecosystem} and \citet{jia2017soil} both note that this phenomena (soils getting progressively dryer through all levels) is a result of two main factors: 1) lower average precipitation with higher evaporation rates in a warmer regional climate and 2) afforestation efforts that largely use non-native vegetation that require more water than that of native vegetation. So, not just \emph{any} plant is beneficial within this specific Loess Plateau environment. While global warming is a more difficult issue to solve on a regional scale, the choice in vegetation is certainly a realistic and crucial measure to consider.

Plants exist in environments through finite balances of resource dynamics. While increasing vegetation cover is attributed to many positive effects mentioned above, it also essentially requires water consumption from the soil. The hydrological cycle within a plant environment thus depends on whether or not plant transpiration, consumption, and transpiration exceed the rate of precipitation and water content \citep{jia2017soil}. Normally, this is balanced by plants that create soil conditions adequate for water retention that overrides the amount of water it consumes. However, since any change gives rise to a response, if roots uptake more water than readily available, negative consequences are inflicted upon the environment. Namely, excessive depletion of water resources, most often by afforestation plots, are correlated with soil desiccation and further ecological degradation. Unfortunately, it also has the potential to change local hydrological dynamics involving eco-transpiration, infiltration, runoff, and groundwater recharge \citep{jia2017soil}. While shallow soil layers are directly impacted by rainfall events, deeper layers are even more influenced by sustained consumption of vegetation \citep{feng2017ecosystem}. Vegetation, for better or for worse, has significant implications into the soil moisture content and greater hydrological cycle. This hydrological balance is ever more important in semi-arid to arid regions such as the Loess Plateau where water is already scarce.

Like any soil development index, SMC is a function of climate (temperature, precipitation, wind), woil properties (texture, OM, porosity, aggregation, bulk density), topography (slope, steepness, face position), and land cover features \citep{jia2017soil}. Since the 1980s, total crop area in Northeast China has greatly increased, including crops such as wheat, potato, and oats \citep{liu2015agriculture}. As can be expected, these intensive crops have been correlated with decreasing trends of SMC in comparison to ``pristine pastures'' that exhibit an opposite positive trend in SMC (Figure~\ref{fig:crops}) \citep{liu2015agriculture}. The clear indication is that vegetation, maybe even over other indices, results in particularly strong implications for SMC. Afforestation vegetation choices are no different.

\begin{figure}
\includegraphics[width=\linewidth]{images/land-use/crops.png}
\caption{Total crop area in northern China from 1980 to 2010 (left); average variation of volumetric soil moisture in topsoil of potato, wheat, and oat crop fields compared to pristine pastures in Northeast China (Wuchuan Agricultural Meteorology Observation Station) from 1983 to 2009 (right) \citep{liu2015agriculture}.}
\label{fig:crops}
\end{figure}

Non-native plants that are deep-rooted and require substantially more water than the environment can provide, especially in a warming dryer climate, dessicate the soil. Under most non-native afforestation types, soil moisture storage therefore consistently decreases within 100cm of the topsoil \citep{jian2015effects}. High-density planting of non-native species ill-fit to arid conditions dessicate soil layers by sucking up all moisture content, which has obvious implications for additional ecosystem services such as C sequestration, limiting vegetation growth, and degenerating the habitat \citep{jia2017soil}. While a broader variety of plant species may be able to improve C sequestration and enhance erosion control, their reforestation efforts are unsustainable if they do not also support water retention. In this way, soil moisture is a strong regulating force behind further ecosystem services, especially in water-limited ecosystems like the Loess Plateau.

While it was aforementioned that SOC and N stocks have been improving within the Loess Plateau, the degree of improvement varies between land use types. In drier regions where they are most adapted, grassland restoration exhibits a greater potential for C and N accumulation. Through an analysis comparing the soil organic carbon, nitrogen, and phosphorus stocks of native grassland and implanted woody ecosystems dominated by the pine and peashrub, \citet{wei2009distribution} confirmed this understanding in that the soil of native grass exhibited the highest concentration of SOC and N stocks. 

The relative superiority of grassland ecosystems within the Loess Plateau over shrub and forestlands, however, is mostly due to the local climate conditions with low mean annual precipitation (MAP). In areas where the MAP is <510mm, grassland ecosystems contribute to higher SOC and N stocks, whereas in areas with a MAP >510mm grassland ecosystems experience losses in nutrient concentrations \citep{tuo2018effects}. Hand et al., 2018 observed a similar trend, just at a different threshold value of 470mm MAP. While these two MAP values are slightly different, they nonetheless show how efficient vegetation cover ultimately depends on local climate conditions and whether or not certain types are more adapted to those conditions than others.

While much of the rhetoric surrounding the ``fight'' against desertification centers around reforestation, regeneration of grasslands, especially if it is the historically native ecosystem, is often overlooked (at least in predominant media). Still, especially in the Loess Plateau, native grasslands are the ecosystems that generally hold a greater potential for carbon and nitrogen stocks, as well as supporting greater soil moisture content, ultimately contributing to more productive soil. Ultimately, grassland systems play an essential role in nutrient cycling and stocks within the Loess Plateau soil environment, and are thus an important measure to consider in relation to land use change.

In general, selection of vegetation types and their placement on slopes are critical factors in ensuring restoration processes are successful in working with the land \citep{feng2017ecosystem}. Once again, a highly localized understanding of the land and its environmental history is important in any land use change scenario, not excluding restoration practices. This information can be available through processes of remote sensing!

\section{How Can Remote Sensing Be Used to Inform Loess Plateau Restoration?}

Just as the ecosystems it attempts to restore, regeneration efforts like that being done in the Plateau are different everywhere. Processes must consider the historical, natural habitat and current conditions--an endeavor made possible through remote sensing. Remote sensing can be used to develop a site history or a place, an especially important tactic when dealing with projects involving development or restoration. Remote sensing can be used to assess an ecosystem and its soil surface conditions, local hydrology, nutrient cycles, vegetation, and overall geology \citep{abdullah2016use}. Especially in terms of restoration sites, remote sensing can be used to identify reference sites with similar characteristics to further evaluate the target site, contributing to a more robust directive towards proper objectives, plans, and action. This was done in modern studies of the Loess Plateau in comparing it to its neighboring region of Sichuan which hosts a rich, lush ecosystem of vegetation and biota. Monitoring is, of course, also important to keep track of long-term changes in vegetation cover and efficacy of restoration methods.

Ultimately, remote sensing can provide critical knowledge of the Loess Plateau, combining all the factors that affect the land from topography, climate, biotic factors, historical geology, soil properties, and anthropogenic forces over a broad span of time. Developing a deep understanding of the Plateau grants us the opportunity for increasingly localized restoration directives that all contribute to more profound insights on how to live well in a place, especially a place where that knowledge might have been put to the wayside. Harking back to tenets of Traditional Chinese Medicine, to truly know a problem is to look at it as a whole and understand the changes in ecosystem balance constantly being shifted--what better way to gain a holistic understanding of the Plateau intricate systems than to do so via remote sensing?

\section{Conclusion}

In this chapter, we have learned about land use change, how it is measured through remote sensing, processes that have impacted the specifical erosional landscape of China's Loess Plateau, and how efforts of restoration continue to shape its ecosystem services.

Now having finished this section, the most important question for us all becomes: \emph{with this information in hand, how can we move beyond the theoretical confines of this chapter to actually initiate sustainable land use change on local, regional, and global scales? How can we utilize the powers of (critical) remote sensing to do so?}


\chapter{Deforestation}\label{ch:deforestation}

\chapterauthor{Xinlan Chen}

\footnote{Statement of Contributions-- For example, ``The chapter was first drafted by Marc Los Huertos (2021). The author recieved valuable feedback from X, and Y and Z to improve the chapter. Slater revised the chapter in 2022 with suggestions from Cater.'' Note: I am still working on the formatting for this to improve it.}

\red{I added my comments from before, it looks like you didn't get them. Be sure to pull my changes}.


\section{Overall}\red{not a very descriptive heading\ldots}

Deforestation is not a small issue in East Asia, since most countries here depend on the money source they got from the wood export industry.\red{I like this as a start} There are\red{let's see if we can avoid passive voice} many issues that could be caused by deforestation with but not limited to air pollution, water pollution and climate change. Upon having looked at all the presentations that people make, I realized that most East Asia countries that was originally covered mostly in forest have significantly reduced in their forest coverage,\red{let's see if we can find some data on forest changes over the last 200+ years} and nevertheless, natural is punishing\red{interesting, let's see if we can define this} them for that. Over-logging is when the level of deforestation exceeds the self-sustaining state of the ecosystem, leading to desertification\red{important concept, let's see if we can define this too}. Excessive cutting of trees can bring a series of troubles and problems. Many countries have started having regulations on business regarding wood exports and forests protection, more have to be done in order to resolve the issue that people have done in the past that resulted in deforestation. \red{as a road map this intro is very nicely structured}

\section{Damage Done by Deforestation – Example of China}

After soil erosion and deforestation\red{let's see how we can use the land use chapter as a basis / foundation for this section}, the bare land cannot withstand the wind and rain. On sunny days, due to the sun exposure, the ground temperature rises, the process of organic matter decomposition into soluble mineral elements is accelerated;\red{great, let's see if we can make a figure for this\ldots} On rainy days, rainwater is washed directly, carrying the fertile topsoil along with mineral elements into rivers.\red{great! and let's see if we can connect to the processed described in the land use chapter so we are not repeating ourselves.} It is \red{passive voice}estimated that more than 5 billion tons of soil are washed into rivers in China every year\red{how do we know if this it alot or too much?}. The siltation of quicksand\red{?} blocked the channel of the reservoir. Sediment concentration in the Yellow River in China is the highest in the world, and when floods come, water and sand are equally divided. Due to the deposition of quicksand\red{I don't think we mean quicksand}, the riverbed in some places in the lower reaches of the Yellow River is 12 meters higher than the land beyond the embankment and even higher than the city wall of Kaifeng City, posing a serious threat to the safety of people's lives and property.\red{to make a paragraph break, just add a line space\ldots}

The deteriorate of environment always accompanies with frequent disasters. The forest coverage rate in Wanning County, Hainan Province, used to be as high as 63 percent.\red{can we map the changes?} Due to forest regulation, there was no drought in the 1940s and 1950s. Since then, natural disasters have been on the rise. From the 1960s to the 1970s, droughts hit an average of six out of every ten years, cutting the flow of 21 rivers, reducing the output of three-quarters of farmland and drying up 25 reservoirs.\red{super interesting!}  In particular, the destruction of the forest, so that some rare animals lost their breeding base\red{let's cite some examples}. Animals there can't survive. China's Hainan slope deer, South China tiger, black - crested gibbon and other precious animals are endangered due to habitat destruction\red{yes, here they are!  nice, I wonder if we can get these into the previous sentance?}. There is less oxygen supply when deforestation issue is present\red{the atmosphere is 21\%, is there any evidence that O2 is impacted?}, consequently there is more carbon dioxide \red{how is this created?} and less natural filters \red{agreed, but we might need to explain this to our readers}. The ecological environment was forcibly occupied, and the law of ecological circulation was destroyed. The temperature can change, causing unstable changes in the climate.

\section{Overall Losses – Example of Indonesia and Thailand}

In the past decade, Southeast Asia has suffered the greatest loss of forest area, with a net loss of more than 900,000 hectares a year.\red{wow! is there a map of this?} Trees are widely used as materials in construction, decoration, paper making, metallurgy\red{weird, how?}, chemical industry. Used in paper making, as furniture, chopsticks and other main daily necessities, wood floor, plywood, ceiling and other industrial supplies. Cutting down trees so that people can use the land in their own benefits. To sum up, people cut down trees mainly for their own needs.\red{not sure how to interpret this statement\ldots, humans generally do everything to meet their own needs, right? or are you implying something else?}

	In some parts of Thailand, inspectors inspect thousands of smuggled trucks at roadblocks every day. Instead of guns and drugs, they are carrying wood. Thailand, once a major teak producer in the world, is now on the verge of disappearing its forest resources, and most of the forest areas have become a mess\red{let's find a better word} of tree stumps. According to government statistics, Thailand's forest cover has almost halved in two decades\red{or -- One half of Thailand's tree cover had been lost}. In 1961, it still accounted for 53\% of China's forest areas, and in 1981, it only accounted for 28\%.\red{why are we talking about China?  I thought this was Thailand} Some experts believe that the actual forest coverage is much less at present\red{not sure what you mean}. According to the latest survey report of the United Nations, in South Asia, Southeast Asia and the Pacific Islands, forests are disappearing at the rate of 12500 hectares per day\red{but Thailand is 900,000/day, seems like something is off here}, that is, 4.5 million hectares per year. The report also points out that Indonesia, the largest producer of tropical hardwood in the world, loses more than 1.2 million hectares of forest land every year.\red{huh?  all these numbers seem in conflict} Thailand is more serious, with an annual deforestation of 800000 hectares, while the land area of this country is only one fourth that of Indonesia.\red{I suggest we make a table} The large-scale disappearance of forests has caused climate change and the death of wild animals and caused ecological disasters. \red{looks like a new topic, should be in a new subsection or paragraph} In Thailand, 42 people died of floods in 1979. If effective measures to protect wildlife areas are not formulated, wild elephants may disappear in 30-40 years.\red{how do we know that?} Another reason for the destruction of forests is that mountain tribes in Thailand still carry out primitive farming.\red{new topic} They cut down trees to burn wasteland and clear land for farming. According to the United Nations survey, 70\% of the forests in the north have been destroyed.\red{and how are they using the forests?} Online Thailand has changed from a major log exporter to an importer, with an annual cost of US \$4400 in foreign exchange.
	Since 1968, Thailand has divided nearly half of the country's forests into more than 500 leased plots for logging. \red{new topic} The annual deforestation rate of Southeast Asian countries is relatively high, and the annual reduction of forest area is increasing year by year. Indonesia has increased from 600000 hectares before the 1990s to 1084000 hectares between 1990 and 1995.\red{yes we need a table to track these. I suggest you create one in word/google docs and then I'll help you create one here} At the same time, Thailand increased from 244000 ha to 329000 ha, Malaysia from 255000 ha to 400000 ha, Philippines from 91000 ha to 262000 ha, Myanmar from 102000 ha to 387000 ha. Compared with the period before the 1990s, the annual average amount of logging in these countries increased by one cup in the first half of the 1990s, and even tripled in some countries. Since Vietnam's reform and opening up, due to a large number of logging and fire, the forest area has also been greatly reduced. From 1990 to 1995, a total of 1.5 million hectares of forest were lost, that is, 26000 hectares per year. In recent years, four of the six countries with the most deforestation in the world are Southeast Asian countries. Their average annual felling rates are 513300 acres in Thailand, 400500 acres in Myanmar, 396000 acres in Malaysia and 316100 acres in the Philippines. From 1969 to 193, about 6000 square kilometers of forests, swamps and grasslands were opened up for agricultural use by immigrants in Indonesia. Although forest is a renewable resource, its growth cycle is decades or even hundreds of years. For a long time, a large number of forests have been cut down in Southeast Asia, but the work of forest restoration and afforestation has not been given due attention. The speed of forest restoration is far behind the speed of deforestation, resulting in the decrease of forest area year by year.\red{this is a great start!, Let's link the deforestation to the market demand and then connect to documented ecological impacts with more specificity and documentation. I think the UN sources are useful, but we need more primary sources.}
	
\begin{figure}
\includegraphics{images/deforestation/Example_image}
\end{figure}

\section{Correlation Between Deforestation and Pandemic}

Deforestation has led to more outbreaks of human infectious diseases, and in 1997, in Indonesia, a haze shrouded the rainforests, an area the size of Pennsylvania \red{why this state? seems odd} was burned to promote agriculture, and droughts exacerbated the fires. Under the smog, the trees are unable to bear fruit, leaving local fruit bats with no choice but to fly elsewhere in search of food, and carry with them a deadly disease\red{super interesting, let's get this cited and put in more details}. Soon after the bats took up residence in the trees in the Malaysian orchards, nearby pigs began to get sick -- \red{probably?? let's put this in more scientific terms, was this tested?} from eating fallen fruit eaten by the bats, and local pig farmers also began to get sick. By 1999, 265 people had severe encephalitis\red{this is an important disease, let's define it and the vector} and 105 had died. This was the first known case of Nipah virus in humans, and since then there have been a series of repeated outbreaks across Southeast Asia\red{interesting, can we learn more about the biology?}. Such infectious diseases are usually confined to wild animals but are spreading to areas\red{open areas? or areas where people live?, I am a bit confused} where forests are being cleared rapidly. Human infections \red{symptoms?} vary from asymptomatic to fatal encephalitis. Infected people initially develop flu-like symptoms: fever, headache, muscle pain, vomiting and sore throat. This may be followed by dizziness, lethargy, confusion, and neurological signs that indicate acute encephalitis. Some people may also develop atypical pneumonia and serious respiratory problems, including acute respiratory distress.\red{interesting! I wonder if we can find a diagram?}In severe cases, encephalitis and epilepsy develop, leading to coma within 24 to 48 hours. The time between infection and onset of symptoms ranges from four to 45 days. Most acute encephalitis survivors who recovered from it are left with residual neurological consequences, such as persistent seizures and personality changes. A small number of people recover and relapse or develop delayed encephalitis. Over the long term, persistent neurological dysfunction has been observed in more than 15\% of the population. The case mortality rate\red{define what this means} is estimated at 40\% to 75\%. Over the past two decades, scientific evidence has mounted that deforestation sets off a complex chain reaction that sets the stage for a host of deadly pathogens to spread to humans, like Nipah and Lassa\red{what's this?} viruses, as well as the parasites that cause malaria and Lyme disease.\red{maybe a table of diseases and their locations and symptoms?} Today, large areas of the Amazon rainforest, parts of Africa and Southeast Asia are still burning, and experts are concerned about the health of people living on the edge of deforestation. They also worry that the next serious pandemic could emerge from the planet's forests.\red{or perhaps did!?} Researchers have identified the correlation between malaria and deforestation, which kills more than a million people a year and is mainly transmitted by the malaria parasite carried by mosquitoes.\red{good, detail the research results here}

\section{Data Proving the Damage of Deforestation as a Broader Picture and Policy Implemented}\red{let's see if we can find a better heading}
In Southeast Asia, the area of forest is gradually decreasing, and the forest coverage rate is decreasing due to the migration and reclamation, deforestation, expansion of farmland, backward farming methods and large-scale commercial forest development. Before the 1970s, the forest coverage rate of Indonesia, Cambodia, Laos and Brunei was as high as 70\%, that of Myanmar and Malaysia was 66\%, and that of Vietnam, Philippines and Thailand was less than 50\%.\red{seems like we covered this already?} By 1995, the forest coverage of these countries had decreased significantly, to 55.7\% in Cambodia, 41.3\% in Myanmar and 47.7\% in Malaysia. 60.6\% in Indonesia, 22.8\% in Thailand and 22.7\% in the Philippines. Forest is the habitat of animals. With the decrease of forest area, the habitat of animals and plants also decreases.
	The rate of net loss is significantly lower than the annual loss of 2.4 million hectares during the period 1990-2000. Despite a net increase in forest area reported at the regional level, the rate of deforestation remains high in many countries. In the past decade, Southeast Asia has suffered the greatest loss of forest area, with a net loss of more than 900,000 hectares\red{now I am even more confused\ldots} a year. However, the rate of net loss is significantly lower than the annual loss of 2.4 million hectares during the period 1990-2000.\red{let's make a table and / or plot of the data, better than text!} Overall, the Asia and Pacific region lost 700,000 hectares of forest per year in the 1990s but gained 1.4 million hectares per year between 2000 and 2010.\red{let's add some references here} This is largely the result of China's massive reforestation\red{back in China? I am pleased by the geographic range, but let's organize it so the reader can follow the trek back and forth with ease} efforts, which saw the country's forest area increase by 2 million hectares a year in the 1990s and an average increase of 3 million hectares a year since 2000. \red{new section}As article 8 of the Forest Law of the People's Republic of China stipulates that the state shall implement the following protective measures for forest resources:\red{create numbered bullets} (1) implement quota cutting for forests, encourage afforestation, close mountains for forest cultivation, and expand the area covered by forests; (2) to give economic support or long-term loans to collectives and individuals for afforestation and forest cultivation in accordance with the relevant provisions of the state and local people's governments; (3) To advocate the comprehensive utilization and economical use of wood and encourage the development and utilization of wood substitutes; (4) Collect forest raising fees and use them exclusively for afforestation and forest raising; (5) In the departments of coal and paper making, a certain amount of funds shall be set aside according to the output of coal, wood pulp and paper to be used exclusively for the construction of pit timber and timber for paper making; (6) Establishing a forestry fund system. The State shall establish a forest ecological benefit compensation fund to be used for the construction, upbringing, protection and management of protection forests that provide ecological benefits and forest resources and trees with special uses. Forest ecological benefit compensation funds must be earmarked for special purposes and may not be misused for other purposes. Specific measures shall be formulated by the State Council.\red{paragraph or section?} Bhutan, India, the Philippines and Vietnam also reported an increase in forest cover over the past 10 years. Despite a net increase in forest area reported at the regional level, the rate of deforestation remains high in many countries. In the past decade, Southeast Asia has suffered the greatest loss of forest area, with a net loss of more than 900,000 hectares a year. China, India and Vietnam have all set targets for large-scale reforestation, as well as incentive programs to encourage small farmers to plant more trees. China plans to increase its plantation area by 50 million hectares to 23 percent by 2020. If the current rate of afforestation continues, it may be possible to achieve this goal by 2015. India has set a goal of 33 percent forest and tree cover by 2012. Forests, other woodlands or land covered with trees accounted for about 25 percent of India's land area in 2010, according to the Global Forest Resources Assessment. The area of unknown rows of trees planted and other "trees outside the forest" should be included in this percentage. The government's goal was to restore forest cover to 43 per cent by 2010, and according to information provided to the Global Forest Resources Assessment, this goal has been met.

\section{Suggestions}\red{are these policy suggestions? if so, we need to be more specific and precise --country agency and funding sources while addressing political capacity and will}
(1) Strengthening the awareness of forest resources protection

\red{read and cite Wenhua, Li. "Degradation and restoration of forest ecosystems in China." Forest Ecology and Management 201.1 (2004): 33-41. and}

\red{Cheng, Shengkui, et al. "Spatial and temporal flows of China's forest resources: Development of a framework for evaluating resource efficiency." Ecological Economics 69.7 (2010): 1405-1415. and} 

\red{Sloan, Sean, and Jeffrey A. Sayer. "Forest Resources Assessment of 2015 shows positive global trends but forest loss and degradation persist in poor tropical countries." Forest Ecology and Management 352 (2015): 134-145.}

The human living environment is increasingly bad, the ecological environment problem has become the focus of globalization. The destruction of forest resources leads to soil erosion, land desertification, poor air quality and even harm to human health. We should realize the importance of forest resources, strengthen the awareness of forest resources protection, the protection of forest resources is everyone's responsibility, starting from ourselves, fundamentally accept and support the forest resources protection policy, dare to report illegal logging behavior, participate in the protection of wild animals and plants. On the basis of not destroying, afforestation can promote the healthy and sustainable development of forestry, mobilize the enthusiasm of the broad masses of people, effectively protect forest resources to the greatest extent, improve the ecological environment, and enable people to live in harmony with nature.
(2) Strengthening the construction of forest resources management institutions
The management and protection of forest resources should be based on the principle of sustainable development and operation, and rational harvesting should be carried out according to the quality and structure of forest resources. Forest resources management agencies should be established and the management system should be improved. According to the actual situation of forest resources, forest management and protection stations should be set up reasonably, and the management and protection personnel should be trained to improve their management and protection ability and legal awareness of forest resources. We should reasonably formulate quotas for forest cutting, conduct management and supervision over the issuance of cutting licenses in accordance with the Forest Law, ensure that there are laws to follow for cutting with vouchers, supervise the progress and management of cutting, and resolutely prohibit arbitrary cutting. Forestry departments should also adapt to the requirements of the current social development situation, under the trend of rapid development of information, increase investment to improve hardware and software infrastructure, establish a perfect information system, improve configuration facilities, improve work efficiency. The perfect forest resources information management system, standardized management data collection and analysis, improve the level and quality of monitoring, scientific management and protection of forest resources.
(3) Strengthening control of diseases and insect pests of trees and protection against forest fires
Natural disasters of forest resources mainly have two aspects, one is pests and diseases, the other is forest fires. These two kinds of harm to the forest destruction and loss are huge. As for the prevention and control of diseases and insect pests, the quarantine of quarantine forest plants and the prevention and control of forest diseases and insect pests should be strengthened. The Quarantine Law should be strictly implemented to prevent the introduction of harmful organisms and put an end to the man-made spread of pests. Monitoring and forecasting of forest diseases and insect pests, taking the initiative to prevent disasters and effectively control the scope and harm of diseases and insect pests; On the other hand, the prevention and control of forest fires has always been the top priority of forest resource management. For a long time, the need to constantly, vigorously adhere to the propaganda of fire prevention work, multimedia, multi-faceted personnel into the forest to strengthen the awareness of fire prevention, has been unable to relax vigilance. Implement the fire prevention responsibility of relevant units, control fire source management, carry out regular joint prevention of responsible households, establish and perfect fire monitoring and forecast system, strengthen the allocation of fire prevention equipment, fundamentally control fire sources, timely control when fire occurs, and reduce the loss of forest resources.
(4) Improve the utilization rate of forest resources
The utilization of forest resources is mainly the development of forestry, and the industrialization of forestry development is its development trend. In the management and management of forestry, we should first cultivate forest resources fundamentally, increase the intensity of forest cultivation, improve the quality of forest resources and improve the ecological environment. On this basis, forestry industrialization, technological production technology improvement, research and development of new wood processing industry, adjust the direction and structure of wood products. At the same time, the comprehensive development of forest resources, the development of breeding, planting, and related processing industries, in many ways to improve the maximum utilization rate of forest resources.

\red{great start, Xinlan! Now that we have a skelton, let's see if we can add images (use the images/deforestration folder) and let's add citations. I think the paper could use some organization and then we can be more specific in how deforestation is caused and it's concequences, which are mentioned, but could have more detail. I am looking forward to the next version}




\section{Overall}
Deforestation is not a small issue in East Asia, since most countries here depend on the money source they got from the wood export industry. There are many issues that could be caused by deforestation with but not limited to air pollution, water pollution and climate change. Upon having looked at all the presentations that people make, I realized that most East Asia countries that was originally covered mostly in forest have significantly reduced in their forest coverage, and nevertheless, natural is punishing them for that. Over-logging is when the level of deforestation exceeds the self-sustaining state of the ecosystem, leading to desertification. Excessive cutting of trees can bring a series of troubles and problems. Many countries have started having regulations on business regarding wood exports and forests protection, more have to be done in order to resolve the issue that people have done in the past that resulted in deforestation. 

\section{Damage Done by Deforestation – Example of China}
After soil erosion and deforestation, the bare land cannot withstand the wind and rain. On sunny days, due to the sun exposure, the ground temperature rises, the process of organic matter decomposition into soluble mineral elements is accelerated; On rainy days, rainwater is washed directly, carrying the fertile topsoil along with mineral elements into rivers. It is estimated that more than 5 billion tons of soil are washed into rivers in China every year. The siltation of quicksand blocked the channel of the reservoir. Sediment concentration in the Yellow River in China is the highest in the world, and when floods come, water and sand are equally divided. Due to the deposition of quicksand, the riverbed in some places in the lower reaches of the Yellow River is 12 meters higher than the land beyond the embankment and even higher than the city wall of Kaifeng City, posing a serious threat to the safety of people's lives and property.
The deteriorate of environment always accompanies with frequent disasters. The forest coverage rate in Wanning County, Hainan Province, used to be as high as 63 percent. Due to forest regulation, there was no drought in the 1940s and 1950s. Since then, natural disasters have been on the rise. From the 1960s to the 1970s, droughts hit an average of six out of every ten years, cutting the flow of 21 rivers, reducing the output of three-quarters of farmland and drying up 25 reservoirs. In particular, the destruction of the forest, so that some rare animals lost their breeding base. Animals there can't survive. China's Hainan slope deer, South China tiger, black - crested gibbon and other precious animals are endangered due to habitat destruction. There is less oxygen supply when deforestation issue is present, consequently there is more carbon dioxide and less natural filters. The ecological environment was forcibly occupied, and the law of ecological circulation was destroyed. The temperature can change, causing unstable changes in the climate.

\section{Overall Losses – Example of Indonesia and Thailand}
In the past decade, Southeast Asia has suffered the greatest loss of forest area, with a net loss of more than 900,000 hectares a year. Trees are widely used as materials in construction, decoration, paper making, metallurgy, chemical industry. Used in paper making, as furniture, chopsticks and other main daily necessities, wood floor, plywood, ceiling and other industrial supplies. Cutting down trees so that people can use the land in their own benefits. To sum up, people cut down trees mainly for their own needs.
	In some parts of Thailand, inspectors inspect thousands of smuggled trucks at roadblocks every day. Instead of guns and drugs, they are carrying wood. Thailand, once a major teak producer in the world, is now on the verge of disappearing its forest resources, and most of the forest areas have become a mess of tree stumps. According to government statistics, Thailand's forest cover has almost halved in two decades. In 1961, it still accounted for 53\% of China's forest areas, and in 1981, it only accounted for 28\%. Some experts believe that the actual forest coverage is much less at present. According to the latest survey report of the United Nations, in South Asia, Southeast Asia and the Pacific Islands, forests are disappearing at the rate of 12500 hectares per day, that is, 4.5 million hectares per year. The report also points out that Indonesia, the largest producer of tropical hardwood in the world, loses more than 1.2 million hectares of forest land every year. Thailand is more serious, with an annual deforestation of 800000 hectares, while the land area of this country is only one fourth that of Indonesia. The large-scale disappearance of forests has caused climate change and the death of wild animals and caused ecological disasters. In Thailand, 42 people died of floods in 1979. If effective measures to protect wildlife areas are not formulated, wild elephants may disappear in 30-40 years. Another reason for the destruction of forests is that mountain tribes in Thailand still carry out primitive farming. They cut down trees to burn wasteland and clear land for farming. According to the United Nations survey, 70\% of the forests in the north have been destroyed. Online Thailand has changed from a major log exporter to an importer, with an annual cost of US \$4400 in foreign exchange.
	Since 1968, Thailand has divided nearly half of the country's forests into more than 500 leased plots for logging. The annual deforestation rate of Southeast Asian countries is relatively high, and the annual reduction of forest area is increasing year by year. Indonesia has increased from 600000 hectares before the 1990s to 1084000 hectares between 1990 and 1995. At the same time, Thailand increased from 244000 ha to 329000 ha, Malaysia from 255000 ha to 400000 ha, Philippines from 91000 ha to 262000 ha, Myanmar from 102000 ha to 387000 ha. Compared with the period before the 1990s, the annual average amount of logging in these countries increased by one cup in the first half of the 1990s, and even tripled in some countries. Since Vietnam's reform and opening up, due to a large number of logging and fire, the forest area has also been greatly reduced. From 1990 to 1995, a total of 1.5 million hectares of forest were lost, that is, 26000 hectares per year. In recent years, four of the six countries with the most deforestation in the world are Southeast Asian countries. Their average annual felling rates are 513300 acres in Thailand, 400500 acres in Myanmar, 396000 acres in Malaysia and 316100 acres in the Philippines. From 1969 to 193, about 6000 square kilometers of forests, swamps and grasslands were opened up for agricultural use by immigrants in Indonesia. Although forest is a renewable resource, its growth cycle is decades or even hundreds of years. For a long time, a large number of forests have been cut down in Southeast Asia, but the work of forest restoration and afforestation has not been given due attention. The speed of forest restoration is far behind the speed of deforestation, resulting in the decrease of forest area year by year.

\section{Correlation Between Deforestation and Pandemic}
Deforestation has led to more outbreaks of human infectious diseases, and in 1997, in Indonesia, a haze shrouded the rainforests, an area the size of Pennsylvania was burned to promote agriculture, and droughts exacerbated the fires. Under the smog, the trees are unable to bear fruit, leaving local fruit bats with no choice but to fly elsewhere in search of food, and carry with them a deadly disease. Soon after the bats took up residence in the trees in the Malaysian orchards, nearby pigs began to get sick -- probably from eating fallen fruit eaten by the bats, and local pig farmers also began to get sick. By 1999, 265 people had severe encephalitis and 105 had died. This was the first known case of Nipah virus in humans, and since then there have been a series of repeated outbreaks across Southeast Asia. Such infectious diseases are usually confined to wild animals but are spreading to areas where forests are being cleared rapidly. Human infections vary from asymptomatic to fatal encephalitis. Infected people initially develop flu-like symptoms: fever, headache, muscle pain, vomiting and sore throat. This may be followed by dizziness, lethargy, confusion, and neurological signs that indicate acute encephalitis. Some people may also develop atypical pneumonia and serious respiratory problems, including acute respiratory distress. In severe cases, encephalitis and epilepsy develop, leading to coma within 24 to 48 hours. The time between infection and onset of symptoms ranges from four to 45 days. Most acute encephalitis survivors who recovered from it are left with residual neurological consequences, such as persistent seizures and personality changes. A small number of people recover and relapse or develop delayed encephalitis. Over the long term, persistent neurological dysfunction has been observed in more than 15\% of the population. The case mortality rate is estimated at 40\% to 75\%. Over the past two decades, scientific evidence has mounted that deforestation sets off a complex chain reaction that sets the stage for a host of deadly pathogens to spread to humans, like Nipah and Lassa viruses, as well as the parasites that cause malaria and Lyme disease. Today, large areas of the Amazon rainforest, parts of Africa and Southeast Asia are still burning, and experts are concerned about the health of people living on the edge of deforestation. They also worry that the next serious pandemic could emerge from the planet's forests. Researchers have identified the correlation between malaria and deforestation, which kills more than a million people a year and is mainly transmitted by the malaria parasite carried by mosquitoes.

\section{Data Proving the Damage of Deforestation as a Broader Picture and Policy Implemented}


Deforestation and long-term destruction of forest resources are the causes of drought and flood disasters. The destruction of forest resources will lead to serious soil and water loss which will lead to the decline of soil fertility, make the land lose the its foundation of sustainable development, and aggravate soil drought. At the same time, soil erosion will also increase the rainstorm runoff, resulting in water and river siltation and aggravate the flood.Large area of continuous and high intensity rainfall is the fundamental reason for the occurrence of flood disasters, and the increasingly serious destruction of forest, soil erosion, river siltation, lake shrinkage, collapse, landslide, debris flow and other non-climatic factors also aggravate the formation and development of flood disasters.Debris flow is a kind of serious natural disaster. It carries the amount of sand and stones with the flood from the height of the dump, and has a devastating damage, endangering human beings and their living environment. It is not difficult to find the relationship between debris flow and forest change. Good forest vegetation can restrain the occurrence of debris flow and weaken the disaster of debris flow within a certain limit. Debris flow is a kind of geological disaster caused by heavy rainfall. When it breaks out, it can devastate villages, homes, living transportation facilities and people's lives. It is often impossible to prevent.

1. Characteristics and causes of debris flow

Debris flow is a sudden collapse of the mountain due to the loss of the protection of the forest and other vegetation on the mountain and the continuous rainfall, the swelling of the forest land, the high degree of weathering of the rock parent material and the comprehensive influence of the slope and slope direction. It mainly occurs in the rainfall concentrated season from July to September, especially in mountainous areas and hilly areas with poor ecological environment and little forest vegetation, which directly causes grave disasters such as buried villages and reservoirs, water hitting and sand pressing farmland, rivers and communication being blocked, as well as a large number of human and animal casualties and missing.

2. Measures for preventing and mitigating debris flow disasters

In the areas with poor vegetation protection, mountains shall be closed off and farmland shall be returned to forests. All management measures and economic compensation for the protection and construction of ecological public welfare forests shall be conscientiously implemented, and illegal cutting and felling of trees on public welfare forest land shall be strictly prohibited. It is strictly prohibited to cut up mountains for rock quarrying, destroy forest land, change the use of woodland and other illegal acts, and gradually restore and enhance the protection function of forest vegetation, which is a long-term plan to prevent and reduce the harm of debris flow. We will implement and do a good job in ecological planning.

Forests can improve soil aggregate structure and increase rainwater infiltration rate. The litter layer under the forest surface has strong water absorption capacity, and when its water absorption is saturated, it can slow the runoff and improve the rainwater infiltration rate, so the forest can reduce the slow mountain flood. The roots of forest plants intersect and form a network in the soil, just as steel does to concrete, consolidating the soil together and enhancing its integrity and stability. If there is good forest vegetation in the development period of the gully accumulation body, the effect of the root system consolidating the soil can reduce the scale of the mountain being washed away by the flood and reduce the risk of collapse. Therefore, good forest vegetation can restrain the occurrence of debris flow, reduce the content of soil and sand, and reduce the viscosity of debris flow, so as to reduce the harm of mountain flood debris flow.

In Southeast Asia, the area of forest is gradually decreasing, and the forest coverage rate is decreasing due to the migration and reclamation, deforestation, expansion of farmland, backward farming methods and large-scale commercial forest development. Before the 1970s, the forest coverage rate of Indonesia, Cambodia, Laos and Brunei was as high as 70\%, that of Myanmar and Malaysia was 66\%, and that of Vietnam, Philippines and Thailand was less than 50\%. By 1995, the forest coverage of these countries had decreased significantly, to 55.7\% in Cambodia, 41.3\% in Myanmar and 47.7\% in Malaysia. 60.6\% in Indonesia, 22.8\% in Thailand and 22.7\% in the Philippines. Forest is the habitat of animals. With the decrease of forest area, the habitat of animals and plants also decreases.
	The rate of net loss is significantly lower than the annual loss of 2.4 million hectares during the period 1990-2000. Despite a net increase in forest area reported at the regional level, the rate of deforestation remains high in many countries. In the past decade, Southeast Asia has suffered the greatest loss of forest area, with a net loss of more than 900,000 hectares a year. However, the rate of net loss is significantly lower than the annual loss of 2.4 million hectares during the period 1990-2000. Overall, the Asia and Pacific region lost 700,000 hectares of forest per year in the 1990s but gained 1.4 million hectares per year between 2000 and 2010. This is largely the result of China's massive reforestation efforts, which saw the country's forest area increase by 2 million hectares a year in the 1990s and an average increase of 3 million hectares a year since 2000. As article 8 of the Forest Law of the People's Republic of China stipulates that the state shall implement the following protective measures for forest resources: (1) implement quota cutting for forests, encourage afforestation, close mountains for forest cultivation, and expand the area covered by forests; (2) to give economic support or long-term loans to collectives and individuals for afforestation and forest cultivation in accordance with the relevant provisions of the state and local people's governments; (3) To advocate the comprehensive utilization and economical use of wood and encourage the development and utilization of wood substitutes; (4) Collect forest raising fees and use them exclusively for afforestation and forest raising; (5) In the departments of coal and paper making, a certain amount of funds shall be set aside according to the output of coal, wood pulp and paper to be used exclusively for the construction of pit timber and timber for paper making; (6) Establishing a forestry fund system. The State shall establish a forest ecological benefit compensation fund to be used for the construction, upbringing, protection and management of protection forests that provide ecological benefits and forest resources and trees with special uses. Forest ecological benefit compensation funds must be earmarked for special purposes and may not be misused for other purposes. Specific measures shall be formulated by the State Council. Bhutan, India, the Philippines and Vietnam also reported an increase in forest cover over the past 10 years. Despite a net increase in forest area reported at the regional level, the rate of deforestation remains high in many countries. In the past decade, Southeast Asia has suffered the greatest loss of forest area, with a net loss of more than 900,000 hectares a year. China, India and Vietnam have all set targets for large-scale reforestation, as well as incentive programs to encourage small farmers to plant more trees. China plans to increase its plantation area by 50 million hectares to 23 percent by 2020. If the current rate of afforestation continues, it may be possible to achieve this goal by 2015. India has set a goal of 33 percent forest and tree cover by 2012. Forests, other woodlands or land covered with trees accounted for about 25 percent of India's land area in 2010, according to the Global Forest Resources Assessment. The area of unknown rows of trees planted and other "trees outside the forest" should be included in this percentage. The government's goal was to restore forest cover to 43 per cent by 2010, and according to information provided to the Global Forest Resources Assessment, this goal has been met.

\section{Suggestions}
(1) Strengthening the awareness of forest resources protection
The human living environment is increasingly bad, the ecological environment problem has become the focus of globalization. The destruction of forest resources leads to soil erosion, land desertification, poor air quality and even harm to human health. We should realize the importance of forest resources, strengthen the awareness of forest resources protection, the protection of forest resources is everyone's responsibility, starting from ourselves, fundamentally accept and support the forest resources protection policy, dare to report illegal logging behavior, participate in the protection of wild animals and plants. On the basis of not destroying, afforestation can promote the healthy and sustainable development of forestry, mobilize the enthusiasm of the broad masses of people, effectively protect forest resources to the greatest extent, improve the ecological environment, and enable people to live in harmony with nature.
(2) Strengthening the construction of forest resources management institutions
The management and protection of forest resources should be based on the principle of sustainable development and operation, and rational harvesting should be carried out according to the quality and structure of forest resources. Forest resources management agencies should be established and the management system should be improved. According to the actual situation of forest resources, forest management and protection stations should be set up reasonably, and the management and protection personnel should be trained to improve their management and protection ability and legal awareness of forest resources. We should reasonably formulate quotas for forest cutting, conduct management and supervision over the issuance of cutting licenses in accordance with the Forest Law, ensure that there are laws to follow for cutting with vouchers, supervise the progress and management of cutting, and resolutely prohibit arbitrary cutting. Forestry departments should also adapt to the requirements of the current social development situation, under the trend of rapid development of information, increase investment to improve hardware and software infrastructure, establish a perfect information system, improve configuration facilities, improve work efficiency. The perfect forest resources information management system, standardized management data collection and analysis, improve the level and quality of monitoring, scientific management and protection of forest resources.
(3) Strengthening control of diseases and insect pests of trees and protection against forest fires
Natural disasters of forest resources mainly have two aspects, one is pests and diseases, the other is forest fires. These two kinds of harm to the forest destruction and loss are huge. As for the prevention and control of diseases and insect pests, the quarantine of quarantine forest plants and the prevention and control of forest diseases and insect pests should be strengthened. The Quarantine Law should be strictly implemented to prevent the introduction of harmful organisms and put an end to the man-made spread of pests. Monitoring and forecasting of forest diseases and insect pests, taking the initiative to prevent disasters and effectively control the scope and harm of diseases and insect pests; On the other hand, the prevention and control of forest fires has always been the top priority of forest resource management. For a long time, the need to constantly, vigorously adhere to the propaganda of fire prevention work, multimedia, multi-faceted personnel into the forest to strengthen the awareness of fire prevention, has been unable to relax vigilance. Implement the fire prevention responsibility of relevant units, control fire source management, carry out regular joint prevention of responsible households, establish and perfect fire monitoring and forecast system, strengthen the allocation of fire prevention equipment, fundamentally control fire sources, timely control when fire occurs, and reduce the loss of forest resources.
(4) Improve the utilization rate of forest resources
The utilization of forest resources is mainly the development of forestry, and the industrialization of forestry development is its development trend. In the management and management of forestry, we should first cultivate forest resources fundamentally, increase the intensity of forest cultivation, improve the quality of forest resources and improve the ecological environment. On this basis, forestry industrialization, technological production technology improvement, research and development of new wood processing industry, adjust the direction and structure of wood products. At the same time, the comprehensive development of forest resources, the development of breeding, planting, and related processing industries, in many ways to improve the maximum utilization rate of forest resources.


\chapter{Invasive Species}

\chapterauthor{Soleil Laurin}

\footnote{Statement of Contributions-- For example, ``The chapter was first drafted by Marc Los Huertos (2021). The author recieved valuable feedback from X, and Y and Z to improve the chapter. Slater revised the chapter in 2022 with suggestions from Cater.'' Note: I am still working on the formatting for this to improve it.}

\section{Introduction}% Avoid putting text between section and subsection headings.


  Trekking through the forests of Borneo, a traveller is overwhelmed by the beauty of this South-Asian Island. Their rainforests are famous for their natural beauty and ancient species. As an island, native species in Borneo have been isolated from mainland Asia, allowing for independent evolution to take place with its species, making Borneo uniquely diverse with species endemic to the island. The traveller may find many different species in just one tree if they look closely. But Borneo's incredible biodiversity, like so many other East-Asian countries, is under threat. In our travels to Borneo, people have brought invasive species to many East-Asian countries that threaten their increased rates of biodiversity. What's more, through the wildlife trade we have also taken East-Asian species from their homes and threatened other areas with East-Asian invasive species. With the influx of invasive species, native biodiversity in East-Asia has been threatened. Invasive species are a driving force behind biodiversity loss in East Asia.
	
\section{Biodiversity}% Avoid putting text between section and subsection headings.
	
\begin{figure}
\includegraphics[width=\linewidth]{images/biodiversity/biodiversity-image}
\caption{An image displaying the biodiversity of grasshoppers.\citep{lanting_2021}}
\label{fig:Lanting}
\end{figure}
	
	Biodiversity (Figure \ref{fig:Lanting}) can be defined in a couple of different ways. It can refer to the genetic diversity of species, the diversity of different species, and the variety of ecosystems those species form (\citep{brooker_2017chapter}, 1261). Biodiversity regulates the Earth's natural cycles and keeps order in ecosystem food chains. For example, in a rainforest ecosystem, such as those found in Borneo, plants help to maintain the water cycle by releasing oxygen and water vapors, which form clouds and cause the rain storms that are characteristic for the ecosystem (\citep{brooker_2017chapter}). The plants in a rainforest also provide several ecosystem services for a rainforest, like providing shelter and food for many species. Herbavores could shelter in the trees and feed on the fruit or leaves that a tree provides, giving them the energy they need to survive (\citep{brooker_2017chapter}). Predators could hunt the herbavores which live in the trees, and the leftover carcasses of their prey could fall to the forest floor to be consumed by the organisms below the tree. Fungi and bacteria could then decompose the carcass, recycling important nutrients back into the soil to be taken in by the tree again, completing the cycle. These natural cycles are critical in a healthy ecosystem, and would not work without biodiversity. 
	
	A biodiverse ecosystem is a healthy ecosystem, meaning the ecosystem is balanced with diversity so no one species has the upper hand and each species population is regulated by other species as well as abiotic factors, like weather or access to resources. But this balance of diversity can be too easily disrupted. This is what David Attenborough discusses in his doccumentary, "David Attenborough: A Life On Our Planet." Attenborough states how fortunate he was to be born at the right time, so he could witness the ecosystems of the world change as human technology and population grew. Attenborough travelled many times to areas of East-Asia, and noticed how the area's biodiversity changed over the years, as it did in other parts of the world. When he was born in 1937, the worldwide human population was 2.3 billion, and the remaining wilderness was 66 percent. In 2020, the world population was 7.8 billion, with only 35 percent remaining wilderness. As human populations expand into the remaining corners of the world, David Attenborough explains how he has watched biodiversity diminish over the years. He says we are going through the sixth mass extinction, as species are being lost at unprecidented rates (\citep{Attenborough2020life}). Reasons for this biodiversity loss include deforestation, habitat loss, and, near the top of the list, invasive species.

  An invasive species can be defined as, "an organism that causes ecological or economic harm in a new environment where it is not native," by the National Oceanic and Atmospheric Administration, or NOAA (NOAA 2021). Invasive species can proliferate in environments where they have no natural predators and can disrupt the food chains in their ecosystems. These invasive species pose a high risk to the diverse ecosystems in East-Asia, and their effects have already been felt in East Asian countries. They can create monocultures, displace native species, alter ecosystems, carry harmful diseases, and compete with native species for resources. 
  
  One of the most contriversial topics around invasive species and around ecology in general is competition. The theory of competition is that species consume or control access to resources in the ecosystem which is limited, and that this negatively impacts other species which also rely on these resources (\citep{Keddy2001competition}). In the case of invasive species, the argument is that invasive species take up valuable resources in a native ecosystem which native species rely on. It is also argued that competition is not influential to ecosystems since they are rarely near equilibrium, where all species and resources would be in balance with each other. This would mean that invasive species would not be throwing off the equilibrium of the ecosystems they invade, which implies that they do not impact the ecosystem in detrimental ways by competing with natives (\citep{Keddy2001competition}). However, there is evidence to contradict this argument. Scientists can measure competition in several ways. They can measure the fitness of species in relation to their population density, and compare this to other species in an ecosystem to measure intraspecific competition, or competition between other species. This method can be particularly compelling when tested in a controlled environment, where one species is tested first by itself in an environment and given all the resources it needs to survive, and second in the same environment with the addition of another species which uses the same or similar resources. If the first species population and fitness is limited by the other species, then it could be argued that the two species are competing for the resources in the environment (\citep{Keddy2001competition}). This experiment could be applied to measuring competition of invasive species, by first measuring the fitness and population trends of an environment before the introduction of an invasive species and then measuring the same factors after the introduction of an invasive species.

\begin{figure}
\includegraphics[width=\linewidth]{images/biodiversity/hot-spot}
\caption{Figure displaying biodiversity hotspots, East Asia region outlined in green.\citep{britannica2020conservation}}
\label{fig:britannica}
\end{figure}
	
	Invasive species have impacted biodiversity on a global scale, but the effects of them have been felt particularly hard in East-Asia. East-Asia is known as a "biodiversity hotspot" since it has a large amount of biodiversity (Figure \ref{fig:britannica}), as well as endemic species which are only found in East-Asia and not found anywhere else in the world (\citep{britannica2020conservation}). Wildlife in East-Asia is already threatened by human actions, such as habitat loss, poaching, and pollution. With the addition of invasive species, which can out-compete native species, prey on native species, spread diseases to native species, and alter native ecosystems, many of these unique species do not have a chance. As native species diminish, the biodiversity of East-Asian ecosystems diminishes, making these ecosystems more vulnerable to human damage, even putting local communities of people at risk.

\section{Invasive Species In East-Asia}

  In East-Asia, when invasive species are introduced, they can displace the native species of the area. Many species introduced do not get to the level of an invasive species, and instead will die off shortly after arriving in the place they are introduced into, or will simply cohabitate with native species. In these cases, the introduced species does not harm the native ecosystem in detrimental or even noticeable ways. However, when an introduced species becomes invasive, it can become a serious threat to natives, especially in areas with high rates of biodiversity and high percentages of endemic species, like East-Asia. 

  Some invasive species in East-Asia create competition with native species for the limited resources in their ecosystems. For Magpie Robins, a bird native to Singapore, tree hollows are important during the breeding season. The hollows provide vital shelter for the bird and its young, making perfect nesting sites. However, tree hollows can be difficult to find since they are a limited resource, and the birds have competition. Common Mynas, an invasive species from Southern Asia, also nest in tree hollows. This increased competition displaces the Magpie Robin from its natural habitat. Since many birds lose the protection of the tree hollows during mating season, their young are at greater risk for predation (\citep{peh2010invasive}, 1086). The Magpie Robin population begins to drop and their population niche is replaced by the Common Myna, disrupting the ecosystem of Singapore and impacting the food chain.

  Other invasive species in East Asia, such as the Yellow Crazy Ant, can also out-compete native species and negatively impact ecosystem functioning. The ant’s invasion in Singapore’s primary forests comes with two outstanding factors that make this invasion unique from others in East Asia. First, it speaks to the declining health of Singapore’s primary forests. The richness of biodiversity in these forests used to be able to protect them from harmful invaders, but with increased habitat fragmentation and climate change, their loss in biodiversity could have made them more vulnerable to invasive species. The second outstanding factor in this case is Singapore's challenges to control invasive species. Singapore is in a region part of a highly fragmented part of East Asia, which includes other biologically unique and diverse countries such as Indonesia, Malaysia, The Phillipines, and others, which each have their own political systems to handle invasive species. This makes managing invasive species particularly difficult, because one country may be able to control their population while another country's population of invasive species spills over into surrounding countries. In the case of the Yellow Crazy Ant, its population could spill over out of Singapore, putting the biodiversity of all Southeast Asian countries at risk \citep[p. 1084]{peh2010invasive}.

  Invasive species in East-Asia also cause ecosystem modification, causing localized biodiversity loss. In the Southeast Asian wetlands, the ecosystem is threatened by the invasive Golden Apple Snail, which feeds on native aquatic plants (Figure \ref{fig:CABI}). The snail was first introduced to areas such as China, the Phillipines, Taiwan, Myanmar, and Vietnam as a potential food source. The snails soon became a major pest in these areas, putting aquatic wildlife and local farms at risk (\citep{cabi2021}).  In 2004, a report was published regarding the effects of the Golden Apple Snail in Southeast Asian wetlands. The report found that the snails consume every plant species in the wetlands except for one particular species, and have little to no predators in the area. As a result, the density of Golden Apple Snails in a given area of wetland negatively correlated with the plant diversity in that area. With a higher density of the invasive snail's population, the plant life was less diverse. This causes detrimental impacts on the ecosystem, since the diverse plant life found in these areas of wetland provide important ecosystem services, such as refuge from predators, food sources, nutrient storage, water quality, and biofiltration.  The report concludes with this point, saying that as the diversity of these plants decreases, the health of the ecosystem decreases as well (\citep{carlsson2004invading}, 1579). Additionally, with the loss of plant diversity in the wetlands, algae proliferates, modifying the wetland ecosystem in negative ways. With the introduction of the Golden Apple Snail, localized biodiversity loss has occurred and the ecosystem has been negatively modified.

\begin{figure}
\includegraphics[width=\linewidth]{images/biodiversity/golden-apple-snail}
\caption{Image of the Golden Apple Snail, a rampant pest in Southeast Asia. \citep{cabi2021}}
\label{fig:CABI}
\end{figure}

  While East-Asian countries are faced with invasive species, they are not alone in their struggle. East-Asian species can become invasive in other parts of the world. Many invasive species which are a problem in East-Asia are also a problem in other regions. These organisms have far reaching impacts on a global scale, and are difficult to control wherever they are introduced.

\section{Invasive Species Effecting Other Regions}

\begin{figure}
\includegraphics[width=\linewidth]{images/biodiversity/burmese-python}
\caption{A Burmese Python caught by a snake hunter in the Everglades \citep{frazier2019thesnakes}.}
\label{fig:python}
\end{figure}

  East Asia has also produced many invasive species, which reak havoc on ecosystems around the world. One such invasive species is the Burmese Python, which has spread to the Florida Everglades through the exotic pet trade. The snakes originate from Southeast Asia, where they are a top predator in a warm tropical environment. They can grow up to twenty feet long and weigh two hundred pounds, with razor sharp teeth that point inwards to draw prey into their bodies (Figure~\ref{fig:python}). The snakes have completely dominated the Everglades with a population of thousands, seriously impacting native wildlife. With no predators to stop them, the only factor limiting the snake's population growth is prey. They started with small mammals: marsh rabbits, raccoons, oppossoms, and other species. However, with this resource exhausted, they have moved on to larger prey, like deer, bobcats, and sometimes alligators and other snakes. To aid in containment, public, private, and tribal groups have joined scientists in hunting and tracking the snakes throughout the Everglades (\citep{frazier2019thesnakes}). While their efforts to curb the growth of these beasts is admirable, there are still hundreds of Pythons in the Everglades, and they have already altered the ecosystem in powerful ways.
  
  Another East Asian invasive species, \emph{Cyprinus Carpio} or more commonly known as Asian Carp or Common Carp, poses a great risk to the unique ecosystem of the Great Lakes. The Great Lakes already face several invaders, including Zebra Mussels, Quagga Mussels, and many plant invaders, but \emph{Cyprinus Carpio} is especially worrisome to the area. The fish was first brought to the Southern United States for fish farming, and was originally not meant to be able to breed (\citep{scott2020}). Somehow, however, they bred and escaped the fish farms and ended up in the Mississippi River where they put many native fish species at risk. They consumed many native species in the river to threatening levels, slowly working their way north until now (\citep{scott2020}). With the invaders so close to the Great Lakes, many people are concerned they could cause irreversible harm to the lakes since the great lakes have already faced regional extinctions from existing invasive species (\citep{nationalwildlifefederation2021}). In order to keep the carp out of the Great Lakes, government and private workers have been working together to preserve the unique biodiversity of the lakes.

\begin{figure}
\includegraphics[width=\linewidth]{images/biodiversity/guam-snake-invasive}
\caption{The Brown Tree Snake, an invasive predator in Guam. \citep{rodda2007biology}}
\label{fig:guam}
\end{figure}

While many invasive species from East-Asia have travelled far through intentional and unintentional introduction, some have remained closer to East-Asia. In Guam, an island nation in the South Pacific, another East-Asian invasive species has begun to take over. Accidentally brought over after World War II from Indonesia, the Brown Tree Snake has taken up residence in Guam's famous forests (\ref{fig:guam}). As an arborial snake, the Brown Tree Snake took advantage of Guam's rainforests, preying on rare bird species which once provided an important source for tourism revenues to the country. The snake caused thirteen of Guam’s twenty-two native bird species to be extirpated (extinct from Guam but native to other areas), as well as several species of native bats and lizards. The Brown Tree Snake has not only impacted the local ecosystem of Guam, but also the people of Guam. It has caused damage to electrical infrastructure, bitten local residents, killed and eaten domestic animals, caused higher costs of shipping from Guam, and threatened the tourist industry of Guam. The snake is estimated to cause a power outage in Guam every other day, costing billions. They have been known to eat domestic animals, such as poultry and pets, and have been known to bite infants as they sleep in their cribs. Since the snake poses a threat to other pacific countries that Guam ships to, cargo to and from Guam is inspected closely to be sure the snake is not stowed away as an unwanted passenger. They have also threatened the tourist industry, by not only killing off local species but also scaring tourists away from the island (\citep{rodda2007biology}). The brown tree snake has impacted the island of Guam’s ecosystem significantly, but also its people and their economy. 

\begin{figure}
\includegraphics[width=\linewidth]{images/biodiversity/crabs-and-ants}
\caption{Red Crab being consumed by Yellow Crazy Ants in Christmas Island. \citep{Baldwin2010great}}
\label{fig:crabs}
\end{figure}

Some invasive species have unknown origins, but affect many regions of the world, such as the Yellow Crazy Ant. As previously discussed, the Yellow Crazy Ant has reached many parts of East-Asia and has out-competed many native species for space and food in that environment. On Christmas Island, however, the Yellow Crazy Ant has become a serious problem for the native Red Crabs, which have a yearly migration back to their ocean breeding grounds. The ants attack the crabs, spraying acid in their eyes and mouths, until the crabs collapse. At this point, the ants go inside of the crab’s bodies and tear them apart, carrying their flesh back to hungry ant larvae (\ref{fig:crabs}). In the 1990s, the ants cut the crab population by a third, from 80 million crabs to 50 million. The Yellow Crazy Ants on Christmas Island have little holding them back (\citep{Baldwin2010great} 8:50-11:43).

	Other invasive species can cause diseases in native species, but this has not yet been recorded in East-Asia. One species could have this effect in East-Asia, since it has caused diseases in other areas it has invaded. This species is the Feral Rock Pigeon, found in many Southeast Asian areas, and is known for transmitting at least 40 diseases to birds in other parts of the world. In the Galapegos, the pigeon has transmitted \emph{Tricomonas gallinae}, a parasite common among birds which can cause necrosis in the mouth and asophagus. It has been fatal to many native pigeons on the islands. Another species, the invasive \emph{Rattus rattus}, or ship rat, could similarly transmit another disease to native East-Asian species. They carry the fatal disease Sarcocystis, which could kill many native rodents. The disease comes from a parasite, which after entering the rodent through ingestion, causes cysts to form in the rodent’s body, eventually killing them (\citep{peh2010invasive}, 1086-1087). These species could not only effect native wildlife, but also local people and their livestock. While none of these cases have been seen in East-Asia yet, it is another risk to consider with invasive species (\citep{peh2010invasive}, 1086-1087).
	
	Invasive species can cause a lot of damage, not only to East-Asian countries, but globally. With increased globalization, environments are more connected than ever before. This makes the risk of invasive species even greater. They are the second leading cause for biodiversity loss, behind habitat loss. The problem of invasive species is something that people need to get under their control, in order to restore the biodiversity of our planet.

\section{Efforts Against Invasive Species}

	While it may seem like the fight against invasive species is hopeless, many scientists and organizations are working tirelessly to remove invasive species and repare the broken ecosystems they have created. In the Everglades, several tactics are being used to remove Burmese Pythons from the area. One tactic is to track specific males, called "sentinel males," and catch the snakes they run into. This is a particularly useful tactic during the mating season between December and March, when the sentinels can lead the scientists to breeding females, often with eggs and other competing males (\citep{frazier2019thesnakes}). Another tactic used with the pythons is to team up with local Floridians to create hunting initiatives. Snake hunters are paid eight dollars and forty-six cents an hour with an additional fifty dollars per snake caught, and twenty-five dollars per foot over four feet. This gives the hunters incentive to catch the pythons; some hunters are full time. Since March of 2017, hunters contracted to the SFWMD have caught over 2,000 pythons, making a significant dent (\citep{frazier2019thesnakes}).
	
	Also in the Southeastern United States, a species of chinese tallow, or Triadica sebifera, has invaded, and could be removed with the risky tactic of introducing a "biological control agent." A biological control agent is introducing a second species, often from the same area as the pest or invasive species, to manage the first one (\citep{vincent2007biological}). This tactic of controlling invasive species is very contriversial, since it poses the question of "where does it end?" If the new species does not work and becomes a pest itself, would more foreign species have to be introduced to manage them as well? And if those do not work? This tactic can easily get out of hand, if conducted improperly. 
	
  One example of this was seen with the ladybird beetle, originally from Northeastern Asia, commonly known as ladybugs. The ladybug was first introduced as a biological control agent to control several agricultural pests, including aphids and mites which feed on corn, apples, and other crops. Since the beetle was so versatile in multiple environments, it was introduced into several areas of the United States and Canada, including California, Nova Scotia, Connecticut, Louisiana, Ohio, and other states. While the beetle was tested on the pests it was introduced to control, it was not tested with many of the other insects in the native areas it would be introduced into. When the ladybug was eventually introduced, it was such an efficient predator that it not only took care of many aphids and mites, but also many non-targeted insects. The ladybug grew in population, and now can be found anywhere in North America between Quebec and Florida (\citep{vincent2007biological}).
\begin{comment}  
  But this tactic has proven to be effective in some cases, when done properly. In lake Victoria in Uganda, this tactic was used against the invasive water hyacinth, using weevils to kill the hyacinth. The weevil was first tested with native plants, to be sure that it would not consume them and only consume the water hyacinth. When this proved true, the weevils were released into the lake, and closely monitored. The hyacinth population was soon under control, restoring the health of the river (\citep{norton2004strange}25:21-32:30). A similar process occurred in Australia with the introduction of parasitic flies to control the four introduced Mediterranean Snails which became invasive. In this case, several different species were tested with the invasive snails as well as with native snails to see how each possible biological control agent would effect native snails. After some trial and error testing, scientists eventually released a small population of S. penicillata, which attacks conical snails specifically. They hope that these flies will help to control the snails after some time (\citep{vincent2007biological}).

\begin{figure}
\includegraphics[width=\linewidth]{images/biodiversity/table}
\caption{Table displaying North American plant species tested with the leaf-rolling weevil. - symbol indicates the weevil did not interfere with the species, ++ indicates the adults consumed this species, +++ indicates adults consumed and layed eggs on this species. \citep{steininger2013biology}}
\label{fig:table}
\end{figure}

  A similar tactic is being considered in the Southern United States to handle Triadica sebifera, using the leaf-rolling weevil. A study was conducted on these weevils as displayed in the table above(\ref{fig:table}), and it found that the weevils only laid eggs and ate Triadica sebifera when given the option of several species of plants within the same genus. In few cases, the adult weevils consumed a native species, but they only ever laid eggs on Triadica sebifera. This indicates that this weevil could be useful in controlling Triadica sebifera’s population in the Southern United States (\citep{steininger2013biology}). While biological control agents can be risky, when done properly, they can lead to proper management and control of invasive species.

\begin{figure}
\includegraphics[width=\linewidth]{images/biodiversity/nesting-box}
\caption{The Magpie Robin, pictured next to its nesting box which contains its eggs.\citep{singh2016nesting}}
\label{fig:nesting}
\end{figure}

Another way to handle invasive species could be to support native species by providing them the resources they need to thrive. In the case of the Magpie Robins in Singapore, Common Mynas were taking up valuable nesting resources as previously discussed. One solution that was studied to aid the Magpie Robins during the breeding season was the use of bird nesting boxes in more densely populated areas of forest (\citep{singh2016nesting}). The bird boxes provided additional shelter and nesting resources to the birds, so they were not so easily out-competed by the Common Mynas. This way, both the Mynas and the Robins can live in a similar area without putting the Magpie Robins at risk for extinction (\citep{singh2016nesting}). With this tactic in place, it would still be important to monitor the population of Common Mynas and to make efforts to control their populations, but the native species most threatened by their presence would be better protected against the effects of the invader.

	In the case of the \emph{Cyprinus Carpio} threatening the Great Lakes, the main goal for invasive species management is to stop the spread. One defense is an electric fence, which fills the water with electric charges telling the fish to turn around. Another defense, recently approved in 2020, is a dam which uses air bubbles to get the fish out from under the boats and sounds to deter them from the lakes (\citep{scott2020}). \emph{Cyprinus Carpio} do not like the sound of motorboats, so the sound can be used to push them away from the lakes (\citep{scott2020}). These defenses could help to protect the Great Lakes from other invasive species as well, which would keep the lakes safe from further damage (\citep{scott2020}).
	
\section{Conclusion}

Invasive species continue to pose a significant threat to ecosystems in East Asia, but it’s not too late for many of East Asia’s ecosystems. Efforts to both curb the populations of invasive species and preserve current wildlife are underway in many parts of East Asia. In Southeast Asia, with the help of the Center for Agriculture and Bioscience International (CABI) and the United Nations Environment Program (UNEP), a project was developed in 2011 to 2016 to protect and conserve forests in Southeast Asia, with a focus on aiding countries with their management of invasive species. The goals of the project were to set up national policies and institutional frameworks, facilitate co-operation between Southeast Asian countries, support and strengthen the country's environmental institutions, facilitate work to monitor, control, and prevent invasive species, and educate the public about invasive species. Their efforts were overall successful, setting up a National Steering Committee on invasive species and formulating the National Invasive Species Strategic Action Plan in Vietnam (\citep{cabi2020}). With continued efforts such as these, invasive species can be monitored, controlled, and prevented in affected areas. Invasive species may be one of the leading causes of biodiversity loss around the world, and continued globalization has only made this issue more difficult. But with efforts like those seen in East Asia, we may be able to stop them.
\end{comment}


\chapter{Flood Pulse System of the Mekong River and Tonle Sap}\label{ch:floodpulse}

\chapterauthor{Kristin Gabriel and Marc Los Huertos\footnote{Gabriel initiated the research and early drafts of the chapter. Los Huertos will be adding content, references, and figures over the summer of 2021.}}

\section{Introducing the Mekong and Tonle Sap}

\subsection{The River-Lake System}

The Mekong River and Tonle Sap Lake function as one of the most important river-lake systems in the world. The river itself runs through six countries: China, Myanmar, Thailand, Lao PDR, Cambodia, and Vietnam (Figure~\ref{fig:tonlesap}). The Mekong river is currently ranked as the 10th largest river by volume on the planet, while the Tonle Sap is the largest freshwater lake in Southeast Asia.

\begin{figure}
\includegraphics[width=\linewidth]{images/floodpulse/TonleSapMap}
\caption{Map of Tonle Sap (Source: Wikicommons).}
\label{fig:tonlesap}
\end{figure}

Throughout the year, the Mekong River experiences significant flow alterations due to the seasonal monsoon (FIND SOURCE). Because of this, the water level in the Tonle Sap is controlled by the water level in the Mekong main stem.

Because of the unique weather patterns of southeast Asia and connection between the Mekong and Tonle Sap, a unique flood pulse system is created. The flood usually lasts an average of 258 days, or 71\% of the year. On average, the flood begins at the end of June, peaks at the beginning of October, and ends in early March, with some variation of course. During the wet season, the Tonle Sap's size expands by about four or five times, and this water then flows into the Mekong during the dry season as its level drops (Evans et al. 2004). This creates a distinctive flow pattern, in which the river's main stem actually flows in opposing directions in the flood season and dry season (SOURCE).

The Tonle Sap ecosystem is also one of the most heavily relied-upon ecosystems in the world. The lake is especially important to those in Cambodia and the lower Mekong Basin; in this area, over one million people depend on the natural resources of the lake . In many areas of the basin, water resources are used intensively at a small-scale loval level. However, the Mekong basin is under rapid development, including the construction of large-scale hydropower dams, deforestation, development, and climate change . However, recent research has found that the construction of dams and other development activities may significantly alter the hydrology, and therefore the health, of the ecosystem.

\section{What is the Flood Pulse System?}

Scientists who study riverive systems, such as water flow, channel morphology dynamics, and plant and animal organisms have tried to characterize streams to better understand them and make predictions about them. 

The Flood-Pulse Concept does exactly this. By acknowledging how rivers can extend well beyond their low flow channels and flood a broad lowlying area on a seasonal basis is a clear pattern of certain rivers, especially those that experience highly seasonal rainfall that is a characteristic of monsoon climates.

\begin{figure}
\includegraphics[width=\linewidth]{images/floodpulse/flood-pulse-concept-of-Bayley-1995.png}
\end{figure}

\subsection{Habitats in the Flood-Pulse System}

The floodplain vegetation is one of the most important elements of the Tonle Sap ecosystem. Because  they are inseparable in terms of water, sediment, and shared organic material, rivers and their respective floodplains are considered the same unit, and they work as a combined river-floodplain system. Floodplains are defined ecologically as areas that are periodically inundated by the lateral overflow of rivers or lakes, and/or by direct precipitation or groundwater (The flood pulse concept in river flood-pulse systems).  The river-floodplain system comprises a permanent moving-water ecosystem (lotic), a permanent marshy, stillwater habitat (lentic), and the floodplain. Despite many researchers’ efforts, it is impossible to apply stable and defined borders between land and water in any floodplain.

\subsection{Organisms}
This p  eriodic flooding produces similar anatomical and physiological adaptations in the organisms inhabiting different areas affected by the same phenomenon, and it produces characteristic community structures. The life cycles of organisms in the floodplain are very related to the timing, duration, and rate of rise and fall of the flood pulse system. As an ecosystem, floodplains are very unique because they have both aquatic and terrestrial phases. This leads to specific adaptations among both aquatic and terrestrial animals. For example, aquatic organisms are selected to populate the floodplain at rising and high levels because of the abundance of food. Conversely, terrestrial organisms have adapted to take advantage of the floodplain at low water levels.

\subsection{Nutrient Patterns}
The  lateral exchange of nutrients between floodplain and river channel has the most direct impact on the biota of the ecosystem. The floodplains receive all nutrients directly from the main channel; however, because of their differing biogeochemical cycles to the Mekong River and Tonle Sap, the floodplains develop a different nutrient profile (The flood pulse concept in river flood-pulse systems). These nutrients include both organic and inorganic compounds, that can be further divided up into gaseous compounds, dissolved solids, and particulate matter.

\subsection{Hydrology}

\subsubsection{Climate, Monsoons, and the Holocene}

@article{penny2006holocene,
  title={The Holocene history and development of the Tonle Sap, Cambodia},
  author={Penny, Dan},
  journal={Quaternary Science Reviews},
  volume={25},
  number={3-4},
  pages={310--322},
  year={2006},
  publisher={Elsevier}
}

\subsubsection{Ecosystem Productivity}

The Tonle Sap ecosystem is widely perceived as an ecosystem with naturally high productivity. The Asian Development Bank even considers the productivity of the lake to be ``among the highest in the world'', and this is widely believed to be because of the flood pulse system itself. However, most of the beliefs surrounding the ecosystem productivity of the Tonle Sap actually refer to fisheries production and productivity. Despite fish catch statistics being incomplete and historically unreliable, there has been no research or specific data collected to determine either the primary or secondary productivity of the ecosystem.

Therefore, fish catch data must be used as an indication of ecosystem productivity in this situation. Another powerful indication of the ecosystem's productivity is the livelihoods that are supported, either directly or indirectly, completely or partially, by the river-lake system.

Because the surrounding communities and other users of the lake’s natural resources are especially dependent on its productivity, the lake's ability to support these communities’ livelihoods is a robust way of determining its productive capacity. This also makes the ecosystem's productivity especially important, in that it is not only a key factor of environmental health, but it is also a key determining factor for social equity and economic welfare.

Ask: what is the best way to talk about fish catch data? It seems very all over the place from family fisheries to large scale operations. Would a description/summarization of the data be helpful, or should I just summarize overall findings and trends?

\section{People and the Flood Pulse System}

\subsection{Anthropology}
	The Tonle Sap's abundance of fish and sustenance has been documented for centuries. It has even been hypothesized that the Tonle Sap's productivity was the basis for the development of the Khmer Empire: The huge productivity of the Tonle Sap sustained a Khmer empire at Angkor from 802 C.E. until the fifteenth century. (Evanssourcefromthissource) 

\subsection{Human Impacts on the Flood Pulse System}

\subsubsection{Hydropower Development}

\subsubsection{Climate Change}

\subsubsection{Urbanization and Development}

Main focuses:
•	Importance of flood patterns
o	Not just a diagram, but also math about how floods are measured/work
o	Some equations/acknowledgement of hydrology that looks at flood patterns and that can also tie in with the climate change section
•	Define lentic and lotic
o	Identify groups of organisms that like the different types of organisms
o	Then we can link that in the productivity section
o	Look at California levee that broke and filled in farmland and the fish were more successful in the flooded zone
o	Carbon input and insects in floodplains look at carbon input in lotic zone and lentic as well
•	River continuum concept
o	Historical development, the flood pulse system was in response to the river continuum concept
o	Ignores the lateral movement of  water
o	This will be helpful in the lotic zone of the river
o	Only covers part of the reality
o	Flood pulse model only covers part of the reality too. So we need to put them together!!
•	Use case studies of specific fish to explain the larger processes going on in the river
o	Can even use case studies from north America and Europe and hypothesize about what could be parallel in the Mekong
•	Don’t pretend like you know all the answers, invite questions, and POSE questions
•	Ecosystem Productivity: tell the reader how fisheries are measured and WHY that’s uncertain
o	Also define primary productivity and secondary productivity and how that’s measured in real life, like how do scientists figure that out
•	Look up river culture: an eco-social approach
•	Values of inland fisheries in the Mekong river basin
•	Influence of built structures on the tonle sap ecosystem
•	Civil society and interdependencies chapter on the tonle sap
•	Ngo doing toilets on the lake, what local ngos are doing
•	Look at Lucas’ article “Design With Nature” for inspiration
•	Should I include the Delta in Vietnam?? Food for thought
•	


Ecosystem Services

Fish stocks

Fisheries

Immigration and emigration


\chapter{The Sorrow of East Asia: Hydropower and Dam Development}

\chapterauthor{Quyen Hoang Ballagh}

\section{Pretext}
\subsection{Former Residents of Balui}

Prior to construction of the Bakun Dam, dozens of different Indigenous communities practiced subsistence farming, hunting, and rich cultural traditions along the Balui River. Indigenous communities such as the Kayan have developed and practiced knowledge of the environmental (Traditional Ecological Knowledge) for centuries in agriculture such as rice cultivation. Traditinoal tattoos, stretched earlobes, and connected apartments known as long houses have been the practice of communities such as the Kayan and others who live[d] along the Balui River until development of a megadam called the Bakun Dam which disrupted their watershed and displaced thousands of communities members from their homes \citep{aiken2015dams}. A series of case studies was conducted with the community forcibly displaced to government-controlled Sungai Asap in September 2002. The following quotes are pulled from anonymously interviewed Indigenous village members on comparing the conditions of their original home and their government controlled resettlement in Sungai Asap \citep{choy2004sustainable}. They are translated to English as directly as is possible. 

\subsection*{On Survival}

\begin{quote}
Life has been difficult for us since we moved to this area a few years ago. Now we are running out of means to buy our foodstuffs to pay our water and electricity bills. You just look at the electricity meters up there, supplies have been cut off as we have not been able to settle our electricity bill for months. \emph{}
\end{quote}
\quad \quad \quad \quad\quad\quad\quad\quad --As quoted in \citet{choy2004sustainable}.

\subsection*{On Social Change and Economic Development}
\begin{quote}
I do not think the Bakun project has brought us social and economic development as claimed by the government. The environment here is just not suitable for us. Last time we did not have to worry about our daily consumption because we got everything from the forest. Now everything is so different. No money, no food. Just look at the three acres of farmland allocated to me; it takes me about 20 to 25 minutes by land cruiser to get there. Yes, the government built the road for us but where is the means of transport? A return trip from here (Sungai Kojan) to my farmland in Sungai Asap cost me RM4 (US \$1.05). That is very costly for me.''
\end{quote}
\quad \quad \quad \quad\quad\quad\quad\quad --As quoted in \citet{choy2004sustainable}.


\subsection*{On Returning Home}
\begin{quote}
 [If given the chance, would you return to your original settlement?] 
 
Of course! [The] former settlement is better. Furthermore, when we are asked to pay for the house installment amounting to RM52,000 (US 13,684), we are ''habits'' [finished]. Where do we have the means to pay? Makan pun susah [Eating is also a problem]. By then, we have no choice but to go back to the jungle.
\end{quote}


\quad \quad \quad \quad\quad\quad\quad\quad --As quoted in \citet{choy2004sustainable}.


\begin{figure}
\includegraphics[width=\linewidth]{images/hydroelectric/borneomap.png}
\caption{Available for Public use\\ This is an image of dams in Malaysia--specifically within the Sarawak Region. Dams that are denoted in red have energy production capacities of over 500 MW. Generally, megadams are considered dams over 15 meters in height and having energy production capacities of over 400 MW so the red dotted dams are considered megadams. In bold, one can see the Bakum Dam which has the largest energy capacity in the region.}
\label{fig:refborneo}
\end{figure}

\section{Introduction}

\subsection{Overview of Hydropower}

Megadams across the globe have forcibly displacement of between 40 and 80 million peopled (Hai, 2016). Hydroelectric dams account for 19\% of the world's electricity needs. In Sarawak, Malaysia on the island of Borneo, dozens of new megadam projects are set to be built with plans to account for 7,000 megawatts of electricity by the end of 2020, and up to 20,000 megawatts of electricity on a long-term scale. For reference, China's annual electricity needs are in the 5.8 billion megawatt range and the U.S.was in the 1.2 billion megawatts range in 2020 \citep{gullen2013statista}.This is estimated at around 13,000 kWh (kilowatt hours) per capita which is converted to 13 megawatts.



Hydroelectricity and mega dam projects are notorious for displacing billions, disrupting Indigenous and traditional modes of livelihood, social structures, and impacting the social-emotional well-being of millions. Mega dam displacement disproportionately impacts rural and low income communities, many of whom are Indigenous to the land they are displaced from. For the purposes of this chapter, discussed displaced populations are local impovershed rural communities but not all of the mentioned statistics are exclusive to Indigenous groups.

Additionally, one of the largest ironies of hydroelectricity is that the benefits of the produced electricity, both the energy itself, and the money made from selling that electricity, goes to those who were not impacted negatively by the construction. Instead, energy produced from mega dams is exported to urban centers and cities and profit made off of it goes to development companies. With the harnessing of hydropower, large construction and deveopment companies profit off of the ecological and humanitarian degradation involved to generate electricity in megadams. Conversely, Indigenous and local populations bear the burden of the development a process termed water grabbing. In most cases, specifically with mega dam construction, the produced electricity is either exported to nearby urban cities--often in an attempt to meet growing energy demand with non--fossil fuel alternatives. The energy demand in rural communities in Sarawak in 2020 was about 972 megawatts with estimated increases to 1,500 megawatts annually. Despite the increasingly energy demand in the region, the dams in the area including the Baram and the Bakun both have approximate averages of 1,000 MW (megawatts) and 2,4000 MW respectively of annual energy production (See Figure~\ref{fig:refborneo}). In most cases, the produced energy far exceeds the regional demand. The energy is exported to other regions or countries and the displaced Indigenous groups are not granted access to the energy \citep{merme2014private}. This is in part due to the funders of dams and who facilitates construction. Prior to the late 1900s--1940 to roughly 1970-1990--dam construction was funded by the World Bank \citep{awojobi2015were}.

One of the reasons that dams were seen as a critical part of ``development'' and subsequently funded by large investment groups such as the World Bank was because at the beginning of the 20th century, the United States and North America began to be used. President Roosevelt of the United States began construction of large dams in the 1930s in the New Deal \citep{billington2006big}. This development continued in post World War II economic and population growth \citep{severnini2014power}. As colonization and imperialism occured in proxy states throughout the later half of the 20th century, countries such as Brazil and China began to encourage hydropower development \citep{folch2019hydropolitics}.

For example, Brazil during the 1970s under Alfredo Stroessner's dictatorship and U.S. imperialism in Latin America, dam construction was viewed as the pinnacle of modernity and development. This view is due in part because dam development was so prominent in the United States in previous centuries \citep{folch2019hydropolitics}. Under U.S. control of Latin America, the military dictatorships and big loans for development projects from funders such as the World Bank enabled mega dam construction such as the Itaipu in Brazil which is the second largest dam in the world to date after the Three Gorges Dam in China \citep{mainardi2017first}.

The World Bank began funding projects such as the Itaparica Dam in Brazil, as well as the Sardar Sarovar Dam in India, the Chonoy in Guatemala, and The Three Gorges Dam--the largest dam to date--as well as countless others. Beginning in the 1970s, criticism arose about the lack of social protection policies for the local and Indigenous communities affected by dam construction \citep{awojobi2015were}. Calls to action condemming the human rights abuses fascilitated an assesment and end the harm caused by the World Bank. The World Commission on Dams was released, outlining the human rights abuses by the World Bank and they withdrew official funding from hydropower projects in the 1990s. Despite this, other corporations have filled the gap in funding, namely actors based in China such as the China Three Gorges Corporation and Sinohydro \citep{cooke2017limits}. 


Since the 1990s, alongside a global push for renewable energy sources, construction of hydroelectric dams surged ({See Table~\ref{tbl:bigdams}})--specifically in East Asia with over 10,000 of the dams. The dams in Asia alone accounts for approximately 28\% of all worldwide dam construction. Even within that disproportionate 28\%, a large number of dams lie in Laos which has been devastated by Chinese foreign investment and funders. Many even refer to Laos as the ``Battery of [Southeast]Asia.'' In cases of water grabbing in hydropower in Laos, energy is exported to other countries or urban cities and the Indigenous or local groups who were impacted by dam development are not granted access to the benefits of the dams that uprooted their lives. Many of these corporate projects have failed international guidelines on human rights and Indigenous rights, as well as ecological harms, which are often tied to subsistence and cultural reciprocity (See Table~\ref{tbl:UNrights}). 


In many cases, the current social protection policies that do exist are lacking in implementation and relocation schemes can lead to justification for displacement. Relocation schemes enable hydropower companies to write off the harm they are causing to communities under the guise of some form of [insufficient] reparation. Furthermore, many displacement structures force Indigenous communities into cramped living conditions with poor sanitation, education, traditional modes of livilihood; including but not limited to subsistence/agriculture, as well as cultural aspects of tradition such as crafts, styles, and familial structure. This, paired with the fact that in most cases of water grabbing in hydropower construction, the Indigenous communities displaced are moved into or not given access to the electricity their lives were destroyed to create. Instead, many Indigenous groups must find their own way to pay for or generate electricity should they desire it in their community.

\subsection{History of Rural Electrification}

Rural electrification in Asia specifically can be traced historically back to the 1950s-1990s when electricity reform took place across the country by national government run programs \citep{williams2004asian}. Through both of the Cold War Superpowers--The United States and USSR-- as well as the World Bank and other foreign investors, states began to develop more wide-spread electricity sectors. Monopolies of electricity generation, transmission, and distribution were created in a features called state-owned utilities which gave the states complete control of production from start to finish (vertical monopolies) over electricity. In the time around and following the Cold War, post-colonial Asian countries saw electricity as representative of modernity and governments made promises to provide electricity nation-wide. Such promises were subsequently used to justify human rights abuses to generate that electricity which we see now in hydroelectric dams \citep{williams2004asian}. 

In terms of rural electrification specifically, many Asian countries made promises to provide electricity to rural areas. The World Bank for example started the Rural Electrification Project Phase I in Laos. But, because of increasing fossil fuel prices and increasing electricity demand, the World Bank's Rural Electrification Project was thought to have high costs with few financial benefits for the institutions paying for the projects \citep{bambawale2011realizing}. Authors such as \citet{van2012model} argue that many governments and investment companies see the high financial costs too high to bring electricity to sparsely populated regions. Because of this, much of the electricity produced by hydroelectric dams is exported to urban centers rather than distributed over complex grids for rural and sparsely populated communities. This export of energy from the dams means that those who are impacted or displaced by the dam construction do not often reap the electrical benefits of the energy production, demonstrating the inequitable process of water--grabbing.  

\begin{table} 

\caption{Selected UN Indigenous Rights Articles that directly pertain to land-use rights and are connected to hydroelectric dams \citep{mackay2002rights}. The UN Articles on Indigenous Rights is a set of guidelines the UN ratified to respect Indigenous Autonomy. Of the 46 total articles, these are the selected ones that pertain most closely to the impacts of dam construction and displacement. One of the problems with the UN and its articles is that it is difficult to enforce the guidelines and we see such violations with mega dam development \citep{gilbert2007indigenous} } 

\label{tbl:UNrights} 

\begin{tabular}{lp{5.5in}}\hline 


Article 3 & Indigenous peoples have the right to send determination\\ 

\rule{0pt}{3ex}

Article 4 & Indigenous peoples, in excercising their right to self-determination, have the right to autonomy 
or self-government\\
\rule{0pt}{3ex}
Article 5 & Indigenous peoples have the right to maintain and strengthen their distinct political, legal, economic, social and cultural institutions, while retaining their right to participate fully, if they so choose, in the political, economic, social and cultural life of the State.\\

\rule{0pt}{3ex}
 Article 7 & Indigenous peoples have the collective right to live in freedom, peace and security as distinct
 peoples and shall not be subjected to any act of genocide or any other act of violence, including forcibly removing children of the group to another group.\\ 
\rule{0pt}{3ex}

Article 8 &  1. Indigenous peoples and individuals have the right not to be subjected to forced assimilationor destruction of their culture. 

2. States shall provide effective mechanisms for prevention of, and redress for:

(a) Any action which has the aim or effect of depriving them of their integrity as distinct peoples,
or of their cultural values or ethnic identities; 

(b) Any action which has the aim or effect of dispossessing them of their lands, territories or resources; 

(c) Any form of forced population transfer which has the aim or effect of violating or undermining any of 
their rights; 

(d) Any form of forced assimilation or integration\\

\rule{0pt}{3ex}
Article 10 & Indigenous peoples shall not be forcibly removed from their lands or territories. No relocation shall take place without the free, prior and informed consent of the indigenous peoples concerned and after agreement on just and fair compensation and, where possible, with the option of return.\\ 


\rule{0pt}{3ex}

Article 11 & 1. Indigenous peoples have the right to practise and revitalize their cultural traditions and customs. This includes the right to maintain, protect and develop the past, present and future manifestations of their cultures, such as archaeological and historical sites, artefacts, designs, ceremonies, technologies and visual and performing arts and literature.\\

\rule{0pt}{3ex}

Article 20 & 1. Indigenous peoples have the right to maintain and develop their political, economic and social systems or institutions, to be secure in the enjoyment of their own means of subsistence and development, and to engage freely in all their traditional and other economic activities.

2. Indigenous peoples deprived of their means of subsistence and development are entitled to just and fair redress. \\ 

\rule{0pt}{3ex}

Article 26 & 1. Indigenous peoples have the right to the lands, territories and resources which they have traditionallyowned, occupied or otherwise used or acquired. 

2. Indigenous peoples have the right to own, use, develop and control the lands, territories and resources that they possess by reason of traditional ownership or other traditional occupation or use, as well as those which they have otherwise acquired. 

3. States shall give legal recognition and protection to these lands, territories and resources. Such recognition shall be conducted with due respect to the customs, traditions and land tenure systems of the indigenous peoples concerned.\\ 

\rule{0pt}{3ex}

Article 29 & ``1. Indigenous peoples have the right to the conservation and protection of the environment and the productive capacity of their lands or territories and resources. States shall establish and implement assistance programmes for indigenous peoples for such conservation and protection, without discrimination.''\\

\hline 

\end{tabular} 
\end{table}

\begin{table} 

%\caption{3 column table example, left justified, right justified, and center justified} 
\caption{These are some of the largest or most famous dams mega dams world wide (energy capacities of more than 400 Megawatts and heights above 15 meters). The Three Gorges Dam is the largest dam globally and produces the most electricity. Dams in East Asia are beginning to be built with more frequency, but other dams such as the Itaipu and Sardar Sarovar are also notorious for their human rights failures. Due to the nature of megadams and the amount of space they need, all of the dams in this chart have caused forced displacement of local or Indigenous populations.}
\label{tbl:bigdams} 

\begin{tabular}{llrrrl}\hline 

Dam Name & Location/Watershed & Reservoir Vol. & Height & Energy Capacity & Dam Type \\
        &                    & (km$^3$)      & (m)     & (MW) & \\ 
        
        \hline\hline 

Three Gorges Dam      & China/Yangtze River             & 39.3 & 181 & 101.6x10$^8$ & Gravity Dam\\ \hline 

Bakun Dam & Malaysia/Balui River & 43.8 & 205 & 2,400 & Embankment Dam \\ \hline

Sardar Sarovar Dam & India/Narmada River&5.8 &139 & 1,450 & gravity dam\\ \hline
Son La Dam&Vietnam/Black River &3.1 &138 &2,400 &Gravity Dam\\ \hline
Hoa Binh Dam&Vietnam/Black River& 1.6 &128  &1,920 & Embankment Dam\\ \hline
Nam Ngum & Laos/Nam Ngun River &4.7 & 70 & 865,000 &Gravity Dam\\ \hline
Itaipu Dam & Brazil \& \footnotesize{Paraguay/Parana R.} & 29 & 196  & 96.6 x 10$^8$ & Gravity and Embankment dam\\ \hline

\end{tabular} 
\end{table} 


\section{Dam Function}

  Hydroelectric dams hold water in artificially created bodies of water, or reservoirs. To generate electricity, water is released through a turbine which spins a metal shaft--propeller--within the turbine to convert the energy of the flowing water into energy from movement, or mechanical energy (USGS). Because energy cannot be created or destroyed, the hydroelectric dams convert one form of energy into another. In this case the energy of the moving water is the converted energy. Watts are distributed to a larger electrical grid for purposes such as energy for a household's lighting \citep{jansen1983dams}. This type of turbine that turns moving water's energy into electricity is known as a hydraulic turbine (Figure~\ref{fig:USBR}) 

\begin{figure}
\includegraphics[width=\textwidth]{images/hydroelectric/diagram.png}
\caption{The reservoir or body of water is often called the headwater. Water travels down to the forebay, then into a sloped area called the penstock where it reaches the turbine then the tailwater is released into the afterbay. The afterbay is lower than the forebay to allow gravity to move the water to generate electricity for the turbine to collect \citep{usbr}.}
\label{fig:USBR}
\end{figure}

\subsection{Hydroelectric Generators}

Hydroelectric generators are based on the ideas of Michael Faraday. He found that certain materials allow electricity to flow more easily. A magnet can move electricity through these conductors which allows the electricity to flow. When direct currents of electricity are circulated through loops of wire around magnetic steel--electromagnets--, poles of electromagnetivity are created--field poles. The field poles are put on the outside of the rotor which is attached to the turbine shaft. When the turned rotates, the field poles move past the conductors and electricity flows, causing an electromotive force, voltage, to develop. 
  
\subsection{Dam Type}

Dams are not created equal, nor are their environmental and human impacts the same.
Function, energy production, and human/environmental impacts of dams depend on the dams structure type. Large dams that are over 15 meters high and have energy capacity's of 400 Megawatts or more are termed mega dams. Two of the most common mega dams are gravity dams, and embankment dams. Another type of dam that is common are run of river dams. Run of river dams can be classified as mega dams or small dams depending on their size. Small dams are also used for hydroelectricity generation although they have less energy production capacity than mega dams.


\subsubsection{Gravity Dams}

Gravity Dams are dams that are made of concrete or stone and often hollow. They use the horizontal pressure of the water from gravity to create energy. Because they are reliant on gravity, they tend to have large heights and require stable foundations to bear the weight of the reservoirs.

If reservoirs are fully filled, the excess water flows over the top in a process called overtopping occurs. Gravity dams are generally resistant to the over-topping flows of the excess water because they are usually made with concrete which is not impacted by the erosion or scouring from the excess water overflow \citep{jansen2012advanced}. Depending on the size of the dam and the required energy production capacity, they can range in height, but in general, higher dams will yield more energy production \citep{lemperiere2017dams}. 

\subsubsection{Embankment Dams}

Embankment dams are usually made from natural minerals such as soil, sand, clay, rock, or occasionally, industrial waste materials. They can also be filled with either compacted earth --hydraulic fill dams--or rocks--rockfill dams--that are smaller than about 3 inches \citep{lemperiere2017dams}. Because of the materials used, embankment dams are generally more susceptible to erosion which means over-topping flows are more of a risk for embankment dams because the excess water can cause erosion of the minerals which puts the dam at risk for collapse or failure because the erosion of the top can cause imbalance of the dam and compromise the structure and the dam could fracture into pieces. Embankment dams are often on some sort of hill or bank with a different levels from high to low elevation \citep{jansen2012advanced}.

\subsubsection{Run of River Dams}

Run of river dams are classified by water flow. If the water flow upstream and downstream of the dam is the same, it is considered a run of river dam. Run of river dams should not cause water stoppage or dam the rivers so drastically to create a reservoir. A lot of literature advocates unequivocably for run of river dams as non destructive to the environment. Despite this, run of river dams actually can cause disruptions to fish migration patterns and buildup of soil, rocks, and particles that sink to the bottom of rivers or bodies of water--sediment--around the dam itself in a process known as sedimentation \citep{csiki2010hydraulic}. 

It has been suggested that some run of river dams are actually the least environmentally impactful option. For example, a run of river dam on the Mekong called the Xayaburi Project invested millions of dollars into research on fish ladders, locks, and elavators to protect downstream fishing. Despite this, the project still caused displacement and moved local populations into government controlled villages. Furthermore, as of 2021, there is still insufficient data to determine if the fish migration protection mechanisms will be effective enough to prevent disruption of the fish migration patterns and thus downstream fishing \citep{ascher_2021}.

\subsection{Energy}

\subsubsection{Kinetic Energy}

  The size of reservoirs for hydroelectric dams can impact the amount of energy the dam can produce because of the pressure which impacts energy of motion, or kinetic energy. It is the amount of energy from the movement of the water which can be estimated with the equation. 
\begin{equation}
KE = 0.5mv%^2%
\end{equation}
In this equation, $m$ is the mass of the object and $v$ is velocity or speed. 

This is important for hydroelectric dams because the strenght at which turbines spin is based on how quickly or strongly water travels--water flow. The mass of water is 1 kg~L$^{-1}$ and the speed depends on multiple factors, including the amount of pressure in the dam. For example, when using a hose to clean dirt from a surface, when the hose is hardly turned on, the flow is weak and slow. The water flow will not clean the surface very well because of the low amounts of moving energy. A hose turned on to its highest setting has larger amounts of water moving through the same sized hose at faster rates. This means the hose would have high amounts of moving energy and clean the dirt more efficiently.
The same applies to hydroelectric dams and how their turbines are propelled to convert the moving energy. The higher speed and pressure means more moving energy which impacts how efficient a dam will be based on size and height. When thinking about a dam with a small reservoir, the amount of water will not create a lot of pressure on the released water, meaning the moving energy is low and the dam will produce fewer amounts of energy. A large dam reservoir will have high amounts of pressure and thus a greater moving or kinetic energy \citep{khawam2006solid}.

\subsubsection{Potential Energy}

Similarly, the height of a dam impacts its energy potential. The higher a dam is, the more the water ``falls'' and the higher the potential energy is. Potential energy can be viewed as the amount of energy an object has or has stored, if moved out of its equilibrium position, so when there is a further height to fall from, the stored or potential energy of the water is greater. This potential energy based on height or gravity can be modeled with the equation:

\begin{equation}
PE_{grav}=m*g*h
\end{equation}

In this, the $PE$ grav is the potential energy, $m$ is the mass, $g$ is the gravitational strength. This is generally based on planet in which it will remain constant at about 9.8N/kg on Earth where N stands for a Newton which is the force needed to accelerate a mass of one kilogram per one meter per second per second. In this equation, $h$ is the height. The mass of water is also consistent at 1~kg/liter on Earth. With this in mind, the gravitational potential energy of water in the case of hydropower is based on the variable of height. The higher the dam, the greater the height, and the greater the potential energy \citep{thorin2014basics}.



\subsubsection{Total Mechanical Energy}

There is also a type of energy that is the energy of an object as the result of motion. In essence, it is the stored potential energy and motion energy when moved out of its equilibrium. A simple model for Total Mechanical Energy in dams that assumes there is no bounce of the object [water] in motion is modeled by the equation 

\begin{equation}
TME=PE+KE
\end{equation}

In this, TME stands for the Total Mechanical Energy, PE is the potential energy, and KE is the kinetic energy \citep{thorin2014basics}. This means that the most basic way to model the total mechanical energy, or total produced energy, of a hydroelectric dam is dependent primarily on two factors, height of the dam and size of the dam. This is part of the reason that mega dams are favored by development companies--they have larger reservoirs and greater heights which allows the dams to produce more electricity, and thus garner larger profits, even though mega dams require exponentially larger volumes and surface areas which requires more local communities to be uprooted and increased deforestation and other ecological consequences. 

\begin{figure}
  \includegraphics[width=\linewidth]{images/hydroelectric/TGDTME.png}
  \caption{This is a graph of the inflow and subsequent water levels of the Three Gorges dam from May to September in 2020. It demonstrates how flooding is both possible and frequent. As the amount of water going into the dam increases dramatically--the inflows--the dam's water levels also increase. This can be seen in each of the floods; for example, flood 2 increased the water levels dramatically on 7/19. }
  \label{fig:reftme}
\end{figure}

  
\subsection{Megawatts and Electricity Usage}

   Hydroelectric Dams produce an average of 42 billion kWh (kilowatt-hour) per year. This is enough energy to meet the needs of a developed residential area of about a million people, or 72 million barrels of oil \citep{bilgen2014structure}. To put this into other perspective of just one megawatt is enough energy to power about 10 cars, or 300 homes an hour. Furthermore, part of the reason hydroelectric dams is touted as so efficient is because unlike alternate sustainable energy sources such as solar or wind, hydropower can run at all hours of the day regardless of weather conditions \citep{butler2009headwinds}.

Another driver for the increase in dam construction--especially in Asia--is the growing electrical demand that is taking place. According to \citet{ascher_2021}, industrialization and the growing need for electricity in SouthEast Asia is growing at one of the fastest rates worldwide. Because of this, he argues that hydroelectric dams are a viable option for alternative energy sources away from fossil fuels to meet the growing demand in the region. One must ask where to draw the line in terms of growing energy demand. If SouthEast Asia specifically has such a need for energy, is it better to explore hydropower at the expense of rural and Indiegnous populations, or to continue using fossil fuels. Are other options beyond hydropower and fossil fuels viable to meet the growing energy needs of regions such as SouthEast Asia? In many rural areas of South East Asia, solar would be inviable--deforestation would need to take place to clear sufficient land for solar farms. Other alternatives such as wind may not be reliable or consistent and may not meet the energy demands of the regions such as cities in SouthEast Asia who rely on energy imports from rural areas from hydropower plants. This leaves options such as nuclear which present a whole problem as well in terms of where to put nuclear waste--the potential for nuclear energy in the region also has limits just as hydropower, solar, and wind \citep{ashourian2013optimal} and \citep{chua2011green}.



\section{Three Gorges Dam}

  Construction of one of the biggest hydroelectric dams, the Three Gorges Dam in the Yichang Hubei Province in China began in 1994 with the aim to prevent flooding, generate electricity, and to allow ocean freighter ship to navigate inland from Shanghai along the Yangtze river basin \citep{jackson2000resettlement}. With a production capacity of approximately 18,600--22,800 Megawatts of hydro powered electricity, the dam accounts for about 10\% of all of China's energy production \citep{xu2011impacts}.

  By 2012, the dam's 32 turbine generators were all operational \citep{fu2010three}. The dam's reservoir holds 5.85 trillion gallons of water and is estimated by the dam's beneficiaries to eliminate flood related damage for up to a century to protect human development on the banks of a river, or a riparian ecosystem. The dam produces about 84 billion kWh of non-fossil fuel energy a year \citep{wang2003three}. The water flow on the Yangtze has also deepened by the dam to allow ships to carry commercial goods to and from Shanghai to reduce shipping costs by 35-37\% and thus increase profits for large commercial enterprises \citep{wang2003three}.
The number of people displaced and affected by Three Gorges Dam is highly controversial. The Dam's developers estimates the amount of people displaced at less than one million but less biased estimates from researches such as \citet{jackson2000resettlement} have estimates at around 10.2 million in 1996, and up to 20 million in recent years. There are also estimates ranging up to 900 million people affected--though not displaced--due to disruptions in watersheds upstream or downstream of the dam \citep{kittinger2009toward}. 

\begin{figure}
\includegraphics[width=\textwidth]{images/tgd.png}
\caption{Usage Rights available for public use\\This is a map of the impacted towns along the Yangtze River that the Three Gorges Dam flooded or impacted. Chongqing is an unflooded urban center upstream of the dam that many displaced populations are forced to move to.}
\label{fig:tgdref}
\end{figure}

  The local region and population around the dam prior to construction was below the poverty line and reliant on agriculture with dense but patchily populated areas \citep{kittinger2009toward}. Literacy rates were lower than the national average. Upon construction of the dam, new townships were created to house the local displaced populations. Many displaced people were forced into urban centers for which they were poorly equipped to find work given their reliance on subsistence farming previously. Displaced populations were often forced into facotry and production jobs--often for foreign export items \citep{cernea1997risks} and \citep{cernea2004impoverishment}. 30,000 hectares of farmland that local populations once relied on were flooded. Water borne illnesses such as schistosomiasis and other diseases such as tuberculosis and pneumonia become more common in relocation areas due to reduced access to sanitation \citep{cernea2004impoverishment}.
  
  Furthermore, psychosocial impacts on the displaced populations were devastating. The complete forced shift away from traditional livelihoods as well as loss of cultural sites and disintegration of social support networks caused traumatic stress, stigmatization, depression, and inter-community violence as a result of the destruction. The Yangtze--a watershed involved in the dam--is known as the ``sorrow of China'' due to the millions of lives the dam project has destroyed both through forced displacement, and even death due to disasters such as dams breaking and flooding \citep{kittinger2009toward} and \citep{jiang2011larger}.
  
  \subsection{Hazards and Flooding}
  
  Another consequence of the Three Gorges Dam poses is the increased risk for disasters such as floods. Due to alterations in the Yangtze's hydrological patterns, sedimentation and saltation of the basin is also altered. As populations were forced to move, many of them moved to areas downstream of the Yangtze river into areas that are now considered floodplains due to the altered hydrology \citep{world2000dams}. Predictions estimated increased flooding and landslides as a result of the dam construction that has since been confirmed in the 21st century (See Figure~\ref{fig:reftme} of water inflows in 2020).
    
  Following the construction of the Three Gorges Dam in the 1990s, the amount of landslides in the region increased. Of the almost 6,000 kilometers of reservoir shoreline, 5\% of it was considered dangerous and at risk for landslides \citep{wu2005comprehensive}. With forced relocation, many people moved into the high-risk shoreline areas, putting them at huge risk for landslides. The increased risk of the landslides is due to the changed siltation and sedimentation which have the potential for loose rocks and sediment \citep{wei1989investigation}. 
  
  Another risk posed by dam construction includes flooding; the alteration of hydrological patterns in the region means that previous flood plains are no longer flooded, and new flood plains are established, sometimes in areas population with humans. Estimates from \citep{jiang2011larger} has documented up to 3,000 people having been killed due to flooding in 1998 as a result of the changed to the Yangtze River Basin due to construction. There also exists the possibility for dam collapse should an earthquake occur, although flooding and landslides are more common \citep{world2000dams}. 

\subsection{Displacement}
    One of the common costs of ``development'' is displacement of local and Indigenous populations--hydroelectric dams are no different. Due to the fact that flood plains are covered in water, the reservoirs themselves take up large amounts of space. In the case of the Three Gorges Dam--an area of over 400 square miles--local hydrology is often drastically altered as a result of dam construction, forced displacement is a common result of dam building. Displacement has further implications for even more construction and development for the displaced populations--new infrastructure such as hospitals, schools, roads, water irrigation, and electricity needs to be set up for displaced populations. Unfortunately, new infrastructure for forcibly displaced peoples often is insufficient or inadequate to sustain populations. Proponents of dam construction such as governments looking to switch to renewable energy and dam construction comapnies argue that the townships and urban centers displaced communities are sent to reduce poverty and encourage development. It is important though to note that this development is ``development'' in the eyes of western industrialization--a switch from agrarian lifestyles to industrial ones is not necessarily ``development'' but rather a change to a non--Indigenous lifestyle which is not necessarily better \citep{world2000dams}.
  
  One measurement of human well-being following construction of the Three Gorges Dam was done by \citep{fekete2010millennium} by the Millenium Ecosystem Assessment and World Resources Institute. They classify human well-being under five categories: 
\begin{enumerate}
  \item access to basic materials
  \item freedom and choice
  \item health
  \item social relations and social capital
  \item security
\end{enumerate}

Due to a number of factors, the MA assessment concluded that the Three Gorges Dam does not meet the requirements to be socially responsible and equitable. Prior to the Millenium Ecosystem Assesment, \citet{ascher_2021} also explains that the UN Human Rights Standards Assesmnet in 1988 concluded violation of human rights in the mega dam construction. It is important to re--emphasize that standards or guidelines by bodies such as the UN (See Table~\ref{tbl:UNrights}) provide a solid foundation for human rights, but there is no framework for implementation. While the 2020 MA and UN 1988 Assesments concluded human rights violations are results of the Three Gorges Dam Cosntruction, such assesments have no safeguards to prevent further damage. They are assesments of previous damage, not solutions or prevention measures for future human rights abuses. Furthermore, beyond some of the social impacts outlined by the MA assesment, another impact of mega dam construction is that of the spread of disease that accompanies dam development and displacement. Factors mentioned above such as poor infrastructure play a key role in the spread of diseases such as schistosomiasis. 

\subsection{Schistosomiasis}

Snail fever, or schistosomiasis is a disease from parasitic worms called schistosomes. Schistosomiasis causes abdominal pain, diarrhea, bloody stool, or blood in urine. Schistosomes typically live in freshwater snails which are vectors of the disease. The parasite comes out of the snail in freshwater and humans can become infected from skin contact with the contaminated freshwater \citep{verjee2019schistosomiasis}. Snails infected with schistosomes live in freshwater dams such as the Three Gorges Dam \citep{zhu2008three}. 
  
Schistosomiasis is found only in certain areas, making the virus endemic, but along with the construction of hydroelectric dams, snails commonly become vectors for the disease, and they bring Schistosomiasis from the endemic regions. The dams have the potential to increase the quantities of snails entering the region every year, making the disease more common \citep{zhu2008three}. Due to displacement of local populations due to dam construction, this often means influxes of people from schistosomiasis endemic zones enter greater populated areas and this may lead to an increase in infections near the flooded areas.

One of the dams in East Asia that experiences cases of schistosomiasis is the Three Gorges Dam. There is a narrow reservoir attached to the dam that is 1084 \kms which extends into the Jianhan Plain in the Hubei Province and the Chengdu Plains in the Sichuan province--both regions are endemic for schistosomiasis \citep{zhou2007epidemiology}.
  
  Dam construction throughout the globe such as Egypt, Senegal, and other continents such as South America caused increased cases of Schistosomiasis. African snails for example, have thin and lightweight shells that make their dispersal throughout regions especially effective and dangerous \citep{zhu2008three}. Following construction of the Diama dam in Senegal, a strain of infection from \textit{S. mansoni} snails occurred due to the stabilized water levels. Inadequate health care services in the area of the Diama dam led to an outbreak infection everyone in the city of Richard-Toll in the dam region \citep{waldstein1997schistosoma}. 

Despite differences in snail type, the Diama Dam serves as a cautionary tale for large dam construction such as the Three Gorges Dam. Healthcare for displaced populations is limited and due to numerous factors, the Three Gorges dam construction poses increased risk for snail dispersal, breeding, and survival. The most common type of snails in East Asian regions such as the Three Gorges Dam is the \textit{O. Hupensis} \citep{shi1990schistosoma}. For the purposes of this case study of the Three Gorges Dam and schistosomiasis, the ``snails'' referred to are of the \textit{O. hupensis} or \textit{S. Japonicum} variety.

\citep{nelwan2019schistosomiasis} and the Center for Disease Control observed that schistosomiasis cases occurred frequently in the Three Gorges area after the construction of the dam, with estimates of up to 100,000 cases of infection. If these cases return to their homes from hospitals, the presence of \textit{O}. \textit{hubensis} in the region could mean their co-existence can lead to increased numbers of infected snails, and later, increased cases in humans. In comparison to other areas of the world, proponents of the Three Gorges Dam argue that there is no correlation between cases of schistosomiasis and the dam, but as evidenced by \citep{zhu2008three} and other authors, The Three Gorges Dam may make the region more susceptible to both \textit{O. hubensis} and vectors for the disease, and thus increased cases among humans.

\subsubsection{Temperature}

  One factor that has the potential to draw snails to the region of the Three Gorges Dam is the ideal temperature for the snails. The compatible range for the snails was approximately 0.9\degree-21.1\degree C. The Three Gorges region's average temperature is 3.9-18.8\degree C which is not quite currently compatible with that of the snails. Despite this, because the dam has a greater heat capacity than naturally occurring bodies of water in the area, the average temperatures have the potential to increase in a range from 0.3-1.2\degree C depending on the season, making the climate changes more compatible with the snail's ideal temperatures which may make the region around the dam increasingly ideal for their presence \citep{wenxiang2006index}. 

\subsubsection{Breeding Patterns and Resettlement}

  As displacement of millions of local and Indigenous communities occurred due to the construction of the dam, migrants often resettled as close to their homes as possible away from the reservoir. This means that slopes or hilly areas near the dam are newly inhabited by farmers and the displaced populations, causing soil erosion in the farmed resettled regions, and it causes increased silt deposits in the Hubei region--recall this is a region endemic to the snails. This causes sedimentation and creation of marshlands in areas such as Hubei which are conducive to snail reproduction \citep{jian2000study}.  
  
Figure~\ref{fig:schis}

\begin{figure}
\includegraphics[width=\textwidth]{images/schis-life.png}
\caption{Usage Rights available for public use\\ This is an image of the lifecylce of schistosomiasis both inside, and outside of the human body. Ideal conditions such as temperature and large bodies of stagnant water enable hatching of the schistomes' eggs. The Miracidia in turn can infect snails in the reservoirs which can infect humans. Eggs develop into Schistosomulae (similar to larva/larvae in more common worm reproduction). The worms develop into adult schistosoma who reproduce within humans and can be excreted to begin the process again. Factors such as poor infrastructure or no access to clean water (common conditions in relocation schemes) encourage the infection of humans within this cycle.}
\label{fig:schis}
\end{figure}

\subsubsection{Snail Dispersal}

  Snails are distributed or dispersed throughout regions in two ways--actively and passively (Wu 2005). An example of passive dispersal is snails attaching to ships that move to new areas. Active dispersal can look like snails moving across regions via water currents. Passive dispersal does not necessarily have a limit to potential distance traveled, but through active dispersal, \citet{zhu2008three} observed a snail can travel distances of up to 168 kilometers. The closest endemic region to the dam itself is 299 kilometers and there have been documented cases on paper-making reeds that snails and snail eggs have lived on which carried the infected snails across large distances into paper factories in the region \citep{wei1989investigation}. Documented cases such as these imply that the snails could have made their way to the dam from endemic regions through passive dispersal such as the paper-making reeds, or even soil brought into the region that contains snail eggs. \citet{zhu2008three} suggests that the Yangtze River facilitates an increase of merchandise imports which in turn increases the chance for infected snails to enter through passive dispersal. 

\begin{figure}
\includegraphics[width=\textwidth]{images/schis-map.png}
\caption{This is a map of schistosomiasis in the Three Gorges Region. As evidenced by the yellow regions, very few amounts of the disease is under control in the region and there is a huge concentration of schistosomiasis just upstream of the dam itself near the flooded regions such as the Dongting and Poyang lake area upstream and east of the dam}
\label{fig:schis2}
\end{figure}

\subsubsection{Health Infrastructure}

  One of the reasons the risk for Schistosomiasis remains high in the Three Gorges Region is because relocated populations are moved into crowded urban areas which make the possibility for outbreaks and spread of disease higher. Furthermore, insufficient infrastructure such as hospitals, clean water, or proper sanitation, is availbale to prevent or treat the disease. Crowded living arrangements with multiple generations forced into crowded housing also does not aid either sanitation, or the prevention of the spread. For this reason, transmission of diseases such as Schistosomiasis is a threat and danger. Other diseases, water borne or otherwise, are also common, including but not limited to hepatitis, pneumonia, diarrhea, and a virus often spread through rodents--Hantavirus. 

\subsection{Eutrophication}


  Outside of Schistosomiasis and the spread of disease, another issue presented by the Three Gorges Dam and other mega dams is altered hydrological flows, increased nutrient sedimentation, and toxin buildup. Processes such as farming and factories often lead to runoff such as fertilizer runoff from farming and industrial chemicals from facotries. This runoff can have serious consequences for local populations because too much nutrients from runoff can be bad for bodies of water such as rivers \citep{glibert2005eutrophication}.

The Three Gorges Reservoir is fed primarily by the Yangtze River. There have been shifts in both water flow, and water levels in the Yangtze as a result and it disrupts the aquatic life, including but not limited to the plankton and microbes of the river \citep{sekiguchi1989case}. An excess of nutrients from runoff leading to dangerously low depleted oxygen levels in the water, a process called eutrophication, has occurred in the Yangtze and its tributaries and watersheds. Agricultural practices that were once not harmful are now causing the eutrophication due to the changed water levels and flows \citep{liu2012effects}.

  Furthermore, industrialization in areas of the Yangtze River watershed has occurred, due in part because of extending urbanization from other areas, as well as urbanization of the displaced populations due to the dam construction. For this reason, the nutrient runoff and toxin runoff associated with industrialization and urbanization has worsened \citep{wilmsen2018damming} and \citep{peng2016spatial}. Because of the excess of nutrients associated with eutrophication, there have been increased number of scum and deleterious--dangerous--toxins such as neurotoxins which are a threat to public health--this growth as a result of nutrient excess of nutrient loading is known as cyanobacterial algal blooms \citep{fristachi2008occurrence}. This buildup of toxins that effects the water quality is especially prevalent in the Three Gorges region because of the amount of subsistence farming that occurs. 
  
  Most of the local populations in the Three Gorges regions live rural agrarian lives based on subsistence. Populations along the Yangtze River that do rely on the subsistence farming are also hugely dependent on the river for a water source. As the water quality decreases with eutrophication and the buildup of toxins, the local populations that rely on the water and soil nearby are exposed to the toxins that are caused upstream due to industrialization. The water quality can impact crops and nutritional needs, but also daily life water usage such as washing and cleaning \citep{kravitz1999quantitative}.
  
\section{Social Systems and Infrastructure}
  As a result of millions of people being forcibly displaced, local communicties are faced with the question of where to relocate. Some populations move downstream or into regions nearby, maintaining some level of autonomy and subsistence that they had prior to their displacement. Despite having lost often culturally significant lands, this seems to be a better option for many than those who are forced into townships or urban areas such as Chongqing--an island city at the mouth of the Yangtze in Shanghai that has been growing in population and size where many displaced populations from the Three Gorges Dam were relocated to. (Figure~\ref{fig:tgdref}) In these cases, the displaced populations are forced into areas with existing political institutions which causes social unrest, resentment, and discontent among populations who were once primarily autonomous and maintained soverignty. 
  
  Within these townships and urban centers, there is insufficient living space or housing, education, health care, sanitation, job opportunities, space for subsistence farming/traditional careers, and ironically, energy or electricity. With the loss of 30,000 ha of farming land and government policies to discourage subsistence farming in the region--known as ``de-farming''--malnutrition has increased. Almost half of the resettled populations will not have access or resources to farm and provide subsistence for themselves, driving them away from agrarian jobs into industrial careers \citep{jim2006local}. Forced career changes can be detrimental to mental health and personal/individual agency when stripped of the ability to provide for oneself through traditional modes of subsistence \citep{jackson2000resettlement}. 

The World Health Organizations reported that similar psychosocial effects are results of displacement such as traumatic stress, depression, and increased violence as a result of the displacement from homelands. Furthermore, social and cultural networks that were once in place in communities collapsed following displacement in many cases. Communities were split up or simply altered so significantly that the social support that was once available has disintegrated. This too causes mental illnesses such as depression. With a lack of health care system for both physical and mental needs, none of the mental health issues are addressed at a health care level. 

Furthermore, the lack of availability of traditional food sources have led to inadequate diets that have documented cases of having caused chronic diseases and ailments \citep{fekete2010millennium}. It would be naive to group all populations impact by hydroelectric dam construction under the same universal or to assume they all rely on the same modes of livelihood, but a large majority of local or Indigenous populations in dam construction regions rely on agrarian lifestyles. Groups such as the community of Long Lawen and neighboring villages in Malaysian Borneo for example rely on agriculture, hunting, and fishing and this is the case with most other villages in numerous villages in East Asia near dams such as the Three Gorges or Bakun dams.
        A member of the Kayan Indigenous community who were former residents of Balui who were forcibly displaced for construction of the Bakun dam wrote about their subsistence before and after the dam.
    
\begin{quote}
``[Before] the change of lifestyle, we live very close to the jungle and we had a large area where we can hunt, rivers for fishing and we have everything in the jungle, not only food but also medicine. [Now], here [in Sangai Asap] there are 15 longhouses, 10,000 people. There are no rivers where to fish, there is no jungle where to hunt. And even if there were, the little fish or wild animals would end very quickly. Here we have no natural resource. This is the main problem.''
 \emph{Alex}
\end{quote}
        
   Elsewhere, Laos or Lao PDR--known as the ``Battery of Asia'' because of the number of hydroelectric dams in the country--relies on agriculture for up to 80\% of employment and almost a quarter of the country lives at or below the international poverty line \citep{costenbader2015drivers}. Using Laos as an example, the threat to subsistence for forcibly displaced or impacted populations. Rural populations in Laos rely on fish for anywhere between 47-80\% of their protein intake in their daily diets, often supplemented by carbohydrates and fats from other sources including, though not limited to, crops from farming. This means that when displacement occurs, or when watersheds are altered. Examples of such alterations are flooding to create reservoirs, eutrophication, or other alterations to water quality. As environmental watersheds are disrupted, so too is subsistence and traditional food sources are disrupted. This often leads to malnutrition and simply insufficient amounts of food. A case study that briefly illustrates this is that of the Ladakh people. Anthropologist and author \citet{norberg2000ancient} did a study of the Indigenous Ladakh people and found that with the introduction of dam construction and western development, the Ladakhi's shifted away from traditional subsistence--both forcibly and willingly. In her time with the Ladakh people, she noticed a shift in diet from the agrarian crops produced, to less nutritious options such as instant noodles which are increasingly popular as an affordable food source for Indigenous populations who no longer have access to farmland or fish. This leads to malnutrition as well as psychosocial impacts from a shift away from traditional nutritious food sources for Indigenous and displaced populations \citep{norberg2000ancient}.

\subsection{Cultural Sites}
  The aforementioned issues from spread of disease to infrastructure and food soverignty are potentially more easily considered in a traditional cost-benefit analysis but there are also losses that displaced populations that are invaluable. 
  Estimates of more than 140 villages were destroyed due the dam and it has been impossible to even count the number of cultural sites lost, as well as their value to communities \citep{liu2013chinese} and \citep{du1996university}. This loss of cultural sites such as traditional alters, ancestral lands, or other culturally significant landmarks, topographical features, homes, burial grounds, and other sites lost to the dam construction is incalculable. It is truly impossible to do an accurate cost benefit analysis when comparing the number of megawatts produced as opposed to the unquantifiable number of cultural sites lost and lives altered.
  Author \citet{ortiz2018indigenous} explains that ``physical land and landmarks become as familiar to us as a relative. If there were ever no more land, there would be no more existence''. In other words, many frameworks of Indigenous Reciprocity view land and cultural sites as invaluble--a price cannot be assigned to a culturally valuable region or homeland so while cost-benefit analyses from development companies can calculate changed literacy rates, availbility of food, altered jobs, or even increased cases of mental illness, the value of physical land lost cannot be calculated through western science or cost-benefit assesments. Indigenous Author \citet{kimmerer2018mishkos} wrote that science and western assesments are ``only one dimension in a decision--making framework that must also include values, ethics, and spiritual perspectives. Science is a suberb tool for answering true/false questions, but it does not have the capacity to address questions of right/wrong'' \citep{kimmerer2018mishkos}.

  

  Entire towns and temples have been flooded by singular dam development. The Three Gorges Dam for example submerged the White Crane Ridge--a hydrology inscription--, the original sites of the Zhangfei Temple--a temple that was removed by archeologists from its original location to prevent it from being completely submerged--, and Dachang--an ancient town that was submerged and had to be replicated/rebuilt downstream \citep{du1996university}. While the Three Gorges Dam is the largest mega dam and thus can be used to illustrate many of the issues of mega dam development, other countries such as Loas and Borneo also present serious and grave ecological and humanitarian consequences. 




\section{Vietnam and the Hoa Binh Dam}

  While countries like China do have mega dams, many other countries in East [and South East] Asia are heavily reliant on hydropower as an electrcity source. One of the country's in South East Asia most reliant on hydroelectric energy is Vietnam due to its Riparian ecosystem that makes it ideal for the dams. Between 1995 and 2009, 20 large scale hydropower dams with capacities of more than 100 Megawatts were constructed in Vietnam (Ty, and Westen, 2011). Most of the dams were built in areas populated by ethnic minorities with more than 144 dam investment projects in the Kon Tum and Dak Nong provinces-- rural areas populated with ethnic minorities \citep{dao2010dam}. 


    \begin{figure}
\includegraphics[width=\linewidth]{images/hydroelectric/vietmap.png}
\caption{This is a map of the Da River Basin in the Northwest of Vietnam. Both the Son La and Hoa Binh disrupted the region's watersheds}
\label{fig:refviet}
\end{figure}

  
  Dak Nong for example is located in Vietnam in the central highlands in a mountainous region with generally nutrient rich soils due to high levels of basalt which makes it ideal for farming and many of the people in the Dak Nong province farm coffee, rubber, cashews, sugar cane, and pepper trees (Figure~\ref{fig:refbasalt}). Around 85\% of the population lives in rural areas and it is one of the poorest provinces in Vietnam. Furthermore, the Vietnamese government made it a priority to bring electricity to Indigenous and rural communities in 2004 to reach 100\% of households by 2020 but these goals were not met. Instead, rampant dam construction has occurred at the expense of the Indigenous populations the Vietnamese government tried [and failed] to bring electricity to and they instead ravaged local ecology in the name of progress and energy \citep{thao2019assessment}.
  
      \begin{figure}
\includegraphics[width=\linewidth]{images/hydroelectric/basalt}
\caption{This is an image of the basalt red soils in the Dak Nong Province. 47\% of the province's land is used for agriculture such as rubber, coffee, tea, pepper, rice, and corn. The Basaltic Red Soil is rich in nutrients because of the high concentration of basaltic stone \citep{daknong}. }
\label{fig:refbasalt}
\end{figure}
  
  The dams in this area caused damage to forests and altered river streams and flow speeds, causing sedimentation and erosion which threatened the biodiversity of the rivers and surrounding areas. An impact statement regarding the sedimentation of the Hoa Binh Dam explained that the dam trapped up to 88\% of annual sediment which cut annual sediment delivery to the Hoa Binh delta downstream in half. The decrease downstream of the dam caused an increase in sedimentation in other river basins and watersheds, including the Haiphong harbor in Cam estuary, and the Van Maren connected to the Ba Lat coastal area. Both regions experienced alterations to sediment upstream and downstream of the dam \citep{vu2014impact}. Furthermore, the 200-kilometer Hoa Binh Dam displaced 58,000 people, 79\% of whom belonged to ethnic minority groups including the Muong, Tay, Dao, Thai Trang and Thai Den \citep{dao2010dam}. The dam destroyed 20,800 hectares of farmland, crop fields, and forests which were crucial to traditional production and subsistence practices of the rural poor in the area \citep{nguyen2013linking}. 
  
  The Muong people for example are the second largest ethnic minority in Vietnam and the largest ethnic minority in the Indochina region. They rely on agrarian farming with crops such as rice. They rely on local resources for subsistence including crops, hunting, and raising animals. They also engage with tourists and western economies to some extent as they gather wood and cinnamon for trade with those outside their communities. The destroyed farmland and crop fields were incredibly destructive for the Muong living in the region impacted by the Hoa Binh dam as it stripped necessary farmland and forests necessary for subsistence as well as products to trade \citep{trang2004local}.
  

  
  Of the 58,000 people displaced by the dam and surrounding destruction to the ecosystem, clean water, education, healthcare, and other basic human needs were unavailable. Up to 34\% of the displaced population did not have access to clean water, and ironically, 20\% of them did not have access to electricity--regardless of the fact that their displacement stemmed from exploitation of their lands to produce electricity for the national grid which they did not have access to). In 2009, poverty rates of the displaced population were 59-60\% compared to the national rate of 14\% \citep{vinh2014impact}. Despite this, it should be noted that the government has touted the overall resettlement project in a positive light because estimates from Vietnamese regulated news sources claim poverty was reduced from almost 74\% prior \citep{laodong}. While conditions still remain worse than the national average, it can be argued there is nuance that accompanies that decrease in poverty in relocation sites. 

\section{Long Lawen and Long Liam}
  Another case study that is important for understanding Indigenous sovereignty alongside the issue of hydroelectric dams is that of Malaysian Borneo, and more specifically, the villages of Long Lawen and Long Liam. Both villages are case studies in which the people protested or refused resettlement and dam construction and they found energy alternatives. This is not to say hydropower development has a ``happy ending'' but Long Lawen and Long Liam are arguably more optimistic case studies for the autonomy and advocacy in Indigenous communities.  

  To begin, Long Lawen set some model of an example for what is called micro hydro projects as termed by Headman Gara Jalong of the village. He has been fighting for many years to protect Long Lawen, often at the cost of imprisonment. In the 1990s, the Bakun company tried to forcibly remove the community of Long Lawen to build a mega-hydropower dam called the Bakun Dam. It is one of the largest dams in Southeast Asia to this day in the rainforest in the Belago District in East Malaysia in Sarawak on the Balui river. The village of Long Lawen lives in this area and their original location was in the way of dam construction. The dam is 14,170 \kms. For reference, is the size of the entire country of Singapore--this is an extremely large dam. It occupies 12\% of Sarawak and its main funder is a Chinese company. During construction of the dam, the company Bakun was practically uninvolved with the relocation scheme which they left to the local government in Borneo \citep{keong2005energy}. This relocation scheme required the removal of the Long Lawen village and was intended to remove them to a government-controlled site, stripping them of their cultural autonomy and land \citep{lee2015compensation}.  

  Gara Jalong, as well the community members refused the government's resettlement plan and the community moved independently in a safe area above the dam construction. While an imperfect solution--which already is flawed as Long Lawen should not have been presented with the requirement to move at all--this did allow Long Lawen to remain somewhat autonomous and independent as opposed to moving into the government-controlled resettlement location \citep{lee2015compensation}.
  
  Long Lawen still remains in the rainforest in longhouses with connected apartments for the community members which is the traditional style of residency, and it is not uncommon for the older generation to maintain many of the traditional practices such as stretched long ears, as well as traditional modes of livelihood such as creating forest products and hunting and fishing. Many members of the younger generation prefer western fashion or living. Most of the younger generation moved out of Long Lawen but they still visit annually.  

  In 2000, the Malaysian Senator Banie Lasimbang helped Long-Lawen build a micro-hydro turbine. The science of these micro-dams remains like that of mega-dams--a turbine utilizes gravity to turn and generate electricity--but a key difference is that this micro-hydro power installation was built by, approved by, and desired by the community of Long-Lawen \citep{doyle2015indigenous}. 

  Through Senator Banie's help, Gara Jalong and the community were able to construct a dam that did not displace them or alter their watershed to detrimental effects. One of the injustices of hydroelectric dams as mega projects by outside contractors is that the local and Indigenous communities they displace almost never receive the benefits of the generated electricity. This is the case with Long Lawen which made the micro-dam so appealing to bring ``cleaner'' and arguably better electricity to the village as it was advocated for by the community members themselves.  

  With that said, micro-hydroelectric dams do have limitations, including but not limited to the amount of energy they produce. For example, during the winter holidays, younger community members and extended family members of the Long Lawen community return home and during this time, the 4-6 kilowatts a day produced by the dam is insufficient for the village. The micro dam was supposed to produce up to 12 kw/day of electricity a day, but the actual numbers have fallen quite short of those estimates \citep{cooke2017limits}.  

  Despite the lower energy production than anticipated, the micro dams have been replicated in almost a dozen other villages in the region including Long Liam who also built a micro dam using Long Lawen as an example. The village of Long Liam also ran into issues with their dam, including technical problems. For example, Long Liam's micro dam is under--performing in electricity production because the pathway to the penstock is not straight which disrupts the strength of the water flow. The turbine is not spun with the efficiency that was anticipated and the dam typically produces less than half of the anticipated 14 kw/day \citep{choy2004sustainable}. 


  This is not an argument in favor of the mega dams, nor for energy sources such as fossil fuel, just an acknowledgement of the inadequacy of micro dams as a sole energy source for a village. Mega dams are so harmful in both the social aspect, as well as the ecological aspect that micro dams may in fact be a potential alternative to the alternative energy source of mega dams. 

  Author of the book Ancient Futures, Helena \citet{norberg2000ancient} argues for a transition from globalization and globally based economies/development such as mega dams, towards localization of agriculture, as well as energy production. She advocates for small--scale energy development projects such as micro dams, but explains that ``when it comes to small-scale projects that truly promote self-reliance, such as village--scale hydroelectric installations or solar ovens and water heaters for the household, the question is immediately asked: ``Can the people pay?'' Nuclear reactors and big dams are heavily subsidized, while small scale--technologies based on renewable energy receive no significant support from any major aid agencies'' \citep{norberg2000ancient}. There is a very fine line to balance between imposition of western technology and aid. Should Indigenous communities or villages desire such village-scale installations of hydroelectric dams or other energy sources, one must ask whether it is the obligation of global aid organizations to subsidize, or even fully fund such projects. 

  An important part of this desire can be explained by Author \citet{bodley2014victims}'s proposal for cultural autonomy for Indigenous nations. He says that its purpose is ``establishing the Indigenous peoples' right to self--determination and helping to secure the future of Indigenous cultures in concurrence with their own efforts and desires: to focus international attention on the contemporary situation of Indigenous peoples, to pressure governments to respect the internationally recognized rights of Indigenous peoples, and to provide financial assistance to Indigenous peoples in support of their self--determination struggle which would permit tribal peoples to choose the degree of isolation they want to maintain'' \citep{bodley2014victims}. These ideas of cultural autonomy for Indigenous community is key in the discussion of both mega and micro hydroelectric dams. Part of the reason most hydropower development so radically violates Indigenous rights is because it ignores land rights and human rights as well as the desires of the Indigenous communities and their right to make decisions for themselves. If village--scale hydropower is an avenue to further explore in terms of electricity generation, it must comply with ideas of cultural autonomy. In other words, any development, even at the micro scale, must permit Indigenous groups to determine the level of involvement and development they would want in their villages.  

\subsection{Conclusion}
Alongside the increase in SouthEast Asian energy demand, the question must be asked of whether we can afford to fuel the growing demand without hydroelectric mega dams. For example, \citet{begum2015co2} explains that the energy demand in Malaysia is increasing due to more industrialization and population growth. If alternative energy sources such as hydropower are not explored, the demand for electricity is not going to decrease. \citet{ascher_2021} argues that many SouthEast Asian countries are vying for global markets that rival western economies such as the United States. Countries such as Malaysia will not sacrifice energy consumption if it means a stop to development. Instead, fossil-fuel energy sources are used to meet the demand which has impacts greenhouse gas emissions. Knowing that energy consumption will continue unrelentingly, we must ask how to  meet demand with more sustainable alternative energy sources while still respecting Indigenous Autonomy and human rights.
  Because mega dams are far from ``clean'' both from an ecological standpoint, as well as from a humanitarian point of view, is it still worth it to continue construction and operation of them? The water--grabbing that occurs in the mega dam development results in clear winners and losers as big construction companies profit off the ecological degradation and human impact of the dams. The displaced populations face loss of culturally important homelands, as well as threats to social systems and modes of livelihood that are clear violations of cultural autonomy and Indigenous rights. We must ask if halting mega dam construction for good is feasible. Can we return and protect Indigenous Autonomy and agency while still finding renewable energy sources to combat the climate crisis and meet growing energy demand both in SouthEast Asia as well as across the globe. While alternatives to mega dam hydropower projects do exist, alternative options such as micro dams have smaller energy capacities and are unlikely to meet growing demand. Run of river dams also have less impact on water flows and local hydrology, but they can still cause sedimentation buildup and still displace local and Indigenous groups \citep{ascher_2021}. Even options such as geothermal cause displacement and destruction of land such as deforestation \citep{edelstein1995cultural}. The human rights abuses and environmental degradation from mega hydroelectric dams leave few choices in terms of sustainable energy options and so we must ask ourselves what we are willing to sacrafice in terms of sustainable energy. Reduction of energy demand itself is one possibility. Another could be serious divestment from fossil fuels and to turn those finances towards researching sustainable alternatives that also comply with environmental justice and human rights standards. While such alternatives would be ideal, it is also unlikely that such divestment from the fossil fuel industry will occur and so we are left with few options outside of hydroelectricity. 




\chapter{The Political Ecology and Economy of Rice in Myanmar}

\chapterauthor{Lucas Florsheim}

\section{Introduction}

  As climate change becomes an increasingly concerning threat to agriculture many countries throughout Southeast Asia are rushing to adopt new agricultural practices that subvert issues like flooding and rising temperatures. According to the IPCC, Southeast Asian food production could fall by up to 30\% By 2050 If certain agricultural adaptations are not implemented \citep{wassmann2004sea}.Although specific agricultural adaptations and solutions differ depending on climate and region, many scientists have outlined the general changes that will need to be implemented to sustain food production in the face of the climate crisis. 


Adaptive options for confronting issues related to increasing temperatures and hydrological disruption include incorporating crops that are heat-resistant and less vulnerable to salinity stress, improving water management, and altering crop management practices. Other potential options are implementing resource-conserving technologies (RCTs), diversifying crops, improving pest management, and creating better weather forecasts. \citep{wassmann2004sea}. While evidence shows that these methods will indeed work to sustain agricultural production in countries throughout Southeast Asia, many of these adaptations may not be feasible implementations for developing countries with complex political structure and history. 

Myanmar certainly fits the bill of having a convoluted government and political history. Since its independence from Great Britain in 1948, the country has fallen victim to several coups and has been ruled by its military for most of the last 70 years. Leading up to the Green Revolution in the 1960s, Myanmar was one of the top rice exporters in the world \citep{naing2008survey}. The deltaic regions of Myanmar had an advantage over other rice producing areas as lucrative deep-water or floating rice systems were routinely utilized. Following the Green Revolution, the country’s military government was not initially invested in implementing new agricultural methods and thus fell behind other leading rice exporters. Just as lack of infrastructure and varying priorities caused Myanmar to lose its spot as the top rice exporter in the latter half of the 20th century, these factors now actively inhibit the country from performing the necessary implementations in response to the agricultural issues posed by the climate crisis.

Myanmar has made tremendous progress towards democracy in the last 20 years. This progress culminated in a democratically elected parliament being sworn in 2016, with the elections of 2015 being the first openly contested elections since 1990. A democratically elected government is necessary to execute the sweeping agricultural adaptions and to introduce the infrastructure needed to combat the effects of climate change. Military goverments prioritize controlling the economy, which often means maintaining the status quo, even in the face of climate change. Earlier this year the military of Myanmar, led by General Min Aung Hlaing declared a national emergency and seized control of the country in the form of a coup de'etat. The military is now in sole control of the county's operations and infrastructure and has spent the last several months attempting to qual both peaceful protests and armed civilian resistance \citep{Goldman.2019}. The overwhelming majority of Burmese people are in opposition to military governance, resulting in daily protests since the coup took place. Since the military seized power on February 1st, over 600 civilians have died at the hands of the government, and thousands more have been injured and arrested \citep{Goldman.2019}.

Combatting the effects of climate change and ensuring food security in Myanmar through sustainable rice production is only possible if the country is politically stable. A history of political unrest and conflict inhibiting proper infrastructure and development indicates that Myanmar will struggle to adopt the agricultural adaptations that experts recommend to mitigate the environmental problems of the coming century. 

In this chapter we will examine the political economy and ecology of rice in Myanmar throughout history as well as the potential climate related threats to rice production. It is through this analysis that we will attempt to understand the implications of the history of rice production in Myanmar, as well as examine the current state of Myanmar’s government as it relates to the future of the country's rice production in the era of climate change. 

\section{Climate and Rice Production}

Being one of the largest countries in Southeast Asia, Myanmar experiences several different climates, varying greatly based upon region. The most northern parts of the country feature a humid sub-tropical climate with hot and humid summers and mild winters. The Northeastern part of the country experiences a temperate oceanic climate meaning warm summers without significant seasonal differences in precipitation. The rain-fed rice producing regions in the south of the country are in either Tropical Savanna or Monsoon climate zones. Tropical Savannah climate classifications are defined by having distinct and differing wet and dry seasons. Monsoon climates are very similar to Tropical Savanna climates, but experience significantly heavier rainfall. Precipitation and minimum and maximum temperatures not only show substantial variation across the country but are also greatly impacted by monsoons. 

\begin{figure}
\includegraphics[width=\linewidth]{images/myanmar/Image1.jpg}
\caption{Average annual precipitation in Asia, the rainfed rice-producing region of Myanmar is circled in red.}
\end{figure}



\begin{figure}
\includegraphics[width=\linewidth]{images/myanmar/Image2.jpg}
\caption{Climate Classifications of Southeast Asia, with the rice-producing region circled in red.}
\end{figure}

Temperatures are generally lower in the mountainous regions to the North, whereas the southern portion of the country is much warmer and wetter and therefore more suitable for rice production \citep{naing2008survey}. Ayeyarwady, Yangon, and Bago are the three major rice growing divisions in Myanmar, all of which are in the Southern part of the country. 

\begin{figure}
\includegraphics[width=.5\linewidth]{images/myanmar/image3newnew.jpg}
\caption{Map of regions in Myanmar}
\end{figure}

The Irrawaddy delta is responsible for 68\% of Myanmar's rice production \citep{brakenridge2017design}. River basins and deltas are the most ideal for rice agriculture, as fertile sediment deposited by water, known as alluvial soil, is naturally provided, and monsoon rainfal is prevalent in these floodplains. \citep{naing2008survey}. Rice cultivation in Myanmar relies heavily on deep-water rice production, the flooding of paddy fields that results in increased growth and higher yields.
 
\section{Agricultural Production and Food Insecurity}

Jobs in food production make up over 80\% of the workforce in Myanmar, 70\% of these jobs are in rice production. Despite relying on an agricultural economy, Myanmar has historically struggled to feed its population. In 2019, the Myanmar Ministry of Health and Sports conducted a random sampling survey that revealed that 1 in every 3 families in Myanmar is food insecure and is facing problems related to undernutrition \citep{hlaing2019chronic}. Because the majority of Burmese people work in agriculture, understanding the source of economic and food insecurties of these individuals is critical. In a survey that involved 22 NGOs in Myanmar, it was concluded that hunger in agricultural communities is heavily tied to farmer's limited access to markets, land tenure rights, and lagging and ineffective agricultural environments and technology \citep{elkharouf2019grassroot}. Many farmers live in isolated communities connected by dilapidated roads and do not own cars or have access to public transportation.This being the case, it is extremely difficult for farmers to reach sellers or traders. In addition, complicated land tenure rules often lead to disputes over agricultural land and leave loopholes for the government to seize farmland. Furthermore, the government's reluctancy to subsidize the costs quality seeds and proper agricultural technology has been economically detrimental to many farming communities. 


\section{Implications of Climate Change on Rice Production}

According to the Asian Development Bank (ADB), the effects of climate change are expected to result in a decrease in rice paddy yield across all Asian sub-regions of up to 20\% by 2050 \citep{sinnarong2019estimating}. A simulation using climate projections from Thailand's Development and Coordination Center for Global Warming and Climate Change, and the Southeast Asia START Regional Center, an environmental organization founded by the National Research Council of Thailand and Chulalongkorn University, created a regional climate modelling system that could be used to predict changes in precipitation and temperature and model the subsequent consequences on the economy. This simulation, which combined macroeconomic theory and crop modeling, revealed that the fluctuations in temperature and precipitation associated with the effects of climate change ultimately result in the instability of prices of rice and other crops across 73 provinces in Thailand \citep{sinnarong2019estimating}. Although the model was specific to a 20 km x 20 km region in Thailand, its implications can certainly be applied to neighboring Myanmar. While the unpredictability of temperature and precipitation pose significant threats to Myanmar's agricultural economy, extreme weather and sea-level rise are also expected to have a negative effect on rice production in deltas such as the Irrawaddy River Basin \citep{furuya2014impacts}. In the following sections the ways in which increasing temperatures, increasing variability in precipitation patterns, and other climate change related factors have been found to decrease average rice yield will be explored.


\subsection{Increasing Temperatures}
One effect of climate change that has a significant impact on rice production is increasing temperatures. In Myanmar, the warmest months occur directly before monsoon season, from March to June. During these months, the Southern coastal regions of the country experience temperatures well above 36\degree Celsius (C), which is beyond the critical limits of heat tolerance for rice (\citep{wassmann2004sea}). A study conducted by Dr. Shao-Bing Peng of Huazhong Agricultural University concluded that rice yield can decrease by up to 10\% for each 1\degree (C) increase in minimum temperatures during the dry-season, which occurs from October to May in Myanmar \citep{peng2004rice}. Rising temperatures are more significant in the winter months, lasting from November to February, than in the summer as it is expected that there will be a more prominent increase in minimum temperatures than maximum temperatures annually (\citep{wassmann2004sea}). 

High heat stresses rice crop during the most vulnerable periods of the plant's production cycle. The growth rate of rice increases linearly as temperatures increase from 22\degree C to 31\degree C, and then decreases significantly beyond that point (\citep{wassmann2004sea}). In the vegetative stage, the period of growth between germination and flowering, temperatures hotter than 35\degree C have an extremely negative affect on photosynthetic carbon fixation. At high temperatures kinetic energy increases and bonds within cellular membranes loosen which creates a reorganization of thylakoid stacks, or grana \citep{karim1997heat}. This heat-induced reaction can reduce photosynthesis by over 35\% \citep{oh2007effects}.

\begin{figure}
\includegraphics[width=.7\linewidth]{images/myanmar/Image4.jpg}
\caption{Comparison of an undisturbed (stacked) chloroplast and the swelling of the grana that is induced by heat}
\end{figure}

 During the reproductive stage the meristem, the tissue on the tips of roots and shoots prodcues new cells, begins producing flowers. It is during this phase when rice plants experience spikelet sterility because of high temperatures (R. \citep{wassmann2004sea}). Spikelet sterility refers to reduced fertility in florets, in this case caused by reduced pollen production as a result of increased temperatures. Regarding the ripening period, also known as grain filling, high temperatures can greatly decrease grain weight and size because of excessive energy consumption that occurs to meet increased respiratory demands that are associated with warm temperatures (R. \citep{wassmann2004sea}). As a result of negative effects that occur at multiple different stages in the production cycle of rice, rice yield can decrease by as much as 39.6\% if the crop experiences temperatures above 35\degree C \citep{xiong2017meta}.

\begin{figure}
\includegraphics[width=\linewidth]{images/myanmar/Image5new.png}
\caption{The effects of high temperature stress plant reproduction. Although high temperatures effects both male and female functions, studies show that high temperature predominantly reduces rice grain size by impairing the production of pollen \citep{wassmann2004sea}.} 
\end{figure}

\subsection{Sea Level Rise}

According to the National Oceanic and Atmospheric Administration (NOAA) from 2018 to 2019 global sea level rise (SLR) was 6.1 mm. Over the last 100 years, the average SLR has been 1.8mm \citep{chen2012climate}. SLR is projected to increase up to 5 m by 2100 as a result of melting glaciers, ice sheets, and ice caps \citep{dasgupta2009sea}. While this melting does not usually occur in tropical regions, the implications of SLR are experienced throughout Southeast Asia. SLR is associated with several environmental issues that are prevalent in Myanmar including coastal erosion, flooding, changes in groundwater quality and subsequent decreases in agricultural production \citep{chen2012climate}.
	
	For each meter of SLR, Myanmar is projected to lose 0.85\% of its rice cropland \citep{chen2012climate}.If long term estimations are correct, a 5-meter increase in sea level could result in a loss of over 6.5\% of Myanmar's available rice cropland and up to a 10\% decline in rice production \citep{chen2012climate}. Under these circumstances the price of rice could jump by as much as 22\%, which would have a massive impact on food security regionally in Myanmar, and worldwide \citep{chen2012climate}. According to the World Bank, Myanmar has a poverty-rate of 37\% (The World Bank, 2021). Moreover, people in Myanmar spend 60\% of their income on food (The World Bank, 2021). meaning the price of a staple food like rice rising could have disastrous implications for the average Burmese person.

\begin{figure}
\includegraphics[width=\linewidth]{images/myanmar/Image6.jpg}
\caption{Percentage of agricultural land lost under projected SLR for a number of vulnerable countries \citep{chen2012climate}.}
\end{figure}

\subsection{Precipitation and flooding}

Rice is predominantly grown in monsoon season in Myanmar, which lasts from May to October. Rice paddies require extremely large amounts of water and are often grown in rain-fed river deltas. Most years during monsoon season these rain-fed deltas produce as much rice as the dry-zones, mountainous and coastal regions combined \citep{myint2018myanmar}. Although monsoon rainfall is intergral to plentiful rice production, excess precipitation and flooding related to tropical storms can also be detrimental. For example, floods tied to monsson rainfall during a La Niña episode in 2003 led toa 14\% decline inrice exports the following year \citep{brakenridge2017design}. While rice cannot be grown without adequate levels of rainfall, heavy precipitation and extreme weather result in flooding which has proven to be an increasingly prevalent problem for Myanmar’s rice farmers over the last 50 years.
 
 There are two main types of flooding in Myanmar. The first type has to do with the way in which monsoons and tropical storms hit river deltas hard, triggering river plains to act as runoff generators \citep{kravtsova2009hydrological}. This first type of flood occurs when extra-local runoff from wetter parts of the delta upstream flows downstream onto drier floodplains \citep{brakenridge2017design}. In addition to extra-local precipitation flowing downstream, a second type of flooding occurs when local rainfall concentrates on naturally wetter lowlands, which are often the sites of agricultural fields  \citep{brakenridge2017design}. This is known as pluvial flooding. Pluvial flooding is often associated with large-scale rainfall events like monsoons, and other types of extreme weather as they are caused by high levels of precipitation on already saturated soil in floodplains \citep{brakenridge2017design}.
	
	Myanmar's unique geomorphology greatly influences precipitation and flood patterns throughout the country. Most of the rivers in Myanmar are classified as anabranching \citep{furuichi2009discharge}. This means that the rivers are made up of multiple channels with divided flows as a result of sediment build-up. As sediment accumulates in riverbeds, the flow of rivers is often diverted into smaller channels \citep{brakenridge2017design}. Over the course of a year these small diversions can result in entire segments of rivers to migrate over 100 meters \citep{brakenridge2017design}.  Accumulated sediment and diverted channels cause unpredictable river flow which can be a serious flooding hazard. 
	
	Throughout the 19th and 20th centuries river embankments and levees were constructed in an effort to protect agricultural land from river channel migration \citep{hajek2014river}. In addition to levees and embankments, reservoirs and dams also influence flood hazards throughout the country. When dams and reservoirs were being constructed throughout the 20th century, flood control and mitigation were not considered. Most of these dams were designed to have a spillway feature, a system in which excess water can pass through the dam in a controlled manner when the reservoir is full \citep{brakenridge2017design}. This spillway design is extremely susceptible to flash floods when the mechanism fails. In addition to issues with spillway dams, large hydroelectric dams also pose flood hazards \citep{gupta2012role}. These dams trap sediment during floods which in turn creates a sinking delta, a phenomenon that occurs when riverbeds erode due to a lack of delivery of sediment and sink, which increase the risk of flood hazards downstream \citep{syvitski2009sinking}. Innovations like levees and dams have had moderate levels of success, but when these structures are breached during massive floods, damage is worse than it would be otherwise \citep{brakenridge2017design}. These constructions do not allow for much local flood storage or local flood mitigation, so when these systems fail, they fail miserably. 

	While geomorphology and human construction certainly play prominent roles in causing flood hazards in Myanmar, factors such as deforestation, mangrove logging, delta subsidence and sea level rise have been identified as other main contributors to what scientists refer to as temporal flood trends \citep{brakenridge2017design}. In the past, flooding has been measured using stationary analysis, a type of flood analysis that relies on the assumption that variables other than time are nonfactors to streamflow \citep{bayazit2015nonstationarity}. In the last decade, this type of analysis has come under increasing scrutiny as scientist begin to acknowledge the significance of variables that are influenced by climate change. In Myanmar, deforestation, especially of mangrove forests, has destabilized coast lines and made deltas more susceptible to storm surges \citep{brakenridge2017design}, which are small tsunamis of seawater that flood coastal land \citep{seekins2009state}. Subsidence, the rapid sinking of land as a result of human activity, has made both coastal land and deltaic agricultural flood plains more vulnerable to flooding \citep{brakenridge2017design}. Futhermore, rising sea levels have increased the frequency of inundation. All of these factors influence flooding temporally, meaning they cause unprecedented stream-flow inflections and are greatly influenced by climate change.
	
La Nina, the cold phase of the El Nino-Southern Oscillation (ENSO) cycle, also heavily influences flooding in Myanmar. The wettest conditions and most extreme weather are often associated with La Niña years in Myanmar. In addition to La Nina, other large-scale weather circulations, defined as tropical or global changes in atmospheric and oceanic patterns and weather, affect precipitation and flooding in Myanmar. The Madden-Julian Oscillation (MJO) is a 30-90 day weather pattern associated with anomalous and intense rainfall in Myanmar \citep{brakenridge2017design}.  The Indian Ocean Dipole (IOD) has the opposite effect as it causes warm and wet conditions in the Western Indian Ocean but acts as an inhibiter to rainfall during monsoon season in Myanmar \citep{brakenridge2017design}. The Pacific Decadal Oscillation (PDO) is also important to consider as its cold phase induced nearly twice the number of tropical storms as its warm phase does \citep{brakenridge2017design}. Weather patterns and seasonal oscillations greatly influence rainfall, flooding, and agricultural planning. Preparing for and predicating these oscillations and relaying information to farmers throughout Myanmar has proven to be a difficult task. In the case of extreme weather events, lack of communication and infrastructure can prove to be economically detrimental and deadly. 



\begin{figure}
%\includegraphics[width=\linewidth]{images/myanmar/Image7.jpg}
\caption{Inundation along the upper Ayeyarwady river in the state of Kachin during normal winter flow (top), monsoon season flow (middle), and unusual flooding—usually tied to extreme weather-- bottom}
\end{figure}

\section{Cyclone Nargis}
Cyclone Nargis was a Category 3 tropical storm that struck Myanmar in May of 2008. Cyclones are low pressure storms that form in the Indian Ocean and are identical to the typhoons that occur in the western Pacific and the hurricanes that occur in the eastern Pacific and Atlantic \citep{seekins2009state}. The regions that were predominantly affected by Cyclone Nargis were the Ayeyarwady Delta, the Yangon region, and other parts of Southern Myanmar, which are all prominent rice producing regions in the country. Upwards of 140,000 people died, and buildings and croplands were devastated by a 4-meter-high storm surge, which is a temporary rise in sea level as a result of a tropical storm \citep{seekins2009state}. 

\begin{figure}
\includegraphics[width=\linewidth]{Graphics/Image8.jpeg}
\caption{Inundation as a result of Nargis (red), compared to typical winter hydrology (dark blue) and average maximum monsoon flood events (medium blue).}
\end{figure}
Nargis is significant to this chapter not only because it had a monumental effect on rice production, but also because it is a prime example of the failures of Myanmar's military government to protect the lives and economic interests of its residents.

  When Nargis first hit Myanmar, the storm surge flooded approximately 5,200 square km of coastal and low-lying land, contaminating thousands of rice fields with salt water. An estimated 404, 858 hectares of cropland were inundated by saline water \citep{seekins2009state}. The storm took over 24 hours to pass through and hit the country on the Western side first, before concentrating over the Southern deltaic region and moving North \citep{brakenridge2017design}. The nature of this storm path was a significant factor in the degree of destructiveness of the storm as the movement triggered a counterclockwise rotation along the shallow southern seabeds of the country, resulting in maximum winds up to 118 km/hour \citep{brakenridge2017design}.

\begin{figure}
\includegraphics[width=\linewidth]{Graphics/Image9.jpeg}
\caption{Storm path of Cyclone Nargis }
\end{figure}

 Just like any other major flooding event, population growth, deforestation and human construction were identified as long-term causes of the damage inflicted by Cyclone Nargis \citep{brakenridge2017design}. Although these factors are significant, it is important to consider the role of the government of Myanmar in permitting unsustainable dam construction and deforestation to occur, but it is even more important to analyze the country's disaster response, and to hold the military government accountable for their failures to protect Burmese people.
 
	In the days following May 3rd, so much chaos and death ensued that the government of Myanmar stopped counting deaths after the toll reached 138,000 \citep{brakenridge2017design}. Although government officials in Myanmar and weather experts worldwide could predict the areas in the projected path of storm, the government failed to communicate the severity of the storm or highlight evacuation routes to those living in the communities most affected in the coastal south \citep{brakenridge2017design}. Shortly after the cyclone touched down weak infrastructure was quickly overwhelmed. Storm shelters reached capacity and already damaged roads trapped residents in inundated villages \citep{brakenridge2017design}. Many of the deaths tied to Nargis occurred in the weeks following the storm as people in rural areas did not receive proper aid and experienced disease and starvation \citep{seekins2009state}.
	
	Cyclone Nargis did not impact all the residents of Myanmar equally. Poor agricultural workers and other disenfranchised groups throughout the southern parts of the country bore the brunt of the damage. According to the Post-Nargis Assessment, a project conducted by the Special Association of Southeast Asia Nations (ASEAN), 61\% percent of those who died due to Cyclone Nargis and its aftermath were women \citep{Macan-Maker.2008}. Furthermore, thousands of impoverished children were orphaned and had to relocate to cities where they were referred to as “Cyclone Kids.” \citep{seekins2009state}. Many of these children became subject to exploitation in new metropolitan environments. Young girls were frequently forced into prostitution and young boys were often recruited into the Tatmadaw, the Burmese armed forces. 
	
	In the aftermath of the cyclone, foreign aid and global responses to the disaster were met with hostility and ignorance by Myanmar's military government. When the United Nations called for over 187 million dollars in relief funds, the State Peace and Development Council (SPDC) responded that they would accept foreign aid but wanted absolute control over distributions of supplies and made it extremely difficult for relief organizations to get boots on the ground \citep{seekins2009state}. As time went on Myanmar's government introduced stricter restrictions on the aid coming into the country. Charity and relief organizations frequently ran into  roadblocks where military forces confiscated food and supplies \citep{seekins2009state}. The SPDC began facilitating propaganda to discourage organizations from interfering with their response to Nargis. The State-run newspaper “The New Light of Myanmar” declared that the residents of Myanmar no longer needed help obtaining food and supplies because they could get vegetables from the fields and fish and frogs from the creeks (the Irrawaddy, 2008).
	
	The corruption and overall inefficiencies that were prevalent in the process of delivering aid remained an issue for different relief efforts and government initiatives throughout the country. The SPDC assigned a government organization called the Union Solidarity and Development Association the task of cleaning up the physical damages to infrastructure in the effected regions. The association quickly devolved into a corrupt and harmful presence in many of the communities they were supposed to rebuild \citep{seekins2009state}. The task of removing debris and repairing structures was often left to be organized by community initiatives \citep{seekins2009state}. Over the following weeks the international spotlight on the crisis dwindled and food shortages and corruption continued. Rice was continuing to be exported out of the country even as thousands of residents of Myanmar were starving domestically  \citep{seekins2009state}.
	
	At first glance the occurrence of Cyclone Nargis may seem like a rarity. Large tropical storms usually do not make landing in the Southern regions of Myanmar, especially compared to regions such as Rakhine, which was hit by severe tropical storms 6 times from 1980 to 2000 \citep{brakenridge2017design}. This being said, an event as severe as Nargis is projected to reoccur increasingly frequently as the climate crisis worsens \citep{brakenridge2017design}. Nargis was extremely damaging to agricultural production in Myanmar, and rice farms in particular were devastated by the event. Much of the agricultural infrastructure that was constructed throughout the 1980s in an effort to improve rice yield was severely damaged or destroyed by the storm.  Construction projects supported by the World Bank and the Asian Development Bank that implemented drainage systems, polders and sluice gates were made all but irrelevant due to damage \citep{omori2020assessment}. Many of the embankments that lined man-made canals or polders were busted as a result of flash floods (Omori et al., 2020). Sluice gates, barriers that are used to control river flow were also overrun in the storm \citep{omori2020assessment}. The very infrastructure that helps sustain high rates of rice production in Myanmar is clearly extremely susceptible to unpredictable weather in this era of climate crisis. Robert Brackenridge, the founder of the Flood Observatory at Dartmouth College, claims that reinforced and sustained flood and sediment management will be necessary to prevent future violent floods in Myanmar \citep{brakenridge2017design}. When considering if this kind of sustained management is possible in Myanmar, one must understand the exploitative role of Myanmar's military government in the history of rice production in Myanmar.
	
	\section{The Modern History of Rice Production in Myanmar}

\subsection{The 19th Century--The Industrialization of Rice}

When the British arrived in Myanmar the rice industry in the country changed forever. Prior to British arrival in 1824, rice was cultivated mostly for home consumption and local trade \citep{siok2012rice}. Occasionally rice was exported to nearby regions in Southeast Asia such as Sumatra in Indonesia, but most of the rice was produced for domestic consumption. In the mid 19th century, rice was becoming a desirable good in Europe as it was a cheap staple food and could be used for brewing, bread-making and as a textile \citep{siok2012rice}. A large percentage of the rice being imported by European countries was sourced from the Carolinas in the United States \citep{siok2012rice}. When the American Civil War broke out in 1861, rice exports from the United States declined rapidly and the European market looked to Southeast Asia to meet the demand for rice paddy \citep{siok2012rice}. 

In an effort to take advantage of this newfound demand, the British launched largescale land reclamation projects with the goal of converting the forestland and flood plains in southern Myanmar into arable land \citep{fujita2016agriculture}. Along with land conversion, the British began building new flood control infrastructure mostly in the form of levees. This economic activity in the south was extremely attractive to farmers in Upper Myanmar, and many agricultural workers migrated south where there were ample job opportunities on farms and in busy ports \citep{siok2012rice}. In addition to farmers migrating down from northern Myanmar, immigrant workers from India came flooding in throughout the ladder half of the 19th century. This new immigrant work force was composed of laborers looking for jobs in port cities and rice mills as well as wealthier land-owning immigrants known as Chettiars. Chettiars came down to Myanmar primarily to work in banking and moneylending, however the British government remained firmly in control of export and the majority of trade operations \citep{fujita2016agriculture}. As a staple food, rice has always been extremely culturally significant in Myanmar. The industrialization of rice production in the southern deltas was transformative as it established rice production as the center of economic, political, and social life in the country.

\subsection{The 20th Century--Rice Production in an Era of Political Transition}

As more infrastructure was established and land was cleared for agriculture, rice production steadily increased throughout the first few decades of the 20th century. When the Great Depression struck in the late 1920s rice prices plummeted as many farmers could not pay off their debts and consequently lost land ownership \citep{fujita2016agriculture}. Revolutionary movements and anti-colonial student-led protests gained traction during these times of economic hardship. Tax protests and hunger strikes led by Buddhist Monks quickly grew to a national insurrection against British rule \citep{smith1991burma}.

 In 1937, Great Britain separated Burma from India and established Burma as an individual British colony with its own prime minister and national assembly \citep{aung2013history}. When World War 2 started Burmese leaders used the opportunity to attempt to strike a deal with the British: military support in return for independence \citep{aung2013history}. This agreement never came into fruition and when the Japanese promised Burma independence, the Burmese government formed the Burma Independence Army (BIA) and permitted the Japanese to occupy the region in 1943 \citep{aung2013history}. Although the Japanese declared Burma was a sovereign nation, the Japanese military remained in control of the country's operations and Burmese leaders became increasingly frustrated. Aung San, the figurehead of the Burmese independence movement and the leader of the BIA, shifted the army's allegiance to Great Britain and expelled Japanese forces just two years after permitting the occupation of the country \citep{aung2013history}.

In 1947 the British granted Burma complete independence, but the state of political stability that had defined the country for the previous 20 years continued. Several ethnic-based communist insurgencies ensued in the 14 years between independence and the creation of the Burma Socialist Programme Party (BSPP), which marked the start of the era of military controlled government in the country \citep{aung2013history}. The constant state of political unrest and the amount of damaged infrastructure from the second world war left Myanmar's agricultural economy devastated.

The 1950s were a transformative period for rice production in Myanmar as land use and land tenure were altered drastically. In 1953, the Agricultural Land Act was enacted, and the government seized any plot of agricultural land larger than 50 acres for redistribution \citep{fujita2016agriculture}. The act also prohibited the sale, rental, mortgage, and transfer of this new nationalized farmland, which essentially cemented the working class as agricultural laborers by preventing them from becoming tenant farmers \citep{fujita2016agriculture}. This program exacerbated a growing class divide as the government avoided redistributing land to those who had never owned land previously as they feared that these farmers would fail to produce sufficient quantities of paddy \citep{fujita2016agriculture}.

When the BSPP formally took control in 1962 they introduced an agricultural. This new procurement system was met with a level of resistance by some agricultural communities, but the BSPP responded by seizing the land from resistors and redistributing it to more cooperative farmers \citep{fujita2016agriculture}. The procurement system was largely ineffective as it caused rice exports to decline, and only by accepting foreign aid was the BSPP able to keep the rice industry afloat throughout the 1970s \citep{fujita2016agriculture}.

In the 1980s the gap between the procurement prices and market prices for rice steadily increased, culminating in large-scale farmer led protests that were part of the larger Democratic Movement of 1988 \citep{fujita2016agriculture}. Amidst massive protests the government briefly liberalized the markets which led to a monumental surge in rice prices \citep{fujita2016agriculture}. When the State Law and Order Restoration Council (SLORC) squashed revolution attempts and seized control of the country's operations, they swiftly reinstated the procurement system. The liberalization of markets followed by the quick reinstitution of the procurement system hit working class agricultural workers extremely hard. Although flawed, the procurement system was acting as a safety net of sorts for many farmers. When this safety net was suddenly taken away during the Democratic Movement, rice prices skyrocketed but wages failed to follow suit and actually declined \citep{fujita2016agriculture}.

Following the Democratic Movement of 1988 farmers lost out in 3 ways. First, the new SLORC, which later became the SPDC, decided they would no longer subsidize fertilizers, pesticides, and oil in the agricultural sector. Farmers now had to pay international prices for said goods without benefitting from the international demand for rice. Second, they continued to lose money as the procurement price remained well below the domestic market prices. Lastly, privatized rice export remained prohibited resulting in lost profits as international prices were substantially higher than domestic ones. The 1990s were not much better for rice farmers. Although, things appeared to get off to a better start when rice exports increased due to the introduction of the Summer Paddy Program (SPP) of 1992 which introduced new growing technologies such as double cropping in order to produce more paddy in the summer season. Many farmers preferred to supplement their rice crop income by growing beans and legumes in the summer and were reluctant to adopt the technologies and guidelines proposed by the SPP. Although the SPP increased rice production, it also contributed to the growing distrust and disapproval of government agricultural projects and programs that is still present in Myanmar today.

The exploitative nature of the procurement system continued through the late 1990s as farmers were prohibited from making any private deals before meeting the state quota \citep{fujita2016agriculture}. Farmer's fatigue of the procurement system became evident as the quality of rice declined severely. Myanmar's export prices for rice were 70\% of what Thailand was fetching, and 25\% of Myanmar's rice was classified as damaged compared to Thailand's 5\% \citep{fujita2016agriculture}. The quality of rice was so poor that many state employees would immediately sell the product as animal feed instead of relaying the goods to domestic and global markets \citep{fujita2016agriculture}.

\subsection{The 21st Century-The Slow Process of Liberalization}

	In 2003 the SPDC once again briefly suspended the procurement system and tested the waters by allowing the private export of rice on the condition that the state kept half of the profits of these private sales \citep{fujita2016agriculture}.  Just like in 1988, rice prices increased rapidly, and the procurement system was reinstated. In 2007 the government began allowing private rice export using a quota system that collected a 10\% export tax, a system that was much more successful than the 2003 attempt \citep{fujita2016agriculture}. Still the procurement system remained problematic as instead of acting as means of food security following the onset of Cyclone Nargis in 2008, Myanmar's government continued to export rice despite a massive food shortage \citep{fujita2016agriculture}.. In 2011, as a part of a wave of democratic reforms that were backed by the military government, the rice market was finally liberalized entirely. This decision was likely made due to narrow price differentials between export and domestic prices, which prevented rice prices from spiking at the time of liberalization \citep{fujita2016agriculture}. 
	
	In addition to liberalizing agricultural markets, the government took steps to both increase rice production and create better food security. The Vacant, Fallow, and Virgin Lands Act of 2012 highlighted unused land that was suitable for agriculture. At this time Myanmar had the most undeveloped land of any country in Southeast Asia, and the hope was that some of this land could be converted to farmland in order to increase economic productivity, even if that meant sacrificing conservation efforts \citep{fujita2016agriculture}. The government also created the National Buffer Stock Committee which launched township-level committees that purchased rice as buffer stock, which helped to stabilize rice prices \citep{fujita2016agriculture}. Throughout the 2010s the increasingly democratic government took steps to provide more stability to the agricultural workers who are the backbone of the economy. Although it is too early to tell the long-term effect that the 2021 military coupe will have on the rice industry, constant political turmoil and violence is damaging to the agricultural economy and the society that has developed alongside the rice industry.


\section{Conclusion}

Being the staple good in Myanmar, rice is of extreme cultural significance. The crop is eaten at practically every meal and more than half of the jobs in the country have something to do with the product. As detailed in the historical section, rice influenced people to immigrate. The crop created cultural and social hubs and is extremely tied to class. Rice is a wage-good, meaning its price has strong implications on the economy, wages, and the well-being of Burmese people \citep{fujita2016agriculture}. 

In his book "The Myanmar Economy" Koichi Fujita utilizes the term "rice trauma." Rice trauma as a concept refers to the trauma that is tied to the rice industry, trauma experienced by Burmese citizens and by members of the Myanmar military government. Rice existing as a wage-crop in Myanmar links the good not just to wages, but also to movements. Protests in the 1930s, the 1988 uprising, and social unrest that followed Cyclone Nargis were all explicitly tied to rising rice prices. Burmese citizens experience rice trauma when prices rise and the economy becomes increasingly unstable, and the government experiences rice trauma because they associate changes in the rice industry as having potential to start political movements. This idea of rice trauma is a prime example of the political economy and ecology of rice. Throughout the history of the country, rice has acted as a political vessel.

As the onset of the climate crisis continues over the next century, scientists have stated that agriculture needs to change. Evidence shows that crops will need to become more resilient and agricultural practices must strive to be increasingly sustainable as the world begins to deal with extreme weather and increased temperatures. While alterations to established agricultural processes may be relatively easily to implement in countries with extensive resources and stable economies, a developing Southeast Asian country such as Myanmar will struggle to build upon weak infrastructure and will face strong opposition in the form of corruption and political instability. 

The 2021 coup is evidence that while it is easy to highlight the agricultural technologies and reform that is needed, Myanmar's complicated political history and continued political instability will hinder its ability to adapt in the face of the climate crisis. However, if a democratic government is successfully established in Myanmar, other Southeast Asian countries may look to Myanmar to lead the way as the country can set a precedent by sustainably developing the large intact landscapes that show economic and ecological promise throughout the country. 


\chapter{Storm Modeling and Tropical Cyclones in the Philippines}\label{ch:Philippines}


\chapterauthor{Ian Horsburgh}



\noindent\fbox{\parbox{\textwidth}{\section{A Storm is Coming}
The day is Wednesday, November 6th 2013. In the Visayan Islands, the central islands of the Philippine Archipelago, the people prepare for a storm. Just days earlier, after a south Asian weather station had begun monitoring a low pressure area east of Micronesia, the storm had been named Haiyan, and been declared a category 5 super typhoon. While Hayian is expected to be the most powerful tropical cyclone ever to hit land, devastating tropical cyclones are not new to the Philippines. One rural Fillipino agricultural worker, Angeles Grefiel, speaks of how past storms have wiped out his crops, leaving him with no money to provide his children with a healthy diet. ``We have generations of children that have grown up without having proper access to the right types of food,'' says Grefiel, a sentiment echoed by Evangeline Aloha, a resident of Leyte Province in the central Philippines, who worries she will have no income if the harvest is wiped out by the storm \citep{valera2014sea}. As certain cash crops are easily wiped out by heavy rain, many farmworkers like Evangeline and Angeles are ``so vulnerable to disasters that when one strikes, it takes them further and further into that cycle of poverty,''\citep{valera2014sea} says a local social worker, also in Leyte Province. To see why certain crops are planted and how they make the country so much more vulnerable to tropical cyclones, we must take a look back on the history of agriculture in the Philippines.}}
  
\section{Colonization, Agriculture, and Cyclone Vulnerability}
Rural Filipino farmers are vulnerable to tropical storms in part due to their integration into worldwide markets. In particular, famine is often a product of the conditions surrounding access to food. ``Famine must be seen not as an absolute scarcity of food in particular regions, but rather as a loss of ones entitlements to food and/or the means of subsistence,'' writes \citet{warren2018typhoons}. As for Fillipino farmers, this loss of entitlement goes back to the colonial era. Throughout history, rice has been a staple crop in the Philippines. Due to the favorable and temperate climate, Fillipino farmers were able to harvest rice twice a year, while simultaneously planting root crops such as sweet potato. When typhoons ``created severe food shortages for those who grew rice, these root crops became...the 'refuge of the poor''\citep{warren2018typhoons}. However, in the nineteenth century, this dynamic began to change. 

\subsection{Spanish Colonization of the Philippines}
	Spanish colonization of the Philippines began in the 1500's, although major agricultural change, especially as it relates to globalization and integration into market economies started in the early 19th century. This began with ``the Spanish practice of rewarding the Catholic orders for their conversion efforts with land,'' which ``turned the church into the largest landlord in the islands''\citep{ventura2016small}. Spanish catholic landlords then divided up their land, and under this system, ``Inquilinos (tenant landlords) paid annual rents for lands they then subdivided among sharecroppers, who often further subdivided their portions, which would be worked by families living in a central town near the fields''\citep{ventura2016small}. This system was a major step towards the integration of Filippino farming into the global economy, due to the fact that ''as estates commercialized, they increasingly shared management with Chinese-Philippine mestizo businessmen,'' who had access to British capital \citep{ventura2016small}. When America assumed colonial control following their defeat of Spain in the Spanish-American war in 1898, there were a number of changes to this system, but integration into global markets continued. 

\begin{figure}
\includegraphics[width=\textwidth]{images/tropical-cyclones/cornfield.jpg}
\caption{A Philippine cornfield is left decimated after a tropical storm in 1899 \citep{warren2018typhoons}}
\label{fig:cornfield}
\end{figure}

\subsection{American Colonial Rule in the Philippines}
	At the  beginning of the United States' rule, the new leadership feigned effort to give farmers independent ownership over their land, but just two years later abandoned this effort, apocryphally citing lack of interest in this initiative by the peasants \citep{ventura2016small}. William Howard Taft, the civilian governor of the Philippines who would go on to become US president, then changed direction entirely, saying that ``easing the homestead law's limitations on corporate ownership to 2,500 acres . . . was a much better path to development.'' Following this reversion to a similar corporate control as implemented by Spain, ``prevailing inequalities of landholding and rural wealth during the Spanish period multiplied under US rule.'' \citet{ventura2016small} summarizes the issue, saying that the United States’ ``failure to establish independent homesteads was akin to other alleged shortcomings in hygiene and sanitation, education, and banking, thus justifying the US presence in the islands,'' and consequently ``ownership for large scale plantation agriculture.'' US Civil Service Advisor Roy Franklin Barton talks of the American reforms in the agricultural region of Ifugao, describing how the new ``availability of wage labor jobs'' makes way for ``the introduction of money into the province replacing the old rice currency,'' and thus ``the integration of a market economy'' \citep{klock1995agricultural}. Thus, Spanish and American policies in the Philippines transitioned the country from local to large scale plantation agriculture.

\subsection{Effects of Incorporation into Markets on Typhoon Vulnerability}
	
  This incorporation into large scale plantation agriculture and world trade had large impacts on the vulnerability of the area to natural disasters, such as typhoons ``There is persuasive evidence that peasants and farm laborers became dramatically more pregnable to natural disasters after 1850 as their local economies were violently incorporated into the world market,'' writes \citet{davis2002late}. ``The vulnerability of tropical agriculturalists to extreme climate events after 1870 was magnified by simultaneous restructurings of household and village linkages to regional production systems.'' In refuting claims that farmers chose to adapt to this new age agriculture because it provided a better life, Davis argues on page 289 that ``Recent scholarship confirms that it was subsistence adversity (high taxes, chronic indebtedness, inadequate acreage, loss of subsidiary employment opportunities, enclosure of common resources, dissolution of patrimonial obligations, and so on), not entrepreneurial opportunity, that typically promoted the turn to cash-crop cultivation.''
  
  This cash crop cultivation became the primary type of farming in the Philippines by the mid 19th century, and even farmers who still owned land were increasingly ``encouraged to plant cash crops of abaca, copra, tobacco and sugar, and were often forced to sell rice below market prices''\citep{warren2018typhoons}. 
By the 20th century,much of the rice still produced by the Philippines was exported to China, while poor Filipinos lived ``almost exclusively on imported rice, tubers and corn'' \citep{warren2018typhoons}. Thus, rather than the pre-colonial model of growing rice and root crops, a model that was fairly robust in the face of typhoons, farmers in the colonial Philippines who were encouraged to grow cash crops and buy imported rice, as well as those who now worked plantations for a wage, were left with no money and thus no food when cash crops were wiped out by typhoons. As many cash crops rely on long fertile growing seasons for harvest, and thus were much more easily disrupted by storms, ``subsistence farmers, who increasingly chose to cultivate cash crops, counting on a better standard of living, %\red{can you make a diagram of this? maybe not, just think about it} 
often found they had no visible means of sustaining themselves, because of the ‘economic predicament’ triggered by typhoons,'' writes \citet{warren2018typhoons}, again. Thus, in the shift from a diversified to monocrop economy, agricultural, and thus economic disaster became ``a fact of life'' In sum, as colonial powers such as the United States and Spain implemented policy that shifted Filippino agriculture toward cash crop cultivation within worldwide markets, and away from a local multicrop system, the country became more vulnerable to typhoons.


\noindent\fbox{\parbox{\textwidth}{\section{Ready or Not}
  As the supertyphoon Hayian approached the Philippine islands, the government scrambled to alert those in danger, but to various degrees of success. While some in danger, like Retche Ycoy, did not receive any warning--``We didn't expect this was going to happen. We were just sitting around in our house and the wind suddenly started''--others, like 62 year old taxi cab driver Eduardo did not realize what the warnings meant \citep{valera2014sea}. Recalling the situation, he said, ``What I understood was that there would be a strong wind. We never understood what a storm surge meant'' \citep{valera2014sea}. Others still, like Celina Camposano of Leyte were told to evacuate. While these responses vary from having heard no warning whatsoever to being evacuated and sent to higher ground, they all share one sentiment: their past experiences with tropical storms did not prepare them for what was about to come.
  
  From educating and warning those in danger to helping those affected find food and shelter, the federal Philippine response to tropical cyclones like Haiyan has many obligations. In this section, we will investigate what this response entails, what areas it is successful in, and how it can improve. While disaster mitigation can mean many things, we will separate it into post disaster recovery, and pre-disaster preparation.}}

\begin{figure}
\includegraphics[width=\textwidth]{images/tropical-cyclones/boat.jpg}
\caption{A Ship is washed ashore in Tacloban City \citep{valera2014sea}}
\label{fig:ShipTacloban}
\end{figure}


\section{Philippine Disaster Mitigation}
\subsection{Post-Disaster Relief}
  The Philippines has a number of programs aimed at disaster relief, as seen in figure ~\ref{reliefprograms} Never were thes programs tested more than in 2013, when Typhoon Haiyan, locally know as Typhoon Yolanda, struck the country. With 195 mph 1-minute sustained wind speeds, Hayian was the most powerful storm ever to make landfall at the time. With 6300 people dead, and a million houses destroyed or damaged, the Philippine government had much work to do to provide relief for those that survived the devastating storm.
  
  Immediately after the storm, the Department of Social Welfare and Development (DSWD) provided shelter assistance to many displaced households. For nearly 60,000 households, this came in the form of emergency shelter kits. Although the kits were helpful in providing emergency shelter, the number of families that received the emergency shelter assistance was limited due to an insufficient supply of kits delivered. \citep{bowen2015social}. In addition to these kits, DSWD helped to coordinate the delivery of 136,267 roofing solutions to help with roofs that had been blown off. Concurrently, other families were sent to shelters or bunkhouses, and by the 5th month of the response, many of the families who were originally living in emergency shelter kit housing were transferred to more substantial bunkhouses as well. However, this relief housing was not sufficient for all. Another program implemented by DSWD was Food for Work.

\begin{figure}
\centering
\includegraphics[width=\textwidth]{images/tropical-cyclones/reliefchart.jpg}
\caption{DSWD Relief Programs \citep{bowen2015social}}
\label{fig:reliefprograms}
\end{figure}


  In the DSWD's food for work program, ``Beneficiaries were given food packs in exchange for the provision of their labor to assist in the repacking and distribution of relief goods'' \citep{bowen2015social}. This both helped to expedite the processing and distribution of food packs, and employ/guarantee food for those who lost their job as a result of the tropical cyclone. As the needs of the workers moved beyond immediate shelter and survival, this Food for Work program was replaced with Cash for Work. In the Cash for Work (CFW) program, which continued long into the relief effort, jobs included ``loading/unloading of goods, repacking of relief goods, food preparation, sorting and inventory of damaged property, clearing of debris, coastal clean-up, and canal dredging, among other things'' \citep{bowen2015social}. By 2014, 15,188 people were participating in CFW, which ``helped to provide much needed additional assistance to DSWD relief programs on the ground, while providing beneficiaries with cash based assistance''\citep{bowen2015social}. Programs like Cash for Work, however, would not have been possible without existing cash transfer infrastructure.
  
  The DSWD, with the help of humanitarian organizations, capitalized on strong pre-existing social welfare programs, especially cash transfer infrastructure, to provide monetary relief in the wake of Hayian. In total, ``Four agencies alone in the inter-agency response distributed around US\$34 million, benefitting 1.4 million disaster-affected people''\citep{bowen2015social}. This money was distributed in various ways, with around 70\% of cash transfers being conditional (Cash for work, etc), and 23 percent being unconditional. Although this system works well, the Phillippine government should learn from and refine this cash distribution process for future disasters, as a number of issues in ``coordination leading to coverage gaps and duplication'' of funds were reported during the Hayian relief period\citep{bowen2015social}.%\red{maybe define the problem then tell how to solve?}
  In addition to government aid, local relief and community driven development  was key to post-Haiyan recovery.%\red{we could create a table of activities and their evaluation and recommended changes, make be better than to have all in text?}
  Community driven development refers primarily to the subsection of the DSWD called the National Community Driven Development (NCDD) program that operates primarily on local levels and was established in 2002 to help alleviate rural poverty, especially surrounding disasters. In addition to implementing general infrastructure, the NCDD is well poised for disaster relief due to its geographical breadth and ``has a well established network of community facilitators and community volunteers on the ground''\citep{bowen2015social}. Following Yolanda, the NCDD played a large part ``in the rebuilding/rehabilitation process'' of affected communities by taking on projects such as rebuilding roads, paths/trails, schools, flood/drainage control structures, water systems, and health stations. Thus, ``The Yolanda experience has also demonstrated the important role that community driven development programs can play in the recovery of poor and vulnerable people from disasters''\citep{bowen2015social}. Finally, the Core Shelter Assistance Program helped those affected by the storm find housing. 
  In an effort to build more secure and resilient housing, the DSWD implemented the Core Shelter Assistance Program (CSAP), which aims to establish permanent safe housing for the rural poor. Following emergency housing after a storm, the CSAP builds ``a standardized two-room structure that is built to withstand 220 kmph wind-speeds''\citep{bowen2015social}. As a testament to the durability of these shelters, a local Social Welfare and Development officer stated that ``The core shelter units built through the CSAP of DSWD are still standing even after the mighty force of Typhoon Yolanda...all 80 units built in 1991, 2000 and 2010 in Barangays Cansuso, San Marcelino, San Sebastian, and San Guillermo remain standing''\citep{bowen2015social}. Thus, this program provides robust housing that not only serves as relief for typhoon victims, but actually serves to mitigate future damage caused by natural disasters. 
  Overall, the Philippine government has established a fairly robust system for addressing typhoon relief, combining immediate food and shelter relief with one of the best social protection programs in the region, which helps many residents with both immediate survival and monetary subsistence. However, despite these systems to address relief, many residents still live close to the poverty line. As seen in the figure ~\ref{povstat}, 44\% of Philippine residents experience poverty at least once, and of those that do, 2 out of 3 are in and out of poverty, often triggered by typhoons like Yolanda. While the systems described above do well to provide relief following a tropical storm, thus making the poverty triggered by the storm only transitory, other adaptation and mitigation measures could help reduce the need for such extensive post disaster support as well as decrease the number of deaths immediately caused by typhoons. While some measures described above, such as robust housing, function in this way, there are a number of other measures that can be taken to reduce the effects of tropical cyclones.
\begin{figure}
\centering
\includegraphics[width=8cm,height=24cm,keepaspectratio]{images/tropical-cyclones/povertystatus.jpg}
\caption{Poverty status of Filippino citizens \citep{bowen2015social}}
\label{fig:povstat}
\end{figure}

\subsection{Pre-Disaster Mitigation}
  Disaster mitigation takes many forms, from long term prevention to recovery. As we have previously discussed recovery and post-disaster mitigation, which the Philippines has demonstrated great competency in, we will now talk about prevention, a sector which the south asian archipelago will need to invest in to mitigate damage caused by future storms. The first form this investment could take is protecting natural land features.%\red{is there a science to disaster relief that we could add?}
	While man made structures such as walls and barriers offer some protection and psychological reassurance, they are not an ideal long term solution to cyclone mitigation \citep{king2010disaster}. This stems from a few factors, the first of which being their high maintenance costs, which leads to neglect, and thus causes a dangerous scenario of false security, as was observed in hurricane Katrina in New Orleans in 2005. Another downside of man made structures is that if they are breached, they often keep water in, creating a ponding effect that ``severely constrains response and recovery''\citep{king2010disaster}. Instead of man made infrastructure, many researchers emphasize the importance of retaining natural land features.
	
	Natural features such as coral reefs, mangroves, and dune ridges ``are extremely effective in controlling storm surge flooding,'' writes \citet{king2010disaster}. While coral reels are able to absorb ``some of the power of tropical cyclone wind-generated waves and surges'' before they hit land, mangroves are crucial to providing relief as they are extremely resilient to tropical storms, and often provide shelter and safety to people and boats around them \citep{williams2007lesson}. While only 20\% of the once 500,000 mangroves in the Philippines remained by the early 1990’s, local and national authorities observed this effect of mangrove protection from tropical storms, and have planted 600,000 mangroves since 1996. In addition to typhoon protection, this has had other benefits such as improved fishing and ecotourism \citep{williams2007lesson}. \citet{williams2007lesson} notes, however, that although Philippine legislature in planting mangroves for cyclone protection could be considered a success story, enforcement is ``often wanting,'' and continuous efforts must be made to maintain the progress the country has made in this respect. The final natural feature that has been observed to help prevent typhoon damage is coastal dunes, behind which ``lagoonal wetlands absorb immundation''\citep{king2010disaster}. Unfortunately, these are in great danger, as ``coastal zones have been cleared, settled, and built over''\citep{king2010disaster}. Thus land use planning and legislature are crucial to maintaining, or many cases such as that of mangroves, rebuilding natural infrastructure to mitigate typhoon damage. In addition to infrastructure, education is imperative to disaster mitigation. 
	Mitigation measures have little to no impact ``if the people do not know the hazard risk'' or ``are unaware of evacuation routes, sheltering strategies, and appropriate response to warnings,'' as demonstrated by Eduardo and Maria \citep{king2010disaster}. ``Many people who may have been through a Category 1 or 2 cyclone have no awareness from that experience of what a Category 3 or 4 will do,''  writes \citet{king2010disaster}, again. As the pre-existing idea of what the storm will look like may contradict official reports, vulnerable populations must be educated on what different categories of storms mean, and what responses are appropriate for the divergent levels. Once these warnings are understood, a robust warning system is extremely helpful in preventing injuries and casualties due to storms. In sum, the Philippine government has created a robust post-disaster relief system that encompasses housing, food, and work. However, to mitigate the damage caused by future storms, the country must invest in protecting natural features such as mangroves, coral reefs, and coastal dunes, as well as strengthen education programs on tropical cyclones.
	
	%\red{Okay, I like this, but let's integrate into the text\ldots, perhaps this could be inerted into the areas where you talk about colonilizatoin and talk about long term land use affecting resilience of mangroves?? Maybe too much to force, but it might break up with history with some ecology in a nice way and demonstrate some vulneratility linked to colonial land use changes -- if that actually happened???}
		
\noindent\fbox{\parbox{\textwidth}{\section{A Changing Game}
  Compounding issues of lack of education on tropical storms is the fact that the nature of these storms is changing. Some residents, such as Maria Flora Orbong of Tacloban City understood what the storm surge meant, yet was still underprepared for storms of Haiyan's magnitude, remarking that ``We knew that a strong typhoon was coming but we didn't really expect the water [levels] to rise that high \ldots our neighbors evacuated but we thought we were safe. We were in the middle, surrounded [by cement houses]'' \citep{valera2014sea}. Less than an hour after the storm hit the Island, however, things were far from expected. ``The waves rose to six or eight metres (20 to 26 feet)...many people started to escape but the ships washed up and many people died,'' recalls Orbong. Celina Camposano of Leyte echoes the novelty of this storm, saying ``We've never experienced something like this before \ldots we've never had to evacuate before'' \citep{valera2014sea}. While some of this gap between expectations and reality is due to lack of education on cyclone threat, another reason locals were unprepared for what typhoon brought was that the nature of these types of storms are changing. To understand this change, we must look at how tropical cyclones work and how they are affected by climate change.}}
  
\section{Tropical Cyclone Modeling and Climate Change}
Tropical cyclones, referred to as typhoons when taking place over the Pacific Ocean, occur in southeast Asia primarily in the late summer and fall. Distinct from the monsoon season, which describes the prevailing wind that causes a predictable rainy season in the summer months, typhoons are isolated and severe events, sometimes accounting for more precipitation than the entire monsoon season brings. Typhoons typically form in warm equatorial ocean waters, as the warm air near the surface rises, leaving an area of lower pressure below. 
Soon, a cycle forms, with surrounding air moving into the lower pressure region before heating up and rising itself. Once the risen air cools off, it forms clouds, which are then fed into the cycle, as seen in figure ~\ref{basicdia}. Other characteristics often accompany this air and cloud flow, such as torrential rains and a storm surge which can elevate the sea surface 20 feet and cause widespread flooding. 

\begin{figure}
\centering
\includegraphics[width=8cm,height=24cm,keepaspectratio]{images/tropical-cyclones/cycloneanatomy.jpg}
\caption{Basic Functioning of a Cyclone}
\label{fig:basicdia}
\end{figure}

  To represent how these tropical cyclones operate, climate scientists often employ the Carnot Engine model, a theoretical thermodynamic cycle that provides an upper limit on how powerful a storm can be \citep{emanuel1987dependence}. To really understand this model and how it gives intuition on the effects of climate change on tropical storms, we take a look at the greeenhouse effect before digging into the physics of the Carnot cycle.

\subsection{The Greenhouse Effect}
  
   A quick aside on the greenhouse effect: You’ve probably heard the term ``the greenhouse effect'' thrown around, but if you forgot what it is or never learned, here’s a brief summary. Of the light radiated by the sun, 30\% is reflected by the clouds or surface and the rest is (mostly) absorbed by the earth. The earth then transmits energy up by radiation and convection currents, some of which are absorbed by certain elements in the atmosphere such as water vapor, methane, and carbon dioxide. These molecules are known as greenhouse gasses because they act on the climate as a greenhouse does on a garden, trapping heat in the atmosphere. For a more in depth reading on the greenhouse effect, see Chapter~\ref{ch:earthbudget}
   
   
  %%The carnot cycle uses this heat flow as an input to estimate the maximum power output that could be produced if the storm was perfectly efficient. Although this perfect efficiency is impossible, the carnot cycle is useful to give an upper bound on storm power, operating as described in figure $10.5$.

\subsection{The Carnot Cycle}
From a physics standpoint, the Carnot sysle really stems from the first law of thermodynamics, $\Delta U = Q - W$. Oberve the diagram below, where green gas mulecules are trapped in a canister by the orange piston. In this diagram $\Delta U$ is the change in internal engergy of the molecules, Q is the heat transferred in and out of the systerm, and W is the net work (energy transferred to or from an object via the application of force) done on the system. Basiclly, $\Delta U$ is how much the speed of the green molecules changes over the course of a process, Q is the heat that comes into the canister, and W is the work the green molecules do on the piston.
\begin{center}
\includegraphics[width=3cm,height=3cm,keepaspectratio]{images/tropical-cyclones/basemod.jpg}
\end{center}

Lets imagine a scenario that could happen to this system. If the gas was allowed to expand, pushing the piston upward, the gas mulecules would have more room and so they would bump into eachother less frequently, and thus move more slowly (bumping into eachoters speeds them up). As U is the sum of the speeds of the molecules, raising the piston thus causes U to decrease, and $\Delta U$ is negative. 

%As this motion, also known as the molecule's kinetic energy, decreases, and the system's internal energy U is given by the sum of the KE of all the molecules, this expansion of the champer results in a decrease in U, or a negative $\Delta U$. Similarly, if the piston is lowered, this compression will cause the mulecules to move more quickly and thus $\Delta U$ is positive. If there is no change in the speed of the green molecule, $\Delta U$ = 0. 
  
%Thus, the raising and lowering of this piston changes the pressure and volume of the gas. To model this change, we use a Pressure-Volume diagram where pressure in the champber is on the y axis and volume on the x. We will use this diagram, along with the diagram of the piston, to describe the Carnot cycle. 

  Now, imagine this piston-cylinder system comes in contact with a massive reservoir that is the same temperature, lets say $T_{1}$, as the gas molecules at point A:
\begin{center}
\includegraphics[width=3cm,height=3cm,keepaspectratio]{images/tropical-cyclones/T1.jpg}
\end{center}
  
  
  This reservoir ensures that the molecules in the cylinder remain the same temperature throughout any process, and so for any process with the reservoir present, $\Delta U$ = 0. Now, we allow the molecules to expand and push the piston up, just as we did before. This time, however, the cylinder is in contact with the reservoir, so even though there is more space, the molecules don't slow down. To make this to happen, the reservoir must transfer heat into the cyliner, keeping it at the same temperature. Therefore, even though the molecules bump intoeachother less frequently, they recieve energy from the reservoir and don't slow down. Thus, $\Delta U $ is 0, and so according to the first law of thermodynamics, $0 = Q - W_{AB}$ or $Q = W_{AB}$. Hence, the heat that the reservoir transfers to the cylinder to keep it the same temperature is equal to the work that the gas does on the piston in the process AB. This process AB is called isothermal expansion, as the gas remains the same temperature and expands.
  
\begin{center}
\includegraphics[width=8cm,height=24cm,keepaspectratio]{images/tropical-cyclones/AB.jpg}
\end{center}
$$\Delta U = Q_{AB} - W_{AB}$$
$$0 = Q_{AB} - W_{AB}$$
$$W_{AB} = Q_{AB}$$


  As this first process AB is the most import in the context of cyclones we will go quickly over the next few quickly (for those interested, chapter 44 of \citep{feynman1965feynman} explains all the steps of the cycle in detail). After isothermal expansion, the gas is allowed to expand again, but this time the reservoir is removed. As there is no heat exchange in this step and the gas is allowed to expand, it called adiabatic expansion. Next, a second reservoir is added and the gas is compressed (isothermal compression) before this reservoir is removed and the gas is again compressed with no transfer of heat (adiabatic compression) which leads the gas back to point A. 

  This model is aptly applied to tropical storms \citep{emanuel2006hurricane}. Looking at figure ~\ref{fig:stormcycle}, the mixtrue of air and water vapor undergoes nearly isothermal expansion as it moves from an area of higher pressure A to an area of lower pressure B at the storm's eye all the while in contact with a massive heat reservoir, the ocean. Note that in this step AB, $\Delta U = 0$ because the ocean is acting as a heat reservoir, so the heat transfer Q into the system is equal to the work W done by the system according to the fist law of thermodynamics. In the context of cyclones, this heat transfer takes the form of vaporization of water, which occurs due to the greenhouse effect: as the ocean must lose heat to the athmosphere to balance the absorption of ratiation from the athmosphere and clouds \citep{emanuel2006hurricanes}, water evaporates, with the heat transfer from evaporation being equal to the work the storm can do. Thus, a higher greenhouse gas concentration in the athmosphere will lead to the ocean losing more heat (by evaporation) to balance this absorption of greenhouse gasses, and therefore raising the limit on the work the storm can do (how powerful the storm is) on its surroundings.

  After reaching the center of the storm the mixture then turns abruptly upwards at the storm center and undergoes nearly adiabatic expansion rising to an area of lower pressure. Heat aquired from the sea surface is then radiated to space in step CD before the mixture undergoes adiabatic compression from D to A, completing the Carnot cycle \citep{emanuel2006hurricane}.
  
\subsection{Power Dissipation index and Climate Modeling}

\begin{figure}
\centering
\includegraphics[width=12cm,height=36cm,keepaspectratio]{images/tropical-cyclones/carnotcycle.jpg}
\caption{The Carnot cycle applied to a cyclone \citep{emanuel2006hurricane}}
\label{fig:stormcycle}
\end{figure}

All this physics on tropical cyclone modeling provides good intuition for why an increase in greenhouse gas concentration in the atmosphere could lead to more powerful storms, but to really quantify this change, we must look to climate simulations and statistics. To investigate one statistic that is particularly pertinent to quantifying cyclone power and potential destructiveness, we look to meteorologist Kerry Emmaneul’s work. In his 2005 article Increasing Destructiveness of Tropical Cyclones, \citet{emanuel2005increasing} noted that ``Basic theory,'' such as what we have looked at with the Carnot cycle, ``establishes a quantitative upper bound on hurricane intensity, as measured by maximum surface wind speed.'' Observing that ``the actual monetary loss in wind storms rises roughly as the cube of the wind speed,'' Emmanuel %\red{would giving a short biography of this person, create more interest in his/her work?}
created the Power Dissipation Index, which he defines as:
$$PDI \equiv \int_0^t \sf{V_{max}}^3 \, dt$$
A quick refresher in calculus: the integral represents the area under a curve, so in this case the curve would be a graph of the maximum sustained wind speed of a storm over time (Velocity cubed), and the integral of V3 would be the area under this curve starting at the beginning of the storm $\sf{t{0}}$ = 0, and ending at time t. This is shown by the 0 and t at the bottom and top, respectively, of the integral symbol. A simplification of a previous statistic that was problematic as it input data such as storm dimensions that was seldom recorded, \citet{emanuel2005increasing} notes that ``this [new] index is a better indicator of tropical cyclone threat than storm frequency or intensity alone.'' Because it does well to estimate the damage of tropical cyclones, only requires one input V that is ubiquitous in storm datasets, and is easy to evaluate, it is often used to represent tropical storm damage.
	Using PDI to gauge storm intensity, numerous studies conclude that tropical cyclone intensity will increase under climate change \citep{zhang2017response}, \citep{emanuel2013downscaling}, \citep{chen2021typhoons}. To apply physics to large scale climatological events, scientists use climate models, which divide up the earth's surface into grid cells, and use complex equations based on fundamental laws of physics, fluid motion, and chemistry to describe how energy and the materials within the grid move through it \citep{noaaclimate.gov}. A visual of this type of model is given in the figure below.

\begin{figure}
\centering
\includegraphics[width=8cm,height=24cm,keepaspectratio]{images/tropical-cyclones/climatemodel.jpg}
\caption{A visual as to how the climate is modeled \citep{noaaclimate.gov}}
\label{fig:climatemod}
\end{figure}

When a model is ``run'', scientists set the variables to certain predictable climate conditions, such as greenhouse gas concentration, and solve the equations for those conditions \citep{noaaclimate.gov}. The results are then plugged into the next grid, and so on, representing the passage of time. To test the models, climatologists run the models back in time, ensuring the results are similar to what has actually been observed before simulating future conditions.

Applied to tropical cyclones and climate change, these models allow scientists to make conclusions about how greenhouse gasses will affect storms. While the effects of climate change on storm frequency are unclear(\citep{emanuel2013downscaling}, \citep{chen2021typhoons}), multiple studies have shown an increase in tropical cyclone intensity due to climate change \citep{zhang2017response}, \citep{emanuel2013downscaling}, \citep{chen2021typhoons}. This finding is in consonance with what the physics of tropical storms predicted \citep{emanuel1987dependence}. Presented graphically, this can be seen by an increase in PDI in climate models set for expected greenhouse gas concentrations over the next century (recall that PDI is calculated using wind speed, one of the physical processes modeled in climate models). In sum, tropical cyclones can be modeled with the Carnot cycle to predict the maximim power output of the storm. This model predicts that an increase in greenhouse gas concentration will cause storms to become more powerful, a prediction that is backed up by climate modeling.

\begin{figure}
\centering
\includegraphics[width=8cm,height=24cm,keepaspectratio]{images/tropical-cyclones/pdigraph.jpg}
\caption{Predicted PDI of tropical cyclones in southeast Asia over the 21st century as modeled by \citet{zhang2017response}}
\label{fig:pdigraph}
\end{figure}


\section{Conclusion}
  In this paper we investigated the past and present of tropical cyclones in the Philippines to make predictions and assessments of how the country should go about addressing storms in the future. We started with colonialism in the Philippines, and saw how the Spanish and US facilitated a transition away from local diversified farming and toward a monocrop plantation based market economy made the country more vulnerable to tropical storms. We then looked at how the Philippines responds to tropical storms today, using Typhoon Hayian as a case study to show the various shelter and food relief programs the country has in place, as well as infrastructure like storm education and preservation of storm-mitigating  natural features that the country will have to improve in the future. Finally, we investigated what this future of tropical cyclones will look like by considering the Carnot model applied to storms, which gives intuition for why climate change might increase the intensity of typhoons. This Carnot-model-fueled intuition is then confirmed by climate modeling which uses complex physical equations to conclude that climate change will indeed cause more powerful storms in the decades to come. Putting these sections together, the history of the  Philippines compounded with its geographical location makes it very vulnerable to tropical storms, but extensive measures, both pre and post disaster, can be taken to mitigate the damage done by these storms, which are getting more powerful with climate change. 

%\red{GREAT chapter, I love the physics and discussion of Carnot cycle, I think we might back up and teach some thermodynamics as another "interuption" into the history of the phillipines, I have some ideas of you want to think about it}


\chapter{Climate Change Adaptation and Infrastructure in Vietnam}

\chapterauthor{Jay Scott}

\section{Introduction}

As a low-lying, coastal nation with heavy dependence on its two river deltas, Vietnam is a country with severe risk factors for climate-related disaster. Even without the added effects of sea level rise, Vietnam frequently experiences typhoons during its wet season, at an average of 4-6 times each year \citep{scff2009climate}. Current dike systems aren’t strong enough and their effectiveness will only worsen with increased storm surges \citep{garschagen2011urban}. An increase in runoff could have a catastrophic impact on rural rice economies, with an estimated reduction in yields of 12\% and 24\% in the Mekong and Red River Deltas, respectively \citep{evers2018adaptation}. Rural residents rely on the rivers as their main source of drinking water, and both rivers are at risk from the construction of hydroelectric dam projects, saltwater intrusion, and increased demand for irrigation \citep{evers2018adaptation}. Vietnam’s urban population is steadily on the rise as well, growing from 20\% of the population to 30\% from 1985 to 2009 \citep{margulis2010economics}. This number is expected to continue to rise, as people migrate to Vietnam’s cities for economic opportunities, with estimates expecting cities to account for 57\% of the country’s population by 2050 \citep{garschagen2011urban}. This unprecedented increase in Vietnam’s urban population has the potential to overwhelm local governments, which have struggled with a simultaneous decentralization and tight control from Vietnam’s federal government \citep{garschagen2011urban}. Additionally, Vietnam does not guarantee its citizens the right to free speech, making community input virtually nonexistent in environmental policy \citep{nguyen2015deltaic}. While climate change is a new issue for Vietnam, it is a country that is uniquely adapted to floods \citep{nguyen2015deltaic}. The future of infrastructure the country will either build on this history, or forge a new path as Vietnam seeks to improve its standing internationally through economic development. 

\section{Climate Change Impacts on Vietnam}

\subsection{Flooding}

Most of Vietnam has a wet and dry season, bar the northernmost regions of the country . Unlike the four distinct seasons experienced in other parts of the world, Vietnam’s close proximity to the equator means its temperature rarely fluctuates, making the idea of “summer” and “winter” inadequate to describe conditions. “Wet” and “dry” are used as descriptors instead, and the seasons directly relate to the sun’s position over either the Northern or Southern hemisphere \citep{centerforscienceeducation}.

During the wet monsoon season, the warm air holds more water droplets, and flooding occurs along the coast and river deltas. Flooding is something that Vietnam has been adapted to over centuries. In the Mekong River Delta, there are a variety of housing types adapted to floods, including boat houses, floating houses, and stilted houses \citep{nguyen2015deltaic}. Farmers almost exclusively planted rice resistant to floods until the 1990s, when funding from the World Bank helped build dikes in order to produce a second rice crop during the rainy season \citep{nguyen2015deltaic}.

While these adaptations have proved sufficient in the past, they may not be enough to protect against current and future conditions. A 2018 study found that 33\% of Vietnam’s population is currently exposed to a 25-year flood, and cautioned that climate change could increase this number of exposed to up to 46\% \citep{bangalore2019exposure}Figure~\ref{fig:floodrisk}

\begin{figure}
\includegraphics[width=\linewidth]{images/Vietnam/floodriskovertimemap}
\caption{This map shows the number of people impacted currently in the case of a 25-year flood on the left, shows a future number of people impacted on the right. \citep{bangalore2019exposure}}
\label{fig:floodrisk}
\end{figure}

Flooding can have a profound impact on health. Incidences of drowning are relatively low compared to indirect health effects of flooding, such as diarrheal and skin diseases \citep{who2004report}. Because floods can decrease access to clean water, families respond by either washing foods less, or using subpar sources of water, hence the effect on disease \citep{who2004report}. In addition, Commune Health Services (CHS), an important part of Vietnamese healthcare, can become damaged during floods, worsening epidemics \citep{who2004report}.

In urban areas, floods have a major economic impact, shutting down roadways and preventing people from leaving their homes \citep{margulis2010economics}. 

\subsection{Drought}

In addition to its wet season, Vietnam has a long dry season that is expected to become even more dry with the addition of climate change. While drought risk is everywhere, it is especially concentrated in Vietnam’s mountainous regions \citep{lohmann2015effect}. 

Drought has a negative impact on human health, that is especially pronounced amongst children and young girls in particular. A 2001 study found that children aged 12-24 months during a drought were an average of 1.5-2cm shorter than children born during average conditions \citep{lohmann2015effect}. A separate 2009 study found that women born during years with higher rainfall were taller and had higher academic achievement than those born under average conditions \citep{lohmann2015effect}. This suggests that there are long-term effects of drought on children that continue after the rains return. 

Many rural homes are constructed of highly flammable materials, such as the aforementioned stilted houses constructed using melaleuca trees, and fires can spread quickly during the dry season \citep{margulis2010economics}. 

Vietnam’s most important crop, rice accounts for 47\% of all agricultural production and is very water intensive \citep{margulis2010economics}. As a staple crop, many rural households' economic stability is highly dependent on the year’s harvest. Evidence shows that in areas most affected by the drought, yields dropped 40\% under drought conditions \citep{lohmann2015effect}. Those who plant successful rice crops that year benefit from rice’s higher selling price, but there is a net negative effect on rice growers \citep{lohmann2015effect}. Aside from rice, aquaculture, specifically of catfish and shrimp, is important to rural economies and drought poses a risk to their cultivation \citep{margulis2010economics}. 

\subsection{Sea Level Rise}

Vietnam’s long coast makes it particularly susceptible to the consequences of sea level rise. The coastline has been rising at about the global average of 3mm per year \citep{huong2013urbanization}. If this rate is stable, Vietnam is expected to experience 75 cm of sea level rise by the end of the 21st century \citep{monre2010climate}. This will have wide ranging effects, one of the most damaging being saltwater intrusion \citep{hens2018sea}. Salinization occurs when sea water, with it’s high salt content, vertically infiltrates through soil and contaminates underground sources of fresh water \citep{hens2018sea}. A higher sea level will bring seawater higher up the water table, causing this effect. Figure~\ref{fig:saltwater}

\begin{figure}[h!]
\includegraphics[width=\linewidth]{images/Vietnam/saltwaterintrusion}
\caption{This graphic by the National Environmental Education Foundation visualizes the process of saltwater intrusion. \citep{bradford}}
\label{fig:saltwater}
\end{figure}

A 1 meter rise in sea level is estimated to affect 11 percent of Vietnam’s population and 5\% of total land area \citep{scff2009climate}. To combat this, the government has invested 280 Million VAT into building and fortifying sea dikes \citep{scff2009climate}. While sea dikes can be important elements of adaptation strategy, especially in Northern Vietnam which does not have a long history of floods, this strategy is not always suitable \citep{nguyen2015deltaic}. In his dissertation “Deltaic Urbanism or Living With Flooding in Southern Vietnam”, Phuong Nga Nguyen argues that the ideology of nation building, along with the Vietnamese’ government’s interest in increasing the productivity of rice, has been a contributing factor in the construction of dams in the Mekong Delta \citep{nguyen2015deltaic}. Nguyen believes the communities along the delta are well-adapted to flooding and don’t require much in the way of physical barriers. 

A rise in sea level could increase the severity of floods \citep{huong2013urbanization}. When sea levels rise, the area flooded can creep inland \citep{hens2018sea}. This exposes areas that previously weren’t exposed to flooding and therefore less adapted to its effects \citep{hens2018sea}. 

\subsection{Urbanization}

As climate events, economic conditions, and other factors force people out of agricultural villages, more Vietnamese are migrating to cities \citep{margulis2010economics}. Generally, urbanization increases flood risk because it concentrates the population into small areas and forces quick land use changes \citep{huong2013urbanization}.  The growing proportion of urban Vietnamese poses an issue for infrastructure already vulnerable to weather events during the monsoon season. 

For example, the rapid development of former wetlands in Ho Chi Minh City (HCMC) has led to poor drainage, exacerbating flooding brought on by storms \citep{vachaud2019flood}. These areas, in particular Phu My Hung and Thu Thiem, are considered undesirable places to live and are occupied almost exclusively by poor migrants, creating environmental inequality \citep{vachaud2019flood}. This issue and others will be discussed later in this chapter in the section on Ho Chi Minh City.

The urbanization of rural areas can often damage local aquifers as new residents drill for drinking water \citep{margulis2010economics}. The urban poor are one of the most vulnerable populations to climate disaster as they often have substandard housing, and rely on jobs in the informal economy that come with varied levels of stability \citep{margulis2010economics}. 

\section{Current Adaptation Plans and Policies}

\subsection{Strengthening Barriers and Existing Infrastructure}

The World Bank and United Nations Development Program have both allocated funds to improve existing physical infrastructure in Vietnam. In 2009, the UNDP funded a 180 million dollar project to enhance climate infrastructure, with 168 million specifically dedicated to erecting barriers like seawalls and dikes \citep{scff2009climate}. 

The United Nations, the World Bank, and Vietnam’s federal government, all prioritize physical infrastructure, in their approach to climate adaptation. While physical infrastructure is crucial in many areas of Vietnam, large-scale projects such as dikes and seawalls can have the effect of evicting the poorest and most vulnerable residents of a community. In HCMC, new plans for a ring dike around the city could displace as many as 1500 people \citep{yarina2018}. Historically, small canals were used to redirect water to the Saigon river, and the city’s residents took advantage of flooding with small rice crops and aquaculture operations \citep{yarina2018}. This less invasive approach to infrastructure is important and should be taken into consideration by governments and NGOs. 

\subsection{Implementing Effective Policy and Encouraging Collaboration}

In addition to improving physical infrastructure, the UNDP also allocated funds to an exhaustive review of existing environmental policy, especially in rural coastal communities. The UNDP identified several social obstacles to effective climate policy in Vietnam \citep{scff2009climate}. 

Vietnam’s climate change policy is handled by three separate government agencies: the Ministry of Agricultural Development (MARD), the Ministry of Construction (MOC), and the Ministry of Natural Resources and the Environment (MONRE). Historically, there has been a lack of collaboration between the three agencies. Construction of climate-resilient infrastructure is handled by MOC, while natural disaster relief is the responsibility of MARD \citep{scff2009climate}. Climate change preparedness and mitigation is under the scope of MONRE \citep{scff2009climate}. These three agencies work, for the most part, independently of each other \citep{garschagen2011urban}Figure~\ref{fig:mocmonremard}. In addition, local MOCs and MONREs exist in each province that are under the purview of provincial governments, not the federal MOC and MONRE. These smaller bureaus often work independently of each other with only minimal communication between them and the larger federal ministries \citep{garschagen2011urban}.

\begin{figure}
\includegraphics[width=\linewidth]{images/Vietnam/mocmonremard}
\caption{This organizational map explains the structure and purpose of the three separate federal agencies responsible for climate infrastructure and adaptation.}
\label{fig:mocmonremard}
\end{figure}

The United Nations Development Program identified that institutional knowledge of climate change was lacking, and that administrators were somewhat unwilling to integrate climate into their policy and operations \citep{scff2009climate}. Local governments were also noted as being indifferent to climate change, not seeing it as a larger threat than already common monsoons and other extreme weather events \citep{scff2009climate}. 

\subsection{Ecosystems Based Adaptation}

Ecosystems Based Adaptation, also known as EBA, is an approach to climate adaptation that prioritizes strengthening existing ecosystems. In Vietnam, this usually means strengthening ecosystems to protect against floods, landslides, and land degradation \citep{nguyen2017integration}. 

Mangrove forests have historically provided protection during storm surges and their revitalization could be a key part of Vietnam’s EBA. Additionally, forest conservation can aid in retaining soil nutrients and prevent landslides \citep{nguyen2017integration}. 

EBA often comes with co-benefits that can be economic, sociocultural, and promote biodiversity \citep{nguyen2017integration}. Despite the effectiveness and affordability of EBA, it is often overlooked in favor of new physical infrastructure \citep{nguyen2017integration}\citep{nguyen2015deltaic}. A lack of coordination between provinces and between the aforementioned MONRE, MOC, and MARD can make it difficult to effectively integrate EBA into policy \citep{garschagen2011urban}.


\subsubsection{Three Facets of Adaptation Policy}

The Vietnamese government has outlined three approaches to climate change: full protection, adaptation, and withdrawal \citep{monre2010climate}. Full protection is the use of physical infrastructure to completely insulate an area. This is seen as an option for important economic centers in cities or cultural landmarks, but can often exclude the poorest residents of these cities \citep{nguyen2015deltaic}\citep{yarina2018}. Adaptation is the prediction and acceptance of some climate-related losses, and the design of new systems compatible with a changed climate. Adaptation is important in the agricultural sector, as farmers find solutions to integrate climate change into their practices \citep{scff2009climate}. Withdrawal is complete avoidance of climate events by vacating an area extremely at-risk for climate impacts \citep{scff2009climate}\citep{monre2010climate}. Withdrawal could result in the unequal displacement of poor Vietnamese.

\section{Case Studies in Two Cities}

\subsection{Can Tho}
Can Tho is the largest city on Vietnam’s Mekong river delta, currently at a population of 1.8 million \citep{huong2013urbanization}.  This is up from 1.2 million just eight years ago in 2013, and this rapid pace of urbanization has produced an urban heat island effect \citep{huong2013urbanization}. As discussed in the section on flooding, warmer air holds more water and in turn increases rainfall. This has already been recorded in Can Tho \citep{huong2013urbanization}. The city sits at a low elevation, an average of only 60-80 cm above sea level \citep{huong2013urbanization}. Can Tho City is unique in its use of water as a primary means of transport and way of life. Its residents are uniquely adapted to living with floods, but also at risk for increasingly worse floods brought on by climate change \citep{nguyen2015deltaic}. 

Currently, there are plans underway to build large concrete barriers along the Can Tho River, which is an important part of city life and culture \citep{nguyen2015deltaic}. There are two floating markets that take place on the river, and many people live in houseboats and floating houses on the river \citep{nguyen2015deltaic}Figure~\ref{fig:floatingmarket}. Barriers would invariably change the way residents interact with the river and move through the city.

\begin{figure}
\includegraphics[width=\linewidth]{images/Vietnam/floatingmarket}
\caption{This is a photo of one of Can Tho's Floating Markets.\citep{isderion_2013}}
\label{fig:floatingmarket}
\end{figure}

Forced evacuation is occurring in some sections of the city, with the government offering plots of land on higher ground to those living closest to the river \citep {evers2018adaptation}\citep{nguyen2015deltaic}. However, there is evidence that residents who are relocated move back to their previous homes. Nguyen (2015) interviewed relocated families and around 60\% of them had moved back to their homes along the riverside. 

The Vietnamese federal government classified Can Tho as a first class city in 2015. This gave the federal government more control over Can Tho’s development, and priorities are firmly on the side of economic development \citep{evers2018adaptation}. Flooding, which long-time residents accept as a way of life, is not conducive to the kind of economic development projects the government wants to undertake in order to attract tourists and foreign companies \citep{nguyen2015deltaic}. As a result, Can Tho is a city being pulled in two directions. On one side are residents who lack political representation, and on the other side is the Vietnamese government, seeking to increase economic opportunity in the country as a whole.

\subsection{Ho Chi Minh City}

Ho Chi Minh City, formerly known as Saigon, is the largest city in Vietnam and its main economic center \citep{margulis2010economics}. Much like Can Tho, the city’s lifeblood is the Saigon river, which serves transportation, recreational, and economic purposes \citep{vachaud2019flood}. During the Nguyen Dynasty from the late 18th to 19th centuries, canals were constructed across the city as flood management tools \citep{vachaud2019flood}. In the early 19th century, France colonized Vietnam and by the mid-1800s, the canals were being filled in and converted to tree-lined boulevards meant to mimic the landscape of the River Seine \citep{vachaud2019flood}\citep{yarina2018}. This was disastrous for flood management and has not been corrected. The remaining canals left from the dynastic era became an open air sewage system, and they still serve this purpose today \citep{vachaud2019flood}.

HCMC is a prime example of the quickly growing population of urban Vietnamese, as mentioned in the introduction. During the war era, HCMC, known as Saigon, was part of South Vietnam. South Vietnamese were encouraged to populate cities, and those original residents from the war era are some of the longest residents of the city \citep{bolay1997sustainable}. When Vietnam was reunified in 1975, this policy was reversed as the new government seeked to relieve pressure on the densely populated urban areas, and rebuild the decimated rural economy \citep{bolay1997sustainable}. This effort was largely unsuccessful, and today, HCMC has a population of almost 9 million (Census 2019). 

Today’s Ho Chi Minh City faces major problems in regards to flooding. 65\% of its land area is less than 1.5 meter above sea level \citep{vachaud2019flood}. Technical solutions call for seawalls and dikes in hopes of fully protecting the city; however, this full protection doesn’t extend to everyone \citep{yarina2018}. The Ho Chi Minh City Adaptation Strategy, produced by MONRE in partnership with the Dutch government, seeks to fashion HCMC in the image of Rotterdam \citep{yarina2018}. A major element of this plan is a 2 billion dollar ring dike ensconcing the heart of the city, a complex system including sluice gates and canals. This proposal drew criticism, however, for its exorbitant cost and its potential to worsen flooding in communities on the unprotected side of the dike \citep{yarina2018}. 

As mentioned in the section on urbanization, HCMC’s historical landscape consisted of wetlands and swamps that provided ecosystem services such as drainage, protection against coastal erosion, and flood control \citep{bolay1997sustainable}\citep{vachaud2019flood}. As the city has expanded, many of these former wetlands have been developed and can no longer serve this purpose. Their extremely low elevation makes them undesirable places to live, and as a result they are occupied by the poorest residents of the city \citep{margulis2010economics}.

In many ways, the environmental problems in Ho Chi Minh City can be seen as a microcosm of the complex challenges facing Vietnam’s growing cities. A lack of investment in basic public services, especially access to clean water and sewage treatment, has persisted in the city. The government also prioritizes the construction of new types of infrastructure over historically used methods of flood control \citep{nguyen2015deltaic}\citep{yarina2018}. 

\section{Climate Vulnerable Groups in Vietnam}

\subsection{Women and Climate}
Women in Vietnam are an especially climate-vulnerable group. 60\% of Vietnamese women rely on agriculture as their primary source of income, compared to a little under 50\% of men \citep{margulis2010economics}. Therefore, heavy rainfall and storms’ impact on agriculture is more severely felt. Furthermore, many women being the sole person carrying the financial burden in their households. Anecdotal evidence points to women being more likely to put other family members first during climate disasters, at the expense of their own well-being \citep{nellemann2011women}. Additionally, many women in Vietnam lack basic swimming skills as young girls are not encouraged to learn to swim. This leads many to die avoidable deaths in survivable flooding conditions \citep{margulis2010economics}. 

\subsection{Children and Climate}
Children are a climate-vulnerable group in Vietnam, not only because of the immediate threat of flooding, but also because of their still developing immune systems that are highly susceptible to water-borne illnesses that spread after floods \citep{margulis2010economics}\citep{pink2016vietnam}. Children’s natural inclination to play outside can expose them to pollutants in the air and water \citep{margulis2010economics}. Extreme weather events can interrupt schooling and impact the success of a child long-term \citep{lohmann2015effect}.

\subsection{Migration and Climate in Vietnam}
Not only is climate a major driver of internal migration in Vietnam, it also exposes migrants to environmental hazards caused by climate change \citep{margulis2010economics}. As Vietnam’s agricultural sector continues to produce diminishing returns, many Vietnamese people are leaving the countryside for large cities, with the hope of securing financial opportunities less reliant on the environment \citep{margulis2010economics}.

 In Vietnam, moving required permission from the federal government until the mid-90s, and the difficulty of receiving this permission led many Vietnamese to migrate to cities without it \citep{bolay1997sustainable}. Today, moving requires registration under the National Household Registration System, and many migrants never go through this step \citep{margulis2010economics}. As a result, many of the rural to urban migrants are considered “undocumented” and are more likely to hold exploitative, dangerous, or unstable jobs in informal economy \citep{margulis2010economics}. 

Migrants often live in substandard housing that is extremely vulnerable to weather events. The poorly managed nature of urban sprawl in Vietnam’s cities can eliminate ecosystem services formerly provided by surrounding wetlands or forests \citep{vachaud2019flood}.

\section{Conclusion}

Having persevered through colonialism and a war that literally split the country in two, Vietnam now faces yet another threat to its survival: climate change. How the country will adapt to climate change has yet to be seen. The government envisions high tech physical infrastructure, but hasn’t yet been able to make meaningful progress in implementing its ambitious ideas. While the lack of free speech in Vietnam makes it difficult to gauge the sentiments of its citizens, it seems many do not have the same vision for the country. What is clear is that unlike many countries that will be heavily impacted by climate change, Vietnam has experience in coping with extreme weather events.


\chapter{Coral Reefs, Ecosystem Services, and Indigenous Peoples}

\chapterauthor{David Chengwen Gorman}

\section{Coral Reef Ecosystem Functioning and Interactions}

\subsection{Rainforests of the Sea}

In November 2018, North Sentinel Island, an island in the Bay of Bengal smaller than 60 square kilometers, drew international attention after its indigenous inhabitants killed an American missionary. The Sentinelese, one of the last uncontacted people groups in the world, have occupied North Sentinel island for the last 60,000 years. Surrounded by shallow, razor-sharp reefs, the Sentinelese have managed to remain isolated from the rapid development of neighboring Southeast Asian countries and still practice traditional customs and ways of life \citep{Smith}.

This hunter-gatherer tribe is just one of the thousands of Southeast Asian communities that rely on coral reefs for both protection and sustenance. While most do not practice the traditional hunter-gatherer harvesting practices of the Sentinelese, coral reefs remain fundamental to the lives of millions across Southeast Asia. Unfortunately, coral reefs face significant threats and at the current rate of degradation, reef survival is unlikely. The declining health, in terms of biodiversity, biomass, and fitness, of coral reefs impacts not only the overall condition of the global environment, but negatively affects the majority of the Southeast Asian population. Of these millions, marginalized groups like the indigenous peoples of Southeast Asia are affected disproportionately. Reef destruction will severely harm both the livelihoods and sovereignty of the Sentinelese and other groups like them. 

The coral reefs have become vital components of many Southeast Asian, coastal communities and indigenous groups alike. This fragile, colorful marine ecosystem boasts high levels of both biodiversity and biomass, which are vital to both reef survival as well as the millions worldwide dependent on reefs for sustainably and economic prosperity. These life sustaining ecosystems are akin to the rainforests of the sea. There are nearly 100,000 square-kilometers of coral reefs in Southeast Asia supporting between 600-800 different coral species. Healthy coral reef ecosystems provide various anthropological benefits, or ecosystem services, to people of all socioeconomic backgrounds. Reefs, with their high levels of biodiversity and biomass, provide food to over a billion people annually, are a major destination for tourism, and hold great promise for biomedicine. 

As a multi-faceted and dynamic ecosystem, the reefs are constantly changing. They face countless threats and stressors, including the effects of climate change, pollution, and overfishing; the reefs are struggling. General reef health serves as an indicator for both global and marine ecosystem health, and unfortunately, the easily observed reef degradation is representative of many other ecosystems worldwide \citep{RAR}.

This chapter describes coral reef ecosystems and current environmental and anthropocentric threats to coral reefs in Southeast Asia.  Because of reefs high value to Earth and indigenous groups, conservation is essential; reef restoration efforts will also be highlighted. The following sections will detail the dynamics, threats, and benefits of the Southeastern Asian coral reefs, while focusing on indigenous interactions and their reef reliance. This chapter will stress the importance of reef protection and restoration, and how it is not only an ecological issue, but an environmental justice one as well.

After reading this chapter, the reader should appreciate:

\renewcommand{\labelitemi}{$\blacksquare$}
 \renewcommand\labelitemii{$\square$}
 \begin{itemize}
   \item Coral reefs are threatened but vital ecosystems which sustain both marine and human life
   \item Threats to corals are threats to the reef ecosystem as a whole, and in turn are threats to the millions that rely on them
   \item Of the millions reliant on reefs, indigenous people are some of the most directly dependent; they are disproportionately affected by reef degradation and destruction
 \end{itemize} 

\subsection{Coral Ecology}


Coral is composed of tiny animals referred to as polyps. Hundreds of thousands of these polyps work together to build the visible coral structures which comprise reef ecosystems. Each coral polyp uses calcium and carbonate ions that have been dissociated in the surrounding water to create calcium-carbonate (limestone) skeletons (Figure~\ref{fig:coralanatomy}). During the day, these vulnerable, nocturnal creatures hide in their limestone skeletons to protect themselves. At night, they extend either six (hexacorals) or eight (octocorals) tentacles to feed, using specialized cells called nematocysts to stun their prey before consumption. Most species of coral are extremely slow growing, and the majority grow by less than an inch every year \citep{coralreefalliance_2021}.

\begin{figure}[h!]
\centering
\includegraphics[width=0.775\linewidth]{images/CoralAnatomy}
\caption{This simplified diagram illustrates the anatomy of each coral, its skeleton, and symbiotic zooxanthellae relationship. The mouth of each coral polyp is surrounded by tenacles, each full of stinging nematocyte cells. The zooxanthellae inhabit the tentacles, and, along the coral, use the limestone skeleton underneath for protection. Hundreds of thousands of these polyps compromise a single ``visible coral,'' all working sharing the same calcium-carbonate skeleton \citep{noaa}.}
\label{fig:coralanatomy}
\end{figure}

The majority of corals, and the ones more essential for the construction of the reef ecosystem, are classified as hard corals. These corals are reef-building corals, or hermatypes. They create the calcium carbonate skeletons which support the coral reef ecosystem.  \citep{coralreefalliance_2021}. The body of each coral is transparent; the intricate colors come from single-celled eukaryotic cells called dinoflagellates, more commonly referred to as algae. This grouping of small, photosynthetic algae live inside the coral skeletons and maintain a close, interactive relationship advantageous to both species referred to as symbiosis. The algae symbiotes within the corals are generally referred to by their colloquial, zooxanthellae. \citep{noaa}. These algae grow inside the coral and photosynthesize, providing the necessary nutrients and oxygen the coral needs to grow and survive. The coral protects the zooxanthellae with its calcium-carbonate skeleton, and via respiration, provides the carbon-dioxide needed for photosynthesis. This symbiotic relationship between coral and algae is delicately balanced, and small disruptions can have catastrophic effects not only for the organisms involved, but also for the entire ecosystem as a whole. \citep{https://doi.org/10.1002/fee.2088}  

Corals are slow growing, as they require a delicate balance of nutrients and environmental conditions. Both nutrient influxes and nutrient deficiencies interrupt the coral-zooxanthellae symbiosis, hurting both species. Most species only grow between 0.2-1 inch per year. A reason for this slow growth is that coral reproduction mechanisms are easily disrupted, either from localized pollutants or globalized climate change related issues. Coral reproductive processes are highly vulnerable to changes, as they are affected by the water temperature, time of the year, tidal cycles, and lunar cycles, and vary by species \citep{coralreefalliance_2021}.

For the duration of this chapter, the term ``coral'' will be used to describe the network of polyps which comprise the visual structures of the reefs. 

\subsection{Reef Organisms and Trophic Interactions}

The coral reef food web (Figure~\ref{fig:ExampleReefFoodWeb}), or the relationship between successive trophic levels of the community, starts as any other food web, with energy being derived from the sun by primary producers via photosynthesis. Primary producers include seaweed, grass, phytoplankton, and perhaps most importantly, the zooxanthellae coral symbiotes. As only about 10\% of the energy in a trophic level is passed to each successive level, primary producers must be the most abundant group. The second trophic levels, the primary consumers, includes zooplankton, mollusks, squirrelfish, urchins, and of course, the coral polyps themselves. Following this are the secondary consumers, made up of triggerfish, Parrotfish, butterfly fish, and more. Finally, tertiary consumers like the reef shark and barracuda finish this food web, with detritivores like sea cucumbers or bacteria decomposing and recycling organic matter back into the ecosystem \citep{https://doi.org/10.1890/15-1492.1}.

\begin{figure}[h!]
\centering
\includegraphics[width=0.775\linewidth]{images/reeffoodweb}
\caption{Food webs like the one displayed above illustrate the many different feeding interactions within the reef ecosystem. No food web can be completely comprehensive, as the reefs are home to thousands of different species, all interacting with each other. Species may fall into multiple trophic levels depending on their diet and feeding interactions. Food webs like the one shown above provide simplified examples of the feeding relationships and energy transfers within the ecosystem.\citep{https://doi.org/10.1890/15-1492.1}.}
\label{fig:ExampleReefFoodWeb}
\end{figure}

A wide range of species in the reef ecosystem depends on coral polyps, thus making them a keystone species. Corals are a specific classification of keystone species known as ecosystem engineers; they provide the entire structure and habitat from many reef-dwelling creatures with their calcium carbonate skeletons. The coral polyps themselves constitute a small percentage of the total biomass of the ecosystem, as the average coral polyp is only around 1.5 cm large, yet their impact is vital to the ecosystem's survival. To the naked eye, observable ``coral'' includes these skeletons, and the symbiotic algae living within them; the coral polyps themselves make up a significantly smaller biomass percentage as one would perceive. 

Many species depend on these coral skeletons for either protection or as hunting grounds. Fish, invertebrate, and other organisms aggregate around these underwater structures, accounting for the ecosystem’s dense biodiversity and high overall biomass. Some argue that the Parrotfish is another, secondary keystone species. This herbivorous fish eats algae latched onto the coral skeletons, which blocks sunlight and limits the photosynthetic abilities of the zooxanthellae. In a sense, its mutualistic niche is to ``clean the coral,'' allowing normal function to continue. The Parrotfish is a highly targeted, overfished species, and because it fulfils an important niche, its removal negatively effects the entire reef ecosystem. This specific niche is just one of many, all which illustrate the complex relationships involved in the coral reef ecosystem and the importance of each species to the ecosystem \citep{https://doi.org/10.1890/15-1492.1}. 

\begin{figure}[h!]
\centering
\includegraphics[width=\linewidth]{images/urchin}
\caption{:Pictured above is a seafloor that has been completely overrun by urchins and cannot effectively sustain life anymore. As anthropologic activities influence organism within ecosystems, the natural checks and balances systems degrade and certain species may quickly become dominant, seen with the urchins and the creation of these urchin barrens. The simple removal of the urchin’s keystone predator, the orange-lined triggerfish, can be enough to cripple entire ecosystems, illustrating the fragile balances within the reef ecosystem \citep{https://doi.org/10.1890/15-1492.1}. }
\label{fig:UrchinBarrenExpanseoveraDyingReef}
\end{figure}

These inter-species relationships can be further illustrated by an examination of the indirect actions which affect entire ecosystems and stem from trophic-level suppression, known as trophic cascades. For Southeast Asian reefs, the most prolific trophic cascade exists between urchins and the orange-lined triggerfish. The triggerfish is a keystone predator and limits uncontrolled urchin expansion. Without these predators, urchin populations would explode and consume everything on the seafloor, creating urchin barrens (Figure~\ref{fig:UrchinBarrenExpanseoveraDyingReef}) incapable of supporting life. These predator-prey interactions are disrupted by anthropocentric activities, mainly overfishing. As the fish are removed from the ecosystem at an unsustainable rate, the detrimental grazer effects of the urchins are exacerbated. While negative human influences will be discussed in future sections, this illustrates how interconnected reef species are the delicate equilibrium between them \citep{https://doi.org/10.1890/15-1492.1}.

\subsection{Necessary Climate and Nutrients} \label{sub:nn}

Coral reefs require a delicate balance of nutrients and external conditions, which is easily disturbed, making them especially vulnerable. As for physical needs, sunlight is a crucial limiting factor of the ecosystem. The zooxanthellae require shallow waters with low turbidity, or high clarity. This means they generally can only survive in waters under 50 meters, mostly free of sediment and debris. The water also cannot have too many nutrients, as rapid, uncontrollable, explosive algae growth can cloud waters and block the necessary sunlight needed for photosynthesis. Corals reefs exist in tropical climates usually, as they also need warmer waters of around 20-32 \degree C (68-90 \degree F). Finally, corals need water with a high salinity, or saltwater concentration. They cannot survive in brackish water or estuaries, or anywhere where freshwater sources like rivers drain into the ocean. The tropical, warm climate of Southeast Asia satisfies the aforementioned necessary reef-building conditions, explaining the abundance of coral reefs in the region \citep{https://doi.org/10.1002/fee.2088}.

\begin{figure}[h!]
\centering
\includegraphics[width=0.70\linewidth]{images/reefcycles}
\caption{The above cycles and reciprocal processes are vital for coral growth and survival. They will be further discussed in the following paragraph. Figure from: \citep{https://doi.org/10.1002/fee.2088}.} 
\label{fig:ReefEcosystemFunctioningandIntrisnsicDrivers}
\end{figure}


The first, blue cycle (Figure~\ref{fig:ReefEcosystemFunctioningandIntrisnsicDrivers}), calcium carbonate dynamics, is directly correlated with the coral skeleton formation. Bioerosion, or the breakdown of hard oceanic substrates, leaves behind calcium and carbonate ions, which coral polyps use to produce their limestone skeletons. The second, green cycle (Figure~\ref{fig:ReefEcosystemFunctioningandIntrisnsicDrivers}), herbivore-algae interactions, involves both primary production and herbivory. Photosynthetic organisms transform sunlight into chemical energy, which is then moved up in trophic levels by herbivores. A balance between the two must be maintained to prevent explosive growth of lower trophic levels, which in turn could suppress crucial processes carried out by other organisms in the ecosystem. Predator-prey interactions, the orange cycle (Figure~\ref{fig:ReefEcosystemFunctioningandIntrisnsicDrivers}), also regulate explosive trophic level growth, as successive trophic level limits the prior. Again, the balance keeps the biomass of one species from dominating and smothering other species. The final process driving success within the coral reef ecosystem is the purple cycle (Figure~\ref{fig:ReefEcosystemFunctioningandIntrisnsicDrivers}), nutrient cycling, or the uptake and release of nutrients among the organisms of the ecosystem. Nutrients must be cyclically moved efficiently and effectively while maintaining an equilibrium between retention and reintegration. \citep{https://doi.org/10.1002/fee.2088}.

\subsection{Southeast Asian Reefs by Country}

Although coral reef locations are relatively limited (as the conditions from \ref{sub:nn} must be satisfied), they can be found in over 100 countries, mainly between the Tropics of Cancer and Capricorn. Many reefs are concentrated in shallow waters surrounding islands of the Indo-Pacific, with Indonesia and the Philippines composing a majority of not only Southeast Asian reefs, but also reefs worldwide (Figure~\ref{fig:MapofSoutheastAsianCoralReefs}). Coral reefs cover 284,300 square kilometers (110,000 square miles) worldwide, yet the overall area of healthy reef is decreasing rapidly \citep{Watlas}.

\begin{figure}[h!]
\centering
\includegraphics[width=0.6\linewidth]{images/reefmap}
\caption{This map not only shows the locations of Southeast Asian coral reefs, but also the relative risk levels associated with each reef. The threat levels are as follows: blue = low, yellow=medium, red=high, maroon=very high. The suggested threat levels describe local threats; global threats like climate change are not accounted for \citep{Watlas}.}
\label{fig:MapofSoutheastAsianCoralReefs}
\end{figure}

\begin{table}[h!]
\centering 
\caption{Southeast Asian Coral Reef Size by Country}
\label{tab:Southeast-Asian-Coral-Reef-Size-by-Country}
\begin{tabular}{lrrr}
Country     & Rank                    & Reef Area         & Percentage of World Total Reef Area  \\ 
\hline\hline
Indonesia   & 1                       & 51,020            & 17.95\% \\ 

Philippines & 3                       & 25,060            & 8.81\%  \\ 

Malaysia    & 17                      & 3,600             & 1.27\%  \\ 

Japan       & 23                      & 2,900             & 1.02\%  \\ 

Thailand    & 26                      & 2,130             & 0.75\%  \\ 

Myanmar     & 27                      & 1,870             & 0.66\%  \\\hline
\end{tabular}
\end{table}

Southeast Asian countries account for over a quarter of all the worlds coral reefs (Table \ref{tab:Southeast-Asian-Coral-Reef-Size-by-Country}). While Indonesia and the Philippines dominate total reef area, many coastal Southeast Asian countries have reefs of their own, all which face similar anthropocentric pressures. \citep{Watlas}.

\section{Climate Change and Its Effects on Coral Reefs}

\subsection{The Indigenous Paradox}

Climate change and global warming are undoubtedly one of the biggest contributors to the loss of biodiversity and overall destruction of coral reefs. The IUCN Red List Index (RLI), a comprehensive list tracking biodiversity loss and extinction risk, identifies this, stating corals as one of the fastest declining species (in terms of diversity and richness) in response to climate change \citep{wwfindex}. Even if all localized reef threats and pollutants were eliminated, the overarching threats of ocean acidification and rising temperatures can eradicate reefs entirely \citep{Keller2009ClimateCC}.

In addition to reef eradication, climate change disproportionately effects lower socioeconomic classes, specifically, indigenous people groups. Many Southeast Asian indigenous coastal peoples are reliant on reefs not only for food and sustenance, but they also have strong cultural ties to the reefs and surrounding waters. Climate change threatens these indigenous people-reef ecosystem interactions, and these threats compound other climate change-associated dangers affecting indigenous groups. As many indigenous groups inhabit harsh and isolated environments, they are highly vulnerable to the rising sea levels and increased storm severity resulting from climate change. They are also often excluded from policy discussions and legislative responses to climate change, further increasing their susceptibility to climate change disturbances \citep{13772149520190801}.

The ecosystem expertise and resilience of Southeast Asian indigenous groups, coupled with a high vulnerability to climate change, creates what some call ``The Indigenous Paradox.'' Indigenous groups are some of the most prepared groups to respond to climate change, while simultaneously disproportionately vulnerable to its negative effects. Indigenous groups are perpetually ransacked by reef degradation, however, plethora of knowledge concerning specific environments and ecosystems is passed down generationally which can be used to curtail and reverse climate change. The highly valuable resources offered by indigenous people are further discussed in Section \ref{sub:ie} \citep{13772149520190801}.

\subsection{Rising Sea Level}

A primary issue when discussing climate change and the ocean is rising sea levels. This is due to both melting polar ice caps and thermal expansion of water molecules caused by the rapid increase in temperature, or thermal heat, in recent years. Sea levels are projected to continue to rise 0.5-1.5 meters by 2100, undoubtedly impacting both submerged ecosystems and coastlines worldwide. \citep{sealevelrise}.

In contrast to the negative impacts, Brian Keller, Regional Science Coordinator of NOAA's Office National Marine Sanctuaries, and his colleagues conducted a study concerning sea level rise and its effect on corals on the Sanya Bay of the South China Sea, where they discovered a potentially unforeseen positive development: the sea level rise promoted coral growth. The previously degraded reef was able to recolonize via asexual fragmentation in response to the 16.2 $\Mypm$ 0.6 cm rise over the previous 30 years. Of the corals the study surveyed, 86\% were under 30 years old, suggesting that the rise in sea level promoted their growth. This phenomenon can be explained by the rise in coral growth accommodation space, as sea level rise increases the vertical space a coral can grow in. As the coral gets closer to the surface, growth is inhibited by temperature, exposure, and sediment, but by increasing the overall depth, growth can occur in a greater area of water. This study is not predictive of universal coral growth, however, and holds significant uncertainties concerning prospective implications. Only specific coral species are suited for a rapid rise in accommodation space, which may select out other species. This leads to an increase in total growth but a decrease in biodiversity, and with organisms as vulnerable as coral, biodiversity is crucial \citep{sealevelrise}. Coral growth capacity is likely to be inhibited by other factors, both regional and global, so an increase in accommodation space may have negligible overall impact \citep{Keller2009ClimateCC}.

As sea levels continue to rise, they pose risks to coastal regions, with the threat of washing entire communities into the ocean.  Such catastrophic events would undoubtedly add sediment and pollutants to the water while also physically damaging corals and other organisms with the marine debris. Community submergence would also displace people groups, to other coastal areas. Thereby concentrating pressure on reefs not yet affected by coastline degradation. Increasing storm surge and flooding associated with climate change significantly influences coastal retreat. Already, over a million people in over 27 places worldwide have been forced to relocate, and this number will only increase as climate change worsens. \citep{Sinay_2020} Indigenous groups and other coastal communities are disproportionately threatened by sea level rise, as they are forced off their lands by both nature and self-serving corporations and governments. As rising sea levels and issues of displacement increase, even benevolent governments struggle to accommodate affected citizens and displaced people groups. These threats to coral reefs may seem distant, but they are quickly approaching realities imposed by climate change \citep{Keller2009ClimateCC}.

\subsection{Ocean Warming and Acidification}

Climate change has already begun to negatively impact reefs in quantifiable ways that can be observed at the molecular level. As greenhouse gases are released and trapped in the atmosphere, they prevent heat escape and also contribute to smog and air pollution. The primary greenhouse gas, carbon dioxide (\carbondioxide), is released in high amounts when fossil fuels like coal and natural gases are burned for energy. Fossil fuel emissions release carbon dioxide into the atmosphere, but scientists believe about one third of the \carbondioxide is absorbed by the ocean. The increase in greenhouse gases emissions from fossil fuel burning contributes to both the warming of the planet and the acidification of the ocean \citep{Keller2009ClimateCC}. 

As atmospheric \carbondioxide dissolves in the ocean, it reacts with water to create carbonic acid. This carbonic acid then dissociates into bicarbonate, releasing protons in the process. These protons, when added to the water, lower the pH, making it more acidic (Figure~\ref{fig:OceanAcidifcationChemicalReactionandItsCarbonateRelation}). Since the Industrial Revolution, a 0.1 pH drop has been observed on the ocean surface, and this drop is expected to increase as global emissions continue and \carbondioxide concentrations rise. In more acidic conditions, corals struggle to efficiently create their hard skeletons \citep{Keller2009ClimateCC}. The carbonate ion saturation, or aragonite saturation, is crucial for skeletal growth, as the carbonate ion is one of the two necessary ingredients needed for calcium-carbonate formation. The bicarbonate ion formed by the carbonic acid-carbonate interactions is unusable to corals, thus slowing skeletal growth ({Figure~\ref{fig:OceanAcidifcationChemicalReactionandItsCarbonateRelation}}). Today, up to 60\% of reefs persist in waters with inadequate aragonite saturation \citep{Ayala_2009}.

\begin{figure}[h!]
\centering
\includegraphics[width=\linewidth]{images/acidification2_med}
\caption{The above figure outlines the chemical reactions which connect ocean acidification with limestone formation. As carbon dioxide dissociates in the water, it creates carbonic acid, or H$_2$CO$_3$. The protons of carbonic acid are easily dissociated by the more polar water molecule (H$_2$O), thus dissociating and combining to form bicarbonate, or HCO$_3$. The dissociated protons may also form bicarbonate by reacting with carbonate ions. This chemical reaction removes carbonate ions from the water, which corals need to construct their calcium carbonate skeletons  \citep{Ayala_2009}.}
\label{fig:OceanAcidifcationChemicalReactionandItsCarbonateRelation}
\end{figure}

The skeletons built by corals in waters lacking proper carbonate ion concentrations not only form slower but are also less robust and weaker. As a result, corals have fewer defenses to fight pressures and, combined with the slowed growth rates, may not be able to overcome the pre-existing stressors they face. Increasing acidification will likely cause coral production rates to fall below destruction rates, decreasing reef size and affecting the ecosystem as a whole. Additional stress from acidification and climate change only adds more threats to coral, creating a detrimental positive feedback loop for the already struggling polyps. The effects of ocean acidification compound on each other. The weaker skeletons stemming from low carbonate ion concentrations cannot adequately protect the corals, resulting in even lower skeletal production and exacerbating the preexisting consequences of ocean acidification on corals. 
 \citep{Ayala_2009}.

In addition to ocean acidification, as greenhouse gas emissions continue to cause earth's surface to warm, the ocean warms with it. This rise in temperature is the main cause of sea level rise, due to both glacial melt and the molecular expansion of water molecules in response to heat. Ocean warming and its effects on coral reefs is still a topic requiring more research, but its effects have already begun to be observed \citep{wwfindex}.

Coral disease transmission is exacerbated by higher temperatures, which will be explored further in Section \ref{sub:cd}. Unusually high-water temperatures are the main cause of coral bleaching, and as ocean temperatures continue to rise, mass bleaching events are expected to increase in both frequency and severity\citep{Keller2009ClimateCC}. Ocean warming will have disastrous future effects on coral reefs.  In 2018, Intergovernmental Panel for Climate Change, or IPCC, reported that a mere two-degree Celsius rise in temperature could completely eradicate the coral reefs, with an estimated loss of around 99\% of reefs worldwide \citep{wwfindex}. 

\subsection{Coral Bleaching}

Most mass coral mortality events are driven by heat waves, which have increased with climate change and global warming. Coral and algae enjoy an obligate mutual relationship; they cannot survive without each other. Coral polyps react to warming water temperatures and other environmental stressors by expelling algae, a process known as coral bleaching (Figure~\ref{fig:CoralBleaching}) which has the potential to rapidly destroy entire reef ecosystems. The disruption of the coral-algae symbiosis caused by the loss of algal endosymbionts causes the corals to pale, as the algae are responsible for coral coloring. If the water does not cool or other stressors persist, algae cannot regrow, starving the coral. Emaciated corals crumble into white, lifeless skeletons \citep{https://doi.org/10.1111/gcb.14871}

The depletion of a singular species shifts the organismal balance and negative ecological effects cascade throughout the ecosystem. Many habitats are destroyed as the coral, the keystone species and ecosystem engineers, are removed. The population balances within the ecosystem in turn shift, for the species more directly dependent on corals are disproportionately affected by the bleaching events.\citep{https://doi.org/10.1111/gcb.14871}.

\begin{figure}[h!]
\centering
\includegraphics[width=0.6\linewidth]{images/coralbleach}
\caption{This above split-image shows the same fire coral before (left) and after (right) coral bleaching has occurred. The vibrant colors on the left image result from the zooxanthellae symbiotes, and when the corals expel them, the ghostly white color seen on the right image is observed. This color change is universal across all coral species; only the color of the calcium carbonate skeleton remains after the algae are expelled \citep{https://doi.org/10.1111/gcb.14871}.}
\label{fig:CoralBleaching}
\end{figure}

Coral bleaching is the net outcome of complex, multifactorial stressors working at both the cellular and ecosystem levels (\ref{des:levels}). Other environmental exacerbators include, but are not limited to, light availability, salinity, oxygen demand, organic nutrient availability, and inorganic nutrient availability. 

\begin{description} \label{des:levels}
\item[Cellular Level]: Reactive oxygen species, or ROS, accumulates and triggers signaling cascades prompting corals to expel their zooxanthellae. These ROS are both produced and accumulated in response to environmental changes and stressors. 
\item[Ecosystem Seascape Level]: The most common stressor is heat waves. The first-ever observed mass coral bleaching event in 1998 was driven by El Ni\~{n}o, and rising temperatures have been accredited with the increasing frequency of bleaching events. The two recent back-to-back bleaching events in both 2016 and 2017 gained national attention, showing the vulnerability of the coral ecosystem to change on a global scale. These events also demonstrated how climate change will only make coral bleaching more frequent and of greater severity.
\end{description}

While a single factor may not cause a mass bleaching event, they have cumulative effects on the fragile, unadaptable coral polyps. Additionally, each factor also exacerbates the corals’ adaptability to rapid heat change. Global warming is the origin of negative effects to this intricate, dynamic system \citep{https://doi.org/10.1111/gcb.14871}.

\section{Unsustainable Fishing Practices}

\subsection{The Live Reef Fish Trade}

Unsustainable fishing practices are arguably the most detrimental local practice to Southeast Asian reefs. Over 55\% of reefs worldwide are affected by overfishing and unsustainable fishing techniques, and this number is higher in Southeast Asian reefs. Overfishing disrupts trophic interactions, since macroalgae, without as many fish as predators, grow disproportionality and smother corals. An overfishing-driven decline in fish stocks has made fish harder to catch, leading to a shift away from sustainable fishing practices. Unsustainable fishing threatens not only fish stocks and ecosystems, but also local economies and the people that rely on fish for food. The methods currently used to catch live fish, like cyanide fishing or blast fishing, have been incredibly destructive to the coral reefs of Southeastern Asia. Both Indonesia and the Philippines offer textbook examples of unsustainable practices which still persist today, despite efforts to regulate and ban them \citep{8767176420130601}.

A major driver of unsustainable fishing practices is the Live Reef Fish Trade, or LRFT, where selective fish are caught, kept alive, and later sold. There are two branches of the LRFT: food and ornamental. The food trade makes up a majority of the LRFT, driven largely by wealthy Asian countries' affinity for particular fish as high-priced delicacies. Certain high-commodity fish, like leopard coral reef fish, are targeted and fished disproportionately, pressuring reef ecosystems and disrupting trophic interactions. These fish can sell for as much as \$60 USD per kg instead of the average \$2 USD per kg for most fish. In locales such as Palawan, Philippines, where the average monthly income is under \$100 USD, it is no surprise the LRFT would flourish. Fishermen explain, ``It's like hitting the jackpot every time.'' Fish are also live caught for ornamental reasons. Vibrant, colorful reef fish are highly desired in aquariums globally, sustaining demand and further fueling the LRFT. \citep{10.2307/40603032}. 

Many of the practices for catching live fish are inherently destructive to reefs. Fish are often caught young and then grown to an edible size. This excessive juvenile capture removes fish before they can reproduce and contribute back to the ecosystem, further reducing the fish stocks of the selected species. Other more violent methods, like cyanide and blast fishing, have horrid effects on reefs, and these will be discussed in the following sections \citep{10.2307/40603032}.

The LRFT has continued to grow in both the Philippines and Indonesia despite both decreasing profitability and fish stocks. This dangerous combination has led to high levels of overfishing that further destroyed coral reefs. Fishermen are catching less fish and must travel farther distances to find them, causing large scale economic losses in both coastal communities and the individuals residing in them. Because the LRFT is poorly regulated and any existing regulations are often not enforced or disregarded, the harmful impact of the LRFT is likely to continue \citep{10.2307/40603032}.

\subsection{Cyanide and Blast Fishing}

Cyanide fishing is as simple as diving with a spray bottle containing crushed sodium cyanide tablets which can be injected into reef crevices or squirted directly into the faces of desired fish (Figure~\ref{fig:CyanideFishing}). While larger animals may only be stunned and can later revive, the same dose is fatal to smaller organisms. Cyanide is most often fatal to small fish, invertebrates, corals, and their symbiotic algae, disrupting ecosystem functioning on multiple levels  \citep{wwfcyanide}. After just 30 seconds of cyanide exposure, coral becomes stressed and begins to lose its ability to function normally. For every live fish caught with cyanide, a square meter of reef is destroyed. 

At the turn of the century, 75\% of all aquarium fish coming from Southeast Asia were estimated to be cyanide-caught. \citep{970313024119970301}. Despite being banned in both Indonesia and the Philippines, cyanide fishing still persists today, with an estimated 75\% of aquarium fish from Southeast Asia being cyanide-caught. Many live fish are caught via cyanide fishing. The minimal regulations and negligible punishments for cyanide fishing enable this destructive practice to continue. In the last six years, only six case files regarding cyanide fishing have resulted in convictions \citep{wwfcyanide}. Approximately two-thirds of all Filipino reefs are estimated to be affected in some way by cyanide poisoning  \citep{970313024119970301}.


\begin{figure}[h!]
\centering
\includegraphics[width=0.7\linewidth]{images/cyanidefishing}
\caption{The diver in this picture is cyanide fishing in a Filipino reef. He uses cheap, makeshift gear which can be either made from scrap or re-purposed. The warm, coastal waters allow him to dive without a wetsuit, only using wood planks nailed to slippers as fins. His bottle contains crushed sodium cyanide tablets dissolved in water, a lethal solution which will decimate the reef he is injecting it into \citep{wwfcyanide}.}
\label{fig:CyanideFishing}
\end{figure}

Blast fishing - detonating explosives in reefs to stun fish -- is another destructive practice in the live trade industry. Like cyanide fishing, smaller organisms are often immediately killed by these blasts. In addition, blasts also break the preexisting coral skeletons, destroying both the habitat and ecosystem. \citep{https://doi.org/10.1890/1051-0761(2006)016[1631:RFBFOC]2.0.CO;2}. 
  
The explosives used are almost always homemade and can be as simple as a kerosene-fertilizer mixture inside a glass soda bottle. Just one 300mL bottle can create an explosion which leaves behind a crater with a 1-meter radius capable of lasting multiple years. Besides the general fatalities from the blasts, one of the biggest issues associated with blast fishing is the leftover coral skeleton debris. Even after five years, the rubble and debris left inside the crater can be 5-10 cm deep. This additional sediment can smother both coral and algae, either blocking light and hindering photosynthesis or burning still-intact corals alive. Larger debris and skeletal pieces can also abrase growing corals, scraping off growing colonies or hindering recruitment by blocking secure surfaces for attachment \citep{https://doi.org/10.1890/1051-0761(2006)016[1631:RFBFOC]2.0.CO;2}. 

Over half of all Southeast Asian reefs are currently threatened by blast fishing practices. Like cyanide fishing, blast fishing is legislatively banned, but this ban is barely enforced. Blast fishing was banned in Indonesia in 1985, yet it still continues today, not only in Indonesia, but also in most other reef-bearing Southeastern countries \citep{https://doi.org/10.1890/1051-0761(2006)016[1631:RFBFOC]2.0.CO;2}. 


\subsection{Economic Motivations for Indigenous Peoples}

The semi-nomadic, indigenous maritime communities of Southeast Asia are highly dependent upon unsustainable reef fishing. These communities are reliant on the sea for subsistence, often inhabiting areas of high biomass and biodiversity, such as near coral reefs, where resources are bountiful. Reef survival corresponds with human survival \citep{boatpeople}. One such group, the Sama-Bajau, reside mainly in the Philippines and eastern Indonesia. Not only do the reefs supply the majority of the Sama-Bajau’s protein intake, but also provide the most economic opportunities. The reef supports a wide range of activities, primarily fishing, and additionally boat building, guided tours, and sea trading. Some groups, like the Sama-Bajau, now live in stilt houses, increasing a sense of community and stability. Unfortunately, many other groups are not officially recognized and lack legal representation. Without recognition, the highly mobile lifestyle deprives these groups of land ownership, healthcare, education, and establishing economic prosperity \citep{boatpeople}.

Due to complex systemic indigenous suppression, the vast majority of the reef fishermen are poor and struggle to sustain themselves through traditional fishing practices. Viewed as primitive and uncivilized, the reef-reliant, seafaring lifestyle is commonly not respected among upper socioeconomic classes. As a result, indigenous children face widespread racial and cultural discrimination, leading them to drop out of school and pursue a less economically prosperous career, like fishing. As reefs are continually degraded by issues such as climate change and local problems (e.g. corporate overfishing and pollution), sustainable fishing practices become less viable. Many indigenous fishermen have turned to more profitable yet unsustainable fishing practices like blast fishing and cyanide fishing. With the high demand of the LRFT and aquarium trade, these destructive practices promise steady, higher payouts than traditional fishing methods but put indigenous groups in a vicious cycle of further reef degradation and economic opportunity loss. Indigenous participation in these activities is not driven by greed, but purely because they lack alternatives to feed their families \citep{boatpeople}.

\section{Reefs Pollutants}

\subsection{Marine Debris and Ocean Contaminants} \label{sub:mdoc}

A significant number of anthropocentric activities contribute to reef pollution. Pollutants can be classified generally in three categories: toxins, sediments, and nutrients. Toxins cause physical harm to corals and other organisms at the cellular level. They can be organic or inorganic and are often found in chemical runoff. Sediments block sunlight and prevent photosynthesis and limit primary production. They also reduce the number of viable locations for coral larvae to attach to, as the loose sediment settles on the previously secure spots on the coral skeleton. Sediment also increases the turbidity of the water, blocking visibility. This can contribute to biodiversity loss, selecting for certain animals which do not primarily rely on eyesight. Finally, nutrients can promote algal blooms which smother corals, or they can disturb organismal balances within the ecosystem, disrupting critical interactions among populations. Excessive nutrient richness, or eutrophication, can also promote pathogenic growth among corals and lead to coral epidemics. A wide range of human activities contribute to reef pollution and is exacerbated with with globalization, population expansion, and development \citep{4884777420100401}.

Marine debris is essentially human trash, which enters the water via boats or from land. (Figure~\ref{fig:marinetrash}). Floating trash can bear resemblance to jellyfish and is often consumed by animals, obstructing their gastrointestinal intestinal tract. Lost nets, lines, and other ghost fishing gear also often entangle corals and other reef organisms, severely hindering their mobility and often killing them. The levels of marine litter in Southeast Asia generally exceed the global average, with new coastal development, ineffective regulatory methods, and heavy shipping traffic throughout the region being main contributors \citep{4884777420100401}.

\begin{figure}[h!]
\centering
\includegraphics[width=0.7\linewidth]{images/plasticsuff}
\caption{Plastics and marine debris commonly snag corals and shade them, so that the zooxanthellae cannot adequately perform photosynthesis, and thus starving the corals. Coral engulfed by plastic bags like the one above are highly susceptible to disease, as the plastic bag creates a warm, hypoxic environment that is ideal for pathogen transmission and infectivity. This trash entanglement also may break off chunks of coral, killing entire colonies \citep{USEPA_2017}.}
\label{fig:marinetrash}
\end{figure}

Plastics and marine debris commonly snag corals and shade them, so that the zooxanthellae cannot adequately perform photosynthesis, and thus starving the corals. This trash entanglement also may break off chunks of coral, killing entire colonies \citep{USEPA_2017}.

The majority of marine trash is plastics, which account for 60-80\% of all marine litter. Plastics take multiple generations to degrade and can act as a skin for chemical pollutants, carrying chemicals and toxins into the water from the land or anything else they were in contact with. At each successive trophic level, plastic can be ingested, as there is significant variation among the size of plastic particles. Plastics travel up the entire food chain, decreasing both energy reserves and feeding capacity. Plastic ingestion also decreases the ability to produce offspring, or fecundity. Oceanic plastic pollution is so high that it is estimated there is not a single ocean organism that is completely unaffected by plastic ingestion, either directly or indirectly \citep{12907334620180601}.

The traditional ``plastic pollutant'' is a macroplastic, that is, visible to the naked eye. In actuality, microplastics, or plastic particles under 5mm, comprise the majority of plastic pollution and are increasing in concentration at alarming rates. These microplastics can enter the surrounding reef waters as fragments from larger plastic debris or in terrestrial runoffs and waste dumps. They affect different species of coral differently, but each varying effect is almost always negative. They may attach to corals and disrupt cellular functioning, or they can cause excessive mucus production and overgrowth. Microplastics are small enough for corals to ingest, and the corals retain the plastic fragments for extended periods of time. Microplastic exposure may trigger signaling cascades related to cleaning and digestive responses among different coral species. In a microplastic-coral interaction study, five out of the six coral species examined displayed serious negative health effects when exposed to microplastics. In areas with higher concentrations of microplastics, bleaching and tissue necrosis was widely observed. To account for this study’s short time period, corals were exposed to microplastic concentrations significantly higher than environmentally observed. While this produces considerable uncertainty, this study still shows that microplastics do negatively impact corals and calls for future studies over longer periods of time with environmentally realistic microplastic concentrations. Microplastic exposure is continually increasing, and poses a serious threat to reefs, corals, and reef organisms alike \citep{12907334620180601}.

Both sewage and wastewater discharge contribute a significant amount of toxins and contaminants to reef water. Chemicals, toxins, bacteria, and pathogens can enter reef ecosystems from cesspools, septic tanks, landfills, sewage treatment plants, and more. These chemicals have obvious negative effects on coral reefs, as many of them can either kill or infect corals and reef inhabitants. In Southeast Asia, over 80\% of sewage deposited in the ocean is left untreated, inevitably depositing high levels of toxins and chemicals into the waters \citep{4884777420100401}. One of the most prevalent chemicals affecting coral reefs, however, comes not from industrial waste and sewage, but from tourists and everyday people. Oxybenzone is a common component of sunscreen, which enters the water whenever people wearing sunscreen do. Along with other harmful chemicals in sunscreen, oxybenzone damages coral DNA and also accumulates in tissues, either causing death and deformities among adolescent coral colonies or inducing bleaching \citep{USEPA_2017}.

Another major source of reef contamination is crude oil, which blocks sunlight, smothering and starving reefs. Oil enters the ocean not only though spills, but also via operational discharge. This oil itself contains toxic components. One of these substance groups, polycyclic aromatic hydrocarbons (PAH), bind to coral DNA and proteins, and in turn disrupt necessary cellular functions \citep{4884777420100401}. While the oils themselves are harmful, the dispersants used to clean up spills are actually more damaging to coral polyps. Surfactants, substances which reduce liquid surface tension when dissolved, are often used, as they dissolve and break apart large floating sheets of oil into smaller droplets. These dispersants are toxic to coral larvae and can kill large sections of young coral fairly quickly. They also prevent the fertilization of mature eggs and hinder coral maturation, or metamorphosis. Oil contamination and surfactant poisoning are increasing in Southeastern coral reefs, as development has continued to increase in Southeast Asian coastal regions \citep{2615280620070801}.

\subsection{Development, Industrialization, and the Agta People}

Many coastal regions in Southeast Asia have seen rapid development in the past couple of decades. This development in itself has negative effects on reefs, as piers and other structures are often built on top of already struggling reefs, negating any chance of regeneration. Resources like sand and limestone are also sometimes extracted from reefs in a process called coral mining; large coral pieces and other reef materials are used as road fills, bricks, or cement components \citep{coralreefalliance_2021}. Much of this coastal development also releases high concentrations of sediment into the water, increasing turbidity and primary production within the reef. Coral reefs are continually polluted by the dredging, dumping, and shipping all associated with development as well. While development is typically a positive economic indicator, it often carries disastrous environmental side effects, as evidenced by the relationship between development and the coral reefs \citep{USEPA_2017}.

Road construction in particular generates many negative consequences for reef growth and survival. Not only does road construction release high levels of toxins into surrounding waters, but it also adds a great deal of sediment as well. Roads, as well as other impervious structures, do not allow liquids to seep into the soil. Instead, they flow across road surfaces and downward, generally towards sources of water. Even after construction is complete, runoff from coastal roads continue to contribute substantial amounts of toxins and sediment to the reefs. This is especially problematic during storms and in areas of high precipitation, as heavy rains pick up toxins and sediments from the ground and flow via roads directly into the ocean. Both Indonesia and the Philippines host monsoon climates, and because this stormwater cannot be filtered by the soil, it carries all of its pollutants into the reefs. Notable toxic substances reaching waters in greater quantities include metals like lead or mercury or organic chemicals like polychlorobiphenyls (PCBs). This particular carcinogenic dioxin not only takes a long period of time to break down, but also affects the growth rates, feeding patterns, defensive responses, and reproductive processes of all coral species. As road, industry, and urban development persist, the runoff will continue to carry chemicals, toxins, sediment, nutrients, and pathogens into surrounding reef water \citep{USEPA_2017}.

In addition to reef degradation, development and industrialization has often displaced indigenous groups. Development indirectly undermines indigenous groups by harming reefs, but also directly hurts them via displacement and relocation. One of these groups, the Agta people of Northeastern Luzon in the Philippines, perfectly exemplify this scenario. Over the past decades, infrastructural development has displaced many of the Agta, forcing them to live as landless peasants on the outskirts of other towns and villages. Despite national legislation ruling against displacement, one Agta group has been involved in a multi-decade struggle, fighting against displacement from their ancestral land, Dimasalansan. They do not legally own the land but instead were given settlement rights in the protected park by the Filipino government. This lack of quantifiable ownership, however, has made it easier for displacement to occur. One standout, displacing developmental project has been the construction of an 82km road through Dimasalansan. This project planned for indigenous displacement, stating that the construction will allow the Agta to adopt ``the culture and ways of the (incoming) migrant population'' and also that ``relocation of livelihood/business and dwelling/camps...shall be formulated.'' Developmental projects often justify indigenous displacement by arguing that they are already semi-nomadic and not place-bound, so relocation is relative \citep{agtadisplace}.

This involuntary displacement forces indigenous groups to adopt differing ways of life. Much of the Agta lifestyle consists of spearfishing the shallow reefs of their native waters, and as they are displaced and pushed further inland, they cannot continue their traditional practices. Instead, they must adopt different livelihoods, and turn away from their ancestral reef reliance, breaking the longstanding cultural ties and connections to the water \citep{RePEc:gam:jsusta:v:12:y:2020:i:19:p:7983-:d:420111}.

\subsection{Deforestation and Agriculture}

As the aforementioned Agta are continually forced inland, they must find new economic activities to sustain themselves. The most commonly chosen new livelihood is small-scale, swidden agricultural cultivation. Slash-and-burn technique is used to clear areas for farming, contributing to both deforestation and agricultural issues in the near-coastal forests. Both are contributors of reef pollutants, further harming the already struggling reef ecosystems and the other indigenous coastal groups who still rely upon them \citep{RePEc:gam:jsusta:v:12:y:2020:i:19:p:7983-:d:420111}.

Deforestation adds high levels of sediment into the surrounding waters, smothering coral. Corals and their endosymbionts in sediment-filled water cannot carry out nightly respiration or daily photosynthesis in an efficient manner. In Indonesia, palm oil plantation growth is a major contributor to deforestation, while in the Philippines, the expansion of logging and agriculture industries is the main cause. In both countries, corruption, poor regulation, and economic greed have exacerbated this issue. One specific tree of interest is the mangrove, which is particularly important to reef protection. Many reef fish raise their young in the protected labyrinth of the mangrove forest, and without it, many offspring become easy prey. In addition to the traditional consequences of deforestation, mangrove destruction also means nursery loss. Mangrove forests also act as filtration systems for sediment, and without them acting as a barrier, more sediment from runoff and other human activities reaches reefs. Three quarters of the mangroves in the Philippines have already been lost, and as agriculture continues to expand, this number is likely to rise \citep{coralreefalliance_2021}.

Agricultural runoff also adds sediment to reef water, but more importantly, agricultural runoff adds excess nitrogen, phosphorus, and other nutrients. Generally, having more nutrients is beneficial, but the reef ecosystem is adapted to survive only at specific, lower-nutrient levels. Nutrient excess selects for other organisms like seagrass, however, which boom in population and outcompete corals for their required nutrients. This eutrophication stems mainly from runoff and discharge containing manure, fertilizers, aquaculture byproducts, and pesticides. Manure and other fertilizers cause phytoplankton booms which not only block sunlight but can also create toxins that are dispersed throughout the ecosystem. These massive algae blooms also require tremendous oxygenic support, causing hypoxic conditions in the water surrounding them. The already struggling organisms often succumb to hypoxia and die off in mass quantities. When pesticides reach the waters, they generally disrupt the symbiotic coral-algae relationship. Herbicides in particular are very dangerous, as they kill the algae and induce bleaching. Finally, when corals are exposed to too many nutrients, bacteria and other pathogenic growth are promoted. Diseases and coral contamination quickly follow \citep{4884777420100401}.

\subsection{Coral Diseases}\label{sub:cd}

Corals are highly susceptible to diseases. They lack virtually any dispersal barriers and spread rapidly due to how close in proximity coral colonies are to each other. Viruses, bacteria, and fungi can cause diseases among coral species, which all proliferate with temperature increases and lower water qualities. As a result, global warming and other anthropocentric activities have led to an increase in coral diseases\citep{Keller2009ClimateCC}. 

Pathogens often enter the reef water through runoff and stormwater containing fecal matter, inadequately treated sewage, and other contaminants. One of the major human sources of coral diseases, however, is plastics. Land-based pathogens attach to micro-holes in plastics, and when they reach the reef, the pathogen can disconnect and then reattach to the polyps. A coral that has been in contact with any plastic is twenty times more likely to be infected with a disease than one that has not. For Southeast Asia, a region with an estimated 11 billion plastic reef entanglements, this is a major issue. Reefs that have no plastic exposure have a 4\% lethal coral disease probability, which, while still high, pales in comparison to the 89\% lethal disease probability of plastic-entangled reefs. Plastic entanglement can potentially create perfect conditions for viruses and bacteria, as sunlight and oxygen may be blocked, and temperatures can increase in sort of a greenhouse effect (Figure~\ref{fig:marinetrash}). As certain coral species are more likely to snag plastics than others, plastic-induced pathogens also contribute to biodiversity loss among coral species \citep{Thompson}.

\begin{figure}[h!]
\centering
\includegraphics[width=0.7\linewidth]{images/coraldisease}
\caption{:This picture shows a brain coral that has been infected with black-band disease. The left side of the coral is how healthy coral appears; the right side is diseased. This particular coral disease’s name is derived from the black circumferential band in the middle of the coral. As the disease progresses, the band travels along the coral, killing polyps and leaving behind a dying, empty skeleton (seen on the right half of the coral) \citep{coraldiseasenoaa}.}
\label{fig:BlackBandDisease}
\end{figure}

Though the methods of transmission are well-theorized, the exact pathogenic causes of coral diseases are not well understood. The effects of coral diseases are nevertheless clear. Diseases always present some sort of indicator, including red, black, white, or yellow bands, discolored spots and blotches, or rapid degradation and general tissue loss (Figure~\ref{fig:BlackBandDisease}). Coral diseases also cause large coral chunks to slough off, exposing the calcium carbonate skeleton underneath. After the area is exposed, small reef creatures inhabit it, using it for shelter and breeding. This over-colonization thus cannot be regrown with new corals, so instead of only the chunk dying, the entire colony passes with it. Very little treatment is available for coral diseases, as currently, black-band disease is the only treatable disease. Coral diseases pose great threats to reefs, as they are not easily combated and can spread rapidly through entire ecosystems \citep{coraldiseasenoaa}.

\section{Reef Ecosystem Services}

\subsection{Biomedicine}

Coral reefs host a vast array of human benefits, which helps to explain why they have been so overused and exhausted. Coral reefs hold massive medical potential, though currently that field of biomedicine is lacking and underdeveloped. As stationary animals, corals need chemical defense mechanisms to protect themselves from predators. These chemical defenses also help to fight diseases and combat environmental stressors. Ideally, one would be able to replicate these chemical processes in drugs for humans to use as defenses to diseases and illnesses. Coral medicine is still in the early stages of development; its importance comes mainly from the potential it holds \citep{AndrewWBruckner_1970}.

Coral reefs have been referred to not just as the medicine cabinet of the sea, but the medicine cabinet of the 21st century. These natural underwater pharmacies are genetic warehouses and hold colossal medical potential. Forty to fifty percent of all drugs currently used have natural origins, and coral reefs are estimated to have 300-400 times higher drug potentials than their terrestrial counterparts \citep{AndrewWBruckner_1970}. New medicines to treat cancer, arthritis, Alzheimer’s, heart disease, viruses, bacterial infection, and more are currently being developed with corals as the centerpiece. One exemplary study conducted by \citet{Tseng_2013} found secosteroids and norcembranoids from the coral \textit{Sinularia nanolobata} inhibited carcinogenic skin cell growth in humans. As coral medical research is continually developed, the medical benefits of differing organic compounds in various coral species will likely be discovered. The medical potential of corals are unimaginably high \citep{AndrewWBruckner_1970}.

With this great potential comes complications, however. Research and development into coral reef biomedicine are lacking and underdeveloped and will require a great deal of time and resources. While potentially preventing the next global pandemic or treating chronic diseases justifies the costs, the threat research may pose to corals is troublesome. Corals, as small and slow growing organisms, struggle to provide the adequate amount of product or a substance needed for research and development. While any type of harvest or removal poses potential risks to the already struggling corals, too much removal is sure to harm corals. Reefs hold massive medical potential, but any research must be conducted in an ethical and sustainable manner \citep{AndrewWBruckner_1970}.

\subsection{Food Sources}

Despite only making up 1\% of the ocean, coral reefs support over a quarter of all fish in the ocean \citep{noaa}. They are densely populated regions which provide food to people around the globe. It is no wonder six million fishermen—a quarter of the world's small-scale fishermen, actively fish the reefs \citep{coralreefalliance_2021}. Coral reefs provide both employment and food for the vast majority of the 350 million people living within 50km of Southeast Asian coastlines. In addition to harvesting food, reefs fisheries are vital economic players in Southeast Asia; fisheries generate over 2.4 billion US dollars every year. Indonesia and the Philippines dominate the majority of this revenue, as the total annual economic benefit is estimated at USD 1.6 billion and USD 1.1 billion, respectively.  Fisheries compose the largest section of these earnings, and their true revenue is often underreported, as small-scale, sustainable fishermen often do not report reef catches in monetary values. These fishermen and their families are generally reliant on reefs for substance, and thus are more impacted by reef degradation \citep{RAR}.

Fish and seafood from the reefs, on average, make up around 40\% of total animal protein intake for Southeast Asian diets, explaining why fishing in the Coral Triangle has continued to increase as global rates have flatlined. The Coral Triangle includes Indonesia, the Philippines, Papua New Guinea, the Solomon Islands, and Timur-Lee, but a majority of the coastal people living in the Coral Triangle reside in either the Philippines or Indonesia. Fish-related employment makes up 2\% of the entire working population in the coral triangle, a whopping 4.6 million people. Over 18 million people, or 5\% of the total Coral Triangle population, directly rely on reefs for food. Fishing has been estimated to account for over 5\% of the gross domestic product of the countries in the Coral Triangle. The densely populated reefs rich with fish provide food for all types of people, from the average consumer shopping at a market to the indigenous person fishing purely for personal sustenance. Nevertheless, reef fisheries are crucial to sustaining Southeast Asia’s large and growing population \citep{coraltriangle}.

All socioeconomic groups across Southeast Asia are reliant in some way on coral reefs for food sources, however, lower classes are often more dependent on reefs for sustenance. The indigenous peoples of Southeast Asia are one of these heavily dependent groups. The Tagbanua People offer a case study on reef reliance representative of many other indigenous groups in the Coral Triangle. The Tagbanua People, or the People of the Village, inhabit Coron Island, Palawan on the Southwestern side of the Philippines (Figure~\ref{fig:TagbanuaFishing}). Thought to be descendants of the Tabon Man (16,500-year-old remains found— the earliest appearance of modern man), the Tagbanua are one of the oldest groups in the Philippines. They occupy a coastal region and have a very close relationship with both the water and the reefs \citep{4826000120100501}.

\begin{figure}[h!]
\centering
\includegraphics[width=0.65\linewidth]{images/tagfish}
\caption{:The 12-15 million indigenous people living in the Philippines make up between 10-15\% of the population. 110 groups are officially recognized, yet there are many more who have not been officially recognized by the Philippines’ government. Above pictured are the Tagbanua, one of these indigenous people groups of the Philippines. They either spearfish or fish from simple wooden boats like seen above. These boats are used to traverse the shallow reefs and harvest fish, lobster, squid, octopus, and other protein sources. Many other indigenous groups maintain a similar reef-relationship, one consisting of simple fishing practices, and transportation methods \citep{4826000120100501}.}
\label{fig:TagbanuaFishing}
\end{figure}

The Tagbanua have been categorized as a semi-nomadic, seafaring native group. After agriculture, fishing is the second main economic activity. A vast majority of fishermen practice either traditional hook-and-line fishing or spearfishing and take only what is needed from the reefs. They fish both for personal sustenance and trade, but they donate excess catches to community members in need whenever possible. Tagbanua fishing is highly sustainable, as they hold high respect for the reefs and its inhabitants. As reef degradation continues, millions across Southeast Asia lose access to food and protein sources, but more directly dependent groups like lower socioeconomic classes and indigenous groups are disproportionately affected \citep{4826000120100501}.

\subsection{Tourism and the Tagbanua}

Reefs attract tourists for various recreational activities. Shallow waters and high levels of biodiversity make for world-class snorkeling and diving. For the same reasons, boat tour charters and submarines also thrive from reef tourism. Coral reefs attract spear fishers as well, as there are many fish species which can be shot and caught, all while in relatively shallow water.  Because reefs are natural breakwaters, they create world-class, incredibly consistent waves, which attract surfers from around the globe. Almost every world-famous wave breaks over the reef, and the shallow reef waters provide fast, hollow waves perfect for barrels—every surfer's dream. Bali, Indonesia, is just one example of a destination which attracts surfers of all skill levels solely to surf the reef-breaking waves \citep{wwfindex}. Over 9\% of all coastal tourism value is attributed to the coral reefs.  They provide many incentives for travel, and this tourism brings both positive and negative effects \citep{reeftourism}.

Globally, reef tourism is estimated to generate USD 36 billion yearly, and the mean value of reef per square hectare is thought to be around USD 96,000. Many recreational reef activities have been commercialized. Coral reef tourism helps to stimulate and support local economies. The general appeal of reefs has developed strong tourist economies in the Southeastern coastal areas. This reef-adjacent tourism, including hotels, rentals, restaurants, and more, has created additional jobs and stimulated economic growth. 30\% of the world’s reefs are involved in the tourism sector, and they provide employment to over one hundred different sectors. The importance of reef tourism has provided economic incentives for reef conservation and sustainable practices. Tourism provides opportunities for conservation education and great awareness, which allows issues to be displayed on a global scale instead of only in local communities. Tourists may also become emotionally attached to reefs, providing additional incentives to support reef restoration and protection efforts, either directly or financially \citep{reeftourism}.

While tourism does aid in economic development, it also can lead to heightened economic disparity and steal wealth from local communities. Foreign corporations and large companies often dominate tourist areas, making it so local people cannot own businesses, but instead must work for them. As tourism can be such a dominating economic force, it can also edge out other economic sectors, and large corporations can easily obtain monopolies on entire regional economies. This allows them to mistreat and underpay their workers, who are typically mainly local people \citep{reeftourism}.

Reef tourism has also contributed to countless culturally insensitive developmental projects, which negatively affect the indigenous groups, including the Tagbanua. They are consistently pressured to lease out their land for tourism purposes, as their ``pristine'' surroundings are easy selling points. Tourists who have come close to indigenous lands have left pollutants and contributed to reef degradation, further harming native populations. As reefs are continually degraded, native peoples are continually harmed. The Tagbanua’s recent self-determination and land reclamation efforts are complicated by tourism and the economic greed that accompanies it. The reefs surrounding indigenous lands are highly desired, as they have not been subject to as many unsustainable practices. It is very difficult for indigenous groups like the Tagbanua to become involved in the ``democratic process however,'' as previously only those who are formally dressed and speak fluent English are allowed to attend and participate in policy dialogues. As a result, indigenous groups can be purposefully underrepresented and their land is commonly encroached on, a practice exacerbated by tourism and perfectly exemplified by the struggles of the Tagbanua \citep{4826000120100501}.

Tourism also contributes heavily to marine degradation and reef destruction. It breeds development, as hotels, piers, roads and other structures are built to accommodate tourists. This development adds sediment and other contaminants to reefs, as previously discussed in Section \ref{sub:mdoc} \citep{reeftourism}. Tourists themselves can also harm reefs during their on-reef recreational activities. They may step on corals while diving and kick up sediment while snorkeling. Tourists often are also not environmentally conscious, and they leave behind trash which pollutes reefs. They may wear chemically harmful sunscreen while in the water, leaving pollutants and chemicals in the reefs \citep{coralreefalliance_2021}. From buying unsustainable coral jewelry to accidentally spearing an endangered fish, tourists themselves can threaten reefs in countless ways. Tourism as a whole can be beneficial or detrimental to reefs, depending on how it is implemented, who holds the power, and how well-educated the tourists are \citep{reeftourism}.

\subsection{Coastline Protection}

Worldwide, over 200 million people depend on coral reefs for protection from storm surges, rough waves, and flooding \citep{wwfindex}. Waves break over patches of shallow water, and because reefs extend vertically up the water column, they serve as natural breakwaters. They reduce the power, size, and energy of waves \citep{Beck}. Reefs have been shown to reduce wave energy by up to 97\%, and wave height by 84\%. Waves first break over the shallowest part of the reef, the reef crest, which alone can dissipate around 86\% of a wave’s energy. Without reefs, waves would break over sandbars closer to the coast, flooding terrestrial settlements and destroying dwellings. \citep{Pelton_2017}.

In addition to acting as breakwaters, reefs provide buffers for coastlines against floods, storms, and erosion. Southeast Asian reefs protect billions of dollars' worth of coastline, as reefs continue to dissipate wave energy during natural disasters. In the monsoon climates of both Indonesia and the Philippines, typhoons occur often, and without reefs as protectors, coastlines would be hit significantly harder. Reefs protect coastlines not just from typhoons, but also from hurricanes and tsunamis. The global cost of storm damage is estimated to double without reef protection. The submerged reefs are crucial natural defense systems, protecting millions of coastal inhabitants in Southeast Asia and around the globe \citep{Beck}.

\fbox{
\begin{minipage}[h!]{0.9\textwidth}
\subsection{The Sentinelese People}

Another Southeast Asian indigenous group in close relation with the reefs are the Sentinelese people of the Indian Ocean’s North Sentinel Island. This group drew international attention when they killed an American tourist who traveled on their land illegally, and then shot arrows and threw spears at the helicopter sent to recover his body. The estimated 80-150 people currently living on the island are some of the last untouched native peoples. Their sovereignty is protected by North Sentinel Island’s lack of natural harbors and the surrounding sharp, shallow reefs. The Indian government has also enacted laws protecting their sovereignty, leaving them very isolated from foreign interference \citep{Smith}.

\includegraphics[width=\linewidth]{images/sentpeeps}

While little is truly known about the Sentinelese, their relationship with the reef is relatively understood. Their hunter-gatherer lifestyle is mainly fueled by the reefs. They use small, narrow, outrigger canoes and long poles to navigate and harvest the shallow reefs in the surrounding waters. A vast majority of their consumed protein comes from the reefs; they fish, harvest, and consume both reef fish and invertebrates. It can be assumed that many other Southeast Asian indigenous groups in coastal areas maintain a similar relationship with coral reefs, though they may lack government protection. Reef degradation is often viewed solely in ecological terms, but we must also consider its anthropologic effects, especially on indigenous populations \citep{Smith}.
\end{minipage}
}

\section{Reef Restoration, Protection, and Conservation}

\subsection{Ecosystem Based Adaptations}

Coral reefs, in their current state, cannot be restored passively by simply eliminating stressors. Active restoration efforts are critical for the survival of the corals and the reef ecosystem. Two highly successful restoration efforts include ecosystem-based adaptations and the formation of marine-protected areas. Reef restoration is a complex task requiring community engagement, government support, foreign aid, and restorative projects that are backed by research \citep{14551496520201201}.

Ecosystem-Based Adaptations, or EbAs, are a form of active coral restoration which aims to rebuild natural capital and ecosystems in order to aid in preservation and protection, while also promoting and defending vulnerable ecosystems, like the coral reefs. The first EbA restorative efforts took place in the reefs of the Philippines, but they have since been expanded for worldwide use. In Southeast Asia, EbA are still very prevalent and have had great success when it is properly implemented. It is mainly carried out at the community level but offers involvement opportunities to a wide range of people and groups, from tourists to NGOs to scientists and more. EbA, especially in Southeast Asian coastal communities, drives co-benefits between people and nature, as reef restoration coincides with economic growth and prosperity \citep{14551496520201201}.

Coral gardening is a widely used EbA composed of two phases. In the first, the nursery phase, corals are grown via asexual coral propagation. Coral colonies that have already been destroyed or broken off, or corals of opportunities, are harvested and brought to nurseries where they are grown for the next 6-12 months. Nurseries can be either ocean-based or land-based. Ocean-based nurseries are cheaper but more vulnerable, while land-based nurseries resemble labs and are more expensive and protected (Figure~\ref{fig:coralrestore}). During this phase, a hundred colonies can easily become thousands, which are then implanted directly into the reef \citep{cgarden}. A newly emerging and highly successful type of coral gardening is microplanning. Newly grown colonies are commonly arranged a few inches apart from each other on a rock-like base and subsequently doused with a growth elixir in a lab. The corals grow up to 25 times faster and connect to form one massive colony, which is then later transplanted into a living reef ecosystem \citep{Morin_2014}.

Direct transplantation, another EbA, involves transplanting broken corals to other locations. This strategy is often employed after storms, as rough waters may break pieces of coral away from their main structure which survive without anchoring. Colonies are then relocated to areas with more inhabitable conditions where odds of survival and reproduction are higher. Corals need stable, secure structures, which are fabricated by securing a colony to a cement base attached to the seafloor with cable ties or by simply affixing cable ties to corals which are driven into the seafloor. Direct transplantation also applies to the second phase of coral gardening. EbAs commonly overlap and build off each other to achieve the best restorative results \citep{areef}.

\begin{figure}[h!]
\centering
\includegraphics[width=0.7\linewidth]{images/coralrestore}
\caption{Coral restoration techniques and EbAs often do not fall into a single category but are a mix of multiple. As seen in this picture, this open-ocean coral nursery is also acting as an artificial reef. As the corals are propagated, they will be either transplanted to preexisting communities or remain as an artificial reef, depending on the condition of the reefs in the surrounding water. EbAs must be adaptive to maximize their potential, as the natural world is a dynamic, ever changing system.\citep{cgarden}}
\label{fig:coralrestore}
\end{figure}

Artificial reef creation, another EbA, has seen high levels of success in reef restoration. For a reef to be artificially made, pre-existing materials like oil rigs and other large steel or concrete structures, are sunken in reef areas and act as anchors or attachment zones for reefs to form (Figure~\ref{fig:coralrestore}). As reef restorative technology has progressed, artificial reef creation has progressed with it. Mineral accretion devices such as Biorock have been very promising and hold great potential. These electrified artificial reefs are essentially metal webs with weak electric current flowing through them, which not only prevents rusting but also causes minerals to precipitate out of the water and collect on the metal. The attached coral colonies can in turn make calcium carbonate skeletons three-to-five times faster than before, as they have an abundance of substrate and easy access to it as well. Mineral accretion devices also protect corals from bleaching and other external disturbances, making them highly effective options for artificial reef promotion. They are expensive and hard to maintain, however. Because of the funding and expertise required, mineral accretion devices are difficult to implement at a local level without an external partner \citep{areef}.

Reef restoration via EbAs is by no means cheap, but it is a necessary step in coral reef protection and restoration. The benefits far outweigh the costs, for future issues stemming from a lack of coral reefs will cost more fiscally and harm already struggling populations. EbA’s require significant government support, which is harder to obtain in the developing coastal countries of Southeast Asia than in developed countries. Developing countries have fiscal advantages, however, as restorative efforts are estimated to be thirty times cheaper than in developed nations, because community and volunteer participation significantly lower costs. Case studies in both Indonesia and the Philippines confirmed that local participation and community involvement not reduced costs of restoration, but also contributed to long-term benefits in sustainability and within the community \citep{14551496520201201}.

\subsection{Marine Protected Areas}

Marine Protected Areas, or MPAs, are areas that purposefully restrict human activities in effort to either promote conservation or restoration. They give fish and other reef populations areas to recover, as they cannot be affected by overfishing and other unsustainable practices. As coral conservation efforts continue, the specific areas of the Indo-Pacific have been made MPAs, which have led to significant positive reef growth and restoration. The Philippines Marine Sanctuary Strategy of 2004 declared that 10\% of coral reefs would be no-take MPAs by 2020, but currently, only 2.7-3.4\% of reefs are MPAs. On top of this, 85\% of the Philippines MPAs are concentrated in only two areas, leaving them highly susceptible to ecological disturbances \citep{10.2307/40603378}. 

MPA networks have been deemed more effective than single, large-area MPAs, as disturbances concentrated in one area, like an oil spill or heat wave, will not destroy the entire implemented MPA system. Smaller networked MPAs are also more easily managed locally, a vital component of MPA effectiveness. A large, singular MPA may stop an entire community from harvesting reef resources, making multiple small MPAs much more suitable options when still considering anthropocentric factors\citep{Keller2009ClimateCC}. Finally, effective MPAs must be adaptive, as reefs are constantly changing, but also must be long-term and well-enforced. MPAs are much more easily enforced when local communities and governments are involved with the decision-making process and implementation, as shown by Locally Managed Marine Areas, or LMMAs. LMMAs are not managed in the traditional top-down style with a national government presiding over the area, but instead are managed at local levels in conjunction with local traditions and community practices. In the Indo-Pacific, regions where LMMAs have been implemented have led to high levels of reef renewal and restoration \citep{cgarden}. 

When MPAs are properly implemented and effectively managed, they have been proven to be highly successful. Indonesia's Convention of Biological Diversity and Sustainable Development Goals pledged to turn 32.5 million ha, or 10\% of the total coral reefs, into MPAs. A study conducted across 622 Indonesian coral reefs spanning 17 geographic regions concluded that successfully implemented MPAs had significantly higher biomasses (1.4 times higher). In general, reefs need a measured biomass of 500-650 kb/ha to sustain functionality, and this was accomplished in around 40\% of all Indonesian MPAs surveyed. In contrast, only 25\% of open reefs were above this threshold. Both no-take and gear-restricted MPAs promote biomass by providing safe spots for struggling reef populations to recover without external, anthropocentric threats. This allows fish and other reef populations to better recover from overfishing and other unsustainable activities, further stressing the importance of MPAs \citep{https://doi.org/10.1111/conl.12698}. 

One less common type of MPA results not from governmental restrictions, but indigenous efforts. To many indigenous groups, particular reef sections are culturally significant, and some waters are even thought to be sacred. One of these groups, the already-discussed Tagbanua, believe some nature spirits, or Panyain, dwell in reefs. Panlalambot, a giant-humanoid octopus, is one of these reef-dwelling Panyains that inhabits a reef section in off Coron Island. As a result, these areas are highly protected and act as fish sanctuaries, allowing certain struggling populations to recover from the many unsustainable practices and pollutants effecting marine ecosystems. While these ``MPAs'' are culturally motivated, they are still highly effective and beneficial to reef restoration efforts \citep{4826000120100501}.

\subsection{Intersectional Efforts} \label{sub:ie}

While local management and grassroots projects have been highly effective, they must be paired with involvement by national government and programs in order to enact effective, long-lasting reef protections and restorative efforts. Intersectional efforts also maintain an environmental justice lens, as both environmental and anthropogenic factors are considered in policy making. This protects marginalized groups like the indigenous people of Southeast Asia, for purely environmental efforts often do not consider the needs of groups directly reliant on specific ecosystems. When consulted in policy making, indigenous groups are not only given their deserved voice, but also offer generational knowledge. These indigenous groups and local communities also do not carry the same financial or lobbying ties and biases as other groups may. Generally, they want what is best for the reefs, as it is best for themselves as well. This anthropologic lens developed through intersectionality promotes an inclusive approach to ecological restoration \citep{13772149520190801}. 

Indonesia’s Coral Reef Rehabilitation and Management Program, COREMAP, demonstrates the positive outcomes possible through collaboration between national government and local communities. COREMAP has successfully implemented over 350 collaborative management plans between local communities and governments. Through COREMAP, rare and endangered species have rebounded. In six of the seven project districts, COREMAP policies and initiatives lead to a 17\% growth in coral cover, as well as a 20\% income growth among district residents, illustrating the correlation between coral restoration and economic prosperity. COREMAP’s national scale and intersectional nature allows for both a broader range of achievements while still accounting for the interests of local communities \citep{Marinescienceresilientcommunities}.

\section{Coral Reefs: Conclusion}

\begin{figure}[h!]
\centering
\includegraphics[width=0.9\linewidth]{images/reefbigchillin}
\caption{Pictured is a healthy Southeast Asian coral reef. Bursting with color and biodiversity, reefs like these support lives for thousands of species, including humans. Reef restoration and protection are crucial to maintaining already thriving reefs like this and also reviving the many degraded reefs of Southeast Asia. Reef conservation will allow these coral reefs to remain rich in biodiversity and continue sustaining the indigenous peoples and coastal communities of Southeast Asia.}
\label{fig:A Healthy Southeast Asian Coral Reef}
\end{figure}

From localized threats, like unsustainable fishing and pollutants, to global stressors such as ocean acidification and climate change, coral reefs are highly threatened. These biodiverse ecosystems support millions across Southeast Asia; not only do they provide food and protection, but many coastal economies are directly reliant on reefs. Threats to reefs are threats to people, especially to already struggling poor coastal communities and indigenous groups of Southeast Asia. Indigenous groups are commonly disproportionately affected by reef destruction, as they are more directly reliant on them. Reef degradation in turn is not simply an ecological issue, but an environmental justice one was well. Pollutants and malpractices must be eliminated, but reefs will also require active restoration efforts, as anthropocentric actions have degraded reefs to the point where they are no longer capable of healing independently. Saving the coral reefs is not saving a few fish species; it is saving millions of organisms and the millions of people who rely on them. Coral reef conservation should not be viewed simply as an ecological issue, but as a complex problem requiring multidisciplinary efforts to achieve success.


\chapter{Air Pollution \& Social Justice in Hong Kong}
\chapterauthor{Neenah Vittum}

\section{Introduction}

As many Asian countries have undergone rapid industrialization and rural--to--urban migrations, air pollution has become a key issue. The consequences of air pollution, like many other environmental hazards, intersect with a number of different social issues. To better understand the complex intersections between environmental justice, marginalization and air pollution's human health effects, we will focus specifically on Hong Kong. Hong Kong is an interesting case study for urban air pollution and environmental justice issues because of its substantial wealth gap and unique housing provisions \citep{fan2012differential}.  

This chapter will begin with an overview and history of Hong Kong that is not strictly relevant to the topic of air pollution, but provides helpful context for understanding discussions of the region's government, economy, and people. I will then summarize specific sources of air pollution and current air pollution issues, and their human health implications. After that, I will discuss Hong Kong's income inequality and housing policies as they relate to air pollution exposure and human health effects. I will also briefly touch on issues related to air pollution including indoor air pollution and open spaces. Finally, I will discuss environmental activism and policy in Hong Kong. 

\section{A Brief Overview of Hong Kong}

\begin{figure}
\includegraphics[width=\linewidth]{images/HK-air-pollution/hong-kong-china-map}
\caption{A map of Hong Kong in relation to China}
\label{HongKongMap}
\end{figure}

Hong Kong is a small special administrative region (SAR) of China consisting of the New Territories, Hong Kong Island, and the Kowloon peninsula, which borders China's southern Guangdong province to the north \ref{Fig. 2}. Despite its small size,	1110.18 km$^2$ \citep{govhk1}, Hong Kong is not a singularly urban area. The territory consists of rural, mountainous regions and densely populated communities concentrated near the coast and ports. Though not administratively defined, Hong Kong's urban population is primarily concentrated in the Victoria City and Kowloon areas. 

\begin{figure}
\includegraphics[width=\linewidth]{images/HK-air-pollution/hk-map}
\caption{Political Map of Hong Kong}
\label{Fig. 2}
\end{figure}

Though Hong Kong is a small physical territory, the area's population as of July 2020 is estimated to be a little over 7.5 million people, making it the fourth most densely populated region in the world \citep{govhk2}. 

\section{Hong Kong's History}

Prior to British arrival in the first half of the 19th century, Hong Kong was home to around 4000 people, most of whom were involved in agriculture or fishing \citep{evans2008introduction}. After the First Opium War, which lasted from 1839--1842, the British Empire acquired Hong Kong Island \citep{mark2009lack}. Hong Kong Island's sparse population around the time of British acquisition led then-prime minister Lord Palmerston to falsely refer to the area as a ''barren rock'' \citep{lupton1965government}. The British Empire took control of the nearby Kowloon Peninsula in 1860, and later leased the New Territories for 99 years beginning in 1888 \citep{mark2009lack}.
	
Despite these small expansions, British colonial policy in Hong Kong was not focused on territory or land. The British Empire was mainly interested in Hong Kong to promote trade and commercial interests \citep{tsang2003modern}. Significant territorial expansion was both too expensive and too difficult; as a result, the British Empire expended funds to simply maintain Hong Kong \citep{lupton1965government}. 
	
The British arrival and subsequent development of the area for trade and commercial purposes initially attracted Chinese immigration to the area, with Hong Kong's population increasing to 70,000 people by the 1850s \citep{evans2008introduction}. Many of these first workers were transient immigrants who moved to Hong Kong for work with the intention to return to mainland China \citep{tsai1993hong}. In order to settle the area, early British housing policy subsidized Chinese settlement \citep{lai2011discriminatory}. 
	
Early cooperation between British and Chinese residents, and the British Empire's economic focus on Hong Kong did not prevent issues of division and discrimination. As the British Empire's territory expanded, first to the Kowloon Peninsula and later to the New Territories, their colonial administrative power and discriminatory tendencies grew \citep{lai2011discriminatory}. 

In 1895, the British Empire's government enacted the Light and Pass Ordinance, restricting Chinese residents to a 9pm curfew unless they had a lantern and written permission from their employer \citep{lai2011discriminatory} \citep{munn2013anglo}. Similar forms of discriminatory policy were present in certain housing laws as well with some contractual laws requiring ``European type houses'' as the architectural blueprint for new constructions \citep{lai2011discriminatory}. Furthermore, The Hill District Reservation Ordinance of 1904 prohibited Chinese people from residing within the Hill District zone, an area near Hong Kong's highest elevation, unless they were servants living with European residents \citep{lai2011discriminatory}. 
	
These policies were repealed and both racial segregation and discrimination eased significantly following World War I \citep{lai2011discriminatory}. In the decades following, Hong Kong underwent a form of decolonization as Britain withdrew significantly from it and other colonies around the world \citep{mark2009lack}. The region began to transition and prepare for its return to China as a SAR at the end of the New Territories' lease in 1997. 
	
Hong Kong's semi-sovereignty and current Chinese control does not mean that its colonial past has completely disappeared. Remnants of British influence are ever present in Hong Kong's market system, and the region's settlement patterns and land ownership are rooted in British colonialism \citep{lai2011discriminatory}. 
 
\section{Air Pollution in Hong Kong}

Hong Kong, like many other cities, experiences air pollution issues. A combination of poor air circulation from tall buildings, and dense activity including both shipping and motor vehicle traffic has underscored Hong Kong's air pollution issues. Hong Kong's air quality is monitored by the Hong Kong Environmental Protection Department (HKEPD), whose network of stations covers Hong Kong Island, Kowloon, and the New Territories \citep{xia2005modelling}. In those regions, the HKEPD has a total of 16 stations, 13 general stations installed on rooftops and 3 roadside stations \citep{govhk3}, which consistently measure common air pollutants including sulfur dioxide (SO$_{2}$), nitric oxide (NO), nitrogen dioxide (NO$_{2}$), carbon monoxide (CO), and ozone (O$_{3}$), and particulate matter such as PM$_{2.5}$ and PM$_{10}$ \citep{xia2005modelling}. The locations of these stations are shown in Figure~\ref{fig:HongKongAQStations}.

\begin{figure}
\includegraphics[width=\linewidth]{images/HK-air-pollution/air-quality}
\caption{Hong Kong's Air Monitoring Stations \citep{govhk3}}
\label{fig:HongKongAQStations}
\end{figure}

Hong Kong's air pollution levels come from a number of different notable sources. The primary source of Hong Kong's air pollution is motor vehicle emissions. Motor vehicle emissions are a major source of air pollution in most major cities throughout the world \citep{xia2005modelling}. Urban areas, especially in recent history, tend to house a larger portion of the world's population. This increase in activity in urban areas has been shown to escalate air pollution issues \citep{wong2019vertical} \citep{gurjar2010human}. Air pollution emitted from motor vehicles includes nitrogen oxides (NO$_{x}$), CO, volatile organic compounds (VOCs), and PM \citep{xia2005modelling}. We will see later on how these concentrated amounts of these pollutants can cause detrimental human health issues. 

Hong Kong also has high concentrations of O3 likely because of its sunny climate, which facilitates the creation of O$_{3}$ from other oxides \citep{jo2013polluted}. 

\begin{figure}
\includegraphics[width=\linewidth]{images/HK-air-pollution/Hong-Kong-Air-Pollution}
\caption{Hong Kong's Skyline. Photo: Nora Tam}
\label{fig:HK-air-pollution-image}
\end{figure}

Another source of air pollution in Hong Kong is its harbor and shipping port. Hong Kong's Victoria Harbor is one of the busiest shipping ports in the world \citep{ng2013policy}. Shipping contributes to about 36\% of SO$_{2}$ emissions in Hong Kong. Regulation in recent years has worked to greatly and effectively reduce shipping emissions, but persistent issues still remain \citep{mason2019air}. 

\begin{figure}
\includegraphics[width=\linewidth]{images/HK-air-pollution/hong-kong-port}
\caption{Hong Kong's shipping port. Photo: Roy Issa}
\label{fig:HongKongPort}
\end{figure}

\section{Air Pollution and Human Health}

The human health effects of air pollutants are part of what classifies certain chemicals as air ``pollution.'' These effects have been widely studied in many different parts of the world. \citet{gurjar2010human} developed a risk of mortality model to examine the human health effects of air pollution in megacities \citep{gurjar2010human}. They found that elevated levels of total suspended particles (a measure of PM), SO$_{2}$, and NO$_{2}$ led to more cases of respiratory mortality. Furthermore, they found a higher number of hospital admissions due to chronic obstructive pulmonary disease (COPD). Patients with COPD have been known to experience more symptoms on days when the air quality is worse \citep{harre1997respiratory}. As we saw in the previous section, Hong Kong has significant problems with all of these pollutants and thus threats to human health are cause for concern. 

A number of Hong Kong-specific studies have looked into the human health effects of air pollution. \citet{kan2010short} included Hong Kong in their study of SO$_{2}$ and daily mortality in a number of cities across Asia. SO$_{2}$ has been known to affect the respiratory system and cause irritation of the eyes (WHO 2011). \citet{kan2010short}'s short-term study found an association between SO$_{2}$ and mortality, concluding that SO2can cause cardiorespiratory health effects, act as a respiratory irritant, and can cause cardiovascular abnormalities by action as a bronchoconstrictor \citep{kan2010short}. Furthermore, SO$_{2}$'s association with shipping has been particularly worrisome for Hong Kong as a shipping port. A 2012 assessment by Hong Kong's Civic Exchange Department concluded that SO$_{2}$ was responsible for 519 premature deaths across the Pearl River Delta, a region encompassing Hong Kong, China's Guangdong Province, and Macau (another SAR of China). Of those 519 deaths, 385 were in Hong Kong \citep{mason2019air}. Experts estimate high exposure for a population as dense as Hong Kong with a probable 3.8 million people living near coastal areas directly exposed to shipping emissions like SO$_{2}$, NO$_{x}$, and PM$_{10}$ \citep{lohvision}. 

Air pollution has also been closely tied to COPD. In Hong Kong specifically, several studies have shown an increased risk of hospitalization associated with air pollution \citep{ko2007temporal} \citep{wong1999air}. Gaseous pollutants tend to exacerbate human health effects and lead to more hospitalization when the weather is cold, which \citet{qiu2013season} confirmed in their 2012 study of Hong Kong. Though Hong Kong's winter months are milder than most cold temperatures associated with increased hospitalizations, they believe that Hong Kong's relatively warmer winter and lack of reliance on central heating encourages residents to open their windows for ventilation leading to prolonged outdoor exposure \citep{qiu2013season}. Furthermore, \citet{qiu2013season} found that humidity plays a role in COPD hospital admissions. Lower humidity saw fewer effects of  PM$_{10}$, SO$_{2}$, and NO$_{2}$ on COPD admissions. They believe that high humidity contributes to the protection of the windpipe and the dissolution of certain elements of gaseous chemicals \citep{qiu2013season}. 

Air pollution has also been associated with Ischaemic heart disease (IHD), one of the leading causes of death worldwide (World Health Stats). In Hong Kong, both mortality and hospital admissions have been associated with PM$_{10}$, PM$_{2.5}$, NO$_{2}$, O$_{3}$, and SO$_{2}$ \citep{san2015association}.

\section{Income Inequality, Housing, and Air Pollution}

The severity of air pollution and its human health effects in a region as densely populated as Hong Kong calls into question issues of unequal exposure and differential environment quality. There is no question that environmental injustice exists in air pollution exposure throughout Hong Kong. A number of studies have documented the severity and nuances of the relationship between air pollution exposure and socioeconomic factors. \citet{li2018air} used a data-driven air pollution model to estimate PM$_{2.5}$ exposure across small geographic regions of Hong Kong, concluding that a positive statistically significant relationship exists between social deprivation and PM$_{2.5}$ exposure. Similarly, \citet{wong2008effects} found that air pollution mortality rates associated with NO$_{2}$ and SO$_{2}$ exposure were higher in areas with more social deprivation. These results are shown in \ref{Fig. 6}. In order to better understand the nature of environmental inequality and ``social deprivation'' in Hong Kong, we will explore a few key aspects of Hong Kong housing policy and income inequality.

\begin{figure}
\includegraphics[width=\linewidth]{images/HK-air-pollution/sd-graphs}
\caption{The results of \citet{wong2008effects}'s study indicating increased mortality rates for higher levels of social deprivation}
\label{fig:HKmortality}
\end{figure}


\subsection{Income Inequality and Zoning Laws}

Regional social deprivation is partially characterized by poverty and low annual incomes. As previously mentioned, Hong Kong's wealth gap is large and income inequality is high. As of the government's latest census in 2016, Hong Kong's GINI coefficient, a measurement of income distribution on a scale from 0 to 1, is 0.539 \citep{wong2018}. For the sake of comparison, Singapore's GINI coefficient was 0.458 in 2016, and the United States and United Kingdom recorded coefficients of 0.394 and 0.358, respectively, a couple years prior \citep{yiu2017}. Many point to the territory's emphasis on free market activity, general lack of government regulation, and history of development as reasons why Hong Kong's income inequality is so high \citep{zhao2005economic}.
	
Hong Kong's income inequality and urban population density issues are inexplicably tied. The government's land-zoning practices prioritize revenue collection through real-estate development, a trade-off of low market taxes, and consequently concentrate the population into a variety of high-density living situations \citep{tang2017distribution}. 


\subsection{Public Housing and Residential Mobility}

Economic mobility is an important consideration in understanding issues of environmental justice. Many traditional environmental justice issues stem from economic and residential mobility; in other words, more affluent residents have the means to move away from environmental hazards and concerns \citep{crowder2010interneighborhood}. Mobility is not as obvious in densely populated areas like Hong Kong. As a result, many researchers have concluded that while differences in exposure to air pollution exist, they are typically at the most extreme levels of wealth and poverty \citep{li2018air} \citep{wong2008effects}. 

Political scientist \citet{stern2003hong}'s 2003 article Hong Kong Haze concisely pointed out that several then-members of Hong Kong's government lived in a neighborhood named Deep Water Bay, which was known as a ``green setting'' characterized by tranquility and low pollution levels \citep{stern2003hong}. It is also one of the wealthiest neighborhoods in the world \citep{olsen2015}. Even with the pervasive and wide-spread nature of air pollution, those with wealth and power are able to buy their way out of potentially dangerous living conditions. Those with higher incomes and higher levels of wealth are able to afford the elevated cost of a clean environment. In other words, they benefit from economic mobility. 

\begin{figure}
\includegraphics[width=\linewidth]{images/HK-air-pollution/deepwater-bay-hong-kong}
\caption{Hong Kong's Deep Water Bay neighborhood. Photo: Wpcpey}
\label{fig:HK-deepwater}
\end{figure}

Private housing, in general, allows for a certain degree of choice that is not afforded to residents of public housing \citep{fan2012differential}. Hong Kong's population is split nearly evenly between private housing provided by real estate developers and public housing subsidized by the government \citep{fan2012differential}. Public housing is a necessity given Hong Kong's extreme income inequality issues. On average, those who live in private housing earn double the amount of those in public housing \citep{fan2012differential}. Public and private housing appear in different concentrations throughout Hong Kong's territory with private housing generally in older areas and public housing in newer regions \citep{fan2012differential}. This distribution is shown in \ref{Fig. 8}.

\begin{figure}
\includegraphics[width=\linewidth]{images/HK-air-pollution/HONG-KONG-PUBPRIV-MAP}
\caption{\citet{fan2012differential} sampling of public and private housing throughout Hong Kong}
\label{fig:HK-pubrprivMap}
\end{figure}

The residential mobility of Hong Kong's private housing market is the cause of more environmental injustice. \citet{fan2012differential} found that differential exposure of vehicular air pollution among different levels of socio-economic status existed in private housing, but did not exist in public housing. This may be because affluent private residents have the capacity to change their housing preferences and prioritize environmental quality \citep{fan2012differential}. 

\section{Indoor Air Pollution}

Hong Kong's air quality pollution issues extend to indoor spaces as well \citep{stern2003hong}. This is an important consideration given the way in which Hong Kong's population density and housing shortage problems have led to increasingly cramped living conditions.

Indoor air quality can be negatively affected by activities like smoking, cooking, or burning incense. Though these activities are not necessarily associated with low socioeconomic status, it is important to consider the volume of space in which they are performed. The level of total suspended particles released through these activities is higher in smaller areas because they have less room to disperse. These high concentrations lead to prolonged exposure. Additionally, those who are unable to afford air conditioning may opt to open windows during the warmer months. This brings the negative health effects of outdoor air pollution into the home \citep{stern2003hong}.


\section{Open Spaces}

Hong Kong's zoning practices also influence the territory's distribution of ``Open Space Zones.'' Open Space zones are designated for public recreation in the midst of Hong Kong's urban areas. These Open Space zones are not as explicitly tied to air pollution issues, but they do reflect general trends in environmental inequity in Hong Kong. A proportionally higher number of Hong Kong's Open Space zones are in expensive low-density neighborhoods. Furthermore, Hong Kong's government has future plans to continue open space zoning and construction in these areas, as well as commercial centers in order to encourage consumption and tourism \citep{tang2017distribution}. Hong Kong's low-middle income neighborhoods, the most densely populated urban areas of the territory are not afforded the same opportunities. 

Furthermore, urban streets can be categorized as either open streets or street canyons. Open streets have built structures on one side, while street canyons have built structures on both sides \citep{wong2019vertical}. Street canyons, depending on their positioning and wind patterns, may lead to less circulation and higher exposure to air pollution \citep{wong2019vertical}. With what we know about Hong Kong's tendency to place open spaces, which would imply open streets, in wealthier low-density neighborhoods, the open space issue becomes even more explicitly tied to air pollution beyond general trends of environmental injustice.

\begin{figure}
\includegraphics[width=\linewidth]{images/HK-air-pollution/hong-kong-open-spaces}
\caption{An open space at Mong Kok's Langham Place Mall. Photo: Wikicommons}
\label{fig:HK-openspace}
\end{figure}

\section{Environmental Activism in Hong Kong}
Based on what we have learned about inequality and air pollution exposure, it is important that Hong Kong's environmental regulations and environmental activism be justice-based. Given Hong Kong's complex administrative and economic situation, this may not be realistically achievable. 

Considering its long history, Hong Kong's environmental movement, which rose to prominence in the 1980s and 1990s, is fairly new \citep{lee2015internet}. Hong Kong's main environmental organizations, Friends of the Earth, Greenpower, and the Conservancy Association, were all established in the 1980s \citep{chan1995green}. 

Hong Kong's corporate presence plays a role in its environmental movement. In the 1990s, many environmental organizations partnered with local government or corporations to sponsor environmental projects \citep{chiu2000dynamics}. This created a certain dependence on government and business for funding, which may cause some environmental organizations' interests to be tied with large corporations and government activity. 

It is also worth noting that the beginning of Hong Kong's environmental movement in the 1990s mostly consisted of highly educated individuals and expatriates \citep{chiu2000dynamics}. Social movements in Hong Kong, in general, have changed significantly since this original characterization. Some sociologists refer to Hong Kong's social movements as ``postmodernist,'' in that they lack central organization, rely on spontaneity, and include a wide variety of participants \citep{so2011development}. That being said, some environmental organizations are still not using modern tactics to their fullest capacity \citep{lee2015internet}. 

Environmental activism gives way to environmental regulation and government policy. Because air pollution is a transboundary issue, regulating it requires cooperation amongst different regions and different governments \citep{jo2013polluted}. Hong Kong's mainland neighbor, the Chinese province of Guangdong, is an important regional consideration in the future of Hong Kong's air pollution regulations and the future of the Pearl River Delta as a whole. Hong Kong outsources a great deal of manufacturing work to Guangdong, whose economy is primarily manufacturing based \citep{jo2013polluted}. More specifically, investment from Hong Kong is associated with 65\% of the factories polluting the Pearl River Delta (Guangdong Provincial Government Report 2011). Hong Kong's economic and political success is dependent upon its corporations, making the development of regional regulation difficult. 

Because economic issues tend to drive policy development in the case of Hong Kong and its neighbors, it may be helpful to monetize the cost of air pollution. I believe some aspects of this cost can be monetized, but others, especially deaths and living conditions, cannot be as easily quantified. Despite that, the Hedley Environmental Index project estimated that productivity loss and health care bills directly caused by air pollution total about  HK\$3,891 million annually \citep{hedleyenvind}. These are tangible costs of air pollution that may outweigh the economic benefits of production and manufacturing and sway policy development toward stricter regulations. 

Additionally, air pollution is a visible issue that politically reflects the success of a government if they are able to reduce and control it; this should be a motivating factor for countries to regulate it \citep{jo2013polluted}. 

Finally, in regards to future environmental policy, it is important to acknowledge that justice is a key part of sustainable development and perceptions of environmental justice are often positively correlated with trust in government \citep{li2019role}. Given the difficulties Hong Kong has faced and will face with government, environmental justice may be a seemingly unachievable goal. That does not make it any less necessary. Hong Kong's extreme wealth gap places the environmental burden of air pollution on its most vulnerable and most socially deprived residents. In order to adequately address this issue, activism and policy must target both the sources of pollution and the sources of inequality and social deprivation.



\chapter{Environmental Economies East Asia}\label{ch:economies}

\chapterauthor{Chhavi Monga}

\section{Air Quality and Economics}

\subsection{COVID-19 and Environmental Health Threat}

The case study below includes snippets from the 2020 World Air Quality report, published by IQAir. The goal here is to get the reader acquainted with one of the major environmental issues countries in East Asia face \citep{air2020report}. 

Among criteria pollutants commonly measured in real time, fine particulate matter (PM2.5) is currently understood to be the most harmful to human health due to its prevalence and far-reaching health risks. Exposure to PM2.5 has been linked to negative health effects like cardiovascular disease, respiratory illness, and premature mortality. PM2.5 is defined as ambient airborne particulates that measure up to 2.5 microns in size. These particles include a range of chemical makeups and come from a range of sources. The most common human-made sources include fossil-fuel powered motor vehicles, power generation, industrial activity, agriculture and biomass burning. The microscopic size of PM2.5 allows these particles to be absorbed deep into the bloodstream upon inhalation, potentially causing far-reaching health effects like asthma, lung cancer, and heart disease. PM2.5 exposure has also been associated with low birth weight, increased acute respiratory infections, and stroke. 

\begin{figure}
\centering
\includegraphics[width=100mm]{images/Picture7pm25.png}
\caption{Comparative size of the human hair, PM10 and PM2.5 in microns or $10^{-6}$ meters \citep{air2020report}.}
\label{fig:PMsize} 
\end{figure}

More than 90\% of the global population breathes dangerously high levels of air pollution. Due to its ubiquity and severity, air pollution constitutes the world's biggest environmental health hazard, contributing to as many as 7 million premature deaths globally per year (more than 3 times higher than deaths associated with COVID-19). Air pollution also burdens the global economy with more than \$5 trillion in welfare losses.

In 2020, the spread of COVID-19 raised new concerns, as exposure to particle pollution was found to increase vulnerability to the virus and its health impacts. Early reporting suggests that the proportion of COVID-19 deaths attributed to air pollution exposure ranges from 7-33\%.

In a year defined by dramatic measures taken around the world to reduce the spread of COVID-19, IQAir published its 2020 World Air Quality Report to raise awareness around air pollution - a silent killer.

The 2020 World Air Quality report aggregates PM2.5 data from 106 countries, collected from ground-based government monitors and a growing network of validated, non-governmental air quality monitors contributed by organizations and individuals in order to learn from the world's largest air pollution database.

\begin{figure}
\centering
\includegraphics{images/Picture8EA.png}
\caption{East Asia is estimated to carry the highest regional share of global outdoor air pollution-related deaths (37\%).20 Air pollution also
costs 7.5\% of this region's annual gross domestic product (GDP)
in welfare losses. While cities from this region comprise 42 of
the 100 most polluted cities globally, PM2.5 concentrations are
trending downward overall \citep{air2020report}. } 
\label{fig:AirPollitonDeaths}
\end{figure}


\begin{figure}
\centering
\includegraphics{images/Picture8EA.png}
\caption{Southeast Asia faces air pollution challenges largely stemming
from rapid population growth and economic development. The
region's energy demand has steeply increased as a result, with
electricity demand increasing at around 6\% per year.\red{36?} The region
mostly relies on fossil fuels for energy, with oil as the leading and
coal the fastest-growing source.
PM2.5 emission sources in Southeast Asia vary by country and
environment. In urban areas, dominant emission sources include
construction, industry, and transportation \citep{air2020report}. }

\label{fig:PM25_country}
\end{figure}

Three East Asian cities observed the greatest reductions in PM2.5 emission from the sample: Singapore (-25\%), Beijing (-23\%), and Bangkok (-20\%) . 

\subsubsection{China Air Quality Changes}

86\% of Chinese cities experienced cleaner air than the year prior, while annual PM2.5 exposure by population fell 11\%. Despite this progress, China continues to dominate the ranking of top 100 most polluted cities globally, with 42 cities represented. Hotan, a desert oasis in Xinjiang province, ranked as the most polluted city globally, with pollution levels 11 times higher than the WHO target for annual pollution exposure (< 10 \micrograms). The city also had the highest monthly PM2.5 averages worldwide from March to June, when weather typically increases the intensity of sandstorms (with March peaking at an average 264.4 \micrograms). Beijing experienced improved air quality for its 8th consecutive year, with air pollution levels falling 11 percent from 2019. Air pollution still remains a dire concern in the Chinese capital, with 58 percent of annual days exceeding daily WHO PM2.5 targets (< 25 \micrograms).

\subsubsection{South Korea Air Quality Changes}

No South Korean cities achieved the WHO target for annual PM2.5 exposure (< 10 \micrograms) in 2020. Only 5 (of 60) cities in South Korea met the country's less stringent standard for annual PM2.5 of < 15 \micrograms. Despite chronically high pollution levels, South Korea did observe dramatic air quality improvements in 2020, with population-weighted PM2.5 exposure falling 21 percent. These improvements, however, are largely attributed to short-term measures intended to reduce the spread of COVID-19 and limit coal factory emissions during the polluted winter months.Long-term policy and changes in human behavior are necessary to further reduce South Korea's PM2.5 levels.

\subsubsection{Southeast Asia Air Quality Changes}

Air pollution remains a severe problem facing the Southeast Asia region: only 10.8 percent of cities here breathe air quality that meets annual PM2.5 exposure targets set by the WHO. Despite the continual health burden, 70 percent of cities in Southeast Asia observed improved air quality in 2020. Cities that did not observe direct improvements over 2019 were predominantly located in northern Thailand, which suffered from huge smoke emissions resulting from the agricultural burning season. The capital cities of Jakarta (39.6 \micrograms) and Hanoi (37.9 \micrograms) again ranked higher than notoriously polluted Beijing (37.5 \micrograms) in 2020.

\subsubsection{Global Trends in Air Quality}

Lock-down measures and changes in human behavior in response to the spread of the novel corona-virus resulted in healthier air overall in 2020. Air quality improvements over 2019 were observed in 84\% of countries (weighted by city population) and 65\% of global cities.

The most significant air quality improvements were observed during the first lock-down period, when countries around the world mandated relatively strict social distancing measures in an effort to contain the virus.

Cities with higher average PM2.5 levels and denser populations tended to observe the most significant PM2.5 reductions from COVID-19 lock-down measures. Delhi (-60\%), Seoul (-54\%) and Wuhan (-44\%), for example, all observed substantial drops during their respective lock-down periods as compared to the same time frame in 2019. Los Angeles experienced a PM2.5 reduction of -31\% during its lock-down period as well as a record-breaking stretch of air quality that met WHO air quality guidelines (< 10\micrograms).

Often, these initial improvements were short-lived. By the end of 2020, rebounds in industry and transport resulted in smaller average annual reductions (Table~\ref{tab:reductions}).

\begin{table}
\centering
\caption{Annual Reductions in Selected Cities}
\label{tab:reductions}
\begin{tabular}{lr}
\hline
City & Change \\\hline\hline
Delhi & -15\% \\
Seoul & -16\% \\
Wuhan & -18\% \\
Los Angeles & +15\% \\ \hline
\end{tabular}
\end{table}

To isolate the effect of corona-virus-motivated changes on air quality, the Centre for Research on Energy and Clean Air (CREA) applied a ''weather correction'' to the report's dataset. The correction removed the influence of weather from observed PM2.5 levels.

Weather can greatly influence observed PM2.5 levels by affecting how pollution coagulates (gathers and falls to the ground), disperses, and transforms as a result of chemical reactions.

While the influence of weather on the dataset varies from city to city, the resulting ''de-weathered'' figure paints a clearer picture of true changes in PM2.5 levels from 2019 to 2020. This could be the result of social distancing measures for COVID-19, new air pollution policy, or changing trends in human behavior.


\section{An Economist's View of the Environment}

The case study presented above shows that environmental issues such as air pollution can negatively impact the overall health of an economy. However, if we look at ''progress'' in particular, we see a general upward trend in most East Asian Countries (see \ref{fig:GDPgrowth}). This is counter-intuitive, given that we just learned about the ongoing trend of bad air quality in East Asia. In what sense, then, are these countries 'growing'? How is this progress measured, and why do nation states not take into account environmental damage when calculating their growth rates? 

This chapter will try to answer these questions and in this process attempt to determine a relationship between the environment and economics. Through discussing terms such as sustainable development, the Environmental Kuznet's Curve and the Social Cost of Carbon, the chapter explores the question of whether economic growth can be sustainable.

\section{GDP as a Primary Measure of Growth}\label{sec:GDP}

Most economies use 'GDP' to calculate growth, and this measure has been at the forefront of economic policy for a long time. GDP stands for Gross Domestic Product and measures the final monetary value of all goods and services produced within an economy in a year. In simpler terms, it measures total economic output or production. The GDP growth over time for countries from around the world is presented in Figure \ref{fig:GDPgrowth}. Over time, East Asia's contribution to the world GDP has increased considerably, which indicates that there has been a boost in the overall economic output of countries within this region see Figure \ref{fig:GDPgrowth}. This of course, does not take into account the extensive environmental damage that has been caused in order to achieve the economic output. 

\begin{figure}
\includegraphics[width=\linewidth]{images/Picture1.png}
\caption{GDP growth over the years \citep{bolt2020maddison}.} 
\label{fig:GDPgrowth}
\end{figure}


Another measure, GDP per capita, provides a slightly more accurate description of the overall prosperity of a nation and is used alongside the GDP. It is calculated by dividing the GDP of a country by its population and shows how much economic production value can be attributed to each individual citizen. GDP per capita hence translates to a measure of national wealth while also serving as a prosperity measure, and can give us an idea about the standard of living of a particular country (see Figure \ref{fig:GDPerCapita}). For example, the overall GDP of China might be higher than that of Japan (see Figure \ref{fig:GDPAsia}), but that does not mean that its people are necessarily better off (monetarily). When you divide the GDP by population, you get a better picture of how much each person is earning on average (Figure \ref{fig:GDPPerCapCJ}).

\begin{figure}
\includegraphics[width=\linewidth]{images/gdpPerCap.png}
\caption{GDP per capita over the years. We see that on average, the East Asian population is economically worse off than Western populations, but is better off than Sub Saharan African populations \citep{bolt2020maddison}.}
\label{fig:GDPerCapita}
\end{figure}


\begin{figure}
\includegraphics[width=\linewidth]{images/gdpAsia.png}
\caption{China's overall GDP higher than Japan's overall GDP \citep{world2017world}.}
\label{fig:GDPAsia}
\end{figure}

\begin{figure}
\includegraphics[width=\linewidth]{images/gdpPerCapCJ.png}
\caption{Japan's GDP per capita  is significantly higher than China's GDP per capita despite China having a higher overall GDP \citep{world2017world}.}
\label{fig:GDPPerCapCJ}
\end{figure}

East Asia, including both Northeast and Southeast Asia, has undergone great industrialization and urbanization during the last decades (Mori 2013). These activities allowed the region to enjoy rapid growth compared to the West and is on its way to becoming the world's center of economic growth. As is evident in figure \ref{fig:GDPgrowth}, the GDP share of East Asia has shown an upward trend and has steadily been contributing more and more to the world GDP. Rapid economic growth has brought a sharp reduction in poverty as well, and this can be seen in Figure \ref{fig:GDPerCapita} showcased by rising per capita GDP.


However, although industrialization and urbanization have contributed significantly to economic growth, they have brought along serious environmental degradation. Industrial plants have increased the discharge of untreated air and water pollutants and solid wastes, increasing energy demands, and numbers of automobiles have made air pollution more serious. High-emitting sectors like power, transport, construction, even agriculture, have served as the backbone in economic growth. With little to no adequate policies to control emissions, pollution from all these new and increasing sources accumulate (see Figure \ref{fig:AsiaCarbon}). ''Since air pollution is such a chronic problem in Asia, many in the region -- including some politicians -- have decided to live with it as a necessary cost in the pursuit of development, which it isn't,'' Kakuko Nagatani-Yoshida, the United Nations Environment Program regional coordinator for chemicals, waste and air quality in Bangkok, told Nikkei \citep{faulden_2021}. The result of that is why more and more cities are recording higher PM2.5 levels every year. As evidence, today, seven of the world's 20 cities with the highest levels of particulate air pollution were located in East Asia, and 19 out of 20 were located in Asia \citep{air2020report}. Although countries like Japan have very low PM2.5 levels, Japanese megabanks like Mizuho Financial Group, Sumitomo Mitsui Group and Mitsubishi UFJ Financial Group continue to fund oil- and gas-powered projects, indirectly contributing to pollution levels across the world \citep{faulden_2021} (Figure~\ref{fig:AsiaCarbon}).


\begin{figure}
\includegraphics[width=.8\linewidth]{images/AsiaCarbon.jpg}
\caption{Asian countries have some of the worst recorded pollution levels. Agricultural burning and forest fires are well recognized as a source of dangerous smog affecting many parts of Asia, compounding the industrial and transport air pollution that is a byproduct of Asia's export-led economic growth.}
\label{fig:AsiaCarbon}
\end{figure}

\section{Environmental Economics and Ecological Economics}

There are two main fields of study dedicated solely to studying how economies interact with the environment and the trade-offs between economic growth and environment protection; Environmental economics and Ecological Economics. 

Environmental Economics developed in its present form in the 1960s as a result of the intensification of pollution and the heightened awareness among the general public in Western countries about the environment and its importance to our existence \citep{al-attili}. It started becoming evident to economists that, for economic growth to be indefinitely 'sustainable', the economic system needs to be held accountable for the way it has been exploiting natural resources in order to maintain high levels of economic growth. Environmental economists therefore view the environment as a form of natural capital which performs life support, amenity, and other functions that cannot be supplied by man-made capital \citep{al-attili}. This stock of natural capital includes natural resources plus ecological systems, land and biodiversity.

The growth of environmental economics in the 1970s was initially within the neo-classical paradigm, according to which consumers, workers, and firms all make a rational calculation as to what is in their own best interests. In general, this approach to the environment is concerned with issues of market failure, inappropriate resource allocation, and how to manage public goods \citep{al-attili}. In other words, little attention was paid towards the underlying relationships between the economy and the environment. 

Concerns about the limits of the environmental economics approach led some environmental economists to develop what is now referred to as ecological economics. Ecological economics views the relationship of the economy and the environment as central and any analysis places economic activity within the environment (as opposed to the two being separate). This distinction is best illustrated with reference to debates concerning sustainable development and the difference between weak and strong sustainability (SOAS). Ecological economics supports the notion of strong sustainability- a view of sustainability that assumes that not all forms of capital (i.e. human and natural) are perfectly substitutable (this will be discussed in the next section). The subject deems it necessary to get a more integrated picture of how human interactions with the environment have been in the past, and make predictions about future interactions. Ecological economists view humans as beings embedded in their ecological life-support system, as opposed to being separate from the environment.


East Asian policies have mostly relied on studies by environmental economists rather than ecological economists, which explains why they did not reflect local conditions of the environment in a particular country as they were taken from the West. Since ecological economics generally tends to support the notion that the natural environment must be nurtured and protected, as, without it, an economy and human life cannot function, this field has a more conservative perspective regarding economic activity and how it impacts the natural environment \citep{dean2014principles}. As a result, ecological economists emphasize that East Asian countries may need to rethink the standard orthodox economic concepts such as endless economic growth and want fulfillment on which their current environmental policies are based.

\section{Sustainability and Sustainable Development}

Sustainable development defined by the Brundtland Commission as development that meets the needs of the present without compromising the ability of future generations to meet their own needs.

In addition, there are two main definitions in use currently --- strong sustainability and weak sustainability. 

Weak sustainability is mostly used by economic growth theorists and environmental economists often within the field of economics. It draws a clear distinction between ''economic capital'' and ''natural capital''. Economic capital consists of land, labor, capital (machines), and human capital (knowledge). Natural capital comprises the environment and natural resources. 

\begin{figure}
\includegraphics[width=\linewidth]{images/weakstrongsus.png}
\caption{Difference between weak sustainability and strong sustainability. (Source: \citet{ir})}
\label{fig:WeakStrong}
\end{figure}


Weak sustainability assumes that natural capital and manufactured capital are essentially substitutable and considers that there are no essential differences between the kinds of well-being they generate \citep{ekins2003framework}; \citep{neumayer2012human}. The only issue that matters is the total value of the aggregate stock of capital (the combination of land, labor, physical and human capital), which should be at least maintained or ideally increased for the sake of future generations \citep{solow1993measurement}. In such a perspective: ''it does not matter whether the current generation uses up nonrenewable resources or dumps \carbondioxide in the atmosphere as long as enough machinery, roads and ports are built in compensation'' \citep{neumayer2003weak}. Such a position leads to maximizing monetary compensations for environmental degradation. In addition, from a weak sustainability perspective, technological progress is assumed to continually generate technical solutions to the environmental problems caused by the increased production of goods and services \citep{ekins2003framework}. 

On the other hand, the strong sustainability emphasizes that natural capital cannot be viewed as a mere stock of resources. Rather, natural capital is seen as a set of complex systems consisting of evolving biotic and abiotic elements that interact in ways that determine the ecosystem's capacity to provide human society directly and/or indirectly with a wide array of functions and services \citep{ekins2003framework}. This approach is very popular among ecological economists.


\section{The Growth Debate}

Having acquainted ourselves with the definitions presented above, we are now equipped to understand and engage with the very important and widely debated concept of whether economic growth can be sustainable. In this process, we will also determine a relationship between economic growth and environmental performance.

''Anyone who believes that exponential growth can go on forever in a finite world is either a madman or an economist,'' is a famous quote by economist Kenneth Boulding.


However, economic growth, measured by GDP has been at the forefront of policy making across the world for the past 70 years. Global output (GDP) is now more than eight times higher than it was in 1950, and if it continues to grow at the same average rate, then the world economy will be 17 times bigger in 2100 than it is today: a staggering 146-fold increase in economic scale in the space of just a few generations \citep{theeconomist}. 

Neo classical economics, a broad theory that focuses on supply and demand as the driving forces behind the production, pricing, and consumption of goods and services, seems to take for granted that economic growth increases social welfare. However there has been some criticism on this assumption in recent times.

Since 'social welfare' is not unambiguously measurable, one can discuss endlessly the so called 'meaningful measures' of welfare. For example, just because a country's global output or GDP is growing, it does not mean that its people are always happier, and that its environment is being looked after. GDP is only measured in terms of money, and does not really take into account, for example, environmental damage, which one would argue is essential for measuring 'welfare'. To elaborate, the last 50 years have been accompanied by the degradation of an estimated 60\% of the world's ecosystems, and the Stockholm Resilience Centre at Stockholm University in 2015 \citep{stockholmresiliencecentre} identified four key areas in which human activity already lies beyond the ''safe operating space'' of the planet: climate change, land-use change, loss of biosphere integrity and overload in biogeochemical cycles (see Figure \ref{fig:planetB}). In other words, the unprecedented increase in economic activity fails to take into account the ecological constraints of a finite planet.

\begin{figure}
\includegraphics[width=\linewidth]{images/planetBound.jpg}
\caption{Estimates of how the different control variables for seven planetary boundaries have changed from 1950 to present. The green shaded polygon represents the safe operating space \citep{stockholmresiliencecentre}.}
\label{fig:planetB}
\end{figure}

\section{Case Study: The Environmental Performance Index (EPI)}

As we have just established, the impact human activities have on the environment has not been positive. However, it could advance in one of the two directions. First, excessive and irresponsible industrial production has resulted in pollution, natural resource depletion, and ecosystem deterioration. But on the other hand, effective environmental governance and eco-friendly technologies alleviate the burden on the ecosystem, reduce environmental risks to human health and promote sustainable growth. 

We can find evidence of this by looking at The 2020 Environmental Performance Index (EPI) \citep{epi}, which provides a data-driven summary of the state of sustainability around the world. It uses 32 performance indicators across 11 issue categories and ranks 180 countries on environmental health and ecosystem vitality. These indicators provide insights at a national scale of how close countries are to achieving the environmental policy targets they establish. The indicators also provide a way to spot problems, set targets, track trends, understand outcomes, and identify best policy practices. This analysis can also help government officials refine their policy agendas, facilitate communications with key stakeholders, and maximize the return on environmental investments. 
Overall EPI rankings indicate which countries are best addressing the environmental challenges that every nation faces. Going beyond the aggregate scores and drilling down into the data to analyze performance by issue category, policy objective, peer group, and country offers even greater value for policymakers (see Figure \ref{fig:EPImeas}). This granular view and comparative perspective can assist in understanding the determinants of environmental progress and in refining policy choices (EPI Yale).

\begin{figure}
\includegraphics[width=\linewidth]{images/EPImeas.png}
\caption{The 2020 EPI Framework. The framework organizes 32 indicators into 11 issue categories and two policy objectives, with weights shown at each level as a percentage of the total score \citep{epi}.}
\label{fig:EPImeas}
\end{figure}

\begin{figure}
\includegraphics[width=\linewidth]{images/EPI.png}
\caption{The relationship between 2020 EPI Score and GDP per capita shows a strong positive correlation between GDP per capita and EPI score, indicating that on average richer countries have a better environment \citep{epi}.}
\label{fig:EPI}
\end{figure}

Figure \ref{fig:EPI} graphs the relationship between GDP per capita of a country, and its EPI score. The higher the EPI score, the better the country is considered at promoting sustainable development and conserving its environment. GDP per capita, as discussed in section \ref{GDP.}, serves as a prosperity measure by measuring how much each person within a country is earning on average. The figure shows a positive correlation between the two variables, indicating that good policy results are associated with wealth (GDP per capita). For example, China has a low GDP per capita and also a low EPI score, whereas Singapore has a high GDP per capita as well as a high EPI score. In other words, economic prosperity makes it possible for nations to invest in policies and programs that promote sustainable development. Therefore, if we consider an increasing GDP per capita to be an indicator of economic growth, then we can conclude that economic growth can be sustainable.



\section{The Environmental Kuznet's Curve: An Approach Used by Environmental Economists}

According to Huang et al, the driving factors that affect environmental performance can be divided into two broad categories. The first category consists of socioeconomic factors including national economic achievement and the advancement in production technologies. The second category is concerned with the effectiveness of government regulation for issues concerning government fiscal commitment to environmental governance and the stringency of environmental regulation.


The Environmental Kuznet's Curve (EKC) attempts to answer the question of whether rising income levels and economic growth improve or deteriorate the environmental performance of a country. The theory essentially emphasizes that different stages of economic development influence a country’s environmental status, and posits inverted-U relationship between pollution and economic development (see figure \ref{fig:EKGCurve}): usually, pollution grows rapidly in the early stage of industrialization when clean air and water are not priorities compared to jobs and growth, and then as an economy becomes wealthier and more concerned with environmental quality, pollution gradually falls to the pre-industrial level \citep{grossman1991environmental}.

\begin{figure}
\includegraphics[width=\linewidth]{images/Picture2.png}
\caption{The Environmental Kuznet's Curve showing how different stages of economic development influence a country’s environmental status. Usually, pollution grows rapidly in the early stage of industrialization when clean air and water are not priorities compared to jobs and growth, and then as an economy becomes wealthier and more concerned with environmental quality, pollution gradually falls to the pre-industrial level \citep{taguchi2013environmental}}
\label{fig:EKGCurve}
\end{figure}



Dasgupta and others \citep{dasgupta2002confronting}, however, point out that the conventional EKC \ref{fig:EKGCurves} has been challenged by numerous critics. The negative view is that the curve will rise to a horizontal line (see figure \ref{fig:EKGCurves}) denoting maximum existing pollution levels, as globalization promotes a ''race to the bottom'' when it comes to upholding environmental standards. Under this scenario, relatively high environmental standards in high-income economies impose high costs on polluters, prompting shareholders to drive firms to relocate to low-income countries with weak or non-existent environmental regulations \citep{taguchi2013environmental}. This results in increased capital outflows, and forces governments in high-income countries to begin relaxing environmental standards . 

The second scenario, entitled New Toxics \ref{fig:EKGCurves}, is also pessimistic: industrial society continuously creates new, unregulated and potentially toxic pollutants, thereby the overall environmental risks from these new pollutants continue to grow even if some sources of pollution are reduced \citep{taguchi2013environmental}.

The scenario of the Revised EKC \ref{fig:EKGCurves} on the other hand is optimistic. It assumes that due to growing public concern and research knowledge about environmental quality and regulation, developing societies can experience an EKC that is lower and flatter than the conventional one would suggest; these societies may develop from low levels of per capita income with little or no degradation in environmental quality \citep{taguchi2013environmental}. In Figure \ref{fig:EKGCurves}, the revised EKG curve is lower than the Conventional EKG, and the Conventional EKG is flatter.

\begin{figure}
\includegraphics[width=\linewidth]{images/Picture3.png}
\caption{The Environmental Kuznet's Curve, different versions \citep{taguchi2013environmental}}
\label{fig:EKGCurves}
\end{figure}

\subsection{EKG in East Asia}

Asian economies are at different stages of development consisting of high-income countries, such as Japan and the Republic of Korea, middle-income countries, such as Malaysia and Thailand, and low-income countries,such as Cambodia and Myanmar, so a particular country is expected to lie on a specified point on the Kuznets curve depending on the stage of development it is in. \citep{taguchi2013environmental} examined whether the scenarios of Conventional EKC, Race to the Bottom and Revised EKC were applicable in Asia to representative environmental indices, namely sulfur emissions and carbon emissions. He found that sulphur emissions (see Figure \ref{fig: EKasia} ) follow the expected inverted U-shape pattern of the conventional EKC, while carbon emissions tend to increase with per capita income in the observed range. As for the Race to the Bottom and Revised EKC scenarios, the latter was verified in sulphur emissions, while the former was not present in sulfur or carbon emissions.


\begin{figure}
\includegraphics[width=\linewidth]{images/Picture4.png}
\caption{The Environmental Kuznet's Curve in East and South-East Asia \citep{taguchi2013environmental}.}
\label{fig: EKasia}
\end{figure}

The reason that sulfur emissions follow the conventional Kuznets curve could be attributed to policy. Local pollutants, such as sulphur emissions, are subject to regulation. In fact, the pollution controls on sulphur emissions have been promoted intensively over a broad area of Asia since the 1970s \citep{taguchi2013environmental}. According to Iwami \citep{iwami2001advantage}, the remarkable reduction of sulphur emissions in Japan from the beginning of 1970s to the mid 1980s comes from environmental regulations reinforced by central and local governments, and technological development for desulphurization and energy efficiency promoted by private companies. International regulations also influenced the sudden action taken by governments. For example, in 1972, the Stockholm Declaration came into being. This declaration represented a first major attempt at considering the global human impact on the environment, and an international attempt to address the challenge of preserving and enhancing the human environment \citep{assessment2002united}. It espouses mostly broad environmental policy goals and objectives rather than detailed normative positions.


On the other hand, global pollutants such as carbon emissions are easily externalized, i.e. their effects are felt by parties other than the ones directly producing them. Thus they are not subject to regulation, mainly because they are so difficult to regulate. For example, China's \carbondioxide emissions will be felt by its neighboring countries and have a negative effect on their air quality. This is why in the Figure \ref{fig: EKasia}, we see that carbon emissions do not follow the Conventional EKG. Regulatory frameworks on greenhouse gas were set domestically and internationally only after the Kyoto Protocol was approved in 1997. Asian countries, with the exception of Japan, are, however, non-Annex I countries and so they have no legal obligation nor incentive to reduce carbon emission \citep{yaguchi2007beyond}. Moreover, even though Japan has policies in place for reducing carbon emissions, it has not been successful in doing so, so no spillover effects (benefits) may be expected for other lower-income East Asian countries.


As for the revised EKG, we see that it pertains to developing countries in East Asia for sulfur concentrations \ref{fig: EKasia}. According to Iwami \citep{iwami2001advantage}, for instance, in the early 1970s, air pollution, particularly sulphur concentration, in the large metropolitan areas of South-East Asia was less prevalent, despite rapid economic growth in their respective countries when compared with metropolitan areas of Japan. This is because their governments and firms implemented initiatives in the early stage of development; from the late 1970s to the early 1980s, Indonesia, Malaysia, the Philippines, and Thailand moved forward with establishing fundamental frameworks for environmental protection including regulations pertaining to sulfur emissions, such as laws, standards, and institutions \citep{taguchi2013environmental}.

To sum up the discussion we may conclude that emission control policies play an important role in lowering emissions, and countries with strong environmental policies and regulation often see a decline in their emissions.


\section{Conclusion}

In this chapter, we have looked at different models and theories in order to understand how countries in East Asia and around the world measure progress and growth. Progress is often measured by versions of GDP, and non-monetary factors are usually not taken into consideration. We also learned that there has been evidence that economic growth can be sustainable, as shown by the EKC and the EPI score.

As countries in Asia continue to pursue growth, finding the right balance between growth and sustainability through effective policies and institutions will be an important theme in policy making in the coming decades.






\chapter{Unpacking Plastic Pollution and Recycling in Japan}\label{ch:recycling}

\chapterauthor{Eleanor Dunn}

\section{History of Plastic in Japan}

\subsection{Introduction}

Japan is a unique case study in plastic consumption and recycling in East Asia. Not only is Japan the second largest consumer of single use plastics in the world after the United States \citep{johnston2020}, it also leads the world in recycling \citep{mccurry_japan_2011} with 84 percent of 2018`s collected plastic ending up recycled \citep{pwmi2019}. In the face of this impressive percentage, defining what exactly the Japanese government means by ``recycling'' is significant in revealing the social justice issues associated with this process. This chapter will analyze the reasons for unmanageable plastic consumption in industrial societies, define what it means to recycle and why reliance on ``recycling'' as it stands is unsustainable given the social and environmental implications of thermal recycling and the global plastic trade, acknowledging solutions and barriers to progress through the lens of Japan. 

\subsection{Urbanization}

	Today, 92 percent of Japan's population lives in an urban environment \citep{worldbank}. Japan's rapid urbanization and industrialization began in the late 1800s during the Meiji Restoration. After World War Two, Japan's economy boomed once again, reaching a high in the 60s and 70s. During this period, grocery and convenience stores became ubiquitous. The 80s and 90s also brought significant economic growth, in a time known as the ``bubble economy'' period \citep{hcswm}. The late 90s saw the rise of the \emph{konbini}, a type of convenience store, as well as a shift from reusable containers to single use packaging \citep{meiselman_packaging_2020}. Today, Japan has the third-largest packaging market in the world \citep{meiselman_packaging_2020}.

\subsubsection{Managing Waste in an Urban Society}

Waste management became a mounting problem in the face of urbanization, particularly beginning after World War Two, when there was little waste management infrastructure and waste was dumped into waterways or left in the open, attracting pests and spreading disease \citep{hcswm}. Landfills began to overflow during industrialization, and illegal dumping caused human and environmental health crises, including the Minamata disease of the 50s. Illegally dumped mercury from a chemical factory contaminated fish and as people ate it, they suffered neurological symptoms impacting eyesight and hearing \citep{hcswm}. 

\begin{figure}
\includegraphics{images/JapanPlastics/wealthwaste.png}
\caption{This chart demonstrates waste production and national disposable income from 1955 to 1980. There is a distinct positive correlation between income and waste production (Source: \citet{hcswm}).}
\label{fig:wealthwaste}
\end{figure}

Japan has the third largest economy in the world \citep{lee2021} despite ranking 62nd in land area \citep{worldometer}, creating a situation of high consumption with minimal disposal space, an issue faced worldwide, but exemplified on such a small archipelago. To cope with issues of waste storage space, Japan began building incinerators inside cities in the 1960s, and today, Japan has the most advanced incineration infrastructure in the world \citep{hcswm}, while only one percent of waste ends up in landfills \citep{sturmer2018}. Incinerators reduce waste volume to one-twentieth of its original size \citep{sturmer2018}.

 Despite blame-placing rhetoric on East Asian countries for the world's plastic crisis, signified by articles titled ``Japan's Plastic Addiction'' \citep{mahoney_japans_nodate} and ``Southeast Asia's plastic 'addiction' blights world's oceans'' \citep{kittisilpa_southeast_2018}, on average individuals in Japan throw away half as much garbage as Americans yearly \citep{harden2008}.  

	

\subsubsection{Commuter Culture and Konbini}

	In recent decades, Japan has gained international attention for its work culture. Workers are hesitant to use their vacation days, and in 2018, over half of paid vacation days went unused \citep{demetriou2020}. People fear being judged or told off for taking days off, many interviewed said that they typically even came in to work if they were sick \citep{demetriou2020}. Stress from this culture of overwork has been linked to adverse health impacts, and even suicide \citep{list_working_2017}. The term ``karoshi'' means death from overwork \citep{whitelaw_konbini-nation_2018} \citep{list_working_2017} and has been used in Japan since the 1970s \citep{whitelaw_konbini-nation_2018}. This culture will likely face impacts from the Covid-19 pandemic, on which more research likely will develop in coming years. With this capitalist culture, a lack of free time combined with commuting make konbini an oasis for workers. Most urban Japanese people frequent konbini daily \citep{whitelaw_konbini-nation_2018}, often purchasing bento lunches, young folks meet with friends and peruse new offerings, and homeless people pick returnable cans from the recycling bins outside \citep{whitelaw_konbini-nation_2018}. These stores, available 24 hours a day, offer ``food, entertainment, money, shelter, anonymity, and, sometimes, intimacy'' \citep{whitelaw_konbini-nation_2018}. They also offer an explanation as to how urbanization, capitalist work culture and consumption can contribute to a mounting waste issue worldwide.  

To this end, anthropologist Gavin Whitelaw decided to only consume food from konbini for a full month. In the first week of this experiment, he already collected 28 plastic bags, 6 plastic straws, 13 chopsticks, 11 plastic spoons of various sizes, a few plastic forks, and two 10-litre bin bags of plastic plates, covers, cellophane wrapping, and PET bottles'' \citep{whitelaw_konbini-nation_2018}.


\begin{figure}[h]
\centering
%\includegraphics[width=\linewidth]{images/JapanPlastics/groceries.png}
\includegraphics{images/JapanPlastics/groceries.png}
\caption{Here are plastic wrapped fruits that you could find in konbini or other grocery stores in Japan. Source: \citep{denyer2019}.}
\label{fig:konbini}
\end{figure}

\subsection{Cultural Significance of Packaging} 

While convenience and food safety are practical reasons for plastic consumption, there is historical and cultural significance to packaging in Japan. This section does not seek to proliferate western ideas of hyper traditionalism and exoticism of Asian countries, but rather to create a more complete overview of Japan's plastic consumption. 

Aesthetics have always been important to Japanese culture, guiding art and daily practices alike \citep{saito_japanese_1999}. ``The allure of the hidden'' is a cultural appeal based in indigenous Shintoism and a reason for historic attention to packaging, even before global ubiquity of plastics. The tactile experience of opening packages is also significant:

``Both Buddhist temples and Shinto Shrines are wrapped with a series of enclosures, consisting of walls,  gates, and vegetation, that regulate our experience of the respective space by offering a progressing    sequence of going from the outer to the inner. The glimpse of the inner gradually unfolds as we keep going through the walls and gates'' \citep{saito_japanese_1999}.

Thus, may influence the consumer products that are layered in plastic as a culturally based marketing technique. An anthropological study of Japanese wrapping offers that ``by carefully wrapping an object, one is apparently expressing politeness and care, care for the object, and therefore care for the recipient'' \citep{ben-ari_unwrapping_1990} These traditions and cultural sensibilities give more depth to the merchandizing outsiders hypocritically criticize. While thoughtful wrapping has roots in traditional Japanese culture, aesthetically pleasing packaging is ubiquitous in the global marketplace, and it's fair to say just about everyone enjoys opening presents. 

\subsection{Internal Plastic Demand}

In 2015, around 71 percent of Japan`s domestic plastic demand came from industries, while household demand accounted for 39 percent \citep{nakatani_revealing_2020}. In terms of plastic production in Japan, 40 percent of consumed plastic is for bags, plastic sheets for construction, and similar forms of plastic \citep{pwmi2019}. Polyethylene and polypropylene usually constitute these items, reference chapter about plastic composition, therefore these specific plastics account for half of all plastic production in Japan \citep{pwmi2019}.  

\section{Problems of Plastic Pollution}

\subsection{Ocean Plastics and Ingestion}

What are the human implications of the plastic pollution? Japan is the largest seafood importer in the world, consuming six percent of global fish harvests \citep{bird_will_nodate}. This dietary fact is concerning given that eighty percent of anchovies in Tokyo Bay contain plastic \citep{denyer2019}. While Japan imports much of its seafood, this statistic speaks to an inevitability of consuming trace amounts of plastic when eating fish. Although researchers have not concluded direct correlation between microplastics in fish and negative health outcomes, some plastics contain endocrine disruptors, which have known negative impacts on human health \citep{royte_we_2018} such as increased risk of cancer, immune and nervous system interruption, and reproductive issues \citep{epa}. 

While microplastics and other ``mismanaged'' plastic waste pose health problems for the world`s people, the 84 percent of Japan`s plastic that is recycled also holds implications for human health and social justice.   
\section{What Does Recycling \emph{Really} Mean?}

\begin{figure}[h]
%\includegraphics[width=\linewidth]{images/JapanPlastics/Rcycle.png}
\includegraphics{images/JapanPlastics/Rcycle.png}
\caption{This graph offers the catergories in which Japan disposes of discarded plastic. As you can see, most discarded plastic ends up in some recycling category. For the purpose of this chapter, lets unpack the ``Incineration'' and ``Exported for recycling'' categories. Source: \citep{denyer2019}.}
\label{fig:rcycle}
\end{figure}

\subsection{Recycling Laws and Public Cooperation}

Japan has plethora laws to ensure waste is treated and recycled properly \ref{fig:legalhis}. Individuals in Japan tend to be fastidious about sorting their recycling into an average of nine categories, most homes even display a calendar of when the municipality will pick up each category \citep{jaramillo}. The time consuming individual efforts like washing, deconstructing, and sorting of recycling \citep{jaramillo} don't always reflect what happens further down the recycling line when waste is incinerated or shipped to other countries. 
\begin{figure}[h]
\includegraphics[width=\linewidth]{images/JapanPlastics/legalhis.png}
\caption{This chart demonstrates Japan's extensive waste management legislation and the major events that inspired it. Source:\citep{hcswm}.}
\label{fig:legalhis}
\end{figure}

\subsection{Incineration}

Japan has used incineration for waste management for around eighty years, though as stated they weren't used in cities until the 1960 \citep{hcswm}. In 2009, Japan had nearly 1500 incinerators, 70 percent of which were stoker furnaces, others included fluidized bed furnaces and gasification fusion resource furnaces. Incinerators generate electricity as the burning waste heats up water, turning it into steam. The steam then propels a turbine generator which creates electricity \citep{useia}. Countries all over the world use incinerators, particularly the Nordic countries of Denmark, Norway, and Sweden, which, like Japan, have high recycling rates \citep{seltenrich2013incineration}.

Incinerators are effective at reducing waste volume, an important benefit to a small island nation where many landfills are at capacity \citep{harden2008}. Additionally, they generate electricity through the process detailed above, or \textbf{thermal recycling}, which is better for the environment than some other forms of steam turbine energy production, like coal burning. Generally, one ton of waste creates as much electricity as one ton of coal. \citep{watson_can_2016}. 

\subsubsection{Health and Environmental Justice}

Most studies on the health and social justice implications of incinerating waste include little data on East Asia. Generally, older incinerator technologies correlate to adverse health impacts. Though Japan has replaced most older incinerators in recent years, reducing incinerator related dioxin emissions by 98 percent since 1997 \citep{hcswm}, not enough time has passed for adequate evaluation of possible health impacts of newer technologies \citep{tait_health_2020}. While demographic evaluations of communities near incinerators in Japan are scarce, in Europe and the United States, incinerators tend to be in areas with minority and lower socioeconomically positioned populations \citep{martuzzi2010}. As a country that recycles 58 percent \ref{fig:rcycle}of its plastic waste through incineration, the government should continue research and offer transparency about the process of choosing where incinerators will be built and who might have suffered disproportionately from the hazardous incineration technologies of the past. 

One of the largest concerns with incinerators is \textbf{dioxin} emissions. Between 1997-1998, a study in Japan found a higher incidence of infant mortality related to congenital birth defects in areas near incinerators with higher soil dioxin levels \citep{tait_health_2020}. Dioxins have a seven-year half-life cite and can persist in soil and the body for multiple times longer cite. Therefore, despite Japan's drastic 98 percent reduction in dioxin emissions since 1997, the harmful chemical compounds are likely still impacting the populations closest to them or who eat food from those areas, even if direct emissions no longer pose a significant threat. Although, a 2005 study in Osaka showed a positive correlation between school incinerator proximity and number of children experiencing ``wheeze, headache, stomachache, and fatigue'' \citep{miyake2005}. These problems could also be attributable to ash or heavy metal contamination, two additional byproducts of thermal recycling.

\begin{figure}[h]
\includegraphics[width=\linewidth]{images/JapanPlastics/incinerator.png}
\caption{Here is a picture of the Maishima Incineration Plant in Osaka. Austrian artist Friedensreich Hundertwasser designed it with curved lines and greenery ``as a symbol of harmony with nature''. Source: \citep{sturmer2018}.}
\label{fig:incinerator}
\end{figure}

Conversely, a study conducted from 2000-2007 reviewed the health and dioxin exposure of incinerator workers in comparison to the rest of Japan's population \citep{yamamoto2015}. The study concluded that this occupation did not have significant impacts on the health or dioxin levels of the workers. This study, conducted with oversight from Japan Industrial Safety and Health \citep{yamamoto2015}, implies the safety of incinerators by concluding that even those who work around incinerators daily do not experience negative impacts.

These contradictions suggest the need for further research into incinerator impacts. A pamphlet published by the Ministry of the Environment, offers that the research and development of more efficient, cleaner incinerators are ongoing \citep{hcswm}, and the government communicates with residents about building plans, negotiating until they have consent \citep{harden2008}. Additionally, modern incinerators in cities like Tokyo and Hiroshima serve as tourist attractions and community gathering spaces, featuring swimming pools, fitness centers, and health clinics for the local elderly population. These incinerators are odorless and have tall smokestacks, so that they expel smoke, greenhouse gasses, and trace toxic compounds above the skyscrapers \citep{harden2008}. 

The height increases of smokestacks is dubious in itself, performative even. US companies began building taller smokestacks after the passing of the Clean Air Act in 1970, with the goal of reducing measurable emissions by putting them higher in the atmosphere. This just dispersed pollutants over state lines \citep{edmonson2011}. In Japan, I wonder if with the increase of tall smokestacks pollutants are finding their way into other countries or the ocean at higher rates.

While incineration is an effective and evolving way to reduce plastic waste volume while generating electricity and modern incinerators in Japan can seem like a magic solution to crushing waste volume, the human and environmental costs require further investigation. Governmental assurances of safety coupled with contradictory studies of dioxin impacts prove incineration to be a complex environmental justice issue as Japan continues to rely on the process for waste management. Consider this issue with regard to the two dominant theories of pollution: matter out of place and thresholds of harm \citep{liboiron2016redefining}. Despite strict emissions regulations \citep{hcswm}, Japan's remarkable incinerators still emit ``carbon dioxide, water vapor and trace amounts of toxic particulates'' \citep{harden2008}.

The matter out of place theory would suggest that any amount of contamination from these chemicals in the soil, air, or human body, is unacceptable. The thresholds of harm theory, which seems to inform current environmental legislation, makes the distinction between ``perceived'' and ``demonstrated harm'' \citep{liboiron2016redefining}, basically, that forms of contamination are non-violent until the moment harm is scientifically proven, not anecdotally or psychosomatically. 

Although Japan's incinerators are a solution to overflowing landfills and contribute significantly \emph{less} dioxins than landfill fires \citep{watson_can_2016}, they still emit. Any toxic emissions can constitute a slow violence to those living and working nearby, and the full extent of human and environmental impacts of new incinerator technologies will likely not be demonstrable for decades, if ever. Also, the extent to which dioxins and contaminants from past, more unclean incineration, persists soil, food systems, and bodies requires more research so that offending actors can compensate those impacted \citep{nrdc}. 

\subsection{The Global Plastic Trade}

Per figure \ref{fig:rcycle}, Japan exports fourteen percent of its recycled plastic waste. China, formerly the primary importer of plastic, halted plastic imports in 2018, increasing the import pressure on less wealthy  countries like Thailand, Malaysia, and Vietnam \citep{bauman2019} and compounding the social justice issues of the global plastic trade. Japan counts these waste exports towards their recycling percentage, even though the countries that are importing the plastic have even more limited infrastructure for recycling and waste management \citep{denyer2019}. Additionally, most of the exported plastic is not in good enough condition for recycling \citep{nippon2019}. Much of the plastic that ends up in the oceans comes from these countries \citep{mahoney_japans_nodate}, spurring international blame on poorer Asian countries for polluting the oceans, despite that wealthier, more consumptive countries like Japan and the United States are the sources of the plastic. It was not until January 1, 2021, that importing countries had to provide consent to receive contaminated plastic waste. The Tokyo Plastic Strategy highlights concerns over this protective international legislation, offering that illegal dumping might increase in Japan as exporting becomes more difficult \citep{tps}. Japan`s opposition to this legislation speaks to a global hegemony of wealthy countries and the role of nationalism in environmental justice. 

\subsubsection{``The Recycling Myth'' - The Global Plastic Trade and Malaysia}

In 2018, Greenpeace Malaysia released a paper called The Recycling Myth, outlining the impacts of plastic imports on Malaysia, problematizing recycling as the developed world has come to understand it. Countries like Japan ship their plastic to Malaysia, where waste management facilities or private disposers sort it into high and low grade. High grade plastics are for recycling, though much of those end up dumped or burned, and low grade, typically single-use items, are slotted for disposal. As low as 30 percent of all plastics imported to Malaysia end up recycled.

Individuals in Malaysia have found illegal economic opportunity burning and dumping plastic waste outside of licensed incineration or waste management facilities. Some Malaysians have protested the plastic waste trade and the harm it has done to their communities with varying success. The government has decided to phase out plastic imports, but Greenpeace suggests more could be done. 

Fatma, a mother from Jenjoram, Malaysia, recounts her son's complaints of the health impacts from unregulated plastic burning: ``Mother, it is always hazy in the mornings at my school.' I said, `Oh, it's fine. It's always hazy, and the haze is from Indonesia.' He said, `No! Every day is hazy… Since the beginning of the year, my eldest son aged 13 years old-he has this health problem: his eyes are red, itchy and tearing''. There are illegal dumpsites near fisheries, contaminating the waters of prawn and other food sources with aluminum levels 300 times higher than the acceptable limit \citep{greenpeace}. This imagery is reminiscent of conditions under which Minamata disease spread in Japan in the 1950s, did they just pass their illegal dumping issue to a less wealthy nation?

The UN Basel Convention, which came into effect the first of January 2021, sets regulations for plastic waste exports. Essentially, they must be clean and easy to recycle \citep{staff2021} and importing countries must give consent to import contaminated waste. As of 2019, Malaysia began sending hundreds of containers of contaminated or poor-quality waste back to offending countries \citep{staff2021}.

\section{Conclusion and Moving Forward}

\subsection{Mitigation Efforts}

\subsubsection{Long Term Goals}
 
Japan's goals for reducing plastic usage on a national level include a 25 percent reduction in single use plastic waste and 60 percent reuse rate, and 2 megaton usage of biobased plastic alternatives by 2030 \citep{nakatani_revealing_2020}. Figure \ref{fig:legalhis} demonstrates the history of waste management laws in Japan thus far. 

\subsubsection{Ecotowns}

 	Most of Japan's citizenry is conscious of their waste. Some small-town residents have taken this interest to new levels, serving as an example for the rest of the country and world. There are 26 \textbf{ecotowns} in Japan \citep{higuchi2008japan}, including Kamikatsu and Shikoku. Both towns aim to codify the Reduce, Reuse, Recycle, or the 3Rs and Japan's national calls for creating a sound material or circular society. Chapter refcirculareconomychapt explains waste management for a circular economy in further detail. Kamikatsu replaced its out-of-date incinerator with a 45-category waste separation center, reusing and upcycling most of their waste. Shikoku has a similar separation center with a swap shop, where one person's trash truly becomes another's treasure. Shikoku residents have reclaimed 11 tons of waste this way \citep{crossleybaxter2020}. Residents dedicate much effort to the proper separation of waste, and voice skepticism that such involved effort may not work in more populated areas, as Kamikatsu only has 1,500 residents \citep{mccoy2019}. 

Focus on local as opposed to national adaptation to climate environmental issues has gained popularity in the past decade given that different locales experience climate impacts in unique ways and municipalities are more in touch with their citizenry's needs \citep{measham2011}.  

Despite these efforts, one Kamikatsu resident offers that ``the recycling system treats a symptom of our dependency on single-use plastic'' \citep, and even further, the recycling system itself is less perfect than it may seem. 
 
\subsubsection{Mottainai Campaign}

The Mottainai Campaign is an NGO that attempts to add a fourth R, respect, into reduce, reuse, recycle \citep{hoy}. This Japanese term, meaning, ``without importance'' /citep{hoy} gained international attention when Kenyan environmentalist and Nobel Prize winner, Wangari Maathi visited Japan in 2005 and learned of how the word was being used to discourage waste\citep{siniawer2014affluence}. She began using this term in international environmental forums. Soon enough, companies including as the Itōchū Corporation, involved in international trade, and Panasonic, the electronics company, began using Mottaini as a \textbf{greenwashing} marketing technique. Japanese politicians too began to promote this phrase, participating in gubernatorial greenwashing of their own environmental issues \citep{siniawer2014affluence}.

The Mottainai Campaign hosts flea markets and craft fairs \citep{hoy}, materializing these concepts in a social setting. Proceeds from these events go towards the Green Belt Movement which aims to reforest Maathi's home country, Kenya\citep{hoy}. While this effort speaks to aspirational international collaboration, it is also problematic. 

Aside from greenwashing issues, in some ways, international attention towards mottainai comes off as \textbf{orientalist}. The title of a BBC article: ``Japan's Ancient Way to Save the Planet''\citep{crossleybaxter2020}, indicates a sort of fascination with or nostalgia for ``ancient'' elements of Asian culture. Politicians like former Prime Minister of Japan Koizumi Jun'ichirō attempted to internationalize the word while stressing how hard it would be for foreigners to understand such an inherently Japanese concept \citep{siniawer2014affluence}. Both of these examples point to a sort of othering. While this term has been internationalized, various actors, internally and externally, stress how \emph{essentially} Japanese it is. 

 \subsection{Conclusion}

Perhaps, then, there are better ways to tackle the global problem of plastics and a dangerous recycling system that focus on what highly consumptive, capitalist countries have in common: disproportionate impacts of their economies on less wealthy nations, government prioritization of corporate rights over environmental justice, and a worldwide issue of space. 

Ecotowns and mottainai, both sparking international inspiration, and both helpful in their own small ways, are not the solution this plastic crisis and the social justice impacts of recycling. 
In the ideal situation, the world's wealthiest nations will agree to phase out plastic production in favor of eco-friendly alternatives, whether those are biodegradables, paper, or reliance on reusable packaging. Also, to tackle industry, local and national governments must enforce plastic reduction goals, making it more economically attractive for companies to use less plastic. 

Three of \textbf{The 17 Principles of Environmental Justice}, ``demand the right to
participate as equal partners at every level of decision making, including needs assessment, planning, implementation, enforcement and evaluation'', ``affirms the right of all workers
to a safe and healthy work environment without being forced to choose between an unsafe livelihood and unemployment. It also affirms the right of those who work at home to be free from environmental hazards'' and ``protects the right of victims of environmental injustice to receive full compensation and reparations for damages as well as quality health care'' \citep{nrdc}.  Ideally, powerful plastic producing and consuming nations will take these principles into account in their plastic reduction plans, not only moving forwad with less plastic, but seeking to repair harm done to effected nations and individuals.



\chapter{Wastewater Management for a Circular Economy}
\chapterauthor{Matias Ceccarelli}

\section{Overview}

The First Law of Thermodynamics states: ``energy can be neither created nor destroyed, but can be transformed from one form to another.'' The energy constituting Earth’s resources is constant but always changing form. By understanding this law we can begin to outline sustainable waste management systems and shift from a linear extraction based/single-use economy, to a regenerative circular economy that sustains human/environmental health. While every human activity generates waste, in nature, waste does not exist. Nature operates as a circular system, turning decay into energy; nature is the model for a circular economy; it inspires biomimetic systems that function as a whole in synergy with the environment.

The majority of the world uses a linear supply chain model, ending with disposal rather than reusal. Circularity is gaining enormous traction as the best way to sustain life on Earth. The United Nations and international governments are echoing the need to transition from a linear to a circular economy. Circularity is a way of life still practiced by first nation peoples around the world: ``do not take more than you need,'' ``replenish what is taken'' \citep{sharp_2019}. 

Waste management is an important yet often forgotten sector underlying all systems. It is the foundation of a circular economy. It’s not within this paper’s scope to cover the intricacies of the supply chain, waste management, or circular economics; covered here is an analysis of linear and circular wastewater treatment systems that process sustainable and unsustainable waste from household, industrial, and agricultural facilities. This paper agrees, linear-waste is unsustainable and a danger to humans and the environment \citep{mustafa2020recent} \citep{braungart_mcdonough_kroese_2011}  \citep{bown}. This paper postulates that if sustainable-waste is processed by bio-regenerative-wastewater-treatment-systems, defined by ingredient recovery apparatus, a circular economy will be more readily realized and humans as well as the environment will prosper. It is hoped this analysis will elucidate the interdependence of all systems, human and nonhuman, through life's fundamental element––water. This analysis is organized by multiple case studies examining distinct linear and circular wastewater treatment systems in SouthEast Asia. Here, linear wastewater treatment includes: open defecation, septic latrines, and the majority of municipal sewage waste treatment facilities; Linear waste includes pesticides and chemicals; Circular wastewater treatment includes: double-vault-compost latrines, anaerobic digesters with biogas recovery, and phytoremediation systems. Linear ends with waste; circular is wasteless. 

\subsection{Linear Economy}

The primary challenge of sustainable development comes from the energy/material stream between humans and the environment. The current and traditional linear production model: ``take, make, [use], waste,'' is unsustainable and originates from the Industrialization Revolution (1760-1840) \citep{braungart_mcdonough_kroese_2011}. The linear-supply-chain is fueled by the desire to make products as fast as possible, produce the greatest number of goods, and deliver them to the highest number of people. This process is often extractive and harmful to the environment (convenience economics). Standardized mass-production regards Earth's limited resources as endless; reuse is not a function, instead, disposal is the outcome. Pollutants generated are harmful to humans and the environment. The linear system is a major source of today’s socio-economic injustices; it is the source of climate change 
\citep{braungart_mcdonough_kroese_2011}. 

The linear system begins with raw resource extraction and ends with waste disposal. Pollution is created, emissions are released, and valuable materials are lost. Since this system doesn’t restore what is extracted, the results are: 1. resource scarcity and 2. dangerous waste substances are released into land, water, and air, causing harm to humans and the environment. As a result of the linear system, quantitative geo analysis shows Earth’s usable surface area is diminishing in size and volume \citep{braungart_mcdonough_kroese_2011}. Additionally, waste emissions are emitting greenhouse gases (GHG), which are expanding deserts, causing sea level rise, changing climates, and causing the reduction of biodiversity and the extinction of species. Furthermore, unsustainable industrial and agricultural chemicals and heavy metal waste are accumulating in the environment, causing harm to humans, animals, and plants; rapid population growth is exacerbating these issues \citep{un}. 

\subsection{Circular Economy}

''Nature operates as a system of nutrients and metabolisms. In nature there is no waste'' \citep{braungart_mcdonough_kroese_2011}. In circular economics (CE), waste is regarded as a resource, just like in nature. Waste treatment is the metabolization process, which produces nutrients that can be repurposed. Materials can be divided into biological and technical mass (technical being industrial). Biological nutrients which contribute to the biosphere return to the Earth, while technical mass becomes nutrients to the ''technosphere'' \citep{braungart_mcdonough_kroese_2011}. Thus, CE treats all forms of waste as food for biological and technical systems. Before being disposed of, materials are recovered for reuse, refurbishment and repair, then for remanufacturing and only later for raw material utilization \citep{braungart_mcdonough_kroese_2011}. However, some materials used today (chemicals, plastics, etc) engage with the bio and technosphere simultaneously, often causing harm to organisms in both spheres; many argue these alien materials should be made bio-technically neutral \citep{braungart_mcdonough_kroese_2011}.

According to CE, combustion for energy comes before landfill disposal, thus energy stored in materials can be used and not ''wasted.'' With resource recovery, a material’s value and quality is sustained for the longest possible. Energy used for resource recovery is expected to be energy efficient. The Circular Economy is intended to utilize present natural systems for preserving materials/energy in a form which nature can use in its own systems. 

With linear and circular systems now in mind, the following sections will present linear systems and materials with circular solutions. 

\section{Water Is Life}

One of the most pressing concerns of the linear model is the pollution of Earth’s water. At the most basic level this is a problem of industry, consumption, lack of effective wastewater-treatment infrastructure, education and policy. Only 2.5\% of Earth’s ecosystem contains freshwater, moreover, only 1\% is potable. Almost all available freshwater originates from groundwater, which sources rivers and wetlands (other sources are lakes and dams constituting 1\% of drinkable water) \citep{misachi_2018}. ``Fresh water is a finite and vulnerable resource, essential to sustain life, development and the environment” \citep{united_nations}. There is a limited amount of freshwater, and a great percentage is contaminated by human and livestock faeces, as well as chemicals and heavy metal runoff \citep{purdue}. In South-east Asia more than 140 million people lack access to safe drinking water; worldwide, 2.1 billion people lack access \citep{arshad_2016} \citep{who}. Water is used everyday, domestically and industrially for waste disposal: ``sullage,” coming from kitchens and bathrooms, and ``sewage” waste, composed of human excreta, sullage, and industrial/agricultural runoff \citep{epa_2018}. Waste and water are inextricably linked; to discuss waste is to discuss water, and vice versa. Highlighted here is the importance of responsible, circular, and cost-effective wastewater treatment. The following passages will present case studies demonstrating linear waste-water-management systems. 

\section{Wastewater Systems} 

\subsection{Open Defecation: Linear}

In China 14 million people defecate in the open; in Cambodia 8.6 million \citep{who}. Open defecation is a linear system because untreated excreta-waste introduces dangerous pathogens (cholera, typhoid, hepatitis, polio, cryptosporidiosis, ascariasis, and schistosomiasis) into surface drinking water sources (1). Additionally, sewage contains valuable materials (nitrogen, phosphorus, and biogas) which could be recovered for fertilizer and fuel (explained later). 

Open defecation is a serious issue that must be addressed by low cost wastewater treatment systems for low-income communities  

\subsection{Septic Systems: Linear}

In the 1860’s a Frenchman by the name of Jean-Louis Mouras invented the earliest known septic system \citep{supeck_2019}. Within a short period of time it expanded throughout Europe, the United States, and parts of Asia, deemed a much better alternative to throwing faeces out the window or defecating in the open. Septic systems are used widely today and are far safer than open defecation, nonetheless, they are a linear waste stream and can pose a danger to humans and the environment. 

Septic systems are underground wastewater treatment structures, commonly used in rural areas where centralized sewage systems are not present. The tank is a water-tight-container made from concrete, fiberglass, or polyethylene \citep{epa_2018}. There are a variety of septic design systems, the majority have naturally occurring aerobic and anaerobic microbes that biodegrade sludge/pathogens, as well as technology. Septic systems often treat wastewater from a house’s bathroom (sewage), kitchen and laundry (sullage) \citep{epa_2018}. A septic system includes a septic tank and a drainfield. The septic tank separates hydrophobic particles (oils and grease) which float to the top, from solids and organic matter (sludge which collects at the bottom). The drainfield is a soil-based system in the yard that releases treated liquid effluent from the septic tank into a number of pipes that release the fluid into the soil, ultimately entering the groundwater \citep{epa_2018}. (If the drainfield is overloaded with too much liquid, sewage will begin to flood posing a health hazard). As filtered wastewater travels through the ground, the soil will filter it again, removing most bacteria and pathogens; the soil cannot treat cleaning products, medication (antibiotics), and other harmful chemicals \citep{bown}. It is important to note rural septic systems may be close to a homeowners well. If toxic chemicals , and sometimes pathogens, seep into the water table, the homeowner’s drinking water is compromised. Chemicals furthermore pose a threat to the rest of the environment \citep{bown}. 

A septic tank must be cleaned every year to ensure successful bacterial decomposition of solid waste and prevent sludge build ups. As stated, a variety of septic systems exist. Some use pumps or gravity sending effluent into the soil, alternatively a system may evaporate the wastewater before releasing it into the Earth. Alternatively a home may use an on site aerobic sewage system which can remove between 85-98 percent of the organic matter and solids from the wastewater \citep{epa_2018}. In the United States, aerobic systems are often efficient. Aerobic systems include: a pre-treatment tank, an aeration/settling chamber, and a chlorination chamber (anaerobic microbes are present). The aerobic system is a miniaturized version of a municipal wastewater treatment facility (both septic, and the majority of municipal wastewater treatment facilities, do not remove harmful GHGs, pharmaceuticals and other chemicals) \citep{usgs} \citep{epa_2020}.  

Figure \ref{fig:Septic Tank}
\begin{figure}
\includegraphics{images/Circular-Images/septictank_image.png}
\caption{Illustration of a septic tank}
\label{fig:Septic Tank}
\end{figure}

Septic systems are a linear form of disposal as they release high concentrations of  (\methane) into the atmosphere. \methane is 30 times more powerful than \carbondioxide and comprises 16\% of global GHG emissions \citep{fu_schleifer_zhong_2017}. A study found 85\% of households in Hanoi, Vietnam; Mandalay, Myanmar; and Kota Surakarta, Indonesia use septic systems \citep{huynh}. While many use different designs, it was found they all release \methane through anaerobic decomposition. The United States Environmental Protection Agency (USEPA) estimated global \methane production from septic systems as 3.0 Tg-\methane per year, which is 10.4\% of global \methane emissions from domestic wastewater \citep{e_p_a}. Because of the study, \methane emissions from septic tanks are now roughly accounted for in net global GHG emissions. Many septic systems in East Asia do not have a drainfield, they process only sewage, while sullage is emptied into the environment. The climate of Southeast Asia (77-97°F) promotes anaerobic digestion conditions increasing GHG emissions \citep{huynh}. The study found high concentrations of \methane, \carbondioxide, and dismissal amounts of \nitrous in septic tanks in Hanoi. The table below illustrates \methane emissions reflective of the amount of time sewage remained in septic tanks. The study found if septic tanks are regularly emptied, there are less GHG emissions.

Figure \ref{fig:Methane}
\begin{figure}
\includegraphics{images/Circular-Images/Methane_levels.png}
\caption{Illustrates \methane levels in septic storage over periods of time}
\label{fig:Methane}
\end{figure}


The \methane found in the average septic tank in Hanoi was oversaturated by 24,444\%, while the \methane in lagoons when measured was 260–128,420\% \citep{huynh}. Therefore, when compounded, septic tanks in Hanoi, and other parts of SE Asia, contribute greatly to atmospheric \methane levels. Additionally, sewage was reported to be released into a non centralized sewer or directly into a waterbody, demonstrating the linearity of present septic systems in Hanoi and much of SE Asia. The study shows only 2-6\% of sewage in Hanoi was treated at a sewage facility, some went to a landfill site, and unknown amounts were disposed of illegally \citep{huynh}. The study found the average septic tank to have \methane: 11.92 ± 4.52 g/capacity/day, while \carbondioxide was 20.24 ± 9.15 g/cap/day. Septic tanks are effective at reducing pathogens, however, their methods of disposal are linear as they do not not capture or resource \methane for energy, thus they release \methane, a very potent GHG into the environment. They also, often, do not resource nitrogen and phosphorus for fertilizer. Additionally, if chemicals are present in the effluent, they are released into the groundwater \citep{huynh}. 

\subsection{Vietnam’s Double-Vault Latrine: A Circular Treatment Comparison To Septic Systems }

For centuries farmers in Vietnam and China used untreated human excreta and urine as a fertiliser for rice and other crops (nitrogen in excreta and phosphorus in urine, are extremely rich materials for fertilizer). However, in 1956, to combat the hazardous practice, Vietnam health authorities instituted the first double-vault composting latrine program, which uses similar principles, but with sanitary measures \citep{pebler}. Since then DVC technology has been widely supported and adopted by other countries. In 2007 the Government of Vietnam assessed 25\% of rural households were without a toilet (to be further referred to as a latrine), and 19\% were found to have an unhygienic latrine \citep{cole_phuc_collett_2009}
. Given the startling figures, Vietnam implemented a goal of constructing 2,600,000 hygienic latrines by 2010 \citep{cole_phuc_collett_2009}. The government compared the double-vault composting latrine (DVC latrine), the septic tank latrine, pour-flush water sealed latrine and ventilated pit latrines, and found the most hygienic, as well as versatile, system to be the DVC latrine, which the Government subsequently funded.

The DVC latrine is simple, easily constructed, requires moderate maintenance, is cost-effective(\$70 us), and creates a valuable byproduct. The DVC is a closed-cycle-system: it saves water, prevents groundwater pollution, and produces a fertilizer/soil conditioner for agriculture. Treated human excreta can be used as a fertilizer to increase the water holding and ion-buffering capacity of soil \citep{cole_phuc_collett_2009}. Proper construction and use of the DVC are important, so as to prevent pathogenic contamination on agricultural produce. 

How it works: The DVC can be built above or in the ground, if in the ground, the water table should not be close. The DVC consists of two vaults, one is in use, the other contains decomposing excreta. Each vault is ideally built two accommodate one year of excreta. Excreta enters the vault free from urine (it is a dry system and urine is collected in jars). Sawdust or ash are thrown into the vault after each use, reducing moisture content. When the vault is 75\% full, it is completely filled with dry Earth and the squat hole is sealed. For the following 6-12 months microorganisms (aerobic and anaerobic) decompose the excreta and pathogens die (the vault is often painted black to increase the rate of dehydration, and solar panels can be used) \citep{nzdl}. During this time the other vault is used. After said time, the decomposing vault is emptied and used for fertilizer (Figure \ref{fig:DVC Latrine}).


\begin{figure}
\includegraphics{images/Circular-Images/DVC_image.png}
\caption{Representation of the DVC Latrine}
\label{fig:DVC Latrine}
\end{figure}

A study found that after emptying decomposed containers, 63\% of households immediately used the contents for fertilizer; the remaining households conducted a secondary form of composting. 61\% of households reported using urine on crops and garden trees. Researchers found 91\% of households were satisfied with the DVC latrine. However, 73\% of households reported emptying the vault 1-2 times per year, prior to rice planting (February and June). This suggests the vaults were often emptied before the recommended 6-12 month storage period \citep{cole_phuc_collett_2009}. Reported problems with the system were fly pestilence and odors. Ultimately, the DVC is a more circular system than the septic latrine as it produces fertilizer and preserves water; nonetheless, it still releases GHGs. 

\subsection{Municipal Utility Sewage Treatment: Linear}

Many parts of SE Asia do not have the funds for large scale wastewater treatment facilities like those present in Europe and the United States. Although developed countries have formidable facilities, the majority use linear processes that refine water (as described by aerated septic systems) but do not capture GHGs produced through anaerobic microbial decomposition. Sludge waste collected through water refinement, is sent to landfills where \methane and other GHGs are furthermore released \citep{usgs, epa_2018}. 

\subsection{The Anaerobic Digester and Biogas: Circular Sewage Treatment}

According to the United Nations, 80\% of wastewater flows into the ecosystem without being treated or reused. This creates a multitude of problems for the environment and human health: great amounts of GHGs are released into the atmosphere and roughly 1.8 billion people drink and use the pathogenic water, risking infection: cholera, dysentery, typhoid and polio, and ingest of toxic chemicals \citep{UNpopulation}. 

Polluted rivers contribute significantly to global warming. The New Territories of Hong Kong may appear clean and lush with agricultural fields, expansive greenery and mountains; however, in reality, the rivers flowing through these areas are saturated with three primary greenhouse gases: \methane, carbon dioxide, and nitrous oxide. A study found the concentration of said gases were 4.5 times greater in the rivers flowing near Hong Kong, than found at atmospheric levels. The cause of pollution is discharge from agricultural livestock (farm waste contributes to 58\% of world \methane emissions) and human effluents \red{citep{smith}--citation broken}. 
It's estimated that rivers and streams release up to 3.9 billion tonnes of carbon each year (around four times the amount of carbon emitted annually by the global aviation industry) \citep{bbc_news_2019}. Moreover, it’s estimated global rivers and lakes are the cause of more than 50\% of atmospheric \methane concentrations, largely due to unrecovered \methane in untreated wastewater. The more polluted the rivers, the higher the emissions. ``When rivers become polluted, their global warming potential (GWP) can increase from two to 10 times'' says Long Tuan Ho, researcher at Ghent University, Belgium \citep{matthew}. Moreover, the \carbondioxide and \nitrous concentrations of water zones close to urban areas were found to be four times higher than those at natural sites; in the case of \methane, this ratio was 25 times higher. Again, the primary culprit is untreated wastewater runoff. When untreated effluents enter bodies of water, biogeochemical reactions occur via anaerobic microorganisms which create GHGs (\carbondioxide, \nitrous, \methane), consequently resulting in a positive (warming) feedback loop. The septic system, double-vault-composting latrine, and the vast majority of municipal wastewater treatment sites release GHGs into the atmosphere. These linear waste management systems are increasing the rate of global warming and untreated wastewater is infecting limited drinking water supply with pathogens. 

Water is precious and so is energy. Anaerobic wastewater digestion with biogas recovery is a circular ``waste-to-energy technology” that can greatly reduce world \methane emission, furthermore, the captured \methane can be used for a variety of human energy needs. The system is low-cost, readily available, carbon-neutral, and reduces environmental degradation caused by fossil fuels. 

How it works: During natural decomposition, anaerobic microorganisms transform biomass (organic matter—sewage, food waste, animal manure) into \methane gas and digestate (the material that remains after anaerobic digestion). Biogas can be converted into electricity or fuel, and then used throughout the energy sector, ie: cooking, lighting, transportation, building conditioning \citep{wilkie}. Digestate is nutrient dense and can be used as a non synthetic fertilizer for crops. Biogas is 40\%–60\% \methane, the remainder is \carbondioxide, small amounts of water vapor, and other gases. Biogas can be burned directly, used to generate electricity, or treated to become biomethane--a low carbon ``pipeline-quality” fuel \citep{eia} \citep{environmental_and_energy_study_institute_(eesi)} \citep{epa_2018} \citep{wilkie}.

When \methane is burned, the potent GHG is transformed into energy and \carbondioxide is released. While \carbondioxide is also a potent GHG, the \carbondioxide being produced is not adding new carbon to the atmosphere, this is the \carbondioxide naturally sequestered by plants––thus the \carbondioxide is carbon neutral. Burning \methane is far better than having it escape into the atmosphere. If biogas is used in place of fossil fuels, the net emissions are ``negligible” in comparison \citep{fu_schleifer_zhong_2017}. While many wastewater treatment centers have anaerobic digesters, the majority do not have energy recovery systems, thus they incinerate sludge or dispose it into landfills, which releases \methane, \carbondioxide and nitrous oxide into the atmosphere; consequently, these linear disposal systems increase the rate of climate change and society loses a valuable source of energy. Wastewater treatment with biogas-recovery systems are an example of circular waste management: they save money, capture energy, and reduce global warming.

A pilot project in Xiangyang, China, illustrates how sludge to energy refinement can power the treatment facility itself, recover valuable nutrients and energy, reduce GHG emissions, prevent land and water pollution, and be cost-effective. Over the course of its lifetime, the treatment facility reduces GHG emissions by more than 95 percent \citep{liu}. A 2015 report found the plant can supply 300 cars with natural gas daily. The system is also profitable making the plant \$ 1.5 million annually. Four large cities in China: Beijing, Changsha, Chengdu and Hefei have implemented, or plan to install, sludge to energy facilities, which will reduce 700,000 tons of \carbondioxide emitted per year and are estimated to produce 40 millions cubic meters of compressed gas to be used by municipal vehicles \citep{wri_2015}.  

Developing countries lacking sewage treatment infrastructure can benefit greatly at the village and city level by adopting sludge to energy systems. This is exemplified at the local level in Indonesia as 80\% of Indonesian households dispose their waste into septic tanks; 60\% of these septic tanks are located less than 32 feet from the household wells which results in high concentrations of E. coli in drinking water \citep{andriani2015review,}. Similar conditions are described in Vietnam and other SE Asian countries. \href{https://www.youtube.com/watch?v=9kKRdlAFuZw}{Small and affordable} (\$ 35-100 usd) decentralized digesters can be used at homes for kitchen waste as well as for livestock and human excreta \citep{worldbank_blog}. The biogas produced can be used directly for cooking and electricity generation. Educating and supporting low-income communities to use anaerobic digester systems will reduce deforestation for firewood, improve water quality, and mitigate climate change. Further research may elucidate if anaerobic digesters can be used in conjunction with the double-vault-compost latrine and septic tanks. 

Biomass to energy systems can be applied to any form of organic waste; therefore, 3.9 trillion kg of human/livestock faeces p/y and the 1.3 billion tonnes of global food wasted each year, could be repurposed and turned into carbon-neutral energy \citep{fao}. The electricity generated from biogas could meet the needs of about 40 million people globally, 22 percent of global electricity consumption” \citep{aquatech}. By 2030 it is estimated the planet will be generating more than one trillion kg of faeces each year, this number is expected only to rise with population growth \citep{berendes_yang_lai_hu_brown_2018} \citep{un}. The rising population with associated life consumption waste (food, agriculture, and excreta), makes biogas sequestration a great circular and sustainable technology for wastewater treatment. According to Niclas Svenningsen, manager for the Global Climate Action team at the United Nations:

``[Biogas] is a win-win-win-win-win industry: win for turning GHG into energy; win by using that energy to replace fossil fuels; win by turning global waste, that releases dangerous levels of \methane gas every day, into a valuable resource; win by creating jobs and contributing to the new low-carbon economy; win by offering a stable energy source that can be built and used even at the household scale in remote areas'' \citep{hughes_2019}.
Biogas is a highly recommended wastewater-nutrient-energy technology for the world and SE Asia.

\subsection{Phytoremediation: Wastewater Circularity}

Phytoremediation, ``plant remedy,'' is considered an effective, aesthetically pleasing, and cost-effective sustainable wastewater treatment technology which is governed by the principles of interdependence and circularity \citep{ali2020application}. Phytoremediation has been used for the last 300 years, but became well known in 1983 \citep{stephenson2014one}. A phytoremediation site can treat a multitude of industrial and municipal pollutants including: pesticides, chlorinated solvents, polycyclic aromatic hydrocarbons (PAHs), Polychlorinated biphenyl (PCBs), petroleum hydrocarbons, radio nucleosides, surfactants, explosive elements, heavy metals\footnote{ Heavy metals are said to cause damage to DNA and produce carcinogenic effects in animals and humans \citep{kaur2020role}. Linear waste activities such as mining, municipal waste, application of fertilizer, discharge of urban effluent, vehicle exhausts, waste incineration, fuel production, and smelting release heavy metal contaminants into the wastewater stream (1).}, as well as sewage and livestock excreta. Phytoremediation prevents soil erosion reducing the spread of contaminants (groundwater leaching can occur) \citep{ali2020application}. The treatment method is cost-effective, and does not require exclusive equipment or highly trained workers. The clean up cost is dismal, giving it a special place among green technologies. The disadvantages include a slower treatment time, possible source of dangerous mosquitoes, and is limited to shallow contaminants \citep{mustafa2020recent}. The aquatic plants (macrophytes) used have the following characteristics: they are fast growing, produce a high biomass yield, transport metals and pollutants into above ground part of the plants, and can tolerate high levels of metal toxicity. 

Plant selection is the most important aspect of phytoremediation. Water hyacinth (Eichhornia crassipes), water lettuce (Pistia stratiotes) and Duckweed (Lemna minor) are prominent phytoremediation plants. These plants create a thriving environment for microorganisms, fungi and bacteria, providing them with enzymes from their roots. The microbes reduce pathogens and metal toxicity; they also provide plants with nutrients, increasing a plant’s phytoremediation ability \citep{kaur2020role}. This plant-microbe mutualism showcases circularity in action. Figure \ref{fig:Phytoremediation Facility and System}

\begin{figure}
\includegraphics{images/Circular-Images/phytoremediation.png}
\caption{Phytoremediation facility and how the biotechnology works}
\label{fig:Phytoremediation Facility and System}
\end{figure}

How phytoremediation works: Wastewater travels through matrices of floating plants, the plant’s roots filter the water (rhizofiltration), absorbing nutrients and heavy metals. The leaves above water sequester carbon and hold pollutants. During photosynthesis, the plants release oxygen, aerating the water for aerobic microbes, which in turn break down faecal pathogens and supply nutrients to plants. 

Water hyacinth (Eichhornia crassipes) has been deemed a problematic and invasive plant because of its high growth rate, but is now used for phytoremediation technology. Water hyacinth is the recommended treatment plant for industrial wastewater, domestic wastewater, sewage effluents, and sludge ponds because it absorbs high rates of organic and inorganic contaminants, can tolerate extreme pollution, has a high biomass production rate, absorbs nitrogen and phosphorous, and can remediate metals like arsenic, zinc, mercury, nickel, copper and lead \citep{mustafa2020recent}. The associated wastewater processing time of water hyacinth is 1-2 months \citep{mustafa2020recent}. Sewage pathogens are broken down by predators, protozoa and bacteriophages, living near the plant’s roots \citep{aguainc_2015}. Nitrogen, a rich element present in sludge, leads to the eutrophication of the environment, while phosphorus (present in urine) can lead to cyanobacterial (algae) blooms causing ecological imbalances \citep{aguainc_2015}. Nitrogen and phosphorus accumulate in the roots of water hyacinth, which can be later harvested for compost, fertilizer, and biofuel \citep{kulkari} \citep{aguainc_2015}. Metals that have accumulated in the plant stem and leaves can also be recovered \citep{aguainc_2015}. Phytoremediation plants can be used in all forms of wastewater treatment, as well as implemented in rivers and other bodies of water. Phytoremediation is an important circular technology, treating a variety of pollutants. It is a sustainable bio-wastewater treatment system that can provide solutions to communities through SE Asia and other parts of the world. 


\begin{figure}
\includegraphics{images/Circular-Images/HandyPod_image.png}
\caption{The HandyPod}
\label{fig:HandyPod}
\end{figure}

In Cambodia, about 100,000 people live in floating communities on the Tonle Sap Lake \citep{wetlands_2019}. During the low-water dry-season, much of the community's ambient water becomes infected with human excreta. Children, especially, are at high risk of pathogenic exposure as they play in the lake’s water. Wetlands Work, a social enterprise, has developed the ``Handy Pod,” a circular, cost-effective, waste water treatment system for these floating communities (Figure \ref{fig:HandyPod}). The system uses microbes and phytoremediation to refine sewage to ``grey water” levels––deemed low risk. The effluent quality tested 28-100 colony forming units (cfu)/100 ml in a one-half cubic meter volume of ambient water; this is significantly less than the Cambodian standard maximum: 1000 cfu/100 for contact recreation water quality. The system reduces E-coli by \href{https://www.youtube.com/watch?v=VmRO6nPIIc8&t=2s}{99.9\%}.

The Handy Pod costs \$ 30US  to assemble. It is then placed beneath the floating house’s latrine. After use, excreta travels into an expandable bag, which is treated by naturally occurring microbes, the remaining effluent is then processed by water hyacinth plants, pathogens stick to their roots and are further broken down by microbes. The treated water then flows into the lake \citep{akpan_2014}. The Pod was tested for three years and has been deemed successful. It has no smells, is aesthetic, there are no mosquitoes, no chemicals, it has demonstrated stability during storms, and requires little to zero maintenance. While problems still face the Handy Pod, like residual pathogens, it is a positive step forward for sanitation in impoverished zones. It illustrates how small circular systems can be made, as well as exemplifying the effectiveness of microbial digestion and phytoremediation. A point of further research is to know if biogas recovery systems could be implemented in conjunction with the Handy Pod.

\section{Linear Chemicals in Wastewater: Pesticides in Taiwan}

Pesticides and chemicals are a concern for wastewater treatment since they pass untreated through the majority of wastewater facilities, collecting in humans, animals, and the environment. Synthetically formulated pesticides became widespread after WWII. They enable large farm yields; nevertheless, pesticides are toxic to humans, animals, and a large number of non-invasive insects. Many pesticides are ``persistent” (not breaking down easily) so they don’t have to be reapplied often, and they are hydrophobic, to not be washed off by rain. Pesticides are a classic example of a linear product that is low cost, used widely, results in great yields, and is of  harm to humans and the environment \citep{oregonstate}. Today synthetic pesticides are still used throughout the world, especially in SouthEast Asia where agriculture is a principal commodity. 

Taiwan is a relatively small island in SE Asia (13,974 mi$^2$), surrounded in the north by the East China Sea and to the south by the Pacific Ocean. Located in the semi-tropics, Taiwan is one of the most densely populated islands in the world with 23.57 million people 
\citep{bbc_news_2019}. 
The following case study examines the presence of toxic pesticides located in the Danshui River (largest river in Taiwan), which ultimately passes through a waste treatment plant; pesticides still flow into the ocean. Figure \ref{fig:Danshui River Pesticides} illustrates points of sampled sediments, the greatest being at D13, the wastewater treatment site.

The island experiences a dry and wet season. During the dry season, pesticides were found in river sediments. During the wet season, characterized by winter and summer/fall monsoons, pesticides were found in the ocean ecosystem and marine organisms. The river is defined as a primary transport process and the ocean as a second transport process. Each section of the river is important and sheds light on understanding the pesticide life-cycle in the environment. The Da-han River (top of the Dan Shui) brings fresh water to agriculturists; the Shin-dan River (mid-point of the Dan Shui) provides water to tea farms in drinking water for citizens in Taipei (2.646 million inhabitants) and neighboring counties; finally, the Keelung River (last leg of the Dan Shui) is distinguished by many industrial plants located within the the river’s drainage basin; The waste-water treatment facility located at the mouth of the estuary (where the river flows into the ocean) was the point of highest pesticide contamination and is where oysters and other sea creatures were found to have pesticides in their bodies \citep{hung2007relationships}. 

\begin{figure}
\includegraphics{images/Circular-Images/Danshui_image.png}
\caption{A representation of the advantages of a Circular Economy}
\label{fig:Danshui River Pesticides}
\end{figure}

The pesticides were found in sediments and the marine ecosystem. Pesticides, which are made from hydrophobic carbon-chain molecules, adhere to carbon-based particles in the water, which eventually fall into river sediments. The pesticides are ``sticky” and layer onto each other. There are many pesticides in use (tetrachlorobenzene, HCHs, aldrin, dieldrin, chlorpyrifos, mirex, DDEs, DDDs, and DDTs) and it becomes hard for scientists to differentiate between them since they mesh together \citep{hung2007relationships}. When the chemicals enter the marine ecosystem they attach to phyto-plankton which are ingested by clams and fish, which die and enter ocean sediments or are consumed by other aquatic organisms, or humans (Huertos, conversation). 
``The sum of pesticides, including tetrachlorobenzene, HCHs, aldrin, dieldrin, chlorpyrifos, mirex, DDEs, DDDs, and DDTs, was found in offshore sediments at station D13, approximately 6e7 km from the Danshui River mouth (Fig. 1). Coincidently, the discharge point, with several output pipelines connecting to the main pipeline, of the marine outfall pipeline from the Pali sewage treatment plant is located near this station (Sinotech, 1997)...... Therefore, the fact that individual pesticide concentrations that are found at the station near the marine outfall pipe are high or highest suggests that the sewage treatment plant is still discharging pesticides from residual activated sludges and/or fine particles'' \citep{hung2007relationships}.
Pesticides from present and past agricultural practices were found in the ecosystem, illustrating pesticide persistence. This study illustrates the importance of understanding where pesticides end up (always into the ecosystem and into organisms) and their main outlet source. The Pali sewage treatment plant illustrates how many municipal wastewater treatment facilities ineffectively treat pesticides. Pesticides are illustrated here as single-use linear products, which pass through waste management systems, accumulating in the environment and harming organisms. This is of great concern as pesticides exposure has been linked with health defects in humans, animals, and plants \citep{sharma2019responses}. This study also raises questions on the quality of drinking water in Taipei. 

\subsection{Pesticides in Taipei Food Toxicology: (further research is needed for drinking water)}

Taiwan ranks third globally in pesticide usage ``per unit of area planted” (Japan is number one). Without pesticides, the annual output of rice would decrease by 15\% the first harvest and 30\% the second (similar results would occur with other produce) \citep{ministry_of_foreign_affairs}. Nonetheless, pesticides are problematic for the environment and detrimental to human and animal health. Acute exposure to humans can cause: stinging eyes, rashes, blisters, blindness, nausea, dizziness, diarrhea and death. Known chronic exposure can result in: cancer, birth defects, reproductive harm, neurological and developmental toxicity, immunotoxicity, and disruption of the endocrine system \citep{nicolopoulou2016chemical}.  
Pesticide awareness is growing and many people throughout SE Asia are raising the red flag. Greenpeace, an NGO in Taiwan, measured pesticide residue on 60 vegetables and fruit sold at major retail outlets throughout Taiwan: RT-Mart, Pxmart, Carrefour, Wellcome, A. Mart, Costco, 7-Eleven and FamilyMart. The study found 73\% (44/60) of the tested specimens contained pesticide residue, 27\% (12/44) were above pesticide safety levels, and five contained prohibited pesticides \citep{ministry_of_foreign_affairs}. These dangerous chemicals were found in oranges and passion fruit from A. Mart, as well as oilseed, lettuce and jujubes from Carrefour. Green beans from Costco were found to contain fungicide residue 69 times higher than the legal amount. Lo Ko-jung, the project manager at Greenpeace said: 
``The solution to pesticide residue in food relies on retailers proposing specific measures to ban pesticide-containing products, but Costco ignores the consumer’s right to know because it has refused to disclose its pesticide management policy'' \citep{ministry_of_foreign_affairs}.
Trade secrecy laws allow manufacturers to exclude specific ingredients or chemical quantities on labels if doing so would decrease profit—illustrated by Costco. 

Pesticides are a waste management issue because agriculture runoff and improper disposal release pesticides into wastewater flow, processed by treatment facilities lacking expensive pesticide removal technology; consequently, pesticides enter humans and the environment. If dangerous substances are passed untreated through wastewater treatment facilities, they are a waste management issue—this is regarded to be true for any synthetic chemical. Most countries in SE Asia lack a pesticide waste management framework, 80\% of pesticides are imported illegally and are uncontrolled, they are cheap and high in toxicity \citep{thuy2012current}. Pesticides are a linear commodity whose end-life produces toxic water, which accumulates in sediment, enters living organisms, or circulates through the hydrologic cycle \citep{thuy2012current}. Pesticides and other chemicals are regarded here as neither technical nor biological nutrients, as they pose a threat to humans and the environment alike. 

\subsection{Circular Biopesticides: An Alternative To Hazardous Synthetic Pesticides}

A circular alternative to pesticides are biopesticides and bioherbicides. They have been used since the 1950’s but are now gaining global attention as they provide a sustainable and safe alternative for agricultural needs stefanski2020potential \citep{hubbard2014biochemistry}. Biocontrol products are produced from naturally occuring funguses, microbes, plant metabolites, and biochemicals, which mitigate unwanted pests and weeds with little or no health/environmental effects. ``Biopesticides from fungi are employed to control weeds, beneficial bacterial pesticides are used to control fungal and bacterial disease and viral pesticides are used to resist insect pests” \citep{usta2013microorganisms}. The most effective biopesticides have been reported to be those which utilize compounds produced by microorganisms. Chemicals created by microbes produce toxic proteins, their production can be optimized through fermentation. Biopesticides target pests and do not negatively impact human health, animals or the environment, they are therefore circular and can be regarded as effective alternatives compared with synthetic pesticides. 

Their advantages and disadvantages include the following: biopesticides are not ``persistent,” they break down into the environment becoming part of the biosphere, thus creating a closed cycle. Because of their short life, more biopesticide application is required, this can potentially boost job markets. Bioinsecticides do not cause air or water quality problems. They often contain Bacillus thuringeinsis, a microbe which targets specific pests without harming other beneficial organisms (pollinators, humans, animals, etc). Nevertheless, if many pests are present, selective methods do not serve the intended purpose of a pesticide (however, given biopesticides come in multiple forms, this caveat may be remediated) \citep{mccoy2020organic}. The creation and application of biopesticides is often more complicated and their use requires more knowledge, for instance, nematodes must be refrigerated and used within two weeks for effectiveness. Thus, education is important for biopesticide implementation. Furthermore, biopesticide microbes utilize ``multiple modes of action” preventing pest resistance; this is advantageous as synthetic pesticides use ``single modes of action,” which pests become resistant to \citep{hubbard2014biochemistry}. While initially more expensive, natural pesticides do not cause health issues, thus saving one from suffering and health expenses in the long run. Additionally, biopesticides have a registration time of less than a year (because they are seen as less dangerous), compared with synthetics which call for three years \citep{mordor}.

More than 225 microbial biopesticides are currently being manufactured in 30 countries in the Organization of Economic Development and Cooperation. The US, Canada, and Mexico account for 45\% of all biopesticides sold; Asia accounts only for 6\% which demonstrates Asia’s lack of organic agriculture practice, yet presents an incredible opportunity for SE Asia to increase its health/environmental wellbeing, while also increasing economic prosperity \citep{hubbard2014biochemistry} \citep{mordor}. A shift in pesticide use is now occurring in Asia as countries like Vietnam are recognizing the advantages of using bioinsecticides. The biopesticide market is expected to reach USD 60.61 million by 2024 \citep{marketwatch_2021}. The International Rice Research Institute (IRRI) has called for a ban of certain synthetic pesticides, increasing awareness and demand for biopesticides in SE Asia. China was recently placed first in the field of biopesticides research and development worldwide \citep{mordor}. Demand for sustainable agriculture is building and biopesticides are an important part of the shift. Further research in biotechnology and chemistry can potentially help lower biopesticide prices and increase their presence throughout the world, especially in SE Asia \citep{hubbard2014biochemistry}.

\section{Further Research Needed: Linear Chemical Waste一``Forever Chemicals''}

An individual chemical can persist in the environment for long periods of time, moreover, they build up. The majority of these man made chemicals are polyfluoroalkyl substances (PFAS: PFOAs and PFOs) (different from pesticides): they are a broad class of more than 5,000 chemicals used in a great number of products including ``clothing, furniture, nonstick cookware, food packaging, firefighting foams, and electrical wire insulation,” furthermore, they are found in seafood, agriculture produce, and drinking water \citep{hersher_2019}. Linda Birnbaum, the director of the National Institute of Environmental Health Sciences and the National Toxicology Program, says: ``We're finding them not only in the environment, but we're finding them in people'' \citep{hersher_2019}. These persistent materials are called ``forever chemicals,” and are resilient to natural degradation processes––chemical, biological and photolytic––because of strong covalent bonds between carbon and fluorine atoms \citep{acs}. PFAS can dissolve in water, and because of their chemical properties, traditional wastewater treatment technologies are not able to remove them \citep{epa_2018} \citep{ipen}. These chemicals often enter into organisms through water or food, bioaccumulating in tissues and binding to proteins; as organisms are consumed, the chemicals cycle through the food chain, just like pesticides. In humans PFAS are often found in blood, liver and kidneys \citep{ipen}. However, recent studies have found PFAS contamination in breast milk in India, Indonesia, Japan, Sri Lanka (bood), Malaysia, and Vietnam which have been found in fetus’ \citep{ipen}.

Epidemiologists have found a ``probable link'' between long-term exposure to the PFAS chemicals and certain medical conditions such as: kidney cancer, thyroid disease, (female and male) infertility and developmental effects, lung defects, and more. These chemicals are also found in the blood, liver and kidneys of wildlife. Sewage treatment plants have been identified as major polluting sources of PFAS, evidenced by Japan, Thailand, Vietnam and other countries \citep{ipen}. These chemicals are linear, and circular solutions are deemed necessary to phase them out internationally \citep{land2018effect}. 

There are many valuable resources present in wastewater; however, these do not include pesticides and ``forever chemicals.” It has been 82 years since industrial chemicals (PFAS and pesticides) were first created, yet they are still causing harm to humans and the environment, and still people don't know what to do with them. Phasing them out and replacing them with alternatives is a priority.

\subsection{Household Products}

The average home throughout the developed world contains a multitude of cleaning products which may contain forever chemicals. Given cleaning products’ wide usage, they are often overlooked as dangerous to human and environmental health. When cleaners are washed down the drain by millions of people, even trace amounts become a serious problem. The problem extends to septic tanks and municipal sewage \citep{bown}. 

Municipal sludge, a valuable fertilizer, often cannot be repurposed because it’s laden with pharmaceuticals, chemicals and PFAS, which make people and the environment sick through fertilizer use \citep{perkins_2019}. Global wastewater can contain hormones, pathogens, heavy metals (lead, cadmium, arsenic and mercury), as well as chemicals: pesticides, pharmaceuticals, PCBs, PFAS, BPAs and other harmful substances \citep{perkins_2019}.  Effective wastewater treatment addresses the conglomerate of elements present in water. Given wastewater treatment plants in developed nations still emit dangerous chemicals, their expensive installation is not an advisable trajectory for developing nations to follow, rather a global phasing out of ``persistent” toxic substances is in order. Further research is needed to understand PFAS wastewater treatment and how to phase them out. 

\section{Conclusion}

The linear production model will not sustain life on Earth. Consumption is a necessity for life, but creating waste is not. Energy, which constitutes all things, can neither be created nor destroyed, it only changes form. It makes sense for humans to live in accordance with this law of the universe; circular systems are a way to do so. Circular economics is being pushed forward as a new alternative, nevertheless circular living has been iterated for thousands of years by first nation people, who to this day live in this way. Through biomimetic systems, society can exist in synergy with the environment. The circular economy is underdevelopment and has a ways to go, but as illustrated by case studies from SE Asia, bio-regenerative-wastewater-treatment-systems with ingredient recovery are fundamental to its realization. 

The double-vault-compost latrine, the anaerobic digester, and phytoremediation use natural microbes to treat wastewater, thus demonstrating how humans can design with nature's existing systems. Valuable nutrients: fertilizer, bioenergy, and metal, can all be recovered through engineered solutions informed by natural chemistry. Similarly, by understanding microorganisms, biopesticides are developed. These case studies demonstrate SE Asia and the world can benefit greatly by adopting bio-regenerative-wastewater-treatment-systems, increasing wellbeing, environmental protection, and economic prosperity (Figure ~\ref{fig:Circularity}).

Further research is needed to understand how to remediate PFAS and other toxic chemicals including pharmaceuticals and cleaners; to understand the low-cost construction details of decentralized anaerobic digesters and how they can be attached to DVC latrines, and septic tanks (Gihiandelli, 1; Mang, 1). Questions to be further researched: what circular biomimic innovations can be made for landfills; how can further research into interconnectivity resolve societies issues; how can existing linear systems and supply-chains be made circular? 


\begin{figure}
\includegraphics{images/Circular-Images/CircularEconomy_image.png}
\caption{A representation of the advantages of a Circular Economy}
\label{fig:Circularity}
\end{figure}



\backmatter

\part{Backmatter}

%The back matter often includes one or more of an index, an afterword, acknowledgments, a bibliography, a colophon, or any other similar item. In the back matter, chapters do not produce a chapter number, but they are entered in the table of contents. If you are not using anything in the back matter, you can delete the back matter TeX field and everything that follows it.

\printglossary

\renewcommand\bibname{References}
\setlength{\bibsep}{2\baselineskip}
\setlength\bibindent{.5in}
%\bibliographystyle{plainnat}
\bibliographystyle{ecology}
\bibliography{References,marctest}

\end{document}
